\subsection{Data reduction: Point data as input.}
\funclabel{s1966}
\begin{minipg1}
To compute a tensor-product spline-approximation of order
           (ik1,ik2) to the rectangular array of idim-dimensional
           points given by ep.
\end{minipg1} \\ \\
SYNOPSIS\\
        \> void s1966(\begin{minipg3}
            {\fov ep}, {\fov im1}, {\fov im2}, {\fov idim}, {\fov ipar}, {\fov epar1}, {\fov epar2}, {\fov eeps}, {\fov nend}, {\fov iopen1}, {\fov iopen2},
            {\fov edgeps}, {\fov afctol}, {\fov iopt}, {\fov itmax}, {\fov ik1}, {\fov ik2},
            {\fov rs}, {\fov emxerr}, {\fov jstat})
                \end{minipg3}\\
                \>\>    double \> {\fov ep}[\,];\\
                \>\>    int    \>  {\fov im1};\\
                \>\>    int    \>  {\fov im2};\\
                \>\>    int    \>  {\fov idim};\\
                \>\>    int    \>  {\fov ipar};\\
                \>\>    double \> {\fov epar1}[\,];\\
                \>\>    double \> {\fov epar2}[\,];\\
                \>\>    double \> {\fov eeps}[\,];\\
                \>\>    int    \>  {\fov nend}[\,];\\
                \>\>    int    \>  {\fov iopen1};\\
                \>\>    int    \>  {\fov iopen2};\\
                \>\>    double \> {\fov edgeps}[\,];\\
                \>\>    double \> {\fov afctol};\\
                \>\>    int    \>  {\fov iopt};\\
                \>\>    int    \>  {\fov itmax};\\
                \>\>    int    \>  {\fov ik1};\\
                \>\>    int    \>  {\fov ik2};\\
                \>\>    SISLSurf    \>  **{\fov rs};\\
                \>\>    double \> {\fov emxerr}[\,];\\
                \>\>    int    \>  *{\fov jstat};\\
\\
ARGUMENTS\\
	\>Input Arguments:\\
        \>\>    {\fov ep}\> - \>  \begin{minipg2}
                     Array (length idim*im1*im2) containing the points
                  to be approximated.
                               \end{minipg2}\\[0.8ex]
        \>\>    {\fov im1}\> - \>  \begin{minipg2}
                    The no. of points in the first parameter.
                               \end{minipg2}\\[0.8ex]
        \>\>    {\fov im2}\> - \>  \begin{minipg2}
                    The no. of points in the second parameter.
                               \end{minipg2}\\[0.8ex]
        \>\>    {\fov idim}\> - \>  \begin{minipg2}
                    The no. of components of each input point.
                  The approximation will be a parametric surface
                  situated in idim-dimensional Euclidean space
                  (usually 3).
                               \end{minipg2}\\[0.8ex]
        \>\>    {\fov ipar}\> - \>  \begin{minipg2}
                     Flag determining the parametrization of the data points:
                               \end{minipg2}\\[0.8ex]
          \>\>\>\>  $= 1$  : Mean accumulated cord-length parameterization.\\
          \>\>\>\>  $= 2$  : Uniform parametrization.\\
          \>\>\>\>  $= 3$  : Parametrization given by epar1 and epar2.\\
        \>\>    {\fov epar1}\> - \>  \begin{minipg2}
                     Array (length im1) containing a parametrization
                  in the first parameter. (Will only
                  be used if $ipar=3$).
                               \end{minipg2}\\[0.8ex]
        \>\>    {\fov epar2}\> - \>  \begin{minipg2}
                     Array (length im2) containing a parametrization
                  in the second parameter. (Will only
                  be used if $ipar=3$).
                               \end{minipg2}\\[0.8ex]
        \>\>    {\fov eeps}\> - \>  \begin{minipg2}
                     Array (length idim) containing the max. permissible
                  deviation of the approximation from the given data
                  points, in each of the components. More specifically,
                  the approximation will not deviate more than eeps(kdim)
                  in component no. kdim, from the bilinear approximation
                  to the data.
                               \end{minipg2}\\[0.8ex]
        \>\>    {\fov nend}\> - \>  \begin{minipg2}
                     Array (length 4) giving the no. of derivatives to be
                  kept fixed along each edge of the bilinear interpolant.
                  The numbering of the edges is the same as for edgeps below.
                  All the derivatives of order $< (nend(i)-1)$ will be kept
                  fixed along the edge $i$. Hence $nend(i)=0$ indicates that
                  nothing is to be kept fixed along edge $i$.
                  To be kept fixed here means to have error less than edgeps.
                  In general, it is impossible to remove any knots and keep
                  an edge completely fixed.
                               \end{minipg2}\\[0.8ex]
        \>\>   {\fov iopen1}\> - \> Open/closed parameter in first direction.\\
          \>\>\>\>  $= 1$  : Produce open surface.\\
          \>\>\>\>  $= 0$ : Produce closed, non-periodic surface if possible.\\
          \>\>\>\>  $= -1$ : Produce closed, periodic surface\\
        \>\>\>\>  \begin{minipg2}
                  NB! The surface will be closed/periodic only if the first 
                      and last column of data points are (approximately) equal.
                               \end{minipg2}\\[0.8ex]
        \>\>   {\fov iopen2}\> - \> Open/closed parameter in second direction.\\
          \>\>\>\>  $= 1$  : Produce open surface.\\
          \>\>\>\>  $= 0$ : Produce closed, non-periodic surface if possible.\\
          \>\>\>\>  $= -1$ : Produce closed, periodic surface\\
        \>\>\>\>  \begin{minipg2}
                  NB! The surface will be closed/periodic only if the first 
                      and last row of data points are (approximately) equal.
                               \end{minipg2}\\[0.8ex]
        \>\>    {\fov edgeps}\> - \>  \begin{minipg2}
                     Array (length idim*4) containing the max. deviation from
                  the bilinear interpolant which is acceptable along the
                  edges of the surface.
                  edgeps(1,i):edgeps(idim,i) gives the tolerance along
                  the edge corresponding to the i-th parameter having
                  one of it`s extremal-values.
                     \end{minipg2}\\[0.8ex]
          \>\>\>\> $i=1$: min value of first parameter.\\ 
          \>\>\>\> $i=2$: max value of first parameter.\\
          \>\>\>\> $i=3$: min value of second parameter.\\
          \>\>\>\> $i=4$: max value of second parameter.\\
          \>\>\>\>  \begin{minipg2}
                  edgeps(kp,i) will only have significance if $nend(i)>0$.
                               \end{minipg2}\\[0.8ex]
        \>\>    {\fov afctol}\> - \>  \begin{minipg2}
                     $0.0 >= afctol <= 1.0$.
                  Afctol indicates how the tolerance is to be shared
                  between the two data-reduction stages. For the linear
                  reduction, a tolerance of $afctol*eeps$ will be used,
                  while a tolerance of $(1.0-afctol)*eeps$ will be used
                  during the final data reduction (similarly for edgeps.)
                  Default is 0.
                               \end{minipg2}\\[0.8ex]
        \>\>    {\fov iopt}\> - \>  \begin{minipg2}
                  Flag indicating the order in which the data-reduction
                  is to be performed:
                               \end{minipg2}\\[0.8ex]
          \>\>\>\> $=1$: Remove knots in parameter 1 only.\\
          \>\>\>\> $=2$: Remove knots in parameter 2 only.\\
          \>\>\>\> $=3$: Remove knots first in parameter 1 and then in 2.\\
          \>\>\>\> $=4$: Remove knots first in parameter 2 and then in 1.\\
        \>\>    {\fov itmax}\> - \>  \begin{minipg2}
                    Max. no. of iterations in the data-reduction..
                               \end{minipg2}\\[0.8ex]
        \>\>    {\fov ik1}\> - \>  \begin{minipg2}
                     The order of the approximation in the first
                  parameter.
                               \end{minipg2}\\[0.8ex]
        \>\>    {\fov ik2}\> - \>  \begin{minipg2}
                     The order of the approximation in the second
                  parameter.
                               \end{minipg2}\\[0.8ex]
\\
	\>Output Arguments:\\
        \>\>    {\fov rs}\> - \>  \begin{minipg2}
                     Pointer to surface.
                               \end{minipg2}\\[0.8ex]
        \>\>    {\fov emxerr}\> - \>  \begin{minipg2}
                     Array (length idim) (allocated outside this routine.)
                  containing the error in the approximation to the data.
                  This is a guaranteed upper bound on the max. deviation
                  in each component, between the final approximation
                  and the bilinear spline- pproximation to the original data.
                               \end{minipg2}\\[0.8ex]
        \>\>    {\fov jstat}     \> - \> Status messages\\
                \>\>\>\>\>              $> 0$   : Warning.\\
                \>\>\>\>\>              $= 0$   : Ok.\\
                \>\>\>\>\>              $< 0$   : Error.\\
\\
EXAMPLE OF USE\\
		\>      \{ \\

                \>\>    double \> {\fov ep}[\,];\\
                \>\>    int    \>  {\fov im1};\\
                \>\>    int    \>  {\fov im2};\\
                \>\>    int    \>  {\fov idim};\\
                \>\>    int    \>  {\fov ipar};\\
                \>\>    double \> {\fov epar1}[\,];\\
                \>\>    double \> {\fov epar2}[\,];\\
                \>\>    double \> {\fov eeps}[\,];\\
                \>\>    int    \>  {\fov nend}[\,];\\
                \>\>    int    \>  {\fov iopen1};\\
                \>\>    int    \>  {\fov iopen2};\\
                \>\>    double \> {\fov edgeps}[\,];\\
                \>\>    double \> {\fov afctol};\\
                \>\>    int    \>  {\fov iopt};\\
                \>\>    int    \>  {\fov itmax};\\
                \>\>    int    \>  {\fov ik1};\\
                \>\>    int    \>  {\fov ik2};\\
                \>\>    SISLSurf    \>  **{\fov rs};\\
                \>\>    double \> {\fov emxerr}[\,];\\
                \>\>    int    \>  *{\fov jstat};\\                \>\>    \ldots \\
        \>\>s1966(\begin{minipg4}
            {\fov ep}, {\fov im1}, {\fov im2}, {\fov idim}, {\fov ipar}, {\fov epar1}, {\fov epar2}, {\fov eeps}, {\fov nend}, {\fov iopen1}, {\fov iopen2},
            {\fov edgeps}, {\fov afctol}, {\fov iopt}, {\fov itmax}, {\fov ik1}, {\fov ik2},
            {\fov rs}, {\fov emxerr}, {\fov jstat});
                \end{minipg4}\\
                \>\>    \ldots \\
		\>      \}
\end{tabbing}
