\subsection{Data reduction: Points and tangents as input.}
\funclabel{s1962}
\begin{minipg1}
To compute the approximation to the data given by the points
           ep and the derivatives (tangents) ev, and represent it as a
           B-spline curve with parametrization determined by the parameter
           ipar. The approximation is determined by first forming the cubic
           hermite interpolant to the data, and then performing knot
           removal on this initial approximation.
\end{minipg1} \\ \\
SYNOPSIS\\
        \> void s1962(\begin{minipg3}
            {\fov ep}, {\fov ev}, {\fov im}, {\fov idim}, {\fov ipar}, {\fov epar}, {\fov eeps}, {\fov ilend}, {\fov irend}, {\fov iopen}, {\fov itmax},
            {\fov rc}, {\fov emxerr}, {\fov jstat})
                \end{minipg3}\\
                \>\>    double \> {\fov ep}[\,];\\
                \>\>    double \> {\fov ev}[\,];\\
                \>\>    int    \>  {\fov im};\\
                \>\>    int    \>  {\fov idim};\\
                \>\>    int    \>  {\fov ipar};\\
                \>\>    double \> {\fov epar}[\,];\\
                \>\>    double \> {\fov eeps}[\,];\\
                \>\>    int    \>  {\fov ilend};\\
                \>\>    int    \>  {\fov irend};\\
                \>\>    int    \>  {\fov iopen};\\
                \>\>    int    \>  {\fov itmax};\\
                \>\>    SISLCurve    \>  **{\fov rc};\\
                \>\>    double \> {\fov emxerr}[\,];\\
                \>\>    int    \>  *{\fov jstat};\\
\\
ARGUMENTS\\
	\>Input Arguments:\\
        \>\>    {\fov ep}\> - \>  \begin{minipg2}
                     Array (length idim*im) comtaining the points
                  to be approximated.
                               \end{minipg2}\\[0.8ex]
        \>\>    {\fov ev}\> - \>  \begin{minipg2}
                     Array (length idim*im) containing the derivatives
                  of the points to be approximated.
                               \end{minipg2}\\[0.8ex]
        \>\>    {\fov im}\> - \>  \begin{minipg2}
                     The no. of data points.
                               \end{minipg2}\\
        \>\>    {\fov idim}\> - \>  \begin{minipg2}
                     The dimension of the euclidean space in which the
                  curve lies.
                               \end{minipg2}\\[0.8ex]
        \>\>    {\fov ipar}\> - \> \begin{minipg2}
                     Flag indicating the type of parameterization to be used:
                               \end{minipg2}\\[0.8ex]
                \>\>\>\>   $= 1$ : Paramterize by accumulated cord length.\\
                \>\>\>\>         (Arc length parametrization for the piecewise\\
                \>\>\>\>         linear interpolant.)\\
                \>\>\>\>   $= 2$ : Uniform parameterization.\\
                \>\>\>\>   $= 3$ : Parametrization given by epar.\\
                \>\>\>\>  If ipar $<1$  or ipar $>3$, it will be set to 1.\\
        \>\>    {\fov epar}\> - \>  \begin{minipg2}
                     Array (length im) containing a parameterization
                  of the given data.
                               \end{minipg2}\\[0.8ex]
        \>\>    {\fov eeps}\> - \>  \begin{minipg2}
                     Array (length idim) giving the desired accuracy of
                  the spline-approximation in each component.
                               \end{minipg2}\\[0.8ex]
        \>\>    {\fov ilend}\> - \>  \begin{minipg2}
                     The no. of derivatives that are not allowed to change
                 at the left end of the curve.
                 The $0,\ldots,(ilend-1)$ derivatives will be kept fixed.
                 If ilend $<0$, this routine will set it to 0.
                 If ilend $<ik$, this routine will set it to ik.
                               \end{minipg2}\\[0.8ex]
        \>\>    {\fov irend}\> - \>  \begin{minipg2}
                     The no. of derivatives that are not allowed to change
                 at the right end of the curve.
                 The $0,\ldots,(irend-1)$ derivatives will be kept fixed.
                 If irend $<0$, this routine will set it to 0.
                 If irend $<ik$, this routine will set it to ik.
                               \end{minipg2}\\[0.8ex]
        \>\>    {\fov iopen}\> - \>  Open/closed parameter\\
            \>\>\>\>  $= 1$  : Produce open curve.\\
            \>\>\>\>  $= 0$ : Produce closed, non-periodic curve if possible.\\
            \>\>\>\>  $= -1$ : Produce closed, periodic curve if possible.\\
                 \>\>\>\>  \begin{minipg2}
                 If a closed or periodic curve is to be produced and the
                 start- and endpoint is more distant than the length of
                 the tolerance, a new point is added. Note that if the
                 parametrization is given as input, the parametrization
                 if the last point will be arbitrary.
                                \end{minipg2}\\
        \>\>    {\fov itmax}\> - \>  \begin{minipg2}
                     Max. no. of iteration.
                               \end{minipg2}\\
\\
	\>Output Arguments:\\
        \>\>    {\fov rc}\> - \>  \begin{minipg2}
                     Pointer to curve.
                               \end{minipg2}\\
        \>\>    {\fov emxerr}\> - \>  \begin{minipg2}
                     Array (length idim) (allocated outside this routine.)
                  containing an upper bound for the pointwise error
                  in each of the components of the spline-approximation.
                               \end{minipg2}\\
        \>\>    {\fov jstat}     \> - \> Status messages\\
                \>\>\>\>\>              $> 0$   : Warning.\\
                \>\>\>\>\>              $= 0$   : Ok.\\
                \>\>\>\>\>              $< 0$   : Error.\\
\\
EXAMPLE OF USE\\
		\>      \{ \\

                \>\>    double \> {\fov ep}[\,];\\
                \>\>    double \> {\fov ev}[\,];\\
                \>\>    int    \>  {\fov im};\\
                \>\>    int    \>  {\fov idim};\\
                \>\>    int    \>  {\fov ipar};\\
                \>\>    double \> {\fov epar}[\,];\\
                \>\>    double \> {\fov eeps}[\,];\\
                \>\>    int    \>  {\fov ilend};\\
                \>\>    int    \>  {\fov irend};\\
                \>\>    int    \>  {\fov iopen};\\
                \>\>    int    \>  {\fov itmax};\\
                \>\>    SISLCurve    \>  **{\fov rc};\\
                \>\>    double \> {\fov emxerr}[\,];\\
                \>\>    int    \>  *{\fov jstat};\\                \>\>    \ldots \\
        \>\>s1962(\begin{minipg4}
            {\fov ep}, {\fov ev}, {\fov im}, {\fov idim}, {\fov ipar}, {\fov epar}, {\fov eeps}, {\fov ilend}, {\fov irend}, {\fov iopen}, {\fov itmax},
            {\fov rc}, {\fov emxerr}, {\fov jstat});
                \end{minipg4}\\
                \>\>    \ldots \\
		\>      \}
\end{tabbing}
