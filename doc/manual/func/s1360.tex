\subsection{Approximate the offset of a curve with a curve.}
\funclabel{s1360}
\begin{minipg1}
  To create a approximation of the offset to a curve within a
  tolerance.
  The output will be represented as a B-spline curve.\\
  With an offset of zero, this routine can be used to approximate any
  NURBS curve, within a tolerance, with a (non-rational) B-spline curve.
\end{minipg1} \\ \\
SYNOPSIS\\
        \>void s1360(\begin{minipg3}
        {\fov oldcurve}, {\fov offset}, {\fov epsge}, {\fov norm}, {\fov max}, {\fov dim}, {\fov newcurve}, {\fov stat})
                \end{minipg3}\\[0.3ex]
                \>\>    SISLCurve       \>      *{\fov oldcurve};\\
                \>\>    double  \>      {\fov offset};\\
                \>\>    double  \>      {\fov epsge};\\
                \>\>    double  \>      {\fov norm}[\,];\\
                \>\>    double  \>      {\fov max};\\
                \>\>    int     \>      {\fov dim};\\
                \>\>    SISLCurve       \>      **{\fov newcurve};\\
                \>\>    int     \>      *{\fov stat};\\
\\
ARGUMENTS\\
        \>Input Arguments:\\
        \>\>    {\fov oldcurve}\> - \> The input curve.\\
        \>\>    {\fov offset}   \> - \> \begin{minipg2}
                        The offset distance. If dim=2, a positive sign on
                        this value put the offset on the side of the positive
                        normal vector, and a negative sign puts the offset on
                        the negative normal vector. If dim=3, the offset direction is
                        determined by the cross product of the tangent
                        vector and the normal vector. The offset distance is
                        multiplied by this cross product.
                                \end{minipg2}\\[0.8ex]
        \>\>    {\fov epsge}    \> - \> \begin{minipg2}
                        Maximal deviation allowed between the true offset
                        curve and the approximated offset curve.
                                \end{minipg2}\\[0.3ex]
        \>\>    {\fov norm}     \> - \> Vector used in 3D calculations.\\
        \>\>    {\fov max}      \> - \> \begin{minipg2}
                        Maximal step length. It is neglected if
                        max$\leq$epsge. If max=0.0, then a maximal step
                        equal to the longest box side of the curve is used.
                                \end{minipg2}\\[0.8ex]
        \>\>    {\fov dim}      \> - \>The dimension of the space must be 2 or 3.\\
\\
NOTE\\
\>\begin{minipg6}
  If the vector norm and the curve tangent are parallel at some point,
  then the curve produced will not be an offset at this point, and it
  will probably move from one side of the input curve to the other side.
\end{minipg6}\\
\newpagetabs
        \>Output Arguments:\\
        \>\>    {\fov newcurve}\> - \> \begin{minipg2}
                        Pointer to the B-spline curve
                        approximating the offset curve.
                                       \end{minipg2}\\[0.8ex]
        \>\>    {\fov stat}     \> - \> Status messages.\\
                \>\>\>\>\>              $> 0$   : Warning.\\
                \>\>\>\>\>              $= 0$   : Ok.\\
                \>\>\>\>\>              $< 0$   : Error.\\
\\
EXAMPLE OF USE\\
                \>      \{ \\
                \>\>    SISLCurve       \>      *{\fov oldcurve};\\
                \>\>    double  \>      {\fov offset};\\
                \>\>    double  \>      {\fov epsge};\\
                \>\>    double  \>      {\fov norm}[3];\\
                \>\>    double  \>      {\fov max};\\
                \>\>    int     \>      {\fov dim} = 3;\\
                \>\>    SISLCurve       \>      *{\fov newcurve};\\
                \>\>    int     \>      {\fov stat};\\
                \>\>    \ldots \\
        \>\>s1360(\begin{minipg4}
        {\fov oldcurve}, {\fov offset}, {\fov epsge}, {\fov norm}, {\fov max}, {\fov dim}, \&{\fov newcurve}, \&{\fov stat});
                        \end{minipg4}\\
                \>\>    \ldots \\
                \>      \} \\
\end{tabbing}
