\subsection{Draw a sequence of straight lines.}
\funclabel{s6drawseq}
\begin{minipg1}
        Draw a broken line as a sequence of straight lines
        described \\ by the array points.
        For dimension 3.
\end{minipg1} \\ \\
SYNOPSIS\\
        \>void s6drawseq(\begin{minipg3}
                {\fov points}, {\fov numpoints})
                \end{minipg3}\\[0.3ex]
                \>\>    double \>       {\fov points}[\,];\\
                \>\>    int    \>       {\fov numpoints};\\
\\
ARGUMENTS\\
        \>Input Arguments:\\
        \>\>    {\fov points}   \> - \> \begin{minipg2}
                                Points stored in sequence.
                                i.e.\ \\
                                $(x_{0},y_{0},z_{0},x_{1},y_{1},z_{1},\ldots)$.
                                \end{minipg2}\\[0.8ex]
        \>\>    {\fov numpoints}\> - \> \begin{minipg2}
                                Number of points in the sequence.
                                \end{minipg2}\\
\\
NOTE\\
\>      \begin{minipg6}
s6drawseq() is device dependent, it calls the empty dummy functions s6move() and
s6line().
Before using it, make sure you have a version of these two functions
interfaced to your graphic package.\\
More about s6move() and s6line() on pages~\pageref{s6move} and~\pageref{s6line}.
\end{minipg6} \\ \\
EXAMPLE OF USE\\
        \>      \{ \\
        \>\>    double \>       {\fov points}[30];\\
        \>\>    int    \>       {\fov numpoints} = 10;\\
        \>\> \ldots \\
        \>\>s6drawseq(\begin{minipg4}
          {\fov points}, {\fov numpoints})
        \end{minipg4}\\
        \>\>    \ldots \\
        \>      \}
\end{tabbing}
