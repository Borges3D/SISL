\subsection{Compute a fillet curve based on points.}
\funclabel{s1608}
\begin{minipg1}
  To calculate a fillet curve between two curves. Points indicate
  between which points on the input curve the fillet is to be produced.
  The output is represented as a B-spline curve.
\end{minipg1} \\ \\
SYNOPSIS\\
        \>void s1608(\begin{minipg3}
        {\fov curve1}, {\fov curve2}, {\fov epsge}, {\fov point1}, {\fov startpt1}, {\fov point2}, {\fov endpt2},
        {\fov filltype}, {\fov dim}, {\fov order}, {\fov newcurve}, {\fov parpt1}, {\fov parspt1}, {\fov parpt2},
        {\fov parept2}, {\fov stat})
                \end{minipg3}\\[0.3ex]
                \>\>    SISLCurve       \>      *{\fov curve1};\\
                \>\>    SISLCurve       \>      *{\fov curve2};\\
                \>\>    double  \>      {\fov epsge};\\
                \>\>    double  \>      {\fov point1}[\,];\\
                \>\>    double  \>      {\fov startpt1}[\,];\\
                \>\>    double  \>      {\fov point2}[\,];\\
                \>\>    double  \>      {\fov endpt2}[\,];\\
                \>\>    int     \>      {\fov filltype};\\
                \>\>    int     \>      {\fov dim};\\
                \>\>    int     \>      {\fov order};\\
                \>\>    SISLCurve       \>      **{\fov newcurve};\\
                \>\>    double  \>      *{\fov parpt1};\\
                \>\>    double  \>      *{\fov parspt1};\\
                \>\>    double  \>      *{\fov parpt2};\\
                \>\>    double  \>      *{\fov parept2};\\
                \>\>    int     \>      *{\fov stat};\\
\\
ARGUMENTS\\
        \>Input Arguments:\\
        \>\>    {\fov curve1}   \> - \> The first input curve.\\
        \>\>    {\fov curve2}   \> - \> The second input curve.\\
        \>\>    {\fov epsge}    \> - \> Geometry resolution.\\
        \>\>    {\fov point1}   \> - \> \begin{minipg2}
                        Point close to curve 1 indicating that the part of the
                        curve lying on this side of startpt1 is
                        not to be replaced by the fillet.
                                \end{minipg2}\\[0.3ex]
        \>\>    {\fov startpt1}\> - \>  \begin{minipg2}
                        Point close to curve 1, indicating where the fillet is
                        to start. The tangent at the start of the fillet will
                        have the same orientation as the curve
                        from point1 to startpt1.
                                \end{minipg2}\\[0.3ex]
        \>\>    {\fov point2}   \> - \> \begin{minipg2}
                        Point close to curve 2 indicating that the part of the
                        curve lying on this side of endpt2 is not
                        to be replaced by the fillet.
                                \end{minipg2}\\[0.8ex]
        \>\>    {\fov endpt2}   \> - \> \begin{minipg2}
                        Point close to curve two, indicating where the fillet
                        is to end. The tangent at the end of the fillet will
                        have the same orientation as the curve
                        from endpt2 to point2.
                                \end{minipg2}\\[0.3ex]
\newpagetabs
        \>\>    {\fov filltype}\> - \>  Indicator of type of fillet.\\
                \>\>\>\>\>      $=1$ : \>\begin{minipg5}
                                Circle, interpolating tangent on first
                                curve, not on curve 2.
                                \end{minipg5}\\[0.3ex]
                \>\>\>\>\>      $=2$ : \>\begin{minipg5}
                                Conic if possible,
                                \end{minipg5}\\
                \>\>\>\>\>      else : \>\begin{minipg5}
                                polynomial segment.
                                \end{minipg5}\\
        \>\>    {\fov dim}      \> - \> Dimension of space.\\
        \>\>    {\fov order}    \> - \> Order of fillet curve, which is not always used.\\
\\
        \>Output Arguments:\\
        \>\>    {\fov newcurve}\> - \> Pointer to the B-spline fillet curve.\\
        \>\>    {\fov parpt1}   \> - \> \begin{minipg2}
                        Parameter value of point {\fov point1} on curve 1.
                                \end{minipg2}\\
        \>\>    {\fov parspt1}  \> - \> \begin{minipg2}
                        Parameter value of point {\fov startpt1} on curve 1.
                                \end{minipg2}\\
        \>\>    {\fov parpt2}   \> - \> \begin{minipg2}
                        Parameter value of point {\fov point2} on curve 2.
                                \end{minipg2}\\
        \>\>    {\fov parept2}  \> - \> \begin{minipg2}
                        Parameter value of point {\fov endpt2} on curve 2.
                                \end{minipg2}\\
        \>\>    {\fov stat}     \> - \> Status messages\\
                \>\>\>\>\>              $> 0$   : warning\\
                \>\>\>\>\>              $= 0$   : ok\\
                \>\>\>\>\>              $< 0$   : error\\
\\
EXAMPLE OF USE\\
                \>      \{ \\
                \>\>    SISLCurve       \>      *{\fov curve1};\\
                \>\>    SISLCurve       \>      *{\fov curve2};\\
                \>\>    double  \>      {\fov epsge};\\
                \>\>    double  \>      {\fov point1}[3];\\
                \>\>    double  \>      {\fov startpt1}[3];\\
                \>\>    double  \>      {\fov point2}[3];\\
                \>\>    double  \>      {\fov endpt2}[3];\\
                \>\>    int     \>      {\fov filltype};\\
                \>\>    int     \>      {\fov dim} = 3;\\
                \>\>    int     \>      {\fov order};\\
                \>\>    SISLCurve       \>      *{\fov newcurve};\\
                \>\>    double  \>      {\fov parpt1};\\
                \>\>    double  \>      {\fov parspt1};\\
                \>\>    double  \>      {\fov parpt2};\\
                \>\>    double  \>      {\fov parept2};\\
                \>\>    int     \>      {\fov stat};\\
                \>\>    \ldots \\
        \>\>s1608(\begin{minipg4}
        {\fov curve1}, {\fov curve2}, {\fov epsge}, {\fov point1}, {\fov startpt1}, {\fov point2}, {\fov endpt2},\linebreak[4]
        {\fov filltype}, {\fov dim}, {\fov order}, \&{\fov newcurve}, \&{\fov parpt1}, \&{\fov parspt1},
        \linebreak[4] \&{\fov parpt2}, \&{\fov parept2}, \&{\fov stat});
                        \end{minipg4}\\
                \>\>    \ldots \\
                \>      \} \\
\end{tabbing}
