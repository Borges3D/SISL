\subsection{Basic graphics routine template - move plotting position.}
\funclabel{s6move}
\begin{minipg1}
  Move the graphics plotting position to a 3D point.
\end{minipg1}\\ \\
SYNOPSIS\\
        \>void s6move(\begin{minipg3}
          {\fov point})
        \end{minipg3}\\[0.3ex]
        \>\>    double \>       {\fov point}[\,];\\
\\
ARGUMENTS\\
        \>Input Arguments:\\
        \>\>    {\fov point}   \> - \> \begin{minipg2}
                                A 3D point, i.e.\ $(x,y,z)$, to move the
                                graphics plotting position to.
                                \end{minipg2}\\[0.8ex]
\\
NOTE\\
\>      \begin{minipg2}
The functionality of s6move() is device dependent, so it is only an empty
({\tt printf()} call) dummy routine. Before using it, make sure you have
a version of s6move() interfaced to your graphic package.
\end{minipg2} \\ \\
EXAMPLE OF USE\\
        \>      \{ \\
        \>\>    double \>       {\fov point}[3];\\
        \>\> \ldots \\
        \>\>s6move(\begin{minipg4}
          {\fov point})
        \end{minipg4}\\
        \>\>    \ldots \\
        \>      \}
\end{tabbing}
