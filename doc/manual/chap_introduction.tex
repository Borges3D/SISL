\chapter{Introduction}
\label{introduction}
SISL is a geometric toolkit to model with curves and surfaces. It is a
library of C functions to perform operations such as the definition,
intersection and evaluation of NURBS (Non-Uniform Rational B-spline)
geometries. Since many applications use implicit geometric
representation such as planes, cylinders, tori etc., SISL can also
handle the interaction between such geometries and NURBS.

\medskip
Throughout this manual, a distinction is made between NURBS (the
default) and B-splines. The term B-splines is used for non-uniform
non-rational (or polynomial) B-splines. B-splines are used only where it
does not make sense to employ NURBS (such as the approximation of a
circle by a B-spline) or in cases where the research
community has yet to develop stable technology for treating NURBS.
A NURBS require more memory space than a B-spline, even when the
extra degrees of freedom in a NURBS are not used. Therefore the routines
are specified to give B-spline output whenever the extra degrees of
freedom are not required.

Transferring a B-spline into NURBS format is done by constructing a new
coefficient vector using the original B-spline coefficients and setting
all the rational weights equal to one (1).
This new coefficient vector is then given as input to the routine for
creating a new curve/surface object while specifying that the object to
be created should be of the NURBS (rational B-spline) type.

To approximate a NURBS by a B-spline, use the offset calculation
routines with an offset of zero.

The routines in SISL are designed to function on curves and surfaces
which are at least continuously differentiable. However many routines
will also handle continuous curves and surfaces, including piecewise
linear ones.

\medskip
SISL is divided into seven modules, partly in order to provide a logical
structure, but also to enable users with a specific application to use
subsets of SISL. There are three modules dealing with curves, three with
surfaces, and one module to perform data reduction on curves and
surfaces (this last module is largely in Fortran). The modules for
curves and surfaces focus on functions for creation and definition,
intersection and interrogation, and general utilities.

The three important data structures used by SISL are SISLCurve,
SISLSurf, and SISLIntcurve. These are defined in the Curve Utilities,
Surface Utilities, and Surface Interrogation modules respectively. It is
important to remember to always free these structures and also to free
internally allocated structures used to pass results to the application,
otherwise strange errors might result.

Each chapter in this manual contains information concerning the top
level functions of each module. Lower level functions not usually
required by an application are not included. Each top level function is
documented by describing the purpose, the input and output arguments and
an example of use. To get you started, this chapter contains an Example
Program.

%\vfill
%\newpage

\section{\label{syntax}C Syntax Used in Manual}
This manual uses the K\&R style C syntax for historic reasons, but both
the ISO/ANSI and the K\&R C standards are supported by the library and
the include files.

\section{\label{dynamic}Dynamic Allocation in SISL}
In the description of all the functions in this manual, a
convention exists on when to declare or allocate arrays/objects outside a
function and when an array is allocated internally.
{\em NB! When memory for output arrays/objects are allocated inside a function you
must remember to free the allocated memory when it is not in use any
more.}

The convention is the following:
\begin{itemize}
\item If $[\,]$ is used in the synopsis and in the example it means
that the array has to be declared or allocated outside the function.
\item If $*$ is used it means that the function requires a
pointer and that the allocation will be done outside the function if necessary.
\item When either an array or an array of pointers or an object is to be
allocated in a function, two or three stars are used in the
synopsis.
To use the function you declare the parameter with one star less and use  \&
in the argument list.
\item For all output variables except arrays or objects
that are declared or allocated  outside the function you have to use \&
in the argument list.
\end{itemize}


\vfill
\newpage
\section{Creating a Program}

In order to access SISL from your program you need only one inclusion, namely
the header file sisl.h. The statement
\begin{verbatim}
#include "sisl.h"
\end{verbatim}
must be written at the top of your main program.
In this header file all types
are defined.
It also contains all the
SISL top level function declarations.


To compile the calling program you merely need to remember to include
the name of the directory where sisl.h resides.
For example, if the directory is called sisldir then,
\begin{verbatim}
$ cc -c -Isisldir prog1.c
\end{verbatim}
will compile the source code prog1.c to produce prog1.o.

In order to build the executable, the $c$ parts of the
SISL library libsislc.a must be included. Thus
\begin{verbatim}
$ cc prog1.o -Lsisldir -lsisl -o prog1
\end{verbatim}
will build the test program prog1. See the next section for an example.

\newpage
\section{An Example Program}

To clarify the previous section here is an example program designed to
test the SISL algorithm for intersecting a cone with
a B-spline curve. The program calls the SISL routines newCurve() and s1373().

\begin{verbatim}
#include "sisl.h"


main()
{

SISLCurve *pc=NULL;

double aepsco,aepsge,top[3],axispt[3],conept[3];

double st[100],stcoef[100],*spar;

int kstat;
int cone_exists=FALSE;

int kk,kn,kdim,ki;
int kpt,kcrv;
SISLIntcurve **qrcrv;
char ksvar[100];

kdim=3;

aepsge=0.001;   /* geometric tolerance */
aepsco=0.000001;   /* computational tolerance */

loop:
printf("\n     cu - define a new B-spline curve");
printf("\n     co - define a new cone");
printf("\n     i  - intersect the B-spline curve with the cone");
printf("\n     q  - quit");
printf("\n> ");
scanf("%s",ksvar);



if (ksvar[0] == 'c' && ksvar[1] == 'u')
{

    printf("\n Give number of vertices, order of curve: ");
    scanf("%d %d", &kn, &kk);
    printf("Give knots values in ascending order: \n");

    for (ki=0;ki<kn+kk;ki++)
    {
        scanf("%lf",&st[ki]);
    }

    printf("Give vertices \n");

    for (ki=0;ki<kn*kdim;ki++)
    {
        scanf("%lf",&stcoef[ki]);
    }

    if(pc) freeCurve(pc);
    pc = newCurve(kn,kk,st,stcoef,1,kdim,1);

}
else if (ksvar[0] == 'c' && ksvar[1] == 'o')
{

    printf("\n Give top point: ");
    scanf("%lf %lf %lf",&top[0],&top[1],&top[2]);

    printf("\n Give a point on the axis: ");
    scanf("%lf %lf %lf",&axispt[0],&axispt[1],&axispt[2]);

    printf("\n Give a point on the cone surface: ");
    scanf("%lf %lf %lf",&conept[0],&conept[1],&conept[2]);

    cone_exists=TRUE;

}


else if (ksvar[0] == 'i' && cone_exists && pc)
{

    s1373(pc,top,axispt,conept,kdim,aepsco,aepsge,
      &kpt,&spar,&kcrv,&qrcrv,&kstat);

    printf("\n kstat %d",kstat);
    printf("\n kpt   %d",kpt);
    printf("\n kcrv  %d",kcrv);
    for (ki=0;ki<kpt;ki++)
    {
        printf("\nIntersection point %lf",spar[ki]);
    }
    if (spar)
    {
        free (spar);
        spar=NULL;
    }
    if (qrcrv)
    {
        freeIntcrvlist(qrcrv,kcrv);
        qrcrv=NULL;
    }
}


else if (ksvar[0] == 'q')
{
    return;
}

goto loop;
}
\end{verbatim}
Note the include statement.
\vfill
\newpage
The program was compiled and built using the commands:
\begin{verbatim}
$ cc -c -Isisldir prog1.c
$ cc prog1.o -Lsisldir -lsisl -o prog1
\end{verbatim}

A sample run of prog1 went as follows:
\begin{verbatim}
$ prog1

     cu - define a new B-spline curve
     co - define a new cone
     i  - intersect the B-spline curve with the cone
     q  - quit
> cu

 Give number of vertices, order of curve: 2 2
Give knots values in ascending order:
0 0 1 1
Give vertices
1 0 0.5
-1 0 0.5

     cu - define a new B-spline curve
     co - define a new cone
     i  - intersect the B-spline curve with the cone
     q  - quit
> co

 Give top point: 0 0 1

 Give a point on the axis: 0 0 0

 Give a point on the cone surface: 1 0 0

     cu - define a new B-spline curve
     co - define a new cone
     i  - intersect the B-spline curve with the cone
     q  - quit
> i

 kstat 0
 kpt   2
 kcrv  0
Intersection point 0.250000
Intersection point 0.750000
     cu - define a new B-spline curve
     co - define a new cone
     i  - intersect the B-spline curve with the cone
     q  - quit
> q
$
\end{verbatim}
SISL found two intersection points given by the parameters
$0.25$ and $0.75$. These parameters correspond to the 3D points
$(-0.5,0,0.5)$ and $(0.5,0,0.5)$ (which could be found by calling
the evaluation routine s1221()). They lie on both
the B-spline curve and the cone --- as expected!

\section{B-spline Curves}

This section is optional reading for those who want to
become acquainted with some of the mathematics of
B-splines curves. For a description of the data structure for
B-spline curves in SISL, see section \ref{curveobject}.

A B-spline curve is defined by the formula
$$ {\bf c}(t) = \sum_{i=1}^{n} {\bf p}_{i} B_{i,k,{\bf t}}(t). $$
The dimension of the curve ${\bf c}$ is equal to that of its
{\it control points} ${\bf p}_i$. For example, if the dimension of the
control points
is one, the curve is a function, if the dimension is two,
the curve is planar, and if the dimension is three,
the curve is spatial.
Usually the dimension of the curve will be at most three,
but SISL also allows higher dimensions.

Thus, a B-spline curve is a linear combination of a sequence of B-splines
$B_{i,k,{\bf t}}$ (called a B-basis)
uniquely determined by a knot vector ${\bf t}$ and
the order $k$. Order is equivalent to polynomial degree plus one.
For example, if the order is two, the degree is one and the B-splines
and the curve $c$ they generate are (piecewise) linear.
If the order is three, the degree is two and the B-splines and
the curve are quadratic. Cubic B-splines and
curves have order 4 and degree 3, etc.

The parameter range of a B-spline curve ${\bf c}$ is the interval
$$ [t_k, t_{n+1}], $$
and so mathematically, the curve is a mapping
${\bf c}: [t_k, t_{n+1}] \to \RR^d$, where $d$ is the Euclidean space
dimension of its control points.

The complete representation of a B-spline curve consists of
\begin{description}
\item[$dim$]: The dimension of the underlying Euclidean space,
              $1,2,3,\ldots$.
\item[$n$]: The number of vertices (also the number of B-splines)
\item[$k$]: The order of the B-splines.
\item[${\bf t}$]: The knot vector of the B-splines.
            ${\bf t} = (t_1, t_2, \ldots, t_{n+k})$.
\item[${\bf p}$]: The control points of the B-spline curve.
           $p_{d,i}\;,\; d=1,\ldots,dim\;,\;
                i=1,\ldots,n.\;\;$
                e.g. when $dim = 3$, we have
                ${\bf p} = (x_1,y_1,z_1,x_2,y_2,z_2,\ldots,x_n,y_n,z_n)$.
\end{description}

We note that arrays in $c$ start at index 0 which means,
for example, that if the array $t$ holds the knot vector,
then $t[0] = t_1,\ldots, t[n+k-1] = t_{n+k}$
and the parameter interval goes from
$t[k-1]$ to $t[n]$. Similar considerations apply to the other arrays.

The data in the representation must satisfy certain conditions:
\begin{itemize}
\item The knot vector must be non-decreasing: $t_i \le t_{i+1}$.
      Moreover, two knots $t_i$ and $t_{i+k}$ must be distinct:
      $t_i < t_{i+k}$.
\item The number of vertices should be greater than or equal
      to the order of the curve: $n \ge k$.
\item There should be $k$ equal knots at the beginning and at the end
      of the knot vector; that is the knot vector ${\bf t}$
      must satisfy the conditions $t_1 = t_2 = \ldots = t_k$
      and $t_{n+1} = t_{n+2} = \ldots = t_{n + k}$.
\end{itemize}

To understand the representation better, we will look at
three parts of the
representation: the B-splines (the basis functions),
the knot vector and the control polygon.

\subsection{B-splines}

A set of B-splines is determined by the order $k$
and the knots. For example,
to define a single B-spline of degree one, we need three knots.
In figure~\ref{curve1} the three knots are marked as dots.
Knots can also be equal as shown in figure
\ref{curve2}.
By taking a linear combination of the three types of B-splines shown
in figures~\ref{curve1} and~\ref{curve2}
we can generate a linear spline function
as shown in figure~\ref{curve3}.
\begin{figure}
        \begin{center}
                \begin{picture}(250,80)(0,0)
                \put(80,0){\circle*{2}}
                \put(150,0){\circle*{2}}
                \put(230,0){\circle*{2}}

                \put(30,4){\vector(0,1){76}}
                \put(28,10){\line(1,0){4}}
                \put(28,70){\line(1,0){4}}
                \put(20,10){\makebox(0,0){0.0}}
                \put(20,70){\makebox(0,0){1.0}}

                \thicklines
                \put(80,10){\line(6,5){70}}
                \put(150,70){\line(4,-3){80}}
                \end{picture}\\
        \end{center}
  \caption{\label{curve1}A linear B-spline (order 2) defined by
                        three knots.}
\end{figure}
\begin{figure}
        \begin{center}
                \begin{picture}(300,85)(0,0)
                \put(40,0){\circle*{2}}
                \put(280,0){\circle*{2}}
                \put(120,5){\circle*{2}}
                \put(200,5){\circle*{2}}
                \put(40,5){\circle*{2}}
                \put(280,5){\circle*{2}}

                \put(20,9){\vector(0,1){76}}
                \put(18,15){\line(1,0){4}}
                \put(18,75){\line(1,0){4}}
                \put(10,15){\makebox(0,0){0.0}}
                \put(10,75){\makebox(0,0){1.0}}

                \put(170,9){\vector(0,1){76}}
                \put(168,15){\line(1,0){4}}
                \put(168,75){\line(1,0){4}}
                \put(160,15){\makebox(0,0){0.0}}
                \put(160,75){\makebox(0,0){1.0}}

                \thicklines
                \put(120,15){\line(-4,3){80}}
                \put(200,15){\line(4,3){80}}
                \end{picture}\\
        \end{center}
  \caption{\label{curve2}Linear B-splines of with multiple knots at one end.}
\end{figure}
\begin{figure}
        \begin{center}
                \begin{picture}(300,90)(0,0)
                \put(0,0){\circle*{2}}
                \put(0,5){\circle*{2}}
                \put(60,5){\circle*{2}}
                \put(120,5){\circle*{2}}
                \put(160,5){\circle*{2}}
                \put(240,5){\circle*{2}}
                \put(300,5){\circle*{2}}
                \put(300,0){\circle*{2}}

                \multiput(0,10)(12,10){5}{\line(6,5){10}}
                \multiput(60,10)(-12,8){5}{\line(-3,2){10}}
                \multiput(60,10)(12,2){5}{\line(6,1){10}}
                \multiput(120,10)(-12,10){5}{\line(-6,5){10}}
                \multiput(120,10)(12,6){3}{\line(2,1){10}}
                \multiput(160,10)(-12,3){3}{\line(-4,1){10}}
                \multiput(160,10)(12,12){6}{\line(1,1){10}}
                \multiput(240,10)(-12,3){6}{\line(-4,1){10}}
                \multiput(240,10)(12,12){5}{\line(1,1){10}}
                \multiput(300,10)(-12,16){5}{\line(-3,4){10}}

                \thicklines
                \put(0,50){\line(6,1){60}}
                \put(60,60){\line(3,-2){60}}
                \put(120,20){\line(4,1){40}}
                \put(160,30){\line(4,3){80}}
                \put(240,90){\line(3,-1){60}}
                \end{picture}\\
        \end{center}
  \caption{\label{curve3}A B-spline curve of dimension 1 as a linear
            combination of a sequence of B-splines.
            Each B-spline (dashed) is scaled by a coefficient.}
\end{figure}

A quadratic B-spline is a linear combination of two linear
B-splines. Shown in figure~\ref{curve4}
is a quadratic B-spline defined by four knots.
A quadratic B-spline is
the sum of two products, the first product between the linear B-spline
on the left and a corresponding line from 0 to 1,
the second product between the linear B-spline
on the right and a corresponding line from 1 to 0;
see figure~\ref{curve4}.
For higher degree B-splines there is a similar definition.
A B-spline of order $k$ is the sum of two B-splines of
order $k-1$, each weighted with weights in the interval [0,1].
In fact we define B-splines of order 1 explicitly as box functions,
$$  B_{i,1}(t) =  \cases{1 & if $t_i \le t < t_{i+1}$; \cr
                              0 &  otherwise, \cr } $$
and then the complete definition of a $k$-th order B-spline is
$$ B_{i,k}(t) = {t - t_i \over t_{i+k-1} - t_i} B_{i,k-1}(t)
             +
             {t_{i+k} - t \over t_{i+k} - t_{i+1}} B_{i-1,k-1}(t). $$

\begin{figure}
        \begin{center}
                \begin{picture}(250,70)(0,0)

                \put(10,4){\vector(0,1){76}}
                \put(8,10){\line(1,0){4}}
                \put(8,70){\line(1,0){4}}
                \put(0,10){\makebox(0,0){0.0}}
                \put(0,70){\makebox(0,0){1.0}}

                \put(35,0){\circle*{2}}
                \put(95,0){\circle*{2}}
                \put(155,0){\circle*{2}}
                \put(215,0){\circle*{2}}

                \multiput(35,10)(18,9){7}{\line(2,1){10}}
                \multiput(215,10)(-18,9){7}{\line(-2,1){10}}

                \thicklines
                \put(35,10){\line(1,1){60}}
                \put(95,10){\line(1,1){60}}
                \put(155,10){\line(-1,1){60}}
                \put(215,10){\line(-1,1){60}}

                \bezier{70}(35,10)(65,10)(95,40)
                \bezier{70}(95,40)(125,70)(155,40)
                \bezier{70}(215,10)(185,10)(155,40)
                \end{picture}\\
        \end{center}
  \caption{\label{curve4}A quadratic B-spline, the two linear
           B-splines and the corresponding lines (dashed)
           in the quadratic B-spline definition.}
\end{figure}

B-splines satisfy some important properties for curve and surface design.
Each B-spline is non-negative and it can be shown that they sum to
one,
$$ \sum_{i=1}^{n}B_{i,k,{\bf t}}(t) = 1. $$
These properties combined mean that B-spline curves
satisfy the {\it convex hull property}: the curve lies in the convex
hull of its control points.
Furthermore, the support of the B-spline $B_{i,k,{\bf t}}$ is
the interval $[t_i,t_{i+k}]$ which means that B-spline
curves has {\it local control}: moving one control point only
alters the curve locally.

Due to the demand of $k$ multiple knots at the ends of the knot
vector, B-spline curves in SISL also have the {\it endpoint property}:
the start point of the B-spline curve
equals the first control point and the end point equals the
last control point, in other words
$$ {\bf c}(t_k) = {\bf p}_1
     \qquad \hbox{and} \qquad
   {\bf c}(t_{n+1}) = {\bf p}_n. $$

\subsection{\label{contrlpoly}The Control Polygon}

The control points ${\bf p}_i$ define the vertices
The {\it control polygon} of a B-spline curve is the polygonal
arc formed by its control points,
${\bf p}_0, {\bf p}_1, \ldots,{\bf p}_n$.
This means that
the control polygon, regarded as a parametric curve,
is itself piecewise linear B-spline curve (order two).
If we increase the order, the distance between the control polygon
and the curve increases (see figure~\ref{curve5}).
A higher order B-spline curve tends to smooth
the control polygon and at the same time mimic its shape.
For example, if the control polygon is convex, so is the B-spline curve.

Another property of the control polygon is that it will get closer
to the curve if it is redefined by inserting knots into the curve
and thereby increasing the number of vertices;
see figure~\ref{curve6}.
If the refinement is infinite then the control polygon converges to the curve.
\begin{figure}
        \begin{center}
                \begin{picture}(240,80)(0,0)
                \thicklines
                \put(0,40){\line(2,1){80}}
                \put(80,80){\line(1,-1){80}}
                \put(160,0){\line(1,1){80}}

                \bezier{120}(0,40)(80,80)(120,40)
                \bezier{120}(120,40)(160,0)(240,80)

                \bezier{120}(0,40)(60,70)(120,40)
                \bezier{120}(120,40)(180,20)(240,80)

                \end{picture}\\
        \end{center}
  \caption{\label{curve5}Linear, quadratic, and cubic B-spline
           curves sharing the same control polygon. The control polygon is
                equal to the linear B-spline curve. The curves
                are planar, i.e. the space dimension is two.}
\end{figure}

\begin{figure}
        \begin{center}
                \begin{picture}(240,80)(0,0)
                \thicklines
                \put(0,40){\line(2,1){40}}
                \put(40,60){\line(5,0){40}}
                \put(80,60){\line(2,-1){60}}
                \put(140,30){\line(6,1){60}}
                \put(200,40){\line(1,1){40}}

                \bezier{120}(0,40)(60,70)(120,40)
                \bezier{120}(120,40)(180,20)(240,80)

                \end{picture}\\
        \end{center}
  \caption{\label{curve6}The cubic B-spline curve with a
                redefined knot vector.}
\end{figure}

\subsection{The Knot Vector}

The knots of a B-spline curve describe the following properties of the curve:
\begin{itemize}
\item The parameterization of the B-spline curve
\item The continuity at the joins between the adjacent polynomial
   segments of the B-spline curve.
\end{itemize}
In figure~\ref{curve7} we have two curves
with the same control polygon and order but
with different parameterization.

This example is not meant as an encouragement to use
parameterization for modelling, rather to make users
aware of the effect of parameterization. Something close to
curve length parameterization is in most cases
preferable. For interpolation, chord-length parameterization
is used in most cases.

The number of equal knots determines the degree
of continuity. If $k$ consecutive internal knots are equal,
the curve is discontinuous.
Similarly if $k-1$ consecutive internal knots are equal,
the curve is continuous but not in general differentiable.
A continuously differentiable curve
with a discontinuity in the second derivative
can be modelled using $k-2$ equal knots etc. (see figure~\ref{curve8}).
Normally, B-spline curves in SISL are expected to be continuous.
For intersection algorithms, curves are usually expected to
be continuously differentiable ($C^1$).

\begin{figure}
        \begin{center}
                \begin{picture}(240,240)(0,0)
                \thicklines
                \put(0,130){\circle*{2}}
                \put(0,135){\circle*{2}}
                \put(0,140){\circle*{2}}
                \put(240,130){\circle*{2}}
                \put(240,135){\circle*{2}}
                \put(240,140){\circle*{2}}

                \put(60,140){\circle*{2}}
                \put(180,10){\circle*{2}}


                \put(0,0){\circle*{2}}
                \put(0,5){\circle*{2}}
                \put(0,10){\circle*{2}}
                \put(240,0){\circle*{2}}
                \put(240,5){\circle*{2}}
                \put(240,10){\circle*{2}}

                \put(0,185){\line(2,1){80}}
                \put(80,225){\line(1,-1){80}}
                \put(160,145){\line(1,1){80}}

                \put(0,55){\line(2,1){80}}
                \put(80,95){\line(1,-1){80}}
                \put(160,15){\line(1,1){80}}

                \bezier{130}(0,185)(80,225)(100,205)
                \bezier{160}(100,205)(160,145)(240,225)

                \bezier{160}(0,55)(80,95)(140,35)
                \bezier{130}(140,35)(160,15)(240,95)
                \end{picture}\\
        \end{center}
  \caption{\label{curve7}Two quadratic B-spline curves with the same
control polygon but different knot vectors. The curves and the control
polygons are two-dimensional.}
\end{figure}

\begin{figure}
        \begin{center}
                \begin{picture}(300,90)(0,0)
                \thicklines
                \put(0,0){\circle*{2}}
                \put(0,5){\circle*{2}}
                \put(0,10){\circle*{2}}

                \put(150,5){\circle*{2}}
                \put(150,10){\circle*{2}}

                \put(300,0){\circle*{2}}
                \put(300,5){\circle*{2}}
                \put(300,10){\circle*{2}}

                \bezier{200}(0,50)(80,90)(150,20)
                \bezier{200}(150,20)(220,80)(300,90)
                \end{picture}\\
        \end{center}
  \caption{\label{curve8}A quadratic B-spline curve with two
equal internal knots.}
\end{figure}

\subsection{NURBS Curves}

A NURBS (Non-Uniform Rational B-Spline) curve is a generalization
of a B-spline curve,
$$ {\bf c}(t) = {\sum_{i=1}^{n} w_i {\bf p}_{i} B_{i,k,{\bf t}}(t)
                 \over
                 \sum_{i=1}^{n} w_i B_{i,k,{\bf t}}(t)} . $$
In addition to the data of a B-spline curve, the NURBS curve
${\bf c}$ has a sequence of weights $w_1,\ldots,w_n$.
One of the advantages of NURBS curves over B-spline curves is that
they can be used to represent conic sections exactly (taking the
order $k$ to be three).
A disadvantage is that NURBS curves depend nonlinearly on their weights,
making some calculations, like the evaluation of derivatives,
more complicated and less efficient than with B-spline curves.

The representation of a NURBS curve is the same as for a B-spline
except that it also includes
\begin{description}
\item[${\bf w}$]: A sequence of weights
            ${\bf w} = (w_1, w_2, \ldots, w_n)$.
\end{description}

In SISL we make the assumption that
\begin{itemize}
\item The weights are (strictly) positive: $w_i > 0$.
\end{itemize}

Under this condition, a NURBS curve, like its B-spline cousin,
enjoys the convex hull property. 
Due to $k$-fold knots at the ends of the knot vector,
NURBS curves in SISL alos have the endpoint 


\section{B-spline Surfaces}

This section is optional reading for those who want to
become acquainted with some of the mathematics of
tensor-product B-splines surfaces. For a description of the data structure for
B-spline surfaces in SISL, see the reference manual.

A tensor product B-spline surface is defined as
\[{\bf s}(u,v) = \sum_{i=1}^{n_1}\sum_{j=1}^{n_2}{\bf p}_{i,j} 
	B_{i,k_1,{\bf u}}(u) B_{j,k_2,{\bf v}}(v) \]
with control points ${\bf p}_{i,j}$ and two variables
(or parameters) $u$ and $v$.
The formula shows that a basis function of a B-spline surface is a
product of two basis functions of B-spline curves (B-splines).
This is why a B-spline surface is called a tensor-product surface.
The following is a list of the components of the representation:
\begin{description}
\item[$dim$]: The dimension of the underlying Euclidean space.
\item[$n_1$]: The number of vertices with respect to the first parameter.
\item[$n_1$]: The number of vertices with respect to the second parameter.
\item[$k_1$]: The order of the B-splines in the first parameter.
\item[$k_2$]: The order of the B-splines in the second parameter.
\item[${\bf u}$]: The knot vector of the B-splines with respect to
                  the first parameter,
                  ${\bf u} = (u_1,u_2,\ldots,u_{n_1+k_1})$.
\item[${\bf v}$]: The knot vector of the B-splines with respect to
                  the second parameter,
                  ${\bf v} = (v_1,v_2,\ldots,v_{n_2+k_2})$.
\item[${\bf p}$]: The control points of the B-spline surface,
           $c_{d,i,j}$, $d=1,\ldots,dim$, $i=1,\ldots,n_1$,
		$j=1,\ldots,n_2$.
	When $dim = 3$, we have
          ${\bf p} = (x_{1,1},y_{1,1},z_{1,1},x_{2,1},y_{2,1},z_{2,1},\ldots$,
                  $x_{n_1,1},y_{n_1,1},z_{n_1,1},\ldots$,
                     $x_{n_1,n_2},y_{n_1,n_2},z_{n_1,n_2})$.
\end{description}

The data of the B-spline surface must fulfill the following requirements:
\begin{itemize}
\item
Both knot vectors must be non-decreasing.
\item
The number of vertices must be greater than or equal to the order
with respect to both parameters: $n_1 \ge k_1$ and $n_2 \ge k_2$.
\item
There should be $k_1$ equal knots at the beginning and end 
of knot vector ${\bf u}$ and $k_2$ equal knots at the beginning and
end of knot vector ${\bf v}$.
\end{itemize}

The properties of the representation of a B-spline surface are
similar to the properties of the representation of a B-spline curve.
The control points ${\bf p}_{i,j}$ form a {\it control net} as shown in
figure~\ref{surf1}.
The control net has similar properties to the control
polygon of a B-spline curve, described in section~\ref{contrlpoly}.
A B-spline surface has two knot vectors, one
for each parameter. In figure~\ref{surf1} we can
see {\it isocurves}, surface curves
defined by fixing the value of one of the parameters.

\begin{figure}
\vspace{50 mm}
  \special{psfile=surf1.ps hoffset=-30 voffset=-170 hscale=90 vscale=60}
  %\special{psfile=surf1.ps hoffset=95 hscale=40 vscale=40}
%  \epsffile{surf1.ps}
  \caption{\label{surf1}
		A B-spline surface and its control net.
		The surface is drawn using isocurves.
		The dimension is 3.}
\end{figure}

\subsection{The Basis Functions}

A basis function of a B-spline surface is the product of two
basis functions of two B-spline curves,
\[ B_{i,k_1,{\bf u}}(u) B_{j,k_2,{\bf v}}(v). \]
Its support is the rectangle $[u_i,u_{i+k_1}] \times [v_j,v_{j+k_2}]$.
If the basis functions in both directions are of degree one and all
knots have multiplicity one, then the surface basis functions are
pyramid-shaped 
(see figure~\ref{surf2}).
For higher degrees, the surface basis functions are bell shaped.

\begin{figure}
	\begin{center}
		\begin{picture}(250,125)(0,0)

		\put(10,39){\vector(0,1){86}}
		\put(8,45){\line(1,0){4}}
		\put(8,120){\line(1,0){4}}
		\put(0,45){\makebox(0,0){0.0}}
		\put(0,120){\makebox(0,0){1.0}}

		\put(50,5){\circle*{2}}
		\put(100,5){\circle*{2}}
		\put(150,5){\circle*{2}}

		\put(190,20){\circle*{2}}
		\put(215,45){\circle*{2}}
		\put(240,70){\circle*{2}}

		\thicklines
		\put(65,20){\line(1,0){100}}
		\put(165,20){\line(1,1){50}}
		\multiput(65,20)(18,18){3}{\line(1,1){10}}
		\multiput(115,70)(20,0){5}{\line(1,0){10}}


		\put(65,20){\line(3,4){75}}
		\put(165,20){\line(-1,4){25}}
		\put(215,70){\line(-3,2){75}}
		\multiput(115,70)(8,16){3}{\line(1,2){5}}
		\end{picture}\\
	\end{center}

  \caption{\label{surf2}
		A basis function of degree one in
		both variables.}
\end{figure}

\subsection{NURBS Surfaces}

A NURBS (Non-Uniform Rational B-Spline) surface is a generalization
of a B-spline surface,
$$ {\bf s}(u,v) = {\sum_{i=1}^{n_1}\sum_{j=1}^{n_2} w_{i,j} {\bf p}_{i,j} 
	    B_{i,k_1,{\bf u}}(u) B_{j,k_2,{\bf v}}(v)  \over
             \sum_{i=1}^{n_1}\sum_{j=1}^{n_2} w_{i,j}
	         B_{i,k_1,{\bf u}}(u) B_{j,k_2,{\bf v}}(v)}. $$
In addition to the data of a B-spline surface, the NURBS surface
has a weights $w_{i,j}$.
NURBS surfaces can be used to exactly represent several common
`analytic' surfaces such as spheres, cylinders, tori, and cones.
A disadvantage is that NURBS surfaces depend nonlinearly on their weights,
making some calculations, like with NURBS curves,
less efficient.

The representation of a NURBS surface is the same as for a B-spline
except that it also includes
\begin{description}
\item[${\bf w}$]: The weights of the NURBS surface,
           $w_{i,j}$, $i=1,\ldots,n_1$, $j=1,\ldots,n_2$, so
          ${\bf w} = (w_{1,1},w_{2,1},\ldots,w_{n_1,1},\ldots$,
                  $w_{1,2},\ldots,w_{n_1,n_2})$.
\end{description}
In SISL we make the assumption that
\begin{itemize}
\item The weights are (strictly) positive: $w_{i,j} > 0$.
\end{itemize}



\vfill
\newpage
%\mbox{}
%\newpage
