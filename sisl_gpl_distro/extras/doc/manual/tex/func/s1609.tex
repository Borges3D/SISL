\subsection{Compute a fillet curve based on radius.}
\funclabel{s1609}
\begin{minipg1}
  To calculate a constant radius fillet curve between two curves
  if possible.
  The output is represented as a B-spline curve.
\end{minipg1} \\ \\
SYNOPSIS\\
        \>void s1609(\begin{minipg3}
        {\fov curve1}, {\fov curve2}, {\fov epsge}, {\fov point1}, {\fov pointf}, {\fov point2}, {\fov radius}, {\fov normal},\linebreak
        {\fov filltype}, {\fov dim}, {\fov order}, {\fov newcurve}, {\fov parend1}, {\fov parspt1}, {\fov parend2},\linebreak
        {\fov parept2}, {\fov stat})
                \end{minipg3}\\[0.3ex]
                \>\>    SISLCurve       \>      *{\fov curve1};\\
                \>\>    SISLCurve       \>      *{\fov curve2};\\
                \>\>    double  \>      {\fov epsge};\\
                \>\>    double  \>      {\fov point1}[\,];\\
                \>\>    double  \>      {\fov pointf}[\,];\\
                \>\>    double  \>      {\fov point2}[\,];\\
                \>\>    double  \>      {\fov radius};\\
                \>\>    double  \>      {\fov normal}[\,];\\
                \>\>    int     \>      {\fov filltype};\\
                \>\>    int     \>      {\fov dim};\\
                \>\>    int     \>      {\fov order};\\
                \>\>    SISLCurve       \>      **{\fov newcurve};\\
                \>\>    double  \>      *{\fov parend1};\\
                \>\>    double  \>      *{\fov parspt1};\\
                \>\>    double  \>      *{\fov parend2};\\
                \>\>    double  \>      *{\fov parept2};\\
                \>\>    int     \>      *{\fov stat};\\
\\
ARGUMENTS\\
        \>Input Arguments:\\
        \>\>    {\fov curve1}   \> - \> The first input curve.\\
        \>\>    {\fov curve2}   \> - \> The second input curve.\\
        \>\>    {\fov epsge}    \> - \> Geometry resolution.\\
        \>\>    {\fov point1}   \> - \> \begin{minipg2}
                        Point indicating that the fillet should be put on the
                        side of {\fov curve1} where {\fov point1} is situated.
                                \end{minipg2}\\[0.3ex]
        \>\>    {\fov pointf}\> - \>    \begin{minipg2}
                        Point indicating where the fillet curve should go.
                        {\fov point1} together with {\fov pointf}
                        indicates the direction of the start tangent of
                        the curve, while pointf together with {\fov point2}
                        indicates the direction of the end tangent of
                        the curve.  If more than one position of the fillet
                        curve is possible, the closest curve to {\fov pointf}
                        is chosen.
                                \end{minipg2}\\[0.3ex]
        \>\>    {\fov point2}   \> - \> \begin{minipg2}
                        Point indicating that the fillet should be put on the
                        side of {\fov curve2} where {\fov point2} is situated.
                                \end{minipg2}\\[0.3ex]
        \>\>    {\fov radius}   \> - \> \begin{minipg2}
                The radius to be used on the fillet if a circular fillet is
                possible, otherwise a conic or a quadratic polynomial
                curve is used, approximating the circular fillet.
                                \end{minipg2}\\[0.3ex]
        \>\>    {\fov normal}   \> - \> \begin{minipg2}
                        Normal to the plane the fillet curve
                        should lie close to. This is only used in 3D
                        fillet calculations, and the fillet centre will
                        be in the direction of the cross product of the
                        curve tangents and the normal.
                                \end{minipg2}\\[0.3ex]
        \>\>    {\fov filltype}\> - \>  Indicator of type of fillet.\\
                \>\>\>\>\>      $=1$ : \>\begin{minipg5}
                                Circle, interpolating tangent on first
                                curve, not on curve 2.
                                \end{minipg5}\\[0.3ex]
                \>\>\>\>\>      $=2$ : \>\begin{minipg5}
                                Conic if possible,
                                \end{minipg5}\\
                \>\>\>\>\>      else : \>\begin{minipg5}
                                polynomial segment.
                                \end{minipg5}\\
        \>\>    {\fov dim}      \> - \> Dimension of space.\\
        \>\>    {\fov order}    \> - \> Order of fillet curve, which is not always used.\\
\\
        \>Output Arguments:\\
        \>\>    {\fov newcurve}\> - \> Pointer to the B-spline fillet curve.\\
        \>\>    {\fov parend1}  \> - \> \begin{minipg2}
                        Parameter value of the end of curve 1 not affected
                        by the fillet.
                                \end{minipg2}\\[0.8ex]
        \>\>    {\fov parspt1}  \> - \> \begin{minipg2}
                        Parameter value of the point on curve 1 where the
                        fillet starts.
                                \end{minipg2}\\[0.8ex]
        \>\>    {\fov parend2}  \> - \> \begin{minipg2}
                        Parameter value of the end of curve 2 not affected
                        by the fillet.
                                \end{minipg2}\\[0.8ex]
        \>\>    {\fov parept2}  \> - \> \begin{minipg2}
                        Parameter value of the point on curve 2 where the
                        fillet ends.
                                \end{minipg2}\\[0.8ex]
        \>\>    {\fov stat}     \> - \> Status messages\\
                \>\>\>\>\>              $> 0$   : warning\\
                \>\>\>\>\>              $= 0$   : ok\\
                \>\>\>\>\>              $< 0$   : error\\
\newpagetabs
EXAMPLE OF USE\\
                \>      \{ \\
                \>\>    SISLCurve       \>      *{\fov curve1};\\
                \>\>    SISLCurve       \>      *{\fov curve2};\\
                \>\>    double  \>      {\fov epsge};\\
                \>\>    double  \>      {\fov point1}[3];\\
                \>\>    double  \>      {\fov pointf}[3];\\
                \>\>    double  \>      {\fov point2}[3];\\
                \>\>    double  \>      {\fov radius};\\
                \>\>    double  \>      {\fov normal}[3];\\
                \>\>    int     \>      {\fov filltype};\\
                \>\>    int     \>      {\fov dim} = 3;\\
                \>\>    int     \>      {\fov order};\\
                \>\>    SISLCurve       \>      *{\fov newcurve};\\
                \>\>    double  \>      {\fov parend1};\\
                \>\>    double  \>      {\fov parspt1};\\
                \>\>    double  \>      {\fov parend2};\\
                \>\>    double  \>      {\fov parept2};\\
                \>\>    int     \>      {\fov stat};\\
                \>\>    \ldots \\
        \>\>s1609(\begin{minipg4}
        {\fov curve1}, {\fov curve2}, {\fov epsge}, {\fov point1}, {\fov pointf}, {\fov point2}, {\fov radius},\\ {\fov normal},
        {\fov filltype}, {\fov dim}, {\fov order}, \&{\fov newcurve}, \&{\fov parend1}, \&{\fov parspt1},\\ \&{\fov parend2},
        \&{\fov parept2}, \&{\fov stat});
                        \end{minipg4}\\
                \>\>    \ldots \\
                \>      \} \\
\end{tabbing}
