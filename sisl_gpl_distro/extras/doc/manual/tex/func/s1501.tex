\subsection{March an intersection curve between a surface and an
\mbox{elliptic} cone.}
\funclabel{s1501}
\begin{minipg1}
  To march an intersection curve described by parameter pairs in an intersection
  curve object, a surface and an elliptic cone.
  The guide points are expected to be found by s1503() described on
  page \pageref{s1503}.
  The generated geometric curves are represented as B-spline curves.
\end{minipg1} \\ \\
SYNOPSIS\\
        \>void s1501(\begin{minipg3}
        {\fov surf}, {\fov basept}, {\fov normdir}, {\fov ellipaxis},
        {\fov alpha}, {\fov ratio}, {\fov dim}, {\fov epsco}, {\fov epsge}, {\fov maxstep},
        {\fov intcurve}, {\fov makecurv}, {\fov graphic}, {\fov stat})
                \end{minipg3}\\[0.3ex]

                \>\>    SISLSurf        \>      *{\fov surf};\\
                \>\>    double  \>      {\fov basept}[\,];\\
                \>\>    double  \>      {\fov normdir}[\,];\\
                \>\>    double  \>      {\fov ellipaxis}[\,];\\
                \>\>    double  \>      {\fov alpha};\\
                \>\>    double  \>      {\fov ratio};\\
                \>\>    int     \>      {\fov dim};\\
                \>\>    double  \>      {\fov epsco};\\
                \>\>    double  \>      {\fov epsge};\\
                \>\>    double  \>      {\fov maxstep};\\
                \>\>    SISLIntcurve\>  *{\fov intcurve};\\
                \>\>    int     \>      {\fov makecurv};\\
                \>\>    int     \>      {\fov graphic};\\
                \>\>    int     \>      *{\fov stat};\\
\\
ARGUMENTS\\
        \>Input Arguments:\\
        \>\>    {\fov surf}\> - \>      \begin{minipg2}
                                Pointer to the surface.
                                \end{minipg2}\\
        \>\>    {\fov basept}\> - \>    \begin{minipg2}
                                Base point of the cone, centre of elliptic base.
                                \end{minipg2}\\
        \>\>    {\fov normdir}\> - \>   \begin{minipg2}
                                Direction of the cone axis, normal to the elliptic base. The default is pointing from the base point to the top point.
                                \end{minipg2}\\[0.3ex]
        \>\>    {\fov ellipaxis}\> - \> \begin{minipg2}
                                One of the axes of the ellipse (major or
                                minor).
                                The other axis will be calculated as
                                $normdir\times ellipaxis$ scaled with
                                {\fov ratio}.
                                \end{minipg2}\\[0.8ex]
        \>\>    {\fov alpha}\> - \>     \begin{minipg2}
                                The opening angle in radians of the cone at the ellipaxis.
                                \end{minipg2}\\[0.3ex]
        \>\>    {\fov ratio}\> - \>     \begin{minipg2}
                                The ratio of the major and minor
                                axes = ellipaxis/otheraxis.
                                \end{minipg2}\\[0.3ex]
        \>\>    {\fov dim}\> - \>       \begin{minipg2}
                                Dimension of the space in which the cone lies.
                                Should be 3.
                                \end{minipg2}\\[0.8ex]
        \>\>    {\fov epsco}\> - \>     \begin{minipg2}
                                Computational resolution (not used).
                                \end{minipg2}\\
        \>\>    {\fov epsge}\> - \>     \begin{minipg2}
                                Geometry resolution.
                                \end{minipg2}\\
        \>\>    {\fov maxstep}\> - \>   \begin{minipg2}
                                Maximum step length allowed. If maxstep $\leq$ epsge,
                                maxstep is neglected. maxstep = 0.0 is recommended.
                                \end{minipg2}\\[0.3ex]
        \>\>    {\fov makecurv}\> - \>          \begin{minipg2}
                                Indicator specifying if a geometric curve is to be made:
                                \end{minipg2}\\
                \>\>\>\>\>      0 -     \>Do not make curves at all.\\
                \>\>\>\>\>      1 -     \>Make only a geometric curve.\\
                \>\>\>\>\>      2 -     \>\begin{minipg5}
                                        Make geometric curve and curve in the parameter
                                        plane
                                        \end{minipg5} \\[0.3ex]
        \>\>    {\fov graphic}\> - \>   \begin{minipg2}
                                Indicator specifying if the function
                                should draw the curve:
                                \end{minipg2}\\
                \>\>\>\>\>      0 -     \>Don't draw the curve.\\
                \>\>\>\>\>      1 -     \>\begin{minipg5}
                                        Draw the geometric curve. If this option
                                        is used see NOTE!
                                        \end{minipg5} \\[0.8ex]
        \>Input/Output Arguments:\\
        \>\>    {\fov intcurve}\> - \>  \begin{minipg2}
                                Pointer to the intersection curve.
                                As input only
                                guide points (points in parameter space)
                                exist. These guide points
                                are used for guiding the marching.
                                The routine adds the
                                intersection curve and curve in the parameter
                                plane to the SISLIntcurve object according to the value
                                of makecurv.
                                \end{minipg2}\\[0.8ex]
        \>Output Arguments:\\
        \>\>    {\fov stat}     \> - \> Status messages\\
                \>\>\>\>\>      $= 3$ : \>      \begin{minipg5}
                                                Iteration stopped due to singular
                                                point or degenerate surface. A part of
                                                an intersection curve may have been
                                                traced out. If no curve is traced out,
                                                the curve pointers in the SISLIntcurve
                                                object point to NULL.
                                                \end{minipg5} \\[0.3ex]
                \>\>\>\>\>      $= 0$   :\> ok\\
                \>\>\>\>\>      $< 0$   :\> error\\
\\
NOTE\\
\>      \begin{minipg6}
If the draw option is used the empty dummy functions s6move() and
s6line() are called.
Thus if the draw option is used, make sure
you have versions of s6move() and s6line() interfaced to your graphic package.
More about s6move() and s6line() on pages~\pageref{s6move} and~\pageref{s6line}.
\end{minipg6}\\
\newpagetabs
EXAMPLE OF USE\\
                \>      \{ \\
                \>\>    SISLSurf        \>      *{\fov surf};\\
                \>\>    double  \>      {\fov basept}[3];\\
                \>\>    double  \>      {\fov normdir}[3];\\
                \>\>    double  \>      {\fov ellipaxis}[3];\\
                \>\>    double  \>      {\fov alpha};\\
                \>\>    double  \>      {\fov ratio};\\
                \>\>    int     \>      {\fov dim} = 3;\\
                \>\>    double  \>      {\fov epsco};\\
                \>\>    double  \>      {\fov epsge};\\
                \>\>    double  \>      {\fov maxstep} = 0.0;\\
                \>\>    SISLIntcurve\>  *{\fov intcurve};\\
                \>\>    int     \>      {\fov makecurv};\\
                \>\>    int     \>      {\fov graphic};\\
                \>\>    int     \>      {\fov stat} = 0;\\
                \>\>    \ldots \\
        \>\>s1501(\begin{minipg4}
                {\fov surf}, {\fov basept}, {\fov normdir}, {\fov ellipaxis}, {\fov alpha}, {\fov ratio}, {\fov dim}, {\fov epsco}, {\fov epsge}, {\fov maxstep},
                {\fov intcurve}, {\fov makecurv}, {\fov graphic}, \&{\fov stat});
                        \end{minipg4}\\
                \>\>    \ldots \\
                \>      \}
\end{tabbing}
