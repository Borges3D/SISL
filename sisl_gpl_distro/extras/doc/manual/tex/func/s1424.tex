\subsection{Compute the position and derivatives of a surface at a given
parameter value pair.}
\funclabel{s1424}
\begin{minipg1}
  Evaluate the surface the parameter value ({\fov parvalue}[0],
  {\fov parvalue}[1]).
  Compute the $der1\times der2$ first derivatives.
  The derivatives that will be computed are $D^{i,j}$,
  $i=0,1,\ldots,der1$, $j=0,1,...,der2$.
\end{minipg1}\\ \\
SYNOPSIS\\
        \>void s1424(\begin{minipg3}
          {\fov surf}, {\fov der1}, {\fov der2}, {\fov parvalue}, {\fov leftknot1}, {\fov leftknot2}, {\fov derive}, {\fov stat})
        \end{minipg3}\\[0.3ex]
        \>\>    SISLSurf        \>      *{\fov surf};\\
        \>\>    int     \>      {\fov der1};\\
        \>\>    int     \>      {\fov der2};\\
        \>\>    double  \>      {\fov parvalue}[\,];\\
        \>\>    int     \>      *{\fov leftknot1};\\
        \>\>    int     \>      *{\fov leftknot2};\\
        \>\>    double  \>      {\fov derive}[\,];\\
        \>\>    int     \>      *{\fov stat};\\
\\
ARGUMENTS\\
        \>Input Arguments:\\
        \>\>    {\fov surf}\> - \>      \begin{minipg2}
                                Pointer to the surface to evaluate.
                                \end{minipg2}\\
        \>\>    {\fov der1}\> - \>      \begin{minipg2}
                                Number (order) of derivatives to be
                                evaluated in
                                first parameter direction, where $0\leq der1$.
                                \end{minipg2}\\[0.8ex]
        \>\>    {\fov der2}\> - \>              \begin{minipg2}
                                Number (order) of derivatives to be
                                evaluated in
                                second parameter direction, where $0\leq
                                der2$.
                                \end{minipg2}\\[0.8ex]
        \>\>    {\fov parvalue}\> - \>          \begin{minipg2}
                                Parameter-value at which to evaluate. The dimension
                                of {\fov parvalue} is 2.
                                \end{minipg2}\\[0.8ex]
\\
        \>Input/Output Arguments:\\
        \>\>    {\fov leftknot1}\> - \> \begin{minipg2}
                                Pointer to the interval in the knot vector in first
                                parameter direction where {\fov parvalue}[0] is
                                found. The relation
                                \[
                                etl[leftknot1]\leq parvalue[0]<etl[leftknot1+1],
                                \]
                                where {\fov etl} is the knot vector, should
                                hold. {\fov leftknot1}  should be set
                                equal to zero at the first call to the
                                routine.
                                Do not change the value of {\fov leftknot1}
                                between calls to the routine.
                                \end{minipg2}\\[0.8ex]
        \>\>    {\fov leftknot2}\> - \> \begin{minipg2}
                                Corresponding to {\fov leftknot1} in the
                                second parameter direction.
                                \end{minipg2}\\[0.8ex]
\newpagetabs
        \>Output Arguments:\\
        \>\>    {\fov derive}\> - \>    \begin{minipg2}
                                Array of size $d(der1+1)(der2+1)$
                                where the position and the derivative
                                vectors of the surface in
                                ({\fov parvalue}[0], {\fov parvalue}[1]) is placed.
                                $d=surf\rightarrow dim$ is the number of
                                elements in each vector and is equal to
                                the geometrical dimension.
                       The vectors are stored in the following order:
                       First the $d$ components of the position vector,
                       then the $d$ components of the $D^{1,0}$ vector,
                       and so on up to the $d$ components of the $D^{der1,0}$
                       vector,
                       then the $d$ components of the $D^{0,1}$ vector etc.
                       If derive is considered to be a
                       three dimensional array, then its declaration in C
                       would be $derive[der2+1][der1+1][d]$.
                                \end{minipg2}\\[0.3ex]
        \>\>    {\fov stat}     \> - \> Status messages\\
                \>\>\>\>\>      $> 0$ : \>      Warning.\\
                \>\>\>\>\>      $= 0$ : \>      Ok.\\
                \>\>\>\>\>      $< 0$ : \>      Error.\\
\\
EXAMPLE OF USE\\
                \>      \{ \\
                \>\>    SISLSurf        \>      *{\fov surf};\\
                \>\>    int     \>      {\fov der1} = 2;\\
                \>\>    int     \>      {\fov der2} = 1;\\
                \>\>    double  \>      {\fov parvalue}[2];\\
                \>\>    int     \>      {\fov leftknot1} = 0;\\
                \>\>    int     \>      {\fov leftknot2} = 0;\\
                \>\>    double  \>      {\fov derive}[18];\\
                \>\>    int     \>      {\fov stat};\\
                \>\>    \ldots \\
        \>\>s1424(\begin{minipg4}
                {\fov surf}, {\fov der1}, {\fov der2}, {\fov parvalue}, \&{\fov leftknot1}, \&{\fov leftknot2}, {\fov derive}, \&{\fov stat});
                        \end{minipg4}\\
                \>\>    \ldots \\
                \>      \}
\end{tabbing}
