\subsection{Data reduction: B-spline surface as input.}
\funclabel{s1965}
\begin{minipg1}
To remove as many knots as possible from a spline surface
           without perturbing the surface more than the given tolerance.
           The error in continuity over the start and end of
           a closed or periodic surface is only guaranteed to be within
           edgeps.
\end{minipg1} \\ \\
SYNOPSIS\\
        \> void s1965(\begin{minipg3}
            {\fov oldsurf},  {\fov eps},  {\fov edgefix},  {\fov iopen1},  {\fov iopen2},  {\fov edgeps},  {\fov opt}, 
	     {\fov itmax},  {\fov newsurf}, {\fov maxerr},  {\fov stat})
                \end{minipg3}\\
                \>\>    SISLSurf \> 	*{\fov oldsurf};\\
                \>\>    double	\> {\fov eps}[\,];\\
                \>\>    int	\> {\fov edgefix}[{\fov 4}];\\
                \>\>    int    \>  {\fov iopen1};\\
                \>\>    int    \>  {\fov iopen2};\\
                \>\>    double	\> {\fov edgeps}[\,];\\
                \>\>    int 	\> {\fov opt};\\
                \>\>    int	\> {\fov itmax};\\
                \>\>    SISLSurf\> 	**{\fov newsurf};\\
                \>\>    double	\> {\fov maxerr}[\,];\\
                \>\>    int	\> *{\fov stat};\\
\\
ARGUMENTS\\
	\>Input Arguments:\\
        \>\>    {\fov oldsurf}\> - \>  \begin{minipg2}
                     pointer to the original spline surface. Note
			 if the polynomial orders of the surface are
			 k1 and k2, then the two knot vectors are
			 assumed to have knots of multiplicity k1 and
			 k2 at the ends.
                               \end{minipg2}\\[0.8ex]
        \>\>    {\fov eps}\> - \>  \begin{minipg2}
                     double array of length dim (the number of
			 components of the surface, typically three)
			 giving the desired accuracy of the
			 final approximation compared to oldcurve.
                        Note that in such comparisons the two
			 surfaces are not reparametrized in any way.
                               \end{minipg2}\\[0.8ex]
        \>\>    {\fov edgefix}\> - \>  \begin{minipg2}
                     integer array of dimension (4) giving the number
			 of derivatives to be kept fixed along each edge
			 of the surface. The numbering of the edges is the
			 same as for edgeps below. All the derivatives of
			 order $< nend(i)-1$ will be kept fixed along
			 edge $i$. Hence $nend(i)=0$ indicates that nothing is
			 to be kept fixed along edge $i$.
                NB! TO BE KEPT FIXED HERE MEANS TO HAVE ERROR LESS THAN
		     EDGEPS. IN GENERAL, IT IS IMPOSSIBLE TO REMOVE KNOTS
                    AND KEEP AN EDGE COMPLETELY FIXED.
                               \end{minipg2}\\[0.8ex]
        \>\>   {\fov iopen1}\> - \> Open/closed parameter in first direction.\\
          \>\>\>\>  $= 1$  : Produce open surface.\\
          \>\>\>\>  $= 0$ : Produce closed, non-periodic surface if possible.\\
          \>\>\>\>  $= -1$ : Produce closed, periodic surface\\
        \>\>   {\fov iopen2}\> - \> Open/closed parameter in second direction.\\
          \>\>\>\>  $= 1$  : Produce open surface.\\
          \>\>\>\>  $= 0$ : Produce closed, non-periodic surface if possible.\\
          \>\>\>\>  $= -1$ : Produce closed, periodic surface\\
        \>\>    {\fov edgeps}\> - \>  \begin{minipg2}
                     double array of length 4*dim ([4,dim]) (dim is
                        the number of components of each coefficient)
			 containing the maximum deviation which is
			 acceptable along the edges of the surface.
                         $edgeps[0]-edgeps[dim-1]$ gives the tolerance along
			 the edge corresponding to x1 (the first parameter)
 			 having it's minimum value.
			 $edgeps[dim]-edgeps[2*dim-1]$ gives the tolerance
			 along the edge corresponding to x1 (the first
			 parameter) having it's maximum value.
              		 $edgeps[2*dim]-edgeps[3*dim-1]$ gives the tolerance
			 along the edge corresponding to x2 (the second
			 parameter) having it's minimum value.
              		 $edgeps[3*dim]-edgeps[4*dim-1]$ gives the tolerance
			 along the edge corresponding to x2 (the second
			 parameter) having its maximum value.
           	 NB! EDGEPS WILL ONLY HAVE ANY SIGNIFICANCE IF THE
		     CORRESPONDING ELEMENT OF EDGEFIX IS POSITIVE.
                               \end{minipg2}\\[0.8ex]
        \>\>    {\fov itmax}\> - \>  \begin{minipg2}
                     maximum number of iterations. The routine will
                        follow an iterative procedure trying to remove
                        more and more knots, one direction at a time.
                        The process will almost always stop after less
                        than 10 iterations and it will often stop after
                        less than 5 iterations. A suitable value for itmax
                        is therefore usually in the region 3-10.
                               \end{minipg2}\\[0.8ex]
        \>\>    {\fov opt}\> - \>  \begin{minipg2}
                     integer indicating the order in which the
	       	         knot removal is to be performed.
                               \end{minipg2}\\[0.8ex]
          \>\>\>\>  1 : remove knots in parameter 1 only.\\
          \>\>\>\>  2 : remove knots in parameter 2 only.\\
          \>\>\>\>  3 : remove knots first in parameter 1 and then 2.\\
          \>\>\>\>  4 : remove knots first in parameter 2 and then 1.\\
\\
	\>Output Arguments:\\
        \>\>    {\fov newsurf}\> - \>  \begin{minipg2}
                     the approximating surface on the reduced knot vectors.
                               \end{minipg2}\\[0.8ex]
        \>\>    {\fov maxerr}\> - \>  \begin{minipg2}
                     double array of length dim containing an upper
		         bound for the pointwise error in each of the
		         components of the spline approximation. The two
                        surfaces oldsurf and newsurf are compared at the
                        same parameter vaues, i.e., if oldsurf is f and
                        newsurf is g then
                              $|f(u,v)-g(u,v)| <= eps$
                        in each of the components.
                               \end{minipg2}\\[0.8ex]
        \>\>    {\fov stat}     \> - \> Status messages\\
                \>\>\>\>\>              $> 0$   : Warning.\\
                \>\>\>\>\>              $= 0$   : Ok.\\
                \>\>\>\>\>              $< 0$   : Error.\\
\\
EXAMPLE OF USE\\
		\>      \{ \\

                \>\>    SISLSurf \> 	*{\fov oldsurf};\\
                \>\>    double	\> {\fov eps}[\,];\\
                \>\>    int	\> {\fov edgefix}[{\fov 4}];\\
                \>\>    int    \>  {\fov iopen1};\\
                \>\>    int    \>  {\fov iopen2};\\
                \>\>    double	\> {\fov edgeps}[\,];\\
                \>\>    int 	\> {\fov opt};\\
                \>\>    int	\> {\fov itmax};\\
                \>\>    SISLSurf	\> **{\fov newsurf};\\
                \>\>    double	\> {\fov maxerr}[\,];\\
                \>\>    int	\> *{\fov stat};\\                \>\>    \ldots \\
        \>\>s1965(\begin{minipg4}
            {\fov oldsurf},  {\fov eps},  {\fov edgefix},  {\fov iopen1},  {\fov iopen2},  {\fov edgeps},  {\fov opt}, 
	     {\fov itmax},  {\fov newsurf}, {\fov maxerr},  {\fov stat});
                \end{minipg4}\\
                \>\>    \ldots \\
		\>      \}
\end{tabbing}
