\section{Compute the Length of a Curve}
\funclabel{s1240}
\begin{minipg1}
  Compute the length of a curve. The length calculated will not deviate
  more than {\fov epsge} divided by the calculated length, from the real
  length of the curve.
\end{minipg1} \\ \\
SYNOPSIS\\
        \>void s1240(\begin{minipg3}
        {\fov curve}, {\fov epsge}, {\fov length}, {\fov stat})
                \end{minipg3}\\[0.3ex]
                \>\>    SISLCurve       \>      *{\fov curve};\\
                \>\>    double  \>      {\fov epsge};\\
                \>\>    double  \>      *{\fov length};\\
                \>\>    int     \>      *{\fov stat};\\
\\
ARGUMENTS\\
        \>Input Arguments:\\
        \>\>    {\fov curve}    \> - \> The curve.\\
        \>\>    {\fov epsge}    \> - \> Geometry resolution.\\
\\
        \>Output Arguments:\\
        \>\>    {\fov length}   \> - \> The length of the curve.\\
        \>\>    {\fov stat}     \> - \> Status messages\\
                \>\>\>\>\>              $> 0$   : Warning.\\
                \>\>\>\>\>              $= 0$   : Ok.\\
                \>\>\>\>\>              $< 0$   : Error.\\
\\
NOTE\\
\>\begin{minipg6}
        The algorithm is based on recursive
        subdivision and will thus for small values
        of {\fov epsge} require long computation time.
\end{minipg6} \\ \\
EXAMPLE OF USE\\
                \>      \{ \\
                \>\>    SISLCurve       \>      *{\fov curve};\\
                \>\>    double  \>      {\fov epsge};\\
                \>\>    double  \>      {\fov length};\\
                \>\>    int     \>      {\fov stat};\\
                \>\>    \ldots \\
        \>\>s1240(\begin{minipg4}
                {\fov curve}, {\fov epsge}, \&{\fov length}, \&{\fov stat});
                        \end{minipg4}\\
                \>\>    \ldots \\
                \>      \}
\end{tabbing}
