\chapter{Surface Definition}
\label{surfacedefinition}
\section{Interpolation}
\subsection{Compute a surface interpolating a set of points,
automatic parameterization.}
\funclabel{s1536}
\begin{minipg1}
  To compute a tensor surface interpolating a set of points,
  automatic parameterization.
  The output is represented as a B-spline surface.
\end{minipg1}\\ \\
SYNOPSIS\\
        \>void s1536(\begin{minipg3}
          {\fov points}, {\fov im1}, {\fov im2}, {\fov idim}, {\fov ipar}, {\fov con1}, {\fov con2}, {\fov con3},
          {\fov con4}, {\fov order1},  {\fov order2},  {\fov iopen1}, {\fov iopen2}, {\fov rsurf}, {\fov jstat})
        \end{minipg3}\\[0.3ex]
        \>\>    double \> {\fov points}[\,];\\
        \>\>    int \> {\fov im1};\\
        \>\>    int \> {\fov im2};\\
        \>\>    int \> {\fov idim};\\
        \>\>    int \> {\fov ipar};\\
        \>\>    int \> {\fov con1};\\
        \>\>    int \> {\fov con2};\\
        \>\>    int \> {\fov con3};\\
        \>\>    int \> {\fov con4};\\
        \>\>    int \> {\fov order1};\\
        \>\>    int \> {\fov order2};\\
        \>\>    int \> {\fov iopen1};\\
        \>\>    int \> {\fov iopen2};\\
        \>\>    SISLSurf \> **{\fov rsurf};\\
        \>\>    int \> *{\fov jstat};\\
\newpagetabs
ARGUMENTS\\
        \>Input Arguments:\\
        \>\>    {\fov points}\> - \>
        \begin{minipg2}
          Array of dimension $idim\times im1\times im2$ containing
          the positions of the nodes (using the same ordering
          as ecoef in the SISLSurf structure).
        \end{minipg2}\\[0.8ex]
        \>\>    {\fov im1}\> - \>
        \begin{minipg2}
          The number of interpolation points in the
          first parameter direction.
        \end{minipg2}\\[0.8ex]
        \>\>    {\fov im2}\> - \>
        \begin{minipg2}
          The number of interpolation points in the
          second parameter direction.
        \end{minipg2}\\[0.8ex]
        \>\>    {\fov idim} \> - \> Dimension of the space we are working in.\\
        \>\>    {\fov ipar} \> - \> Flag showing the desired
                                    parametrization to be used:\\
                \>\>\>\>\> $= 1$ : \begin{minipg5}
                                     Mean accumulated cord-length
                                     parameterization.
                                   \end{minipg5}\\[0.3ex]
                \>\>\>\>\> $= 2$ : Uniform parametrization.\\
        \>\>\>\>\begin{minipg2}
                Numbering of surface edges:\\
                \begin{center}
                  \begin{picture}(180,110)(0,0)
                    \put(50,15){\framebox(80,80)}
                    \put(40,55){\makebox(0,0){3}}
                    \put(140,55){\makebox(0,0){4}}
                    \put(90,5){\makebox(0,0){1}}
                    \put(90,105){\makebox(0,0){2}}

                    \put(60,20){\vector(1,0){40}}
                    \put(85,28){\makebox(0,0){$(i)$}}
                    \put(55,25){\vector(0,1){40}}
                    \put(65,50){\makebox(0,0){$(ii)$}}
                  \end{picture}\\
                  $(i) \; \; \;$ first parameter direction of surface.\\
                  $(ii)$   second parameter direction of surface.\\
                \end{center}
              \end{minipg2}\\ \\
        \>\>    {\fov con1} \> - \> Additional condition along edge 1:\\
                      \>\>\>\>\> $= 0$ : No additional condition.\\
                      \>\>\>\>\> $= 1$ : Zero curvature.\\
        \>\>    {\fov con2} \> - \> Additional condition along edge 2:\\
                      \>\>\>\>\> $= 0$ : No additional condition.\\
                      \>\>\>\>\> $= 1$ : Zero curvature.\\
        \>\>    {\fov con3} \> - \> Additional condition along edge 3:\\
                      \>\>\>\>\> $= 0$ : No additional condition.\\
                      \>\>\>\>\> $= 1$ : Zero curvature.\\
        \>\>    {\fov con4} \> - \> Additional condition along edge 4:\\
                      \>\>\>\>\> $= 0$ : No additional condition.\\
                      \>\>\>\>\> $= 1$ : Zero curvature.\\
        \>\>    {\fov order1}\> - \> Order of surface in first parameter
                                     direction.\\
        \>\>    {\fov order2}\> - \> Order of surface in second\\
        \>\>    {\fov iopen1}\> - \>
                     \begin{minipg2}
                       Open/closed/periodic in first parameter direction.
                     \end{minipg2}\\
                      \>\>\>\> $= 1$ \> : Open surface.\\
                      \>\>\>\> $= 0$ \> : Closed surface.\\
                      \>\>\>\> $= -1$ \> : Closed and periodic
                      surface.\\
\newpagetabs
        \>\>    {\fov iopen2}\> - \>
                     \begin{minipg2}
                       Open/closed/periodic in second parameter direction.
                     \end{minipg2}\\
                      \>\>\>\> $= 1$ \> : Open surface.\\
                      \>\>\>\> $= 0$ \> : Closed surface.\\
                      \>\>\>\> $= -1$ \> : Closed and periodic surface.\\
\\
        \>Output Arguments:\\
        \>\>    {\fov rsurf}\> - \> Pointer to the B-spline surface produced.\\
        \>\>    {\fov jstat} \> - \> Status message\\
                \>\>\>\>\> $< 0$ : Error.\\
                \>\>\>\>\> $= 0$ : Ok.\\
                \>\>\>\>\> $> 0$ : Warning.\\
\\
EXAMPLE OF USE\\
        \>      \{ \\
        \>\>    double \> {\fov points}[300];\\
        \>\>    int    \> {\fov im1} = 10;\\
        \>\>    int    \> {\fov im2} = 10;\\
        \>\>    int    \> {\fov idim} = 3;\\
        \>\>    int    \> {\fov ipar};\\
        \>\>    int    \> {\fov con1};\\
        \>\>    int    \> {\fov con2};\\
        \>\>    int    \> {\fov con3};\\
        \>\>    int    \> {\fov con4};\\
        \>\>    int    \> {\fov order1};\\
        \>\>    int    \> {\fov order2};\\
        \>\>    int    \> {\fov iopen1};\\
        \>\>    int    \> {\fov iopen2};\\
        \>\>    SISLSurf \> *{\fov rsurf};\\
        \>\>    int    \> {\fov jstat};\\
        \>\>    \ldots \\
        \>\>s1536(\begin{minipg4}
          {\fov points}, {\fov im1}, {\fov im2}, {\fov idim}, {\fov ipar}, {\fov con1}, {\fov con2}, {\fov con3},
          {\fov con4}, {\fov order1},  {\fov order2},  {\fov iopen1},  {\fov iopen2}, \&{\fov rsurf}, \&{\fov jstat});
        \end{minipg4}\\
        \>\>    \ldots \\
        \>      \}
\end{tabbing}

\pgsbreak
\subsection{Compute a surface interpolating a set of points,
parameterization as input.}
\funclabel{s1537}
\begin{minipg1}
  Compute a tensor surface interpolating a set of points,
  parameterization as input.
  The output is represented as a B-spline surface.
\end{minipg1} \\ \\
SYNOPSIS\\
        \>void s1537(\begin{minipg3}
          {\fov points}, {\fov im1}, {\fov im2}, {\fov idim}, {\fov par1},  {\fov par2}, {\fov con1}, {\fov con2}, {\fov con3},
          {\fov con4}, {\fov order1},  {\fov order2}, {\fov iopen1},
          {\fov iopen2}, {\fov rsurf}, {\fov jstat})
        \end{minipg3}\\[0.3ex]
        \>\>    double \> {\fov points}[\,];\\
        \>\>    int    \> {\fov im1};\\
        \>\>    int    \> {\fov im2};\\
        \>\>    int    \> {\fov idim};\\
        \>\>    double \> {\fov par1}[\,];\\
        \>\>    double \> {\fov par2}[\,];\\
        \>\>    int    \> {\fov con1};\\
        \>\>    int    \> {\fov con2};\\
        \>\>    int    \> {\fov con3};\\
        \>\>    int    \> {\fov con4};\\
        \>\>    int    \> {\fov order1};\\
        \>\>    int    \> {\fov order2};\\
        \>\>    int    \> {\fov iopen1};\\
        \>\>    int    \> {\fov iopen2};\\
        \>\>    SISLSurf \> **{\fov rsurf};\\
        \>\>    int    \> *{\fov jstat};\\
\\
ARGUMENTS\\
        \>Input Arguments:\\
        \>\>    {\fov points}\> - \>
        \begin{minipg2}
          Array of dimension $idim\times im1\times im2$ containing
          the positions of the nodes (using the same ordering
          as ecoef in the SISLSurf structure).
        \end{minipg2}\\[0.8ex]
        \>\>    {\fov im1} \> - \>
        \begin{minipg2}
          The number of interpolation points in the
          first parameter direction.
        \end{minipg2}\\[0.8ex]
        \>\>    {\fov im2} \> - \>
        \begin{minipg2}
          The number of interpolation points in the
          second parameter direction.
        \end{minipg2}\\[0.8ex]
        \>\>    {\fov idim}\> - \> Dimension of the space we are working in.\\
        \>\>    {\fov par1}\> - \> Parametrization in first parameter direction.\\
        \>\>    {\fov par2}\> - \> Parametrization in second parameter
                                   direction.\\
\newpagetabs
        \>\>\>\>\begin{minipg2}
                Numbering of surface edges:\\
                \begin{center}
                  \begin{picture}(180,110)(0,0)
                    \put(50,15){\framebox(80,80)}
                    \put(40,55){\makebox(0,0){3}}
                    \put(140,55){\makebox(0,0){4}}
                    \put(90,5){\makebox(0,0){1}}
                    \put(90,105){\makebox(0,0){2}}

                    \put(60,20){\vector(1,0){40}}
                    \put(85,28){\makebox(0,0){$(i)$}}
                    \put(55,25){\vector(0,1){40}}
                    \put(65,50){\makebox(0,0){$(ii)$}}
                  \end{picture}\\
                  $(i) \; \; \;$ first parameter direction of surface.\\
                  $(ii)$   second parameter direction of surface.\\
                \end{center}
              \end{minipg2}\\ \\
        \>\>    {\fov con1} \> - \> Additional condition along edge 1:\\
                      \>\>\>\>\> $= 0$ : No additional condition.\\
                      \>\>\>\>\> $= 1$ : Zero curvature.\\
        \>\>    {\fov con2} \> - \> Additional condition along edge 2:\\
                      \>\>\>\>\> $= 0$ : No additional condition.\\
                      \>\>\>\>\> $= 1$ : Zero curvature.\\
        \>\>    {\fov con3} \> - \> Additional condition along edge 3:\\
                      \>\>\>\>\> $= 0$ : No additional condition.\\
                      \>\>\>\>\> $= 1$ : Zero curvature.\\
        \>\>    {\fov con4} \> - \> Additional condition along edge 4:\\
                      \>\>\>\>\> $= 0$ : No additional condition.\\
                      \>\>\>\>\> $= 1$ : Zero curvature.\\
        \>\>    {\fov order1}\> - \> Order of surface in first parameter
                                     direction.\\
        \>\>    {\fov order2}\> - \> Order of surface in second
                                     parameter direction.\\
        \>\>    {\fov iopen1}\> - \>
                     \begin{minipg2}
                       Open/closed/periodic in first parameter direction.
                     \end{minipg2}\\
                      \>\>\>\>\> $= 1$ \> : Open surface.\\
                      \>\>\>\>\> $= 0$ \> : Closed surface.\\
                      \>\>\>\>\> $= -1$ \> : Closed and periodic
                      surface.\\
        \>\>    {\fov iopen2}\> - \>
                     \begin{minipg2}
                       Open/closed/periodic in second parameter direction.
                     \end{minipg2}\\
                      \>\>\>\>\> $= 1$ \> : Open surface.\\
                      \>\>\>\>\> $= 0$ \> : Closed surface.\\
                      \>\>\>\>\> $= -1$ \> : Closed and periodic surface.\\
\\
        \>Output Arguments:\\
        \>\>    {\fov rsurf}\> - \> Pointer to the B-spline surface produced.\\
        \>\>    {\fov jstat} \> - \> Status message\\
                \>\>\>\>\> $< 0$ : Error.\\
                \>\>\>\>\> $= 0$ : Ok.\\
                \>\>\>\>\> $> 0$ : Warning.\\
\newpagetabs
EXAMPLE OF USE\\
        \>      \{ \\
        \>\>    double \> {\fov points}[300];\\
        \>\>    int    \> {\fov im1} = 10;\\
        \>\>    int    \> {\fov im2} = 10;\\
        \>\>    int    \> {\fov idim} = 3;\\
        \>\>    double \> {\fov par1}[10];\\
        \>\>    double \> {\fov par2}[10];\\
        \>\>    int    \> {\fov con1};\\
        \>\>    int    \> {\fov con2};\\
        \>\>    int    \> {\fov con3};\\
        \>\>    int    \> {\fov con4};\\
        \>\>    int    \> {\fov order1};\\
        \>\>    int    \> {\fov order2};\\
        \>\>    int    \> {\fov iopen1};\\
        \>\>    int    \> {\fov iopen2};\\
        \>\>    SISLSurf \> *{\fov rsurf};\\
        \>\>    int    \> {\fov jstat};\\
        \>\>    \ldots \\
        \>\>s1537(\begin{minipg4}
          {\fov points}, {\fov im1}, {\fov im2}, {\fov idim}, {\fov par1},  {\fov par2}, {\fov con1}, {\fov con2}, {\fov con3},
          {\fov con4}, {\fov order1},  {\fov order2}, {\fov iopen1},
          {\fov iopen2}, \&{\fov rsurf}, \&{\fov jstat});
        \end{minipg4}\\
        \>\>    \ldots \\
        \>      \}
\end{tabbing}

\pgsbreak
\subsection{Compute a surface interpolating a set of points,
derivatives as input.}
\funclabel{s1534}
\begin{minipg1}
  To compute a surface interpolating a set of points, derivatives
  as input.
  The output is represented as a B-spline surface.
\end{minipg1}\\ \\
SYNOPSIS\\
      \>void s1534(\begin{minipg3}
        {\fov points}, {\fov der10}, {\fov der01}, {\fov der11}, {\fov im1}, {\fov im2}, {\fov idim}, {\fov ipar}, {\fov con1}, {\fov con2}, {\fov con3},
        {\fov con4}, {\fov order1},  {\fov order2}, {\fov rsurf}, {\fov jstat})
      \end{minipg3}\\[0.3ex]
      \>\>    double \> {\fov points}[\,];\\
      \>\>    double \> {\fov der10}[\,];\\
      \>\>    double \> {\fov der01}[\,];\\
      \>\>    double \> {\fov der11}[\,];\\
      \>\>    int    \> {\fov im1};\\
      \>\>    int    \> {\fov im2};\\
      \>\>    int    \> {\fov idim};\\
      \>\>    int    \> {\fov ipar};\\
      \>\>    int    \> {\fov con1};\\
      \>\>    int    \> {\fov con2};\\
      \>\>    int    \> {\fov con3};\\
      \>\>    int    \> {\fov con4};\\
      \>\>    int    \> {\fov order1};\\
      \>\>    int    \> {\fov order2};\\
      \>\>    SISLSurf \> **{\fov rsurf};\\
      \>\>    int    \> *{\fov jstat};\\
\\
ARGUMENTS\\
        \>Input Arguments:\\
        \>\>    {\fov points} \> - \>
        \begin{minipg2}
          Array of dimension $idim\times im1\times im2$ containing the
          positions of the nodes (using the same ordering as ecoef in
          the SISLSurf structure).
        \end{minipg2}\\[0.8ex]
        \>\>    {\fov der10}  \> - \>
        \begin{minipg2}
          Array of dimension $idim\times im1\times im2$ containing
          the first derivatives in the first parameter direction.
        \end{minipg2}\\
        \>\>    {\fov der01}  \> - \>
        \begin{minipg2}
          Array of dimension $idim\times im1\times im2$ containing the
          first derivatives in the second parameter direction.
        \end{minipg2}\\[0.8ex]
        \>\>    {\fov der11}  \> - \>
        \begin{minipg2}
          Array of dimension $idim\times im1\times im2$ containing the
          cross derivatives (the twists).
        \end{minipg2}\\[0.8ex]
        \>\>    {\fov im1}  \> - \>
        \begin{minipg2}
          The number of interpolation points in the first parameter
          direction.
        \end{minipg2}\\[0.8ex]
        \>\>    {\fov im2}  \> - \>
        \begin{minipg2}
          The number of interpolation points in the second parameter
          direction.
        \end{minipg2}\\[0.8ex]
        \>\>    {\fov idim} \> - \> Dimension of the space we are working in.\\
        \>\>    {\fov ipar} \> - \> Flag showing the desired
                                    parametrization to be used:\\
                 \>\>\>\>\> $= 1$ : \begin{minipg5}
                                      Mean accumulated cord-length
                                      parameterization.
                                    \end{minipg5}\\[0.3ex]
                 \>\>\>\>\> $= 2$ : Uniform parametrization.\\
%\newpagetabs
        \>\>\>\>\begin{minipg2}
                Numbering of surface edges:\\
                \begin{center}
                  \begin{picture}(180,110)(0,0)
                    \put(50,15){\framebox(80,80)}
                    \put(40,55){\makebox(0,0){3}}
                    \put(140,55){\makebox(0,0){4}}
                    \put(90,5){\makebox(0,0){1}}
                    \put(90,105){\makebox(0,0){2}}

                    \put(60,20){\vector(1,0){40}}
                    \put(85,28){\makebox(0,0){$(i)$}}
                    \put(55,25){\vector(0,1){40}}
                    \put(65,50){\makebox(0,0){$(ii)$}}
                  \end{picture}\\
                  $(i) \; \; \;$ first parameter direction of surface.\\
                  $(ii)$   second parameter direction of surface.\\
                \end{center}
              \end{minipg2}\\ \\
        \>\>    {\fov con1} \> - \> Additional condition along edge 1:\\
                      \>\>\>\>\> $= 0$ : No additional condition.\\
                      \>\>\>\>\> $= 1$ : Zero curvature.\\
        \>\>    {\fov con2} \> - \> Additional condition along edge 2:\\
                      \>\>\>\>\> $= 0$ : No additional condition.\\
                      \>\>\>\>\> $= 1$ : Zero curvature.\\
        \>\>    {\fov con3} \> - \> Additional condition along edge 3:\\
                      \>\>\>\>\> $= 0$ : No additional condition.\\
                      \>\>\>\>\> $= 1$ : Zero curvature.\\
        \>\>    {\fov con4} \> - \> Additional condition along edge 4:\\
                      \>\>\>\>\> $= 0$ : No additional condition.\\
                      \>\>\>\>\> $= 1$ : Zero curvature.\\
        \>\>    {\fov order1} \> - \> Order of surface in first
                                      parameter direction.\\
        \>\>    {\fov order2} \> - \> Order of surface in second
                                      parameter direction.\\
\\
        \>Output Arguments:\\
        \>\>    {\fov rsurf}  \> - \> Pointer to the B-spline surface produced.\\
        \>\>    {\fov jstat} \> - \> Status message\\
                \>\>\>\>\> $< 0$ : Error.\\
                \>\>\>\>\> $= 0$ : Ok.\\
                \>\>\>\>\> $> 0$ : Warning.\\
\newpagetabs
EXAMPLE OF USE\\
        \>      \{ \\
        \>\>    double \> {\fov points}[300];\\
        \>\>    double \> {\fov der10}[300];\\
        \>\>    double \> {\fov der01}[300];\\
        \>\>    double \> {\fov der11}[300];\\
        \>\>    int    \> {\fov im1} = 10;\\
        \>\>    int    \> {\fov im2} = 10;\\
        \>\>    int    \> {\fov idim} = 3;\\
        \>\>    int    \> {\fov ipar};\\
        \>\>    int    \> {\fov con1};\\
        \>\>    int    \> {\fov con2};\\
        \>\>    int    \> {\fov con3};\\
        \>\>    int    \> {\fov con4};\\
        \>\>    int    \> {\fov order1};\\
        \>\>    int    \> {\fov order2};\\
        \>\>    SISLSurf \> *{\fov rsurf};\\
        \>\>    int    \> {\fov jstat};\\
        \>\>    \ldots \\
        \>\>s1534(\begin{minipg4}
          {\fov points}, {\fov der10}, {\fov der01}, {\fov der11}, {\fov im1}, {\fov im2}, {\fov idim}, {\fov ipar}, {\fov con1}, {\fov con2}, {\fov con3},
          {\fov con4}, {\fov order1},  {\fov order2}, \&{\fov rsurf}, \&{\fov jstat});
        \end{minipg4}\\
        \>\>    \ldots \\
        \>      \}
\end{tabbing}

\pgsbreak
\subsection{Compute a surface interpolating a set of points,
derivatives and parameterization as input.}
\funclabel{s1535}
\begin{minipg1}
  Compute a surface interpolating a set of points, derivatives
  and parameterization as input.
  The output is represented as a B-spline surface.
\end{minipg1}\\ \\
SYNOPSIS\\
      \>void s1535(\begin{minipg3}
        {\fov points}, {\fov der10}, {\fov der01}, {\fov der11}, {\fov im1}, {\fov im2}, {\fov idim}, {\fov par1},  {\fov par2},
        {\fov con1}, {\fov con2}, {\fov con3}, {\fov con4}, {\fov order1},  {\fov order2}, {\fov rsurf}, {\fov jstat})
      \end{minipg3}\\[0.3ex]
      \>\>    double \> {\fov points}[\,];\\
      \>\>    double \> {\fov der10}[\,];\\
      \>\>    double \> {\fov der01}[\,];\\
      \>\>    double \> {\fov der11}[\,];\\
      \>\>    int    \> {\fov im1};\\
      \>\>    int    \> {\fov m2};\\
      \>\>    int    \> {\fov idim};\\
      \>\>    double \> {\fov par1}[\,];\\
      \>\>    double \> {\fov par2}[\,];\\
      \>\>    int    \> {\fov con1};\\
      \>\>    int    \> {\fov con2};\\
      \>\>    int    \> {\fov con3};\\
      \>\>    int    \> {\fov con4};\\
      \>\>    int    \> {\fov order1};\\
      \>\>    int    \> {\fov order2};\\
      \>\>    SISLSurf \> **{\fov rsurf};\\
      \>\>    int    \> *{\fov jstat};\\
\\
ARGUMENTS\\
        \>Input Arguments:\\
        \>\>    {\fov points} \> - \>
        \begin{minipg2}
          Array of dimension $idim\times im1\times im2$ containing
          the positions of the nodes (using the same ordering
          as {\fov ecoef} in the SISLSurf structure).
        \end{minipg2}\\[0.8ex]
        \>\>    {\fov der10} \> - \>
        \begin{minipg2}
          Array of dimension $idim\times im1\times im2$ containing the
          first derivatives in the first parameter direction.
        \end{minipg2}\\[0.8ex]
        \>\>    {\fov der01} \> - \>
        \begin{minipg2}
          Array of dimension $idim\times im1\times im2$ containing the
          first derivatives in the second parameter direction.
        \end{minipg2}\\[0.8ex]
        \>\>    {\fov der11} \> - \>
        \begin{minipg2}
          Array of dimension $idim\times im1\times im2$ containing the
          cross derivatives (the twists).
        \end{minipg2}\\[0.8ex]
        \>\>    {\fov im1} \> - \>
        \begin{minipg2}
          The number of interpolation points in the
          first parameter direction.
        \end{minipg2}\\[0.8ex]
        \>\>    {\fov im2} \> - \>
        \begin{minipg2}
          The number of interpolation points in the
          second parameter direction.
        \end{minipg2}\\[0.8ex]
        \>\>    {\fov idim} \> - \> Dimension of the space we are working in.\\
        \>\>    {\fov par1} \> - \> Parametrization in first parameter direction.\\
        \>\>    {\fov par2} \> - \> Parametrization in second parameter direction.\\
        \>\>\>\>\begin{minipg2}
                Numbering of surface edges:\\
                \begin{center}
                  \begin{picture}(180,110)(0,0)
                    \put(50,15){\framebox(80,80)}
                    \put(40,55){\makebox(0,0){3}}
                    \put(140,55){\makebox(0,0){4}}
                    \put(90,5){\makebox(0,0){1}}
                    \put(90,105){\makebox(0,0){2}}

                    \put(60,20){\vector(1,0){40}}
                    \put(85,28){\makebox(0,0){$(i)$}}
                    \put(55,25){\vector(0,1){40}}
                    \put(65,50){\makebox(0,0){$(ii)$}}
                  \end{picture}\\
                  $(i) \; \; \;$ first parameter direction of surface.\\
                  $(ii)$   second parameter direction of surface.\\
                \end{center}
              \end{minipg2}\\ \\
        \>\>    {\fov con1} \> - \> Additional condition along edge 1:\\
                      \>\>\>\>\> $= 0$ : No additional condition.\\
                      \>\>\>\>\> $= 1$ : Zero curvature.\\
        \>\>    {\fov con2} \> - \> Additional condition along edge 2:\\
                      \>\>\>\>\> $= 0$ : No additional condition.\\
                      \>\>\>\>\> $= 1$ : Zero curvature.\\
        \>\>    {\fov con3} \> - \> Additional condition along edge 3:\\
                      \>\>\>\>\> $= 0$ : No additional condition.\\
                      \>\>\>\>\> $= 1$ : Zero curvature.\\
        \>\>    {\fov con4} \> - \> Additional condition along edge 4:\\
                      \>\>\>\>\> $= 0$ : No additional condition.\\
                      \>\>\>\>\> $= 1$ : Zero curvature.\\
        \>\>    {\fov order1} \> - \> Order of surface in first
                                      parameter direction.\\
        \>\>    {\fov order2} \> - \> Order of surface in second
                                      parameter direction.\\
\\
        \>Output Arguments:\\
        \>\>    {\fov rsurf} \> - \> Pointer to the B-spline surface produced.\\
        \>\>    {\fov jstat} \> - \> Status message\\
                \>\>\>\>\> $< 0$ : Error.\\
                \>\>\>\>\> $= 0$ : Ok.\\
                \>\>\>\>\> $> 0$ : Warning.\\
\newpagetabs
EXAMPLE OF USE\\
        \>      \{ \\
        \>\>    double \> {\fov points}[300];\\
        \>\>    double \> {\fov der10}[300];\\
        \>\>    double \> {\fov der01}[300];\\
        \>\>    double \> {\fov der11}[300];\\
        \>\>    int    \> {\fov im1} = 10;\\
        \>\>    int    \> {\fov im2} = 10;\\
        \>\>    int    \> {\fov idim} = 3;\\
        \>\>    double \> {\fov par1}[10];\\
        \>\>    double \> {\fov par2}[10];\\
        \>\>    int    \> {\fov con1};\\
        \>\>    int    \> {\fov con2};\\
        \>\>    int    \> {\fov con3};\\
        \>\>    int    \> {\fov con4};\\
        \>\>    int    \> {\fov order1};\\
        \>\>    int    \> {\fov order2};\\
        \>\>    SISLSurf \> *{\fov rsurf};\\
        \>\>    int    \> {\fov jstat};\\
        \>\>    \ldots \\
        \>\>s1535(\begin{minipg4}
          {\fov points}, {\fov der10}, {\fov der01}, {\fov der11}, {\fov im1}, {\fov im2}, {\fov idim}, {\fov par1},  {\fov par2},
          {\fov con1}, {\fov con2}, {\fov con3}, {\fov con4}, {\fov order1},  {\fov order2}, \&{\fov rsurf}, \&{\fov jstat});
        \end{minipg4}\\
        \>\>    \ldots \\
        \>      \}
\end{tabbing}

\pgsbreak
\subsection{\sloppy Compute a surface by Hermite interpolation, automatic parameter\-ization.}
\funclabel{s1529}
\begin{minipg1}
  Compute the cubic Hermite surface interpolant to the data given.
  More specifically, given positions, (u',v), (u,v'), and (u',v')
  derivatives at points of a rectangular grid,
  the routine
  computes a cubic tensor-product B-spline interpolant to
  the given data with double knots at each data (the first
  knot vector will have double knots at all interior points
  in epar1, quadruple knots at the first and last points,
  and similarly for the second knot vector).
  The output is represented as a B-spline surface.
\end{minipg1}\\ \\
SYNOPSIS\\
        \>void s1529(\begin{minipg3}
          {\fov ep}, {\fov eder10}, {\fov eder01}, {\fov eder11},
          {\fov im1}, {\fov im2}, {\fov idim}, {\fov ipar}, {\fov rsurf}, {\fov jstat})
        \end{minipg3}\\[0.3ex]
        \>\>    double \>  {\fov ep}[\,];\\
        \>\>    double \>  {\fov eder10}[\,];\\
        \>\>    double \>  {\fov eder01}[\,];\\
        \>\>    double \>  {\fov eder11}[\,];\\
        \>\>    int    \>  {\fov im1};\\
        \>\>    int    \>  {\fov im2};\\
        \>\>    int    \>  {\fov idim};\\
        \>\>    int    \>  {\fov ipar};\\
        \>\>    SISLSurf \>  **{\fov rsurf};\\
        \>\>    int    \>  *{\fov jstat};\\
\\
ARGUMENTS\\
        \>Input Arguments:\\
        \>\>    {\fov ep}     \> - \>
        \begin{minipg2}
          Array of dimension $idim\times im1\times im2$ containing the
          positions of the nodes (using the same ordering as ecoef in
          the SISLSurf structure).
        \end{minipg2}\\[0.8ex]
        \>\>    {\fov eder10} \> - \>
        \begin{minipg2}
          Array of dimension $idim\times im1\times im2$ containing the
          first derivative in the first parameter direction.
        \end{minipg2}\\[0.8ex]
        \>\>    {\fov eder01} \> - \>
        \begin{minipg2}
          Array of dimension $idim\times im1\times im2$ containing the
          first derivative in the second parameter direction.
        \end{minipg2}\\[0.8ex]
        \>\>    {\fov eder11} \> - \>
        \begin{minipg2}
          Array of dimension $idim\times im1\times im2$ containing the
          cross derivative (twist vector).
        \end{minipg2}\\[0.8ex]
        \>\>    {\fov ipar}   \> - \>
          Flag showing the desired parametrization to be used:\\
          \>\>\>\>$= 1$\>:
          \begin{minipg5}
            Mean accumulated cord-length para\-meter\-ization.
          \end{minipg5}\\[0.8ex]
          \>\>\>\>$= 2$\>: Uniform parametrization.\\
        \>\>    {\fov im1}    \> - \>
        \begin{minipg2}
          The number of interpolation points in the first parameter
          direction.
        \end{minipg2}\\[0.8ex]
        \>\>    {\fov im2}    \> - \>
        \begin{minipg2}
          The number of interpolation points in the second parameter
          direction.
        \end{minipg2}\\[0.8ex]
        \>\>    {\fov idim}   \> - \> Spatial dimension.\\
\\
        \>Output Arguments:\\
        \>\>    {\fov rsurf}  \> - \> Pointer to the B-spline surface produced.\\
        \>\>    {\fov jstat}  \> - \> Status message\\
                \>\>\>\>\> $< 0$ : Error.\\
                \>\>\>\>\> $= 0$ : Ok.\\
                \>\>\>\>\> $> 0$ : Warning.\\
\\ %\newpagetabs
EXAMPLE OF USE\\
        \>      \{ \\
        \>\>    double \>  {\fov ep}[300];\\
        \>\>    double \>  {\fov eder10}[300];\\
        \>\>    double \>  {\fov eder01}[300];\\
        \>\>    double \>  {\fov eder11}[300];\\
        \>\>    int    \>  {\fov im1} = 10;\\
        \>\>    int    \>  {\fov im2} = 10;\\
        \>\>    int    \>  {\fov idim} = 3;\\
        \>\>    int    \>  {\fov ipar};\\
        \>\>    SISLSurf \> *{\fov rsurf} = NULL;\\
        \>\>    int    \>  {\fov jstat} = 0;\\
        \>\>    \ldots \\
        \>\>s1529(
        \begin{minipg4}
          {\fov ep}, {\fov eder10}, {\fov eder01}, {\fov eder11},
          {\fov im1}, {\fov im2}, {\fov idim}, {\fov ipar}, \&{\fov rsurf}, \&{\fov jstat});
        \end{minipg4}\\
        \>\>    \ldots \\
        \>      \}
\end{tabbing}

\pgsbreak
\subsection{Compute a surface by Hermite interpolation, parameter\-ization as input.}
\funclabel{s1530}
\begin{minipg1}
  To compute the cubic Hermite interpolant to the data given.
  More specifically, given positions, 10, 01, and 11
  derivatives at points of a rectangular grid, the routine
  computes a cubic tensor-product B-spline interpolant to
  the given data with double knots at each data point (the first
  knot vector will have double knots at all interior points
  in epar1, quadruple knots at the first and last points,
  and similarly for the second knot vector).
  The output is represented as a B-spline surface.
\end{minipg1} \\ \\
SYNOPSIS\\
        \>void s1530(\begin{minipg3}
          {\fov ep}, {\fov eder10}, {\fov eder01}, {\fov eder11}, {\fov epar1}, {\fov epar2},
          {\fov im1}, {\fov im2}, {\fov idim}, {\fov rsurf}, {\fov jstat})
        \end{minipg3}\\[0.3ex]
        \>\>    double \> {\fov ep}[\,];\\
        \>\>    double \> {\fov eder10}[\,];\\
        \>\>    double \> {\fov eder01}[\,];\\
        \>\>    double \> {\fov eder11}[\,];\\
        \>\>    double \> {\fov epar1}[\,];\\
        \>\>    double \> {\fov epar2}[\,];\\
        \>\>    int    \> {\fov im1};\\
        \>\>    int    \> {\fov im2};\\
        \>\>    int    \> {\fov idim};\\
        \>\>    SISLSurf \> **{\fov rsurf};\\
        \>\>    int    \> *{\fov jstat};\\
\\
ARGUMENTS\\
        \>Input Arguments:\\
        \>\>    {\fov ep} \> - \>
        \begin{minipg2}
          Array of dimension $idim\times im1\times im2$ containing
          the positions of the nodes (using the same ordering
          as {\fov ecoef} in the SISLSurf structure).
        \end{minipg2}\\[0.8ex]
        \>\>    {\fov eder10} \> - \>
        \begin{minipg2}
          Array of dimension $idim\times im1\times im2$ containing the
          first derivative in the first parameter direction.
        \end{minipg2}\\[0.8ex]
        \>\>    {\fov eder01} \> - \>
        \begin{minipg2}
          Array of dimension $idim\times im1\times im2$ containing the
          first derivative in the second parameter direction.
        \end{minipg2}\\[0.8ex]
        \>\>    {\fov eder11} \> - \>
        \begin{minipg2}
          Array of dimension $idim\times im1\times im2$ containing
          the cross derivative (twist vector).
        \end{minipg2}\\[0.8ex]
        \>\>    {\fov epar1} \> - \>
        \begin{minipg2}
          Array of size {\fov im1} containing the
          parametrization in the first direction.
        \end{minipg2}\\[0.8ex]
        \>\>    {\fov epar2} \> - \>
        \begin{minipg2}
          Array of size {\fov im2} containing the
          parametrization in the first direction.
        \end{minipg2}\\[0.8ex]
        \>\>    {\fov im1} \> - \>
          The number of interpolation points in the 1st param. dir.\\
        \>\>    {\fov im2} \> - \>
          The number of interpolation points in the 2nd param. dir.\\
        \>\>    {\fov idim} \> - \> Dimension of the space we are working in.\\
\\

        \>Output Arguments:\\
        \>\>    {\fov rsurf} \> - \> Pointer to the B-spline surface produced.\\
        \>\>    {\fov jstat} \> - \> Status message\\
                \>\>\>\>\> $< 0$ : Error.\\
                \>\>\>\>\> $= 0$ : Ok.\\
                \>\>\>\>\> $> 0$ : Warning.\\
\\
EXAMPLE OF USE\\
        \>      \{ \\
        \>\>    double \> {\fov ep}[30];\\
        \>\>    double \> {\fov eder10}[30];\\
        \>\>    double \> {\fov eder01}[30];\\
        \>\>    double \> {\fov eder11}[30];\\
        \>\>    double \> {\fov epar1}[2];\\
        \>\>    double \> {\fov epar2}[5];\\
        \>\>    int    \> {\fov im1 = 2};\\
        \>\>    int    \> {\fov im2 = 5};\\
        \>\>    int    \> {\fov idim = 3};\\
        \>\>    SISLSurf \>  *{\fov rsurf};\\
        \>\>    int    \> {\fov jstat};\\
        \>\>    \ldots \\
        \>\>s1530(\begin{minipg4}
          {\fov ep}, {\fov eder10}, {\fov eder01}, {\fov eder11}, {\fov epar1}, {\fov epar2},
          {\fov im1}, {\fov im2}, {\fov idim}, \&{\fov rsurf}, \&{\fov jstat});
        \end{minipg4}\\
        \>\>    \ldots \\
        \>      \}
\end{tabbing}

\pgsbreak
\subsection{Create a lofted surface from a set of B-spline
input curves.}
\funclabel{s1538}
\begin{minipg1}
  To create a lofted surface from a set of B-spline (i.e.\ NOT rational)
  input curves.
  The output is represented as a B-spline surface.
\end{minipg1} \\ \\
SYNOPSIS\\
        \>void s1538(\begin{minipg3}
          {\fov inbcrv}, {\fov vpcurv}, {\fov nctyp}, {\fov astpar}, {\fov iopen}, {\fov iord2},
          {\fov iflag}, {\fov rsurf}, {\fov gpar}, {\fov jstat})
        \end{minipg3}\\[0.3ex]
        \>\>    int       \> {\fov inbcrv};\\
        \>\>    SISLCurve \> *{\fov vpcurv}[\,];\\
        \>\>    int       \> {\fov nctyp}[\,];\\
        \>\>    double    \> {\fov astpar};\\
        \>\>    int       \> {\fov iopen};\\
        \>\>    int       \> {\fov iord2};\\
        \>\>    int       \> {\fov iflag};\\
        \>\>    SISLSurf  \> **{\fov rsurf};\\
        \>\>    double    \> **{\fov gpar};\\
        \>\>    int       \> *{\fov jstat};\\
\\
ARGUMENTS\\
        \>Input Arguments:\\
        \>\>    {\fov inbcrv} \> - \> Number of B-spline curves in the curve set.\\
        \>\>    {\fov vpcurv} \> - \>
        \begin{minipg2}
          Array (length {\fov inbcrv}) of pointers to the
          curves in the curve-set.
        \end{minipg2}\\[0.8ex]
        \>\>    {\fov nctyp}  \> - \>
        \begin{minipg2}
          Array (length {\fov inbcrv}) containing the types
          of curves in the curve-set.
        \end{minipg2}\\[0.8ex]
                \>\>\>\> $=1$ \> : Ordinary curve.\\
                \>\>\>\> $=2$ \> : Knuckle curve. Treated as an ordinary curve.\\
                \>\>\>\> $=3$ \> : Tangent to next curve.\\
                \>\>\>\> $=4$ \> : Tangent to prior curve.\\
                \>\>\>\> ($=5$ \> : Second derivative to prior curve.)\\
                \>\>\>\> ($=6$ \> : Second derivative to next curve.)\\
                \>\>\>\> $=13$ \> : Curve giving start of tangent to next curve.\\
                \>\>\>\> $=14$ \> : Curve giving end of tangent to prior curve.\\
        \>\>    {\fov astpar} \> - \> Start parameter for spline lofting
                                      direction.\\
        \>\>    {\fov iopen} \> - \>
        \begin{minipg2}
          Flag telling if the resulting surface should be open, closed
          or periodic in the lofting direction (i.e.\ not the curve direction).
        \end{minipg2}\\[0.3ex]
        \>\>\>\>  $= 1$ \> : Open.\\
        \>\>\>\>  $= 0$ \> : Closed.\\
        \>\>\>\>  $= -1$ \> : Closed and periodic.\\
        \>\>    {\fov iord2} \> - \>
        \begin{minipg2}
          Maximal order of the surface in the lofting direction.
        \end{minipg2}\\[0.8ex]
\newpagetabs
        \>\>    {\fov iflag} \> - \>
        \begin{minipg2}
          Flag telling if the size of the tangents in the
          derivative curves should be adjusted or not.
        \end{minipg2}\\[0.3ex]
                      \>\>\>\> $= 0$ \> : Do not adjust tangent sizes.\\
                      \>\>\>\> $= 1$ \> : Adjust tangent sizes.\\
\\
        \>Output Arguments:\\
        \>\>    {\fov rsurf} \> - \> Pointer to the B-spline surface produced.\\
        \>\>    {\fov gpar}  \> - \>
        \begin{minipg2}
          The input curves are constant parameter lines
          in the parameter-plane of the produced surface.
          The $i$-th element in this array contains the (constant) value
          of this parameter of the $i$-th. input curve.
        \end{minipg2}\\
        \>\>    {\fov jstat} \> - \> Status message\\
                \>\>\>\> $< 0$ \> : Error.\\
                \>\>\>\> $= 0$ \> : Ok.\\
                \>\>\>\> $> 0$ \> : Warning.\\
\\
EXAMPLE OF USE\\
        \>      \{ \\
        \>\>    int       \> {\fov inbcrv};\\
        \>\>    SISLCurve \> *{\fov vpcurv}[3];\\
        \>\>    int       \> {\fov nctyp}[3];\\
        \>\>    double    \> {\fov astpar};\\
        \>\>    int       \> {\fov iopen};\\
        \>\>    int       \> {\fov iord2};\\
        \>\>    int       \> {\fov iflag};\\
        \>\>    SISLSurf  \> *{\fov rsurf} = NULL;\\
        \>\>    double    \> *{\fov gpar} = NULL;\\
        \>\>    int       \> {\fov jstat} = 0;\\
        \>\>    \ldots \\
        \>\>s1538(\begin{minipg4}
          {\fov inbcrv}, {\fov vpcurv}, {\fov nctyp}, {\fov astpar}, {\fov iopen}, {\fov iord2},
          {\fov iflag}, \&{\fov rsurf}, \&{\fov gpar}, \&{\fov jstat});
        \end{minipg4}\\
        \>\>    \ldots \\
        \>      \}
\end{tabbing}

\pgsbreak
\subsection{Create a lofted surface
               from a set of B-spline input curves and parametrization.}
\funclabel{s1539}
\begin{minipg1}
To create a spline lofted surface
               from a set of input curves. The parametrization
               of the position curves is given in epar.
\end{minipg1} \\ \\
SYNOPSIS\\
        \> void s1539(\begin{minipg3}
            {\fov inbcrv}, {\fov vpcurv}, {\fov nctyp}, {\fov epar}, {\fov astpar}, {\fov iopen}, {\fov iord2},
            {\fov iflag}, {\fov rsurf}, {\fov gpar}, {\fov jstat})
                \end{minipg3}\\
                \>\>    int    \>  	{\fov inbcrv};\\
                \>\>    SISLCurve    \>  *{\fov vpcurv}[\,];\\
                \>\>    int    \>  	{\fov nctyp}[\,];\\
                \>\>    double    \>  {\fov epar}[\,];\\
                \>\>    double	\> {\fov astpar};\\
                \>\>    int    \>  	{\fov iopen};\\
                \>\>    int    \>  	{\fov iord2};\\
                \>\>    int    \>  	{\fov iflag};\\
                \>\>    SISLSurf    \>  **{\fov rsurf};\\
                \>\>    double 	\> **{\fov gpar};\\
                \>\>    int    \>  	*{\fov jstat};\\
\\
ARGUMENTS\\
	\>Input Arguments:\\
        \>\>    {\fov inbcrv}\> - \>  \begin{minipg2}
                    set.
                               \end{minipg2}\\
        \>\>    {\fov vpcurv}\> - \>  \begin{minipg2}
                     Array (length inbcrv) of pointers to the
                       curves in the curve-set.
                               \end{minipg2}\\[0.8ex]
        \>\>    {\fov nctyp}\> - \>  \begin{minipg2}
                     Array (length inbcrv) containing the types
                       of curves in the curve-set.
                               \end{minipg2}\\[0.8ex]
                \>\>\>\> $=1$ \> : Ordinary curve.\\
                \>\>\>\> $=2$ \> : Knuckle curve. Treated as an ordinary curve.\\
                \>\>\>\> $=3$ \> : Tangent to next curve.\\
                \>\>\>\> $=4$ \> : Tangent to previous curve.\\
                \>\>\>\> ($=5$ \> : Second derivative to previous curve.)\\
                \>\>\>\> ($=6$ \> : Second derivative to next curve.)\\
                \>\>\>\> $=13$ \> : Curve giving start of tangent to next curve.\\
                \>\>\>\> $=14$ \> : Curve giving end of tangent to previous curve.\\
        \>\>    {\fov epar}\> - \>  \begin{minipg2}
                     Array containing the wanted parametrization. Only
                       parametervalues corresponding to position
                       curves are given. For closed curves, one additional
                       parameter value must be spesified. The last entry
                       contains the parametrization of the repeted start
                       curve. (if the endpoint is equal to the startpoint
                       of the interpolation the lenght of the array should
                       be equal to inpt1 also in the closed case). The
                       number of entries in the array is thus equal to
                       the number of position curves (number plus one
                       if the curve is closed).
                               \end{minipg2}\\
        \>\>    {\fov astpar}\> - \>  \begin{minipg2}
                    parameter for spline lofting direction.
                               \end{minipg2}\\
        \>\>    {\fov iopen}\> - \>  \begin{minipg2}
                     Flag saying whether the resulting surface should
                       be closed or open.
                               \end{minipg2}\\
        \>\>\>\>  $= 1$ \> : Open.\\
        \>\>\>\>  $= 0$ \> : Closed.\\
        \>\>\>\>  $= -1$ \> : Closed and periodic.\\
        \>\>    {\fov iord2}\> - \>  \begin{minipg2}
                    spline basis in the
                       lofting direction.
                               \end{minipg2}\\[0.8ex]
        \>\>    {\fov iflag}\> - \>  \begin{minipg2}
                     Flag saying whether the size of the tangents in the
                       derivative curves should be adjusted or not.
                               \end{minipg2}\\
                      \>\>\>\> $= 0$ \> : Do not adjust tangent sizes.\\
                      \>\>\>\> $= 1$ \> : Adjust tangent sizes.\\
\\
	\>Output Arguments:\\
        \>\>    {\fov rsurf}\> - \>  \begin{minipg2}
                     Pointer to the surface produced.
                               \end{minipg2}\\
        \>\>    {\fov gpar}  \> - \>
        \begin{minipg2}
          The input curves are constant parameter lines
          in the parameter-plane of the produced surface.
          The $i$-th element in this array contains the (constant) value
          of this parameter of the $i$-th. input curve.
        \end{minipg2}\\
        \>\>    {\fov jstat} \> - \> Status message\\
                \>\>\>\> $< 0$ \> : Error.\\
                \>\>\>\> $= 0$ \> : Ok.\\
                \>\>\>\> $> 0$ \> : Warning.\\
\\
EXAMPLE OF USE\\
		\>      \{ \\

                \>\>    int    \>  	{\fov inbcrv};\\
                \>\>    SISLCurve    \>  *{\fov vpcurv}[\,];\\
                \>\>    int    \>  	{\fov nctyp}[\,];\\
                \>\>    double    \>  {\fov epar}[\,];\\
                \>\>    double	\> {\fov astpar};\\
                \>\>    int    \>  	{\fov iopen};\\
                \>\>    int    \>  	{\fov iord2};\\
                \>\>    int    \>  	{\fov iflag};\\
                \>\>    SISLSurf    \>  **{\fov rsurf};\\
                \>\>    double 	\> **{\fov gpar};\\
                \>\>    int    \>  	*{\fov jstat};\\                \>\>    \ldots \\
        \>\>s1539(\begin{minipg4}
            {\fov inbcrv}, {\fov vpcurv}, {\fov nctyp}, {\fov epar}, {\fov astpar}, {\fov iopen}, {\fov iord2},
            {\fov iflag}, {\fov rsurf}, {\fov gpar}, {\fov jstat});
                \end{minipg4}\\
                \>\>    \ldots \\
		\>      \}
\end{tabbing}

\pgsbreak
\subsection{Create a rational lofted surface from a set of rational input-curves}
\funclabel{s1508}
\begin{minipg1}
To create a rational lofted surface from a set of rational input-curves.
\end{minipg1} \\ \\ 
SYNOPSIS\\
      \> void s1508(\begin{minipg3}
	   {\fov inbcrv}, {\fov vpcurv}, {\fov par\_arr}, {\fov rsurf}, {\fov jstat})
	       \end{minipg3}\\
               \>\> int \> {\fov inbcrv};\\
	       \>\> SISLCurve \> *{\fov vpcurv[\,]};\\
	       \>\> double \> {\fov par\_arr[\,]};\\
	       \>\> SISLSurf \> **{\fov rsurf};\\
	       \>\> int \> *{\fov jstat};\\
\\
ARGUMENTS\\
      \> Input Arguments:\\
      \>\> {\fov inbcrv} \> - \> \begin{minipg2}
	           Number of NURBS-curves in the curve set.
		                 \end{minipg2}\\
      \>\> {\fov vpcurv} \> - \> \begin{minipg2}
	           Array (length {\fov inbcrv}) of pointers to the curves in the curve-set.
		                 \end{minipg2}\\
      \>\> {\fov par\_arr} \> - \> \begin{minipg2}
	           The required parametrization, must be strictly increasing, 
		   length {\fov inbcrv}.
		                 \end{minipg2}\\

      \> Output Arguments: \\
      \>\> {\fov rsurf} \> - \> \begin{minipg2}
	           Pointer to the NURBS surface produced.
		                \end{minipg2}\\
      \>\> {\fov jstat} \> - \> status message \\
                 \>\>\>\>\>   $ < 0 $ : Error.\\
		 \>\>\>\>\>   $ = 0 $ : Ok. \\
		 \>\>\>\>\>   $ > 0 $ : Warning.\\
\\
EXAMPLE OF USE\\
               \>      \{ \\
	       
	       \>\>    int \> {\fov inbcrv};\\
	       \>\>    SISLCurve \> *{\fov vpcurv[3]};\\
	       \>\>    double \> {\fov par\_arr[3]};\\
	       \>\>    SISLSurf  \> *{\fov rsurf} = NULL;\\
	       \>\>    int \> {\fov jstat} = 0;\\
	       \>\>    \ldots \\
	       \>\>    s1508({\fov inbcrv}, {\fov vpcurv}, {\fov par\_arr}, \&{\fov rsurf}, \&{\fov jstat}); \\
	       \>\>    \ldots \\
	       \>      \}


\end{tabbing}
      

\pgsbreak
\subsection{Compute a rectangular blending surface from a set of
\mbox{B-spline} input curves.}
\funclabel{s1390}
\begin{minipg1}
  Make a 4-edged blending surface between 4 B-spline (i.e.\ NOT
  rational) curves where each curve is associated with a number of
  cross-derivative B-spline (i.e.\ NOT rational) curves.
  The output is represented as a B-spline surface.
  The input curves are numbered successively around the blending
  parameter, and the directions of the curves are expected to be as
  follows when this routine is entered:
\begin{center}
\begin{picture}(180,110)(0,0)
        \put(50,15){\framebox(80,80)}
        \put(40,55){\vector(0,1){20}}
        \put(140,55){\vector(0,1){20}}
        \put(90,5){\vector(1,0){20}}
        \put(90,105){\vector(1,0){20}}
        \put(40,45){\makebox(0,0){4}}
        \put(140,45){\makebox(0,0){2}}
        \put(80,5){\makebox(0,0){1}}
        \put(80,105){\makebox(0,0){3}}

        \put(60,20){\vector(1,0){40}}
        \put(85,28){\makebox(0,0){$(i)$}}
        \put(55,25){\vector(0,1){40}}
        \put(65,50){\makebox(0,0){$(ii)$}}
\end{picture}\\
$(i) \; \; \;$ first parameter direction of the surface.\\
$(ii)$   second parameter direction of the surface.\\
\end{center}
        NB!     The cross-derivatives are always pointing into the patch, and note the
                directions in the above diagram.
\end{minipg1}\\ \\
SYNOPSIS\\
        \>void s1390(\begin{minipg3}
                {\fov curves}, {\fov surf}, {\fov numder}, {\fov stat})
                \end{minipg3}\\[0.3ex]
                \>\>    SISLCurve       \>      *{\fov curves}[\,];\\
                \>\>    SISLSurf        \>      **{\fov surf};\\
                \>\>    int     \>      {\fov numder}[\,];\\
                \>\>    int     \>      *{\fov stat};\\
\\
ARGUMENTS\\
        \>Input Arguments:\\
        \>\>    {\fov curves}   \> - \>
        \begin{minipg2}
          Pointers to the boundary B-spline curves:\\
          $curves[i], i=0,\ldots,numder[0]-1,$
          are pointers to position and cross-derivatives along the first edge.
        \end{minipg2}\\[0.3ex]
        \>\>\>\>
        \begin{minipg2}
          $curves[i],$\\
          $i=numder[0],\ldots,numder[0]+numder[1]-1,$
          are pointers to position  and cross-derivatives
          along the second edge.
        \end{minipg2}\\[0.3ex]
        \>\>\>\>
        \begin{minipg2}
          $curves[i], i=numder[0]+numder[1],\ldots,$\\
          $numder[0]+numder[1]+numder[2]-1,$
          are pointers to position and cross-derivatives
          along the third edge.
        \end{minipg2}\\[0.3ex]
\newpagetabs
        \>\>\>\>
        \begin{minipg2}
          $curves[i],$\\
          $i=numder[0]+numder[1]+numder[2],\ldots,$\\
          $numder[0]+numder[1]+numder[2]+numder[3]-1,$
          are  pointers to position
          and cross-derivatives along the fourth edge.
        \end{minipg2}\\[0.3ex]
        \>\>    {\fov numder}   \> - \> \begin{minipg2}
                                Array of length 4, numder[i] gives the
                                 number of curves on edge number $i+1$.
                                \end{minipg2}\\[0.3ex]
\\
        \>Output Arguments:\\
        \>\>    {\fov surf}\> - \>      \begin{minipg2}
                                Pointer to the blending B-spline surface.
                                \end{minipg2}\\
        \>\>    {\fov stat}     \> - \> Status messages\\
                \>\>\>\>\>              $> 0$   : warning\\
                \>\>\>\>\>              $= 0$   : ok\\
                \>\>\>\>\>              $< 0$   : error\\
\\
EXAMPLE OF USE\\
                \>      \{ \\
                \>\>    SISLCurve       \>      *{\fov curves}[8];\\
                \>\>    SISLSurf        \>      *{\fov surf};\\
                \>\>    int     \>      {\fov numder}[4];\\
                \>\>    int     \>      {\fov stat};\\
                \>\>    \ldots \\
        \>\>s1390(\begin{minipg4}
                {\fov curves}, \&{\fov surf}, {\fov numder}, \&{\fov stat})
                        \end{minipg4}\\
                \>\>    \ldots \\
                \>      \}
\end{tabbing}

\pgsbreak
\subsection{Compute a first derivative continuous blending surface set,
over a 3-, 4-, 5- or 6-sided region in space, from a set of B-spline
input curves.}
\funclabel{s1391}
\begin{minipg1}
  To create a first derivative continuous blending surface set over a
  3-, 4-, 5- and 6-sided region in space. The boundary of the
  region are B-spline (i.e.\ NOT rational) curves and the cross boundary
  derivatives  are given as B-spline (i.e.\ NOT rational) curves.
  This function automatically
  preprocesses the input cross tangent curves in order to
  make them suitable for the blending. Thus, the cross tangent
  curves should be taken as the cross tangents of the
  surrounding surface. It is not necessary and not advisable
  to match tangents etc. in the corners.
  The output is represented as a set of B-spline surfaces.
\end{minipg1}\\ \\
SYNOPSIS\\
        \>void s1391(\begin{minipg3}
          {\fov pc}, {\fov ws}, {\fov icurv}, {\fov nder}, {\fov jstat})
        \end{minipg3}\\[0.3ex]
        \>\>    SISLCurve \> **{\fov pc};\\
        \>\>    SISLSurf  \> ***{\fov ws};\\
        \>\>    int       \>  {\fov icurv};\\
        \>\>    int       \>  {\fov nder}[\,];\\
        \>\>    int       \>  *{\fov jstat};\\
\\
ARGUMENTS\\
        \>Input Arguments:\\
        \>\>    {\fov pc} \> - \>
        \begin{minipg2}
          Pointers to boundary B-spline curves. All curves must
          have same parameter direction around the patch,
          either clockwise or counterclockwise.
          $pc1[i], i=0,\dots nder[0]-1$ are pointers to position
          and cross-derivatives along first edge.
          $pc1[i], i=nder[0],\dots nder[1]-1$ are pointers
          to position and cross-derivatives along second edge.\\
          \hspace*{4em}$\vdots$\\
          \[
            pc1[i], i=nder[0]+\dots+nder[icurv-2],\dots, nder[icurv-1]-1
          \]
          are pointers to position and cross-derivatives along fourth edge.
        \end{minipg2}\\[0.8ex]
        \>\>    {\fov icurv} \> - \> Number of boundary curves (3, 5, 4 or 6).\\
        \>\>    {\fov nder} \> - \>
        \begin{minipg2}
          {\fov nder[i]} gives number of curves on edge number
          $i+1$. These numbers has to be equal to 2.
          The vector is of length {\fov icurv}.
        \end{minipg2}\\[0.8ex]
\\
\newpagetabs
        \>Output Arguments:\\
        \>\>    {\fov ws} \> - \>
        \begin{minipg2}
          These are pointers to the blending B-spline surfaces. The vector is of
          length {\fov icurv}.
        \end{minipg2}\\[0.8ex]
        \>\>    {\fov jstat} \> - \> Status message\\
                \>\>\>\>\> $< 0$ : Error.\\
                \>\>\>\>\> $= 0$ : Ok.\\
                \>\>\>\>\> $> 0$ : Warning.\\
\\
EXAMPLE OF USE\\
        \>      \{ \\
        \>\>    SISLCurve \> **{\fov pc};\\
        \>\>    SISLSurf  \> **{\fov ws} = NULL;\\
        \>\>    int       \> {\fov icurv} = 5;\\
        \>\>    int       \> {\fov nder}[5];\\
        \>\>    int       \> {\fov jstat} = 0;\\
        \>\>    \ldots \\
        \>\>s1391(\begin{minipg4}
          {\fov pc}, \&{\fov ws}, {\fov icurv}, {\fov nder}, \&{\fov jstat});
        \end{minipg4}\\
        \>\>    \ldots \\
        \>      \}
\end{tabbing}

\pgsbreak
\subsection{Compute a surface, representing a Gordon patch, from a set
of B-spline input curves.}
\funclabel{s1401}
\begin{minipg1}
  Compute a Gordon patch, given position and cross tangent conditions as
  B-spline (i.e.\ NOT rational) curves at the boundary of a squared
  region and the twist vector in the corners.
  The output is represented as a B-spline surface.
\end{minipg1} \\ \\
SYNOPSIS\\
        \>void s1401(\begin{minipg3}
          {\fov vcurve}, {\fov etwist}, {\fov rsurf}, {\fov jstat})
        \end{minipg3}\\[0.3ex]
        \>\>    double    \> {\fov etwist}[\,];\\
        \>\>    SISLCurve \> *{\fov vcurve}[\,];\\
        \>\>    int       \> *{\fov jstat};\\
        \>\>    SISLSurf  \> **{\fov rsurf};\\
\\
ARGUMENTS\\
        \>Input Arguments:\\
        \>\>    {\fov vcurve} \> - \>
        \begin{minipg2}
          Position and cross-tangent B-spline curves around the square
          region. For each edge of the region position and cross-tangent
          curves are given. The dimension of the array is 8.
        \end{minipg2}\\[0.8ex]
        \>\>\>\>\begin{minipg2}
                  The orientation is as follows:\\
                  \begin{center}
                    \begin{picture}(180,110)(0,0)
                      \put(50,15){\framebox(80,80)}
                      \put(40,55){\vector(0,1){20}}
                      \put(140,55){\vector(0,1){20}}
                      \put(90,5){\vector(1,0){20}}
                      \put(90,105){\vector(1,0){20}}
                      \put(40,45){\makebox(0,0){4}}
                      \put(140,45){\makebox(0,0){2}}
                      \put(80,5){\makebox(0,0){1}}
                      \put(80,105){\makebox(0,0){3}}

                      \put(60,20){\vector(1,0){40}}
                      \put(85,28){\makebox(0,0){$(i)$}}
                      \put(55,25){\vector(0,1){40}}
                      \put(65,50){\makebox(0,0){$(ii)$}}
                    \end{picture}\\
                    $(i) \; \; \;$ first parameter direction of the surface.\\
                    $(ii)$   second parameter direction of the surface.\\
                  \end{center}
                \end{minipg2}\\ \\
        \>\>    {\fov etwist} \> - \>
        \begin{minipg2}
          Twist-vectors of the corners of the vertex region. The first
          element of the array is the twist in the corner before the
          first edge, etc. The dimension of the array is 4 times the
          spatial dimension of the input curves (currently only 3D).
        \end{minipg2}\\[0.8ex]
\newpagetabs
        \>Output Arguments:\\
        \>\>    {\fov rsurf} \> - \> Gordons-patch represented as a
                                     B-spline surface.\\
        \>\>    {\fov jstat} \> - \> Status message\\
                \>\>\>\>\> $< 0$ : Error.\\
                \>\>\>\>\> $= 0$ : Ok.\\
                \>\>\>\>\> $> 0$ : Warning.\\
\\
EXAMPLE OF USE\\
        \>      \{ \\
        \>\>    int       \> {\fov idim} = 3;\\
        \>\>    double    \> {\fov etwist}[4*{\fov idim}];\\
        \>\>    SISLCurve \> *{\fov vcurve}[8];\\
        \>\>    int       \> {\fov jstat} = 0;\\
        \>\>    SISLSurf  \> *{\fov rsurf} = NULL;\\
        \>\>    \ldots \\
        \>\>s1401(\begin{minipg4}
          {\fov vcurve}, {\fov etwist}, \&{\fov rsurf}, \&{\fov jstat});
        \end{minipg4}\\
        \>\>    \ldots \\
        \>      \}
\end{tabbing}

\pgsbreak
\section{Approximation}
Two kinds of surfaces are treated in this section. The first is
approximation of special shape properties like rotation or sweeping. The
second is offsets to surfaces.

All functions require a tolerance for use in the approximation. It is
useful to note that there is a close relation between the size of the
tolerance and the amount of data for the surface.
\subsection{Compute a surface using the input points as
control vertices, automatic parameterization.}
\funclabel{s1620}
\begin{minipg1}
  To calculate a surface using the input points as
  control vertices. The parametrization is calculated
  according to {\fov ipar}.
  The output is represented as a B-spline surface.
\end{minipg1} \\ \\
SYNOPSIS\\
        \>void s1620(\begin{minipg3}
          {\fov epoint}, {\fov inbpnt1}, {\fov inbpnt2}, {\fov ipar},
          {\fov iopen1}, {\fov iopen2}, {\fov ik1}, {\fov ik2}, {\fov idim}, {\fov rs}, {\fov jstat})
        \end{minipg3}\\[0.3ex]
        \>\>    double \> {\fov epoint}[\,];\\
        \>\>    int    \> {\fov inbpnt1};\\
        \>\>    int    \> {\fov inbpnt2};\\
        \>\>    int    \> {\fov ipar};\\
        \>\>    int    \> {\fov iopen1};\\
        \>\>    int    \> {\fov iopen2};\\
        \>\>    int    \> {\fov ik1};\\
        \>\>    int    \> {\fov ik2};\\
        \>\>    int    \> {\fov idim};\\
        \>\>    SISLSurf \> **{\fov rs};\\
        \>\>    int    \> *{\fov jstat};\\
\\
ARGUMENTS\\
        \>Input Arguments:\\
        \>\>    {\fov epoint} \> - \>
        \begin{minipg2}
          The array containing the points to be used as
          controlling vertices of the B-spline surface.
        \end{minipg2}\\
        \>\>    {\fov inbpnt1} \> - \> The number of points in first
                                       parameter direction.\\
        \>\>    {\fov inbpnt2} \> - \> The number of points in second
                                       parameter direction.\\
        \>\>    {\fov ipar} \> - \> Flag showing the desired
                                    parametrization to be used:\\
                 \>\>\>\> $= 1$ \>:
                 \begin{minipg5}
                   Mean accumulated cord-length
                   parameterization.
                 \end{minipg5}\\[0.8ex]
                 \>\>\>\> $= 2$ \>: Uniform parametrization.\\
        \>\>    {\fov iopen1} \> - \>
          Open/close condition in the first parameter direction:\\
          \>\>\>\> $=1$ \>: Open.\\
          \>\>\>\> $=0$ \>: Closed. \\
          \>\>\>\> $=-1$ \>: Closed and periodic.\\
\newpagetabs
        \>\>    {\fov iopen2} \> - \>
          Open/close condition in the second parameter direction:\\
          \>\>\>\> $=1$ \>: Open.\\
          \>\>\>\> $=0$ \>: Closed. \\
          \>\>\>\> $=-1$ \>: Closed and periodic.\\
        \>\>    {\fov ik1} \> - \> The order of the surface in first direction.\\
        \>\>    {\fov ik2} \> - \> The order of the surface in second direction.\\
        \>\>    {\fov idim} \> - \> The dimension of the space.\\
\\
        \>Output Arguments:\\
        \>\>    {\fov rs} \> - \> Pointer to the B-spline surface.\\
         \>\>    {\fov jstat} \> - \> Status message\\
                \>\>\>\> $< 0$ \>: Error.\\
                \>\>\>\> $= 0$ \>: Ok.\\
                \>\>\>\> $> 0$ \>: Warning.\\
\\
EXAMPLE OF USE\\
        \>      \{ \\
        \>\>    double \> {\fov epoint}[300];\\
        \>\>    int    \> {\fov inbpnt1} = 10;\\
        \>\>    int    \> {\fov inbpnt2} = 10;\\
        \>\>    int    \> {\fov ipar};\\
        \>\>    int    \> {\fov iopen1};\\
        \>\>    int    \> {\fov iopen2};\\
        \>\>    int    \> {\fov ik1};\\
        \>\>    int    \> {\fov ik2};\\
        \>\>    int    \> {\fov idim} = 3;\\
        \>\>    SISLSurf \> *{\fov rs} = NULL;\\
        \>\>    int    \> {\fov jstat} = 0;\\
        \>\>    \ldots \\
        \>\>s1620(\begin{minipg4}
          {\fov epoint}, {\fov inbpnt1}, {\fov inbpnt2}, {\fov ipar},
          {\fov iopen1}, {\fov iopen2}, {\fov ik1}, {\fov ik2}, {\fov idim}, \&{\fov rs}, \&{\fov jstat});
        \end{minipg4}\\
        \>\>    \ldots \\
        \>      \}
\end{tabbing}

\pgsbreak
\subsection{Compute a linear swept surface.}
\funclabel{s1332}
\begin{minipg1}
  To create a linear swept surface by making the tensor-product of two curves.
\end{minipg1} \\ \\
SYNOPSIS\\
        \>void s1332(\begin{minipg3}
                                {\fov curve1}, {\fov curve2}, {\fov epsge}, {\fov point}, {\fov surf}, {\fov stat})
                \end{minipg3}\\[0.3ex]
                \>\>    SISLCurve       \>      *{\fov curve1};\\
                \>\>    SISLCurve       \>      *{\fov curve2};\\
                \>\>    double  \>      {\fov epsge};\\
                \>\>    double  \>      {\fov point}[\,];\\
                \>\>    SISLSurf        \>      **{\fov surf};\\
                \>\>    int     \>      *{\fov stat};\\

\\
ARGUMENTS\\
        \>Input Arguments:\\
        \>\>    {\fov curve1}   \> - \> \begin{minipg2}
                                Pointer to curve 1.
                                \end{minipg2}\\
        \>\>    {\fov curve2}   \> - \> \begin{minipg2}
                                Pointer to curve 2.
                                \end{minipg2}\\
        \>\>    {\fov epsge}\> - \>     \begin{minipg2}
                                Maximal deviation allowed between the true swept
                                surface and the generated surface.
                                \end{minipg2}\\[0.3ex]
        \>\>    {\fov point}    \> - \> \begin{minipg2}
                Point near the curve to sweep along. The vertices of the new surface are made by adding the vector from point to each of the vertices on the sweep curve, to each of the vertices on the other curve.
                                \end{minipg2} \\[0.8ex]
\\
        \>Output Arguments:\\
        \>\>    {\fov surf}     \> - \> \begin{minipg2}
                                Pointer to the surface produced.
                                \end{minipg2}\\
        \>\>    {\fov stat}     \> - \> Status messages\\
                \>\>\>\>\>              $>0$    : warning\\
                \>\>\>\>\>              $=0$    : ok\\
                \>\>\>\>\>              $<0$    : error\\
\\
EXAMPLE OF USE\\
                \>      \{ \\
                \>\>    curve   \>      *{\fov curve1};\\
                \>\>    curve   \>       *{\fov curve2};\\
                \>\>    double  \>      {\fov epsge};\\
                \>\>    double  \>      {\fov point}[3];\\
                \>\>    SISLSurf        \>      *{\fov surf};\\
                \>\>    int     \>      {\fov stat};\\
                \>\>    \ldots \\
        \>\>s1332(\begin{minipg4}
                {\fov curve1}, {\fov curve2}, {\fov epsge}, {\fov point}, \&{\fov surf}, \&{\fov stat});
                \end{minipg4}\\
                \>\>    \ldots \\
                \>      \}
\end{tabbing}

\pgsbreak
\subsection{Compute a rotational swept surface.}
\funclabel{s1302}
\begin{minipg1}
  To create a rotational swept surface by rotating a curve
  a given angle around the axis defined by {\fov point}[\,] and
  {\fov axis}[\,].
  The maximal deviation allowed between the true rotational surface and the
  generated surface, is {\fov epsge}.
  If {\fov epsge} is set to 0, a NURBS surface is generated and if
  $epsge >0$, a B-spline surface is generated.
\end{minipg1} \\ \\
SYNOPSIS\\
        \>void s1302(\begin{minipg3}
                        {\fov curve}, {\fov epsge}, {\fov angle}, {\fov point}, {\fov axis}, {\fov surf}, {\fov stat})
                \end{minipg3}\\[0.3ex]
                \>\>    SISLCurve       \>      *{\fov curve};\\
                \>\>    double  \>      {\fov epsge};\\
                \>\>    double  \>      {\fov angle};\\
                \>\>    double  \>      {\fov point}[\,];\\
                \>\>    double  \>      {\fov axis}[\,];\\
                \>\>    SISLSurf        \>      **{\fov surf};\\
                \>\>    int     \>      *{\fov stat};\\
\\
ARGUMENTS\\
        \>Input Arguments:\\
        \>\>    {\fov curve}    \> - \> \begin{minipg2}
                                Pointer to the curve that is to be rotated.
                                \end{minipg2}\\
        \>\>    {\fov epsge}    \> - \> \begin{minipg2}
                                Maximal deviation allowed between the true rotational
                                surface and the generated surface.
                                \end{minipg2}\\[0.3ex]
        \>\>    {\fov angle}    \> - \> \begin{minipg2}
                        The rotational angle. The angle is counterclockwise around axis. If the absolute
                        value of the angle is greater than $2\pi$ then a rotational surface that is
                        closed in the rotation direction is made.
                                \end{minipg2}\\[0.8ex]
        \>\>    {\fov point}    \> - \> \begin{minipg2}
                                Point on the rotational axis.
                                \end{minipg2} \\
        \>\>    {\fov axis}     \> - \> \begin{minipg2}
                                Direction of rotational axis.
                                \end{minipg2} \\
\\
        \>Output Arguments:\\
        \>\>    {\fov surf}     \> - \> \begin{minipg2}
                                        Pointer to the produced surface.
                                        This will be a NURBS (i.e.\
                                        rational) surface if $epsge=0$
                                        and a \mbox{B-spline} (i.e.\
                                        non-rational) surface if $epsge>0$.
                                \end{minipg2}\\
        \>\>    {\fov stat}     \> - \> Status messages\\
                \>\>\>\>\>              $>0$    : warning\\
                \>\>\>\>\>              $=0$    : ok\\
                \>\>\>\>\>              $<0$    : error\\
\newpagetabs
EXAMPLE OF USE\\
                \>      \{ \\
                \>\>    SISLCurve       \>      *{\fov curve};\\
                \>\>    double  \>      {\fov epsge};\\
                \>\>    double  \>      {\fov angle};\\
                \>\>    double  \>      {\fov point}[3];\\
                \>\>    double  \>      {\fov axis}[3];\\
                \>\>    SISLSurf        \>      *{\fov surf};\\
                \>\>    int     \>      {\fov stat};\\
                \>\>    \ldots \\
        \>\>    s1302(\begin{minipg4}
                        {\fov curve}, {\fov epsge}, {\fov angle}, {\fov point}, {\fov axis}, \&{\fov surf}, \&{\fov stat});
                        \end{minipg4}\\
                \>\>    \ldots\\
                \>      \}
\end{tabbing}

\pgsbreak
\subsection{Compute a surface approximating the offset of a surface.}
\funclabel{s1365}
\begin{minipg1}
  Create a surface approximating the offset of a surface.
  The output is represented as a B-spline surface.\\
  With an offset of zero, this routine can be used to approximate any
  NURBS (rational) surface with a B-spline (non-rational) surface.
\end{minipg1} \\ \\
SYNOPSIS\\
        \>void s1365(\begin{minipg3}
          {\fov ps}, {\fov aoffset}, {\fov aepsge}, {\fov amax}, {\fov idim}, {\fov rs}, {\fov jstat})
        \end{minipg3}\\[0.3ex]
        \>\>    SISLSurf \> *{\fov ps};\\
        \>\>    double   \> {\fov aoffset};\\
        \>\>    double   \> {\fov aepsge};\\
        \>\>    double   \> {\fov amax};\\
        \>\>    int      \> {\fov idim};\\
        \>\>    SISLSurf \> **{\fov rs};\\
        \>\>    int      \> *{\fov jstat};\\
\\
ARGUMENTS\\
        \>Input Arguments:\\
        \>\>    {\fov ps} \> - \> The input surface.\\
        \>\>    {\fov aoffset} \> - \>
        \begin{minipg2}
          The offset distance.
          If $idim=2$ a positive signe on this value put the
          offset on the side of the positive normal vector,
          and a negative sign puts the offset on the sign
          of the negative normal vector.
          If $idim=3$ the offset is determined by the cross
          product of the tangent vector and the anorm vector.
          The offset distance is multiplied by this vector.
        \end{minipg2}\\[0.8ex]
        \>\>    {\fov aepsge} \> - \>
        \begin{minipg2}
          Maximal deviation allowed between true offset surface
          and the approximated offset surface.
        \end{minipg2}\\[0.8ex]
        \>\>    {\fov amax} \> - \>
        \begin{minipg2}
          Maximal stepping length. Is negleceted if $amax\leq aepsge$.
          If $amax=0$ then a maximal step length of the longest box side
          is used.
        \end{minipg2}\\[0.8ex]
        \>\>    {\fov idim} \> - \> The dimension of the space (2 or 3).
\\
        \>Output Arguments:\\
        \>\>    {\fov rs} \> - \>
        \begin{minipg2}
          The approximated offset represented as
          a \mbox{B-spline} surface.
        \end{minipg2}\\[0.8ex]
        \>\>    {\fov jstat} \> - \> Status message\\
                \>\>\>\>\> $< 0$ : Error.\\
                \>\>\>\>\> $= 0$ : Ok.\\
                \>\>\>\>\> $> 0$ : Warning.\\
\newpagetabs
EXAMPLE OF USE\\
        \>      \{ \\
        \>\>    SISLSurf \> *{\fov ps};\\
        \>\>    double   \> {\fov aoffset};\\
        \>\>    double   \> {\fov aepsge};\\
        \>\>    double   \> {\fov amax};\\
        \>\>    int      \> {\fov idim};\\
        \>\>    SISLSurf \> *{\fov rs};\\
        \>\>    int      \> {\fov jstat};\\
        \>\>    \ldots \\
        \>\>s1365(\begin{minipg4}
          {\fov ps}, {\fov aoffset}, {\fov aepsge}, {\fov amax}, {\fov idim}, \&{\fov rs}, \&{\fov jstat});
        \end{minipg4}\\
        \>\>    \ldots \\
        \>      \}
\end{tabbing}

\pgsbreak
%\section{Mirror a Surface} Moved into func/s1601.tex
\section{Mirror a Surface}
\funclabel{s1601}
\begin{minipg1}
  Mirror a surface about a plane.
\end{minipg1} \\ \\
SYNOPSIS\\
        \>void s1601(\begin{minipg3}
          {\fov psurf}, {\fov epoint}, {\fov enorm}, {\fov idim}, {\fov rsurf}, {\fov jstat})
        \end{minipg3}\\[0.3ex]
        \>\>    SISLSurf \> *{\fov psurf};\\
        \>\>    double   \> {\fov epoint}[\,];\\
        \>\>    double   \> {\fov enorm}[\,];\\
        \>\>    int      \> {\fov idim};\\
        \>\>    SISLSurf \> **{\fov rsurf};\\
        \>\>    int      \> *{\fov jstat};\\
\\
ARGUMENTS\\
        \>Input Arguments:\\
        \>\>    {\fov psurf} \> - \> The input surface.\\
        \>\>    {\fov epoint}\> - \> A point in the plane.\\
        \>\>    {\fov enorm} \> - \> The normal vector to the plane.\\
        \>\>    {\fov idim}  \> - \>
        \begin{minipg2}
          The dimension of the space, must be the same as the surface.
        \end{minipg2}\\[0.8ex]
        \>Output Arguments:\\
        \>\>    {\fov rsurf} \> - \> Pointer to the mirrored surface.\\
        \>\>    {\fov jstat} \> - \> Status message\\
                \>\>\>\>\> $< 0$ : Error.\\
                \>\>\>\>\> $= 0$ : Ok.\\
                \>\>\>\>\> $> 0$ : Warning.\\
EXAMPLE OF USE\\
        \>      \{ \\
        \>\>    SISLSurf \> *{\fov psurf};\\
        \>\>    double   \> {\fov epoint}[3];\\
        \>\>    double   \> {\fov enorm}[3];\\
        \>\>    int      \> {\fov idim} = 3;\\
        \>\>    SISLSurf \> *{\fov rsurf} = NULL;\\
        \>\>    int      \> {\fov jstat} = 0;\\
        \>\>    \ldots \\
        \>\>s1601(\begin{minipg4}
          {\fov psurf}, {\fov epoint}, {\fov enorm}, {\fov idim}, \&{\fov rsurf}, \&{\fov jstat});
        \end{minipg4}\\
        \>\>    \ldots \\
        \>      \}
\end{tabbing}

\pgsbreak
\section{Conversion}
\subsection{Convert a surface of order up to four to a mesh of Coons
patches.}
\funclabel{s1388}
\begin{minipg1}
  To convert a surface of order less than or equal to 4 in both
  directions to a mesh of Coons patches with uniform parameterization.
  The function assumes that the surface is $C^1$ continuous.
\end{minipg1} \\ \\
SYNOPSIS\\
        \>void s1388(\begin{minipg3}
                        {\fov surf}, {\fov coons}, {\fov numcoons1}, {\fov numcoons2}, {\fov dim}, {\fov stat})
                \end{minipg3}\\[0.3ex]

                \>\>    SISLSurf        \>      *{\fov surf};\\
                \>\>    double  \>      **{\fov coons};\\
                \>\>    int     \>      *{\fov numcoons1};\\
                \>\>    int     \>      *{\fov numcoons2};\\
                \>\>    int     \>      *{\fov dim}\\
                \>\>    int     \>      *{\fov stat};\\
\\
ARGUMENTS\\
        \>Input Arguments:\\
        \>\>    {\fov surf}     \> - \> \begin{minipg2}
                                Pointer to the surface that is to be converted
                                \end{minipg2}\\[0.8ex]
        \>Output Arguments:\\
        \>\>    {\fov coons}    \> - \> \begin{minipg2}
                                Array containing the (sequence of) Coons patches.
                                The total number of patches is
                                $numcoons1\times numcoons2$. The patches
                                are stored in sequence with $dim\times
                                16$ values for each patch. For each
                                corner of the patch we store in
                                sequence, positions, derivative in first
                                direction, derivative in second
                                direction, and twists.
                                \end{minipg2}\\[0.3ex]
        \>\>    {\fov numcoons1}\> - \> \begin{minipg2}
                                Number of Coons patches in first
                                parameter direction.
                                \end{minipg2}\\[0.3ex]
        \>\>    {\fov numcoons2}\> - \> \begin{minipg2}
                                Number of Coons patches in second
                                parameter direction.
                                \end{minipg2}\\[0.3ex]
        \>\>    {\fov dim}      \> - \> \begin{minipg2}
                                The dimension of the geometric space.
                                \end{minipg2}\\
        \>\>    {\fov stat}     \> - \> Status messages\\
                \>\>\>\>\>      $= 1$   :\>\begin{minipg5}
                                        Order too high, surface interpolated.
                                        \end{minipg5}\\
                \>\>\>\>\>      $= 0$   :\> Ok.\\
                \>\>\>\>\>      $< 0$   :\> Error.\\
\newpagetabs
EXAMPLE OF USE\\
                \>      \{ \\
                \>\>    SISLSurf        \>      *{\fov surf};\\
                \>\>    double  \>      *{\fov coons};\\
                \>\>    int     \>      {\fov numcoons1};\\
                \>\>    int     \>      {\fov numcoons2};\\
                \>\>    int     \>      {\fov dim}\\
                \>\>    int     \>      {\fov stat};\\
                \>\>    \ldots \\
        \>\>s1388(\begin{minipg4}
                {\fov surf}, \&{\fov coons}, \&{\fov numcoons1}, \&{\fov numcoons2}, \&{\fov dim}, \&{\fov stat});
                        \end{minipg4}\\
                \>\>    \ldots \\
                \>      \}
\end{tabbing}

\pgsbreak
\subsection{Convert a surface to a mesh of Bezier surfaces.}
\funclabel{s1731}
\begin{minipg1}
  To convert a surface to a mesh of Bezier surfaces.
  The Bezier surfaces are stored in a surface with all knots having
  multiplicity equal to the order of the surface in the
  corresponding parameter direction.
  If the input surface is rational, the generated Bezier surfaces will be
  rational too (i.e.\ there will be rational weights in the
  representation of the Bezier surfaces).
\end{minipg1} \\ \\
SYNOPSIS\\
        \>void s1731(\begin{minipg3}
                                {\fov surf}, {\fov newsurf}, {\fov stat})
                \end{minipg3}\\[0.3ex]

                \>\>    SISLSurf        \>      *{\fov surf};\\
                \>\>    SISLSurf        \>      **{\fov newsurf};\\
                \>\>    int     \>      *{\fov stat};\\
\\
ARGUMENTS\\
        \>Input Arguments:\\
        \>\>    {\fov surf}\> - \>      \begin{minipg2}
                                Surface to convert.
                                \end{minipg2}\\
\\
        \>Output Arguments:\\
        \>\>    {\fov newsurf}\> - \>   \begin{minipg2}
                                The new surface
                                storing the Bezier represented\\ surfaces.
                                \end{minipg2}\\[0.8ex]
        \>\>    {\fov sta}t     \> - \> Status messages\\
                \>\>\>\>\>              $> 0$   : warning\\
                \>\>\>\>\>              $= 0$   : ok\\
                \>\>\>\>\>              $< 0$   : error\\
\\
EXAMPLE OF USE\\
                \>      \{ \\
                \>\>    SISLSurf        \>      *{\fov surf};\\
                \>\>    SISLSurf        \>      *{\fov newsurf};\\
                \>\>    int     \>      {\fov sta}t;\\
                \>\>    \ldots \\
        \>\>s1731(\begin{minipg4}
                {\fov surf}, \&{\fov newsurf}, \&{\fov stat});
                        \end{minipg4}\\
                \>\>    \ldots \\
                \>      \}
\end{tabbing}

\pgsbreak
\subsection{Pick the next Bezier surface from a surface.}
\funclabel{s1733}
\begin{minipg1}
  To pick the next Bezier surface from a surface. This function
  requires a surface represented as the result of s1731(). See page
  \pageref{s1731}.
  This routine does not check that the surface is correct.
  If the input surface is rational, the generated Bezier surfaces will be
  rational too (i.e.\ there will be rational weights in the
  representation of the Bezier surfaces).
\end{minipg1} \\ \\
SYNOPSIS\\
        \>void s1733(\begin{minipg3}
                {\fov surf}, {\fov number1}, {\fov number2}, {\fov startpar1}, {\fov endpar1}, {\fov startpar2},
                 \linebreak  {\fov endpar2}, {\fov coef}, {\fov stat})
                \end{minipg3}\\[0.3ex]
                \>\>    SISLSurf        \>      *{\fov surf};\\
                \>\>    int     \>      {\fov number1};\\
                \>\>    int     \>      {\fov number2};\\
                \>\>    double  \>      *{\fov startpar1};\\
                \>\>    double  \>      *{\fov endpar1};\\
                \>\>    double  \>      *{\fov startpar2};\\
                \>\>    double  \>      *{\fov endpar2};\\
                \>\>    double  \>      {\fov coef}[\,];\\
                \>\>    int     \>      *{\fov stat};\\
\\
ARGUMENTS\\
        \>Input Arguments:\\
        \>\>    {\fov surf}\> - \>      \begin{minipg2}
                                The surface to convert.
                                \end{minipg2}\\
        \>\>    {\fov number1}\> - \>
        \begin{minipg2}
          The number of the Bezier patch to pick in the horizontal
          direction, where $0\leq number1<in1/ik1$ of the surface.
        \end{minipg2}\\[0.8ex]
        \>\>    {\fov number2}\> - \>
        \begin{minipg2}
          The number of the Bezier patch to pick in the vertical
          direction, , where $0\leq number2<in2/ik2$ of the surface.
        \end{minipg2}\\[0.8ex]
\\
        \>Output Arguments:\\
        \>\>    {\fov startpar1}        \> - \> \begin{minipg2}
                                The start parameter value of the Bezier
                                patch in the horizontal direction.
                                \end{minipg2}\\[0.8ex]
        \>\>    {\fov endpar1}  \> - \> \begin{minipg2}
                                The end parameter value of the Bezier
                                patch in the horizontal direction.
                                \end{minipg2}\\[0.8ex]
        \>\>    {\fov startpar2}        \> - \> \begin{minipg2}
                                The start parameter value of the Bezier patch                           in the vertical direction.
                                \end{minipg2}\\[0.8ex]
        \>\>    {\fov endpar2}  \> - \> \begin{minipg2}
                                The end parameter value of the Bezier patch                             in the vertical direction.
                                \end{minipg2}\\[0.8ex]
        \>\>    {\fov coef}     \> - \>
        \begin{minipg2}
          The vertices of the Bezier patch.
          Space must be allocated with a size of $(idim+1)\times ik1\times ik2$
          as given by the surface (this is done for reasons of efficiency).
        \end{minipg2}\\
\newpagetabs
        \>\>    {\fov stat}     \> - \> Status messages\\
                \>\>\>\>\>              $> 0$   : warning\\
                \>\>\>\>\>              $= 0$   : ok\\
                \>\>\>\>\>              $< 0$   : error
\\
EXAMPLE OF USE\\
                \>      \{ \\
                \>\>    SISLSurf        \>      *{\fov surf};\\
                \>\>    int     \>      {\fov number1};\\
                \>\>    int     \>      {\fov number2};\\
                \>\>    double  \>      {\fov startpar1};\\
                \>\>    double  \>      {\fov endpar1};\\
                \>\>    double  \>      {\fov startpar2};\\
                \>\>    double  \>      {\fov endpar2};\\
                \>\>    double  \>      {\fov coef}[48];\\
                \>\>    int     \>      stat;\\
                \>\>    \ldots \\
        \>\>s1733(\begin{minipg4}
                {\fov surf}, {\fov number1}, {\fov number2}, \&{\fov startpar1}, \&{\fov endpar1},
                        \&{\fov startpar2}, \&{\fov endpar2}, {\fov coef}, \&{\fov stat});
                        \end{minipg4}\\
                \>\>    \ldots \\
                \>      \}
\end{tabbing}

\pgsbreak
\subsection{Express a surface using a higher order basis.}
\funclabel{s1387}
\begin{minipg1}
  To express a surface as a surface of higher order.
\end{minipg1} \\ \\
SYNOPSIS\\
        \>void s1387(\begin{minipg3}
                                {\fov surf}, {\fov order1}, {\fov order2}, {\fov newsurf}, {\fov stat})
                \end{minipg3}\\[0.3ex]

                \>\>    SISLSurf        \>      *{\fov surf};\\
                \>\>    int     \>      {\fov order1};\\
                \>\>    int     \>      {\fov order2};\\
                \>\>    SISLSurf        \>      **{\fov newsurf};\\
                \>\>    int     \>      *{\fov stat};\\
\\
ARGUMENTS\\
        \>Input Arguments:\\
        \>\>    {\fov surf}\> - \>              \begin{minipg2}
                                Surface to raise the order of.
                                \end{minipg2}\\
        \>\>    {\fov order1}\> - \>            \begin{minipg2}
                                New order in the first parameter direction.
                                \end{minipg2}\\
        \>\>    {\fov order2}\> - \>            \begin{minipg2}
                                New order in the second parameter direction.
                                \end{minipg2}\\
        \>Output Arguments:\\
        \>\>    {\fov newsurf}\> - \>   \begin{minipg2}
                                The resulting order elevated surface.
                                \end{minipg2}\\
        \>\>    {\fov stat}     \> - \> Status messages\\
                \>\>\>\>\>      $= 1$   : \> \begin{minipg5}
                                                Input order equal to
                                                order of surface.
                                                Pointer set to input.
                                        \end{minipg5}\\[0.8ex]
                \>\>\>\>\>      $= 0$   : \>Ok.\\
                \>\>\>\>\>      $< 0$   : \>Error.\\
\\
EXAMPLE OF USE\\
                \>      \{ \\
                \>\>    SISLSurf        \>      *{\fov surf};\\
                \>\>    int     \>      {\fov order1};\\
                \>\>    int     \>      {\fov order2};\\
                \>\>    SISLSurf        \>      *{\fov newsurf};\\
                \>\>    int     \>      {\fov stat};\\
                \>\>    \ldots \\
        \>\>s1387(\begin{minipg4}
                                {\fov surf}, {\fov order1}, {\fov order2}, \&{\fov newsurf}, \&{\fov stat});
                        \end{minipg4}\\
                \>\>    \ldots \\
                \>      \}
\end{tabbing}

\pgsbreak
\subsection{Express the ``i,j''-th derivative of an open surface as a \mbox{surface}.}
\funclabel{s1386}
\begin{minipg1}
  To express the $(der1, der2)$-th derivative of an open surface as a surface.
\end{minipg1} \\ \\
SYNOPSIS\\
        \>void s1386(\begin{minipg3}
          {\fov surf}, {\fov der1}, {\fov der2}, {\fov newsurf}, {\fov stat})
        \end{minipg3}\\[0.3ex]
        \>\>    SISLSurf        \>      *{\fov surf};\\
        \>\>    int     \>      {\fov der1};\\
        \>\>    int     \>      {\fov der2};\\
        \>\>    SISLSurf        \>      **{\fov newsurf};\\
        \>\>    int     \>      *{\fov stat};\\
ARGUMENTS\\
        \>Input Arguments:\\
        \>\>    {\fov surf}     \> - \> \begin{minipg2}
                                Surface to differentiate.
                                \end{minipg2}\\
        \>\>    {\fov der1}     \> - \> \begin{minipg2}
                                The derivative to be produced in the first
                                parameter direction: $0\leq der1$
                                \end{minipg2}\\[0.8ex]
        \>\>    {\fov der2}     \> - \> \begin{minipg2}
                                The derivative to be produced in the second
                                parameter direction: $0\leq der2$
                                \end{minipg2}\\[0.8ex]
        \>Output Arguments:\\
        \>\>    {\fov newsurf}  \> - \> \begin{minipg2}
                                The result of the (der1, der2) differentiation of surf.
                                \end{minipg2}\\
        \>\>    {\fov stat}     \> - \> Status messages\\
                \>\>\>\>\>              $> 0$   : warning\\
                \>\>\>\>\>              $= 0$   : ok\\
                \>\>\>\>\>              $< 0$   : error\\[1.0ex]
\\
EXAMPLE OF USE\\
                \>      \{ \\
                \>\>    SISLSurf        \>      *{\fov surf};\\
                \>\>    int     \>      {\fov der1};\\
                \>\>    int     \>      {\fov der2};\\
                \>\>    SISLSurf        \>      *{\fov newsurf};\\
                \>\>    int     \>      {\fov stat};\\
                \>\>    \ldots \\
        \>\>s1386(\begin{minipg4}
                {\fov surf}, {\fov der1}, {\fov der2}, \&{\fov newsurf}, \&{\fov stat});
                        \end{minipg4}\\
                \>\>    \ldots \\
                \>      \}
\end{tabbing}

\pgsbreak
\subsection{Express the octants of a sphere as a surface.}
\funclabel{s1023}
\begin{minipg1}
  To express the octants of a sphere as a surface. This can also
  be used to describe the complete sphere.
  The sphere/the octants of the sphere will be geometrically exact.
\end{minipg1} \\ \\
SYNOPSIS\\
        \>void s1023(\begin{minipg3}
          {\fov centre},  {\fov axis},  {\fov equator},  {\fov latitude},  {\fov longitude},  {\fov sphere},  {\fov stat})
        \end{minipg3}\\[0.3ex]
        \>\>    double \> {\fov centre}[\,];\\
        \>\>    double \> {\fov axis}[\,];\\
        \>\>    double \> {\fov equator}[\,];\\
        \>\>    int    \> {\fov latitude};\\
        \>\>    int    \> {\fov longitude};\\
        \>\>    SISLSurf \> **{\fov sphere};\\
        \>\>    int    \> *{\fov stat};\\
\\
ARGUMENTS\\
        \>Input Arguments:\\
        \>\>    {\fov centre}\> - \> Centre point of the sphere.\\
        \>\>    {\fov axis}\> - \> Axis of the sphere (towards the north pole).\\
        \>\>    {\fov equator}\> - \> Vector from centre to start point
                                      on the equator.\\
        \>\>    {\fov latitude} \> - \>
        \begin{minipg2}
          Flag indicating number of
          octants in north/south direction:
        \end{minipg2}\\[0.3ex]
          \>\>\>\>\> $= 1$ : Octants in the northern hemisphere.\\
          \>\>\>\>\> $= 2$ : Octants in both hemispheres.\\
        \>\>    {\fov longitude} \> - \>
        \begin{minipg2}
          Flag indicating number of octants along the equator.
          This is counted counterclockwise from equator.
        \end{minipg2}\\[0.3ex]
          \>\>\>\>\> $= 1$ : Octants in 1.\ quadrant.\\
          \>\>\>\>\> $= 2$ : Octants in 1.\ and 2.\ quadrant.\\
          \>\>\>\>\> $= 3$ : Octants in 1., 2.\ and 3.\ quadrant.\\
          \>\>\>\>\> $= 4$ : Octants in all quadrants.\\
\\
        \>Output Arguments:\\
        \>\>    {\fov sphere} \> - \> The sphere produced.\\
        \>\>    {\fov stat}     \> - \> Status messages\\
                \>\>\>\>\>              $> 0$   : warning\\
                \>\>\>\>\>              $= 0$   : ok\\
                \>\>\>\>\>              $< 0$   : error
\newpagetabs
EXAMPLE OF USE\\
        \>      \{ \\
        \>\>    double \> {\fov centre}[3];\\
        \>\>    double \> {\fov axis}[3];\\
        \>\>    double \> {\fov equator}[3];\\
        \>\>    int    \> {\fov latitude};\\
        \>\>    int    \> {\fov longitude};\\
        \>\>    SISLSurf \> *{\fov sphere} = NULL;\\
        \>\>    int    \> {\fov stat} = 0;\\
        \>\>    \ldots \\
        \>\>s1023(\begin{minipg4}
          {\fov centre},  {\fov axis},  {\fov equator},  {\fov latitude},  {\fov longitude},  \&{\fov sphere},  \&{\fov stat});
        \end{minipg4}\\
        \>\>    \ldots \\
        \>      \}
\end{tabbing}

\pgsbreak
\subsection{Express a truncated cylinder as a surface.}
\funclabel{s1021}
\begin{minipg1}
  To express a truncated cylinder as a surface. The cylinder can be
  elliptic.
  The cylinder will be geometrically exact.
\end{minipg1}\\ \\
SYNOPSIS\\
        \>void s1021(\begin{minipg3}
        {\fov bottom\_pos}, {\fov bottom\_axis}, {\fov ellipse\_ratio}, {\fov axis\_dir},
        {\fov height}, {\fov cyl}, {\fov stat})
      \end{minipg3}\\[0.3ex]
      \>\>    double \> {\fov bottom\_pos}[\,];\\
      \>\>    double \> {\fov bottom\_axis}[\,];\\
      \>\>    double \> {\fov ellipse\_ratio};\\
      \>\>    double \> {\fov axis\_dir}[\,];\\
      \>\>    double \> {\fov height};\\
      \>\>    SISLSurf \> **{\fov cyl};\\
      \>\>    int    \> *{\fov stat};\\
\\
ARGUMENTS\\
        \>Input Arguments:\\
        \>\>    {\fov bottom\_pos}  \> - \> Center point of the bottom.\\
        \>\>    {\fov bottom\_axis} \> - \> One of the bottom axis
                                            (major or minor).\\
        \>\>    {\fov ellipse\_ratio} \> - \> Ratio between the other
                                              axis and bottom\_axis.\\
        \>\>    {\fov axis\_dir}    \> - \> Direction of the cylinder axis.\\
        \>\>    {\fov height}       \> - \> Height of the cone, can be negative.\\
\\
        \>Output Arguments:\\
        \>\>    {\fov cyl}  \> - \> Pointer to the cylinder produced.\\
        \>\>    {\fov stat}     \> - \> Status messages\\
                \>\>\>\>\>              $> 0$   : Warning.\\
                \>\>\>\>\>              $= 0$   : Ok.\\
                \>\>\>\>\>              $< 0$   : Error.\\
\\
EXAMPLE OF USE\\
        \>      \{ \\
        \>\>    double \> {\fov bottom\_pos}[3];\\
        \>\>    double \> {\fov bottom\_axis}[3];\\
        \>\>    double \> {\fov ellipse\_ratio};\\
        \>\>    double \> {\fov axis\_dir}[3];\\
        \>\>    double \> {\fov height};\\
        \>\>    SISLSurf \> *{\fov cyl} = NULL;\\
        \>\>    int    \> {\fov stat} = 0;\\
        \>\>    \ldots \\
        \>\>s1021(\begin{minipg4}
        {\fov bottom\_pos}, {\fov bottom\_axis}, {\fov ellipse\_ratio}, {\fov axis\_dir},
        {\fov height}, \&{\fov cyl}, \&{\fov stat})
      \end{minipg4}\\
      \>\>    \ldots \\
      \>      \}
\end{tabbing}

\pgsbreak
\subsection{Express the octants of a torus as a surface.}
\funclabel{s1024}
\begin{minipg1}
  To express the octants of a torus as a surface. This can also be used
  to describe the complete torus.
  The torus/the octants of the torus will be geometrically exact.
\end{minipg1}\\ \\
SYNOPSIS\\
        \>void s1024(\begin{minipg3}
          {\fov centre}, {\fov axis}, {\fov equator}, {\fov minor\_radius}, {\fov start\_minor}, {\fov end\_minor},
          {\fov numb\_major}, {\fov torus}, {\fov stat})
        \end{minipg3}\\[0.3ex]
        \>\>    double \> {\fov centre}[\,];\\
        \>\>    double \> {\fov axis}[\,];\\
        \>\>    double \> {\fov equator}[\,];\\
        \>\>    double \> {\fov minor\_radius};\\
        \>\>    int    \> {\fov start\_minor};\\
        \>\>    int    \> {\fov end\_minor};\\
        \>\>    int    \> {\fov numb\_major};\\
        \>\>    SISLSurf \> **{\fov torus};\\
        \>\>    int    \> *{\fov stat};\\
\\
ARGUMENTS\\
        \>Input Arguments:\\
        \>\>    {\fov centre} \> - \> Centre point of the torus.\\
        \>\>    {\fov axis}   \> - \> Normal to the torus plane.\\
        \>\>    {\fov equator}\> - \> \begin{minipg2}
                                        Vector from centre to start point
                                        on the major circle.
                                      \end{minipg2}\\[0.8ex]
        \>\>    {\fov minor\_radius} \> - \> Radius of the minor circle.\\
        \>\>    {\fov start\_minor}  \> - \> \begin{minipg2}
                                               Start quadrant on the
                                               minor circle (1,2,3 or 4).
                                               This is counted clockwise
                                               from the extremum in the
                                               direction of axis.
                                             \end{minipg2}\\[0.8ex]
        \>\>    {\fov end\_minor}    \> - \> \begin{minipg2}
                                               End quadrant on the minor
                                               circle (1,2,3 or 4). This
                                               is counted clockwise from
                                               the extremum in the
                                               direction of axis.
                                             \end{minipg2}\\[0.8ex]
        \>\>    {\fov numb\_major}   \> - \> \begin{minipg2}
                                               Number of quadrants on
                                               the major circle (1,2,3
                                               or 4). This is counted
                                               counterclockwise from
                                               equator.
                                             \end{minipg2}\\[0.8ex]
\\
        \>Output Arguments:\\
        \>\>    {\fov torus} \> - \> Pointer to the torus produced.\\
        \>\>    {\fov stat}     \> - \> Status messages\\
                \>\>\>\>\>              $> 0$   : Warning.\\
                \>\>\>\>\>              $= 0$   : Ok.\\
                \>\>\>\>\>              $< 0$   : Error.\\
\newpagetabs
EXAMPLE OF USE\\
        \>      \{ \\
        \>\>    double \> {\fov centre}[3];\\
        \>\>    double \> {\fov axis}[3];\\
        \>\>    double \> {\fov equator}[3];\\
        \>\>    double \> {\fov minor\_radius};\\
        \>\>    int    \> {\fov start\_minor};\\
        \>\>    int    \> {\fov end\_minor};\\
        \>\>    int    \> {\fov numb\_major};\\
        \>\>    SISLSurf \> *{\fov torus} = NULL;\\
        \>\>    int    \> {\fov stat} = 0;\\
        \>\>    \ldots \\
        \>\>s1024(\begin{minipg4}
          {\fov centre}, {\fov axis}, {\fov equator}, {\fov minor\_radius}, {\fov start\_minor}, {\fov end\_minor},
          {\fov numb\_major}, \&{\fov torus}, \&{\fov stat})
        \end{minipg4}\\
        \>\>    \ldots \\
        \>      \}
\end{tabbing}

\pgsbreak
\subsection{Express a truncated cone as a surface.}
\funclabel{s1022}
\begin{minipg1}
  To express a truncated cone as a surface. The cone can be elliptic.
  The cone will be geometrically exact.
\end{minipg1}\\ \\
SYNOPSIS\\
        \>void s1022(\begin{minipg3}
        {\fov bottom\_pos}, {\fov bottom\_axis}, {\fov ellipse\_ratio}, {\fov axis\_dir},
        {\fov cone\_angle}, {\fov height}, {\fov cone}, {\fov stat})
      \end{minipg3}\\[0.3ex]
      \>\>    double \> {\fov bottom\_pos}[\,];\\
      \>\>    double \> {\fov bottom\_axis}[\,];\\
      \>\>    double \> {\fov ellipse\_ratio};\\
      \>\>    double \> {\fov axis\_dir}[\,];\\
      \>\>    double \> {\fov cone\_angle};\\
      \>\>    double \> {\fov height};\\
      \>\>    SISLSurf \> **{\fov cone};\\
      \>\>    int    \> *{\fov stat};\\
\\
ARGUMENTS\\
        \>Input Arguments:\\
        \>\>    {\fov bottom\_pos}  \> - \> Center point of the bottom.\\
        \>\>    {\fov bottom\_axis} \> - \> One of the bottom axis (major or minor).\\
        \>\>    {\fov ellipse\_ratio} \> - \> Ratio between the other axis and bottom\_axis.\\
        \>\>    {\fov axis\_dir}    \> - \> Direction of the cone axis.\\
        \>\>    {\fov cone\_angle}  \> - \> \begin{minipg2}
                                              Angle between axis\_dir
                                              and the cone at the end of
                                              bottom\_axis, positive if
                                              the cone is sloping
                                              inwards.
                                            \end{minipg2}\\[0.8ex]
        \>\>    {\fov height}       \> - \> Height of the cone, can be negative.\\
\\
        \>Output Arguments:\\
        \>\>    {\fov cone} \> - \> Pointer to the cone produced.\\
        \>\>    {\fov stat}     \> - \> Status messages\\
                \>\>\>\>\>              $> 0$   : Warning.\\
                \>\>\>\>\>              $= 0$   : Ok.\\
                \>\>\>\>\>              $< 0$   : Error.\\
\newpagetabs
EXAMPLE OF USE\\
        \>      \{ \\
        \>\>    double \> {\fov bottom\_pos}[3];\\
        \>\>    double \> {\fov bottom\_axis}[3];\\
        \>\>    double \> {\fov ellipse\_ratio};\\
        \>\>    double \> {\fov axis\_dir}[3];\\
        \>\>    double \> {\fov cone\_angle};\\
        \>\>    double \> {\fov height};\\
        \>\>    SISLSurf \> *{\fov cone} = NULL;\\
        \>\>    int    \> {\fov stat} = 0;\\
        \>\>    \ldots \\
        \>\>s1022(\begin{minipg4}
          {\fov bottom\_pos}, {\fov bottom\_axis}, {\fov ellipse\_ratio}, {\fov axis\_dir},
          {\fov cone\_angle}, {\fov height}, \&{\fov cone}, \&{\fov stat})
        \end{minipg4}\\
        \>\>    \ldots \\
        \>      \}
\end{tabbing}

\vfill
\newpage
