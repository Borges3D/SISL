\section{Intersection Curves}
Intersection curves are tied to two objects where at
least one is a surface or a curve.
The representation of the intersection curves in the SISLIntcurve structure has two levels.
The first level is guide points which are
points in the parametric space and on the intersection
curve. In every case
there must be at least one guide point, but there is no
upper bound. This will be the result from the topology routines.
The second level is curves,
one curve in the geometric space and one curve in each
parameter plane if each surface is parametric. This will be the result from the marching routines.

\subsection{Intersection curve object.}

In the library an intersection curve is stored in a struct SISLIntcurve
containing the following:
\typelabel{SISLIntcurve}
 \> int 	\>{\fov ipoint};	\>\>Number of guide points defining the curve.\\
 \> double 	\>*{\fov epar1};   	\>\> \begin{minipg2}
				Pointer to the parameter values of the points in the first object. 
				\end{minipg2}\\[0.3ex]
 \> double	\>*{\fov epar2};	\>\> \begin{minipg2}
				Pointer to the parameter values of the points in the second object.
				\end{minipg2}\\[0.3ex]
 \> int		\>{\fov ipar1};		\>\> Number of parameter directions of first object.\\
 \> int		\>{\fov ipar2};		\>\>Number of parameter directions of second object.\\
 \> SISLCurve 	\>*{\fov pgeom}; 	\>\> \begin{minipg2}
				Pointer to the intersection curve in the
                           	geometry space. If the curve is not 
                           	computed, pgeom points to NULL.
				\end{minipg2}\\[0.3ex]
 \> SISLCurve 	\>*{\fov ppar1};  	\>\> \begin{minipg2}
				Pointer to the intersection curve in the
                           	parameter plane of the first object. If 
                           	the curve is not computed, ppar1 points 
                           	to NULL.
				\end{minipg2}\\[0.3ex]
 \> SISLCurve 	\>*{\fov ppar2};  	\>\> \begin{minipg2}
				Pointer to the intersection curve in the
                           	parameter plane of the second object. If 
                           	the curve is not computed, ppar2 points 
                           	to NULL.
				\end{minipg2}\\[0.3ex]
 \> int 	\>{\fov itype};	\>\>Type of curve:\\
	\>\>\>\>\>	$= 1$ :\> Straight line.\\
	\>\>\>\>\>	$= 2$ :\> Closed loop. No singularities.\\
	\>\>\>\>\>	$= 3$ :\> Closed loop. One singularity. Not used.\\
	\>\>\>\>\>	$= 4$ :\> Open curve. No singularity.\\
	\>\>\>\>\>	$= 5$ :\> \begin{minipg5}
				Open curve. Singularity at the 
                           	beginning of the curve.
				\end{minipg5}\\[0.8ex]
	\>\>\>\>\>	$= 6$ :\> \begin{minipg5}
				Open curve. Singularity at the end
                            	of the curve.
				\end{minipg5}\\[0.8ex]
	\>\>\>\>\>	$= 7$ :\> \begin{minipg5}
				Open curve. Singularity at the beginning
                            	and end of the curve.
				\end{minipg5}\\[0.8ex]
	\>\>\>\>\>	$= 8$ :\> An isolated singularity. Not used.\\
\end{tabbing}

Singularities are points on the intersection curve where, in an intersection between a curve and a surface, the tangent
of the curve lies in the tangent plane of the surface, or in an intersection between two surfaces, the tangent plane
of the surfaces coincide.
\pgsbreak
\subsection{Create a new intersection curve object.}
\funclabel{newIntcurve}
\begin{minipg1}
Create and initialize a SISLIntcurve-instance. Note that the arrays
{\fov guidepar1} and {\fov guidepar2} will be freed by freeIntcurve. In most cases the SISLIntcurve objects will be generated internally in the SISL intersection routines.
\end{minipg1} \\ \\
SYNOPSIS\\
        \>SISLIntcurve *newIntcurve(\begin{minipg3}
        {\fov numgdpt}, {\fov numpar1}, {\fov numpar2}, {\fov guidepar1},\\ {\fov guidepar2}, type)
                \end{minipg3}\\[0.3ex]
                \>\>    int    \>       {\fov numgdpt};\\
                \>\>    int    \>       {\fov numpar1};\\
                \>\>    int    \>       {\fov numpar2};\\
                \>\>    double \>       {\fov guidepar1}[\,];\\
                \>\>    double \>       {\fov guidepar2}[\,];\\
                \>\>    int    \>       {\fov type};\\
\\
ARGUMENTS\\
        \>Input Arguments:\\
        \>\>    {\fov numgdpt}  \> - \> \begin{minipg2}
                                Number of guide points that describe the curve.
                                \end{minipg2}\\
        \>\>    {\fov numpar1} \> - \> \begin{minipg2}
                                Number of parameter directions of first object
                                involved in the intersection.
                                \end{minipg2}\\[0.8ex]
        \>\>    {\fov numpar2}  \> - \> \begin{minipg2}
                                Number of parameter directions of second object
                                involved in the intersection.
                                \end{minipg2}\\[0.8ex]
        \>\>    {\fov guidepar1}\> - \> \begin{minipg2}
                                Parameter values of the guide points in the parameter
                                area of the first object.
                                NB! The epar1 pointer is set to point to this
                                array. The values are not copied.
                                \end{minipg2}\\[0.3ex]
        \>\>    {\fov guidepar2}\> - \> \begin{minipg2}
                                Parameter values of the guide points in the parameter
                                area of the second object.
                                NB! The epar2 pointer is set to point to this
                                array. The values are not copied.
                                \end{minipg2}\\[0.3ex].
        \>\>    {\fov type} \> - \> \begin{minipg2}
                                Kind of curve, see type SISLIntcurve on
                                page \pageref{SISLIntcurve}
                                \end{minipg2}\\
\\
        \>Output Arguments:\\
        \>\>    {\fov newIntcurve} \> \> \begin{minipg2}
                                 Pointer to new SISLIntcurve. If it is impossible
                                 to allocate space for the SISLIntcurve, newIntcurve
                                 returns NULL.
                                \end{minipg2}\\
\newpagetabs
EXAMPLE OF USE\\
                \>      \{ \\
                \>\>    SISLIntcurve    \>      *{\fov intcurve = NULL};\\
                \>\>    int    \>       {\fov numgdpt} = 2;\\
                \>\>    int    \>       {\fov numpar1} = 2;\\
                \>\>    int    \>       {\fov numpar2} = 2;\\
                \>\>    double \>       {\fov guidepar1}[4];\\
                \>\>    double \>       {\fov guidepar2}[4];\\
                \>\>    int    \>       {\fov type} = 4;\\
                \>\>    \ldots \\
        \>\>{\fov intcurve} = newIntcurve(\begin{minipg4}
                {\fov numgdpt}, {\fov numpar1}, {\fov numpar2}, {\fov guidepar1},\\ {\fov guidepar2}, type);
                        \end{minipg4}\\
                \>\>    \ldots \\
                \>      \} \\
\end{tabbing}

\pgsbreak
\subsection{Delete an intersection curve object.}
\funclabel{freeIntcurve}
\begin{minipg1}
  Free the space occupied by a SISLIntcurve.\\
  Note that the arrays {\fov guidepar1} and {\fov guidepar2} will be freed as well.
\end{minipg1} \\ \\
SYNOPSIS\\
        \>void freeIntcurve(\begin{minipg3}
        intcurve)
                \end{minipg3}\\[0.3ex]
                \>\>    SISLIntcurve \> *{\fov intcurve};\\
\\
ARGUMENTS\\
        \>Input Arguments:\\
        \>\>    {\fov intcurve} \> - \> Pointer to the SISLIntcurve to delete.\\
\\
EXAMPLE OF USE\\
                \>      \{ \\
                \>\>    SISLIntcurve    \>      *{\fov intcurve} = NULL;\\
                \>\>    int    \>       {\fov numgdpt} = 2;\\
                \>\>    int    \>       {\fov numpar1} = 2;\\
                \>\>    int    \>       {\fov numpar2} = 2;\\
                \>\>    double \>       {\fov guidepar1}[4];\\
                \>\>    double \>       {\fov guidepar2}[4];\\
                \>\>    int    \>       {\fov type} = 4;\\
                \>\>    \ldots \\
        \>\>{\fov intcurve} = newIntcurve(\begin{minipg4}
                {\fov numgdpt}, {\fov numpar1}, {\fov numpar2}, {\fov guidepar1},\\ {\fov guidepar2}, {\fov type});
                        \end{minipg4}\\
                \>\>    \ldots \\
        \>\>freeIntcurve(\begin{minipg4}
                {\fov intcurve});
                        \end{minipg4}\\
                \>\>    \ldots \\
                \>      \}
\end{tabbing}

\pgsbreak
\subsection{Free a list of intersection curves.}
\funclabel{freeIntcrvlist}
\begin{minipg1}
      Free a list of SISLIntcurve.
\end{minipg1} \\ \\
SYNOPSIS\\
        \> void freeIntcrvlist(\begin{minipg3}
            {\fov vilist}, {\fov icrv})
                \end{minipg3}\\
                \>\>    SISLIntcurve **{\fov vilist};\\
                \>\>    int    \>  {\fov icrv};\\
\\
ARGUMENTS\\
        \>Input Arguments:\\
        \>\>    {\fov vilist}\> - \>  \begin{minipg2}
                     Array of pointers to pointers to instance
                           of Intcurve.
                               \end{minipg2}\\
        \>\>    {\fov icrv}\> - \>  \begin{minipg2}
                     number of SISLIntcurves in the list.
                               \end{minipg2}\\
\\
        \>Output Arguments:\\
        \>\>    {\fov None}\> - \>  \begin{minipg2}
                None.
                               \end{minipg2}\\
\\
EXAMPLE OF USE\\
                \>      \{ \\

                \>\>    SISLIntcurve **{\fov vilist};\\
                \>\>    int    \>  {\fov icrv};\\                \>\>    \ldots \\
        \>\>freeIntcrvlist(\begin{minipg4}
            {\fov vilist}, {\fov icrv});
                \end{minipg4}\\
                \>\>    \ldots \\
                \>      \}
\end{tabbing}

