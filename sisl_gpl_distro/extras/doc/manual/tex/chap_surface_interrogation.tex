\chapter{Surface Interrogation}
\label{surfaceinterrogation}
This chapter describes the functions in the Surface Interrogation module.

\section{Intersection Curves}
Intersection curves are tied to two objects where at
least one is a surface or a curve.
The representation of the intersection curves in the SISLIntcurve structure has two levels.
The first level is guide points which are
points in the parametric space and on the intersection
curve. In every case
there must be at least one guide point, but there is no
upper bound. This will be the result from the topology routines.
The second level is curves,
one curve in the geometric space and one curve in each
parameter plane if each surface is parametric. This will be the result from the marching routines.

\subsection{Intersection curve object.}

In the library an intersection curve is stored in a struct SISLIntcurve
containing the following:
\typelabel{SISLIntcurve}
 \> int 	\>{\fov ipoint};	\>\>Number of guide points defining the curve.\\
 \> double 	\>*{\fov epar1};   	\>\> \begin{minipg2}
				Pointer to the parameter values of the points in the first object. 
				\end{minipg2}\\[0.3ex]
 \> double	\>*{\fov epar2};	\>\> \begin{minipg2}
				Pointer to the parameter values of the points in the second object.
				\end{minipg2}\\[0.3ex]
 \> int		\>{\fov ipar1};		\>\> Number of parameter directions of first object.\\
 \> int		\>{\fov ipar2};		\>\>Number of parameter directions of second object.\\
 \> SISLCurve 	\>*{\fov pgeom}; 	\>\> \begin{minipg2}
				Pointer to the intersection curve in the
                           	geometry space. If the curve is not 
                           	computed, pgeom points to NULL.
				\end{minipg2}\\[0.3ex]
 \> SISLCurve 	\>*{\fov ppar1};  	\>\> \begin{minipg2}
				Pointer to the intersection curve in the
                           	parameter plane of the first object. If 
                           	the curve is not computed, ppar1 points 
                           	to NULL.
				\end{minipg2}\\[0.3ex]
 \> SISLCurve 	\>*{\fov ppar2};  	\>\> \begin{minipg2}
				Pointer to the intersection curve in the
                           	parameter plane of the second object. If 
                           	the curve is not computed, ppar2 points 
                           	to NULL.
				\end{minipg2}\\[0.3ex]
 \> int 	\>{\fov itype};	\>\>Type of curve:\\
	\>\>\>\>\>	$= 1$ :\> Straight line.\\
	\>\>\>\>\>	$= 2$ :\> Closed loop. No singularities.\\
	\>\>\>\>\>	$= 3$ :\> Closed loop. One singularity. Not used.\\
	\>\>\>\>\>	$= 4$ :\> Open curve. No singularity.\\
	\>\>\>\>\>	$= 5$ :\> \begin{minipg5}
				Open curve. Singularity at the 
                           	beginning of the curve.
				\end{minipg5}\\[0.8ex]
	\>\>\>\>\>	$= 6$ :\> \begin{minipg5}
				Open curve. Singularity at the end
                            	of the curve.
				\end{minipg5}\\[0.8ex]
	\>\>\>\>\>	$= 7$ :\> \begin{minipg5}
				Open curve. Singularity at the beginning
                            	and end of the curve.
				\end{minipg5}\\[0.8ex]
	\>\>\>\>\>	$= 8$ :\> An isolated singularity. Not used.\\
\end{tabbing}

Singularities are points on the intersection curve where, in an intersection between a curve and a surface, the tangent
of the curve lies in the tangent plane of the surface, or in an intersection between two surfaces, the tangent plane
of the surfaces coincide.
\pgsbreak
\subsection{Create a new intersection curve object.}
\funclabel{newIntcurve}
\begin{minipg1}
Create and initialize a SISLIntcurve-instance. Note that the arrays
{\fov guidepar1} and {\fov guidepar2} will be freed by freeIntcurve. In most cases the SISLIntcurve objects will be generated internally in the SISL intersection routines.
\end{minipg1} \\ \\
SYNOPSIS\\
        \>SISLIntcurve *newIntcurve(\begin{minipg3}
        {\fov numgdpt}, {\fov numpar1}, {\fov numpar2}, {\fov guidepar1},\\ {\fov guidepar2}, type)
                \end{minipg3}\\[0.3ex]
                \>\>    int    \>       {\fov numgdpt};\\
                \>\>    int    \>       {\fov numpar1};\\
                \>\>    int    \>       {\fov numpar2};\\
                \>\>    double \>       {\fov guidepar1}[\,];\\
                \>\>    double \>       {\fov guidepar2}[\,];\\
                \>\>    int    \>       {\fov type};\\
\\
ARGUMENTS\\
        \>Input Arguments:\\
        \>\>    {\fov numgdpt}  \> - \> \begin{minipg2}
                                Number of guide points that describe the curve.
                                \end{minipg2}\\
        \>\>    {\fov numpar1} \> - \> \begin{minipg2}
                                Number of parameter directions of first object
                                involved in the intersection.
                                \end{minipg2}\\[0.8ex]
        \>\>    {\fov numpar2}  \> - \> \begin{minipg2}
                                Number of parameter directions of second object
                                involved in the intersection.
                                \end{minipg2}\\[0.8ex]
        \>\>    {\fov guidepar1}\> - \> \begin{minipg2}
                                Parameter values of the guide points in the parameter
                                area of the first object.
                                NB! The epar1 pointer is set to point to this
                                array. The values are not copied.
                                \end{minipg2}\\[0.3ex]
        \>\>    {\fov guidepar2}\> - \> \begin{minipg2}
                                Parameter values of the guide points in the parameter
                                area of the second object.
                                NB! The epar2 pointer is set to point to this
                                array. The values are not copied.
                                \end{minipg2}\\[0.3ex].
        \>\>    {\fov type} \> - \> \begin{minipg2}
                                Kind of curve, see type SISLIntcurve on
                                page \pageref{SISLIntcurve}
                                \end{minipg2}\\
\\
        \>Output Arguments:\\
        \>\>    {\fov newIntcurve} \> \> \begin{minipg2}
                                 Pointer to new SISLIntcurve. If it is impossible
                                 to allocate space for the SISLIntcurve, newIntcurve
                                 returns NULL.
                                \end{minipg2}\\
\newpagetabs
EXAMPLE OF USE\\
                \>      \{ \\
                \>\>    SISLIntcurve    \>      *{\fov intcurve = NULL};\\
                \>\>    int    \>       {\fov numgdpt} = 2;\\
                \>\>    int    \>       {\fov numpar1} = 2;\\
                \>\>    int    \>       {\fov numpar2} = 2;\\
                \>\>    double \>       {\fov guidepar1}[4];\\
                \>\>    double \>       {\fov guidepar2}[4];\\
                \>\>    int    \>       {\fov type} = 4;\\
                \>\>    \ldots \\
        \>\>{\fov intcurve} = newIntcurve(\begin{minipg4}
                {\fov numgdpt}, {\fov numpar1}, {\fov numpar2}, {\fov guidepar1},\\ {\fov guidepar2}, type);
                        \end{minipg4}\\
                \>\>    \ldots \\
                \>      \} \\
\end{tabbing}

\pgsbreak
\subsection{Delete an intersection curve object.}
\funclabel{freeIntcurve}
\begin{minipg1}
  Free the space occupied by a SISLIntcurve.\\
  Note that the arrays {\fov guidepar1} and {\fov guidepar2} will be freed as well.
\end{minipg1} \\ \\
SYNOPSIS\\
        \>void freeIntcurve(\begin{minipg3}
        intcurve)
                \end{minipg3}\\[0.3ex]
                \>\>    SISLIntcurve \> *{\fov intcurve};\\
\\
ARGUMENTS\\
        \>Input Arguments:\\
        \>\>    {\fov intcurve} \> - \> Pointer to the SISLIntcurve to delete.\\
\\
EXAMPLE OF USE\\
                \>      \{ \\
                \>\>    SISLIntcurve    \>      *{\fov intcurve} = NULL;\\
                \>\>    int    \>       {\fov numgdpt} = 2;\\
                \>\>    int    \>       {\fov numpar1} = 2;\\
                \>\>    int    \>       {\fov numpar2} = 2;\\
                \>\>    double \>       {\fov guidepar1}[4];\\
                \>\>    double \>       {\fov guidepar2}[4];\\
                \>\>    int    \>       {\fov type} = 4;\\
                \>\>    \ldots \\
        \>\>{\fov intcurve} = newIntcurve(\begin{minipg4}
                {\fov numgdpt}, {\fov numpar1}, {\fov numpar2}, {\fov guidepar1},\\ {\fov guidepar2}, {\fov type});
                        \end{minipg4}\\
                \>\>    \ldots \\
        \>\>freeIntcurve(\begin{minipg4}
                {\fov intcurve});
                        \end{minipg4}\\
                \>\>    \ldots \\
                \>      \}
\end{tabbing}

\pgsbreak
\subsection{Free a list of intersection curves.}
\funclabel{freeIntcrvlist}
\begin{minipg1}
      Free a list of SISLIntcurve.
\end{minipg1} \\ \\
SYNOPSIS\\
        \> void freeIntcrvlist(\begin{minipg3}
            {\fov vilist}, {\fov icrv})
                \end{minipg3}\\
                \>\>    SISLIntcurve **{\fov vilist};\\
                \>\>    int    \>  {\fov icrv};\\
\\
ARGUMENTS\\
        \>Input Arguments:\\
        \>\>    {\fov vilist}\> - \>  \begin{minipg2}
                     Array of pointers to pointers to instance
                           of Intcurve.
                               \end{minipg2}\\
        \>\>    {\fov icrv}\> - \>  \begin{minipg2}
                     number of SISLIntcurves in the list.
                               \end{minipg2}\\
\\
        \>Output Arguments:\\
        \>\>    {\fov None}\> - \>  \begin{minipg2}
                None.
                               \end{minipg2}\\
\\
EXAMPLE OF USE\\
                \>      \{ \\

                \>\>    SISLIntcurve **{\fov vilist};\\
                \>\>    int    \>  {\fov icrv};\\                \>\>    \ldots \\
        \>\>freeIntcrvlist(\begin{minipg4}
            {\fov vilist}, {\fov icrv});
                \end{minipg4}\\
                \>\>    \ldots \\
                \>      \}
\end{tabbing}


\pgsbreak

\section{Find the Intersections}
\subsection{\sloppy Intersection between a curve and a straight line or a plane.}
\funclabel{s1850}
\begin{minipg1}
        Find all the intersections between a curve and a plane (if
        curve dimension and $dim=3$)
        or a curve and a line (if curve dimension and $dim=2$).
\end{minipg1} \\ \\
SYNOPSIS\\
        \>void s1850(\begin{minipg3}
        {\fov curve}, {\fov point}, {\fov normal}, {\fov dim}, {\fov epsco}, {\fov epsge}, {\fov numintpt},
        {\fov intpar},\linebreak {\fov numintcu}, {\fov intcurve}, {\fov stat})
                \end{minipg3}\\[0.3ex]
                \>\>    SISLCurve       \>      *{\fov curve};\\
                \>\>    double  \>      {\fov point}[\,];\\
                \>\>    double  \>      {\fov normal}[\,];\\
                \>\>    int     \>      {\fov dim};\\
                \>\>    double  \>      {\fov epsco};\\
                \>\>    double  \>      {\fov epsge};\\
                \>\>    int     \>      *{\fov numintpt};\\
                \>\>    double  \>      **{\fov intpar};\\
                \>\>    int     \>      *{\fov numintcu};\\
                \>\>    SISLIntcurve \> ***{\fov intcurve};\\
                \>\>    int     \>      *{\fov stat};\\
\\
ARGUMENTS\\
\\
        \>Input Arguments:\\
        \>\>    {\fov curve}    \> - \> Pointer to the curve.\\
        \>\>    {\fov point}    \> - \> Point in the plane/line.\\
        \>\>    {\fov normal}   \> - \>
        \begin{minipg2}
          Normal to the plane or any normal to the direction of the
          line.
        \end{minipg2}\\[0.8ex]
        \>\>    {\fov dim}      \> - \> \begin{minipg2}
                                Dimension of the space in which the
                                curve and the plane/line lies, {\fov
                                  dim} must be equal to two or three.
                                \end{minipg2}\\[0.8ex]
        \>\>    {\fov epsco}    \> - \> Computational resolution (not used).\\
        \>\>    {\fov epsge}    \> - \> Geometry resolution.\\
\\
        \>Output Arguments:\\
        \>\>    {\fov numintpt}\> - \>  Number of single intersection points.\\
        \>\>    {\fov intpar}   \> - \> \begin{minipg2}
                        Array containing the parameter values of the
                        single intersection points in the parameter
                        interval of the curve. The points lie in sequence.
                        Intersection curves are stored in intcurve.
                                \end{minipg2}\\[0.8ex]
        \>\>    {\fov numintcu}\> - \>Number of intersection curves.\\
\newpagetabs
        \>\>    {\fov intcurve}\> - \>  \begin{minipg2}
                        Array of pointers to SISLIntcurve objects
                        containing description of the intersection
                        curves. The curves are only described by start
                        points and end points in
                        the parameter interval of the curve. The
                        curve pointers point
                        to nothing.
                                \end{minipg2}\\[0.8ex]
        \>\>    {\fov stat}     \> - \> Status messages\\
                \>\>\>\>\>              $> 0$   : warning\\
                \>\>\>\>\>              $= 0$   : ok\\
                \>\>\>\>\>              $< 0$   : error\\
\\
EXAMPLE OF USE\\
                \>      \{ \\
                \>\>    SISLCurve       \>      *{\fov curve};\\
                \>\>    double  \>      {\fov point}[3];\\
                \>\>    double  \>      {\fov normal}[3];\\
                \>\>    int     \>      {\fov dim} = 3;\\
                \>\>    double  \>      {\fov epsco};\\
                \>\>    double  \>      {\fov epsge};\\
                \>\>    int     \>      {\fov numintpt};\\
                \>\>    double  \>      *{\fov intpar};\\
                \>\>    int     \>      {\fov numintcu};\\
                \>\>    SISLIntcurve \> **{\fov intcurve};\\
                \>\>    int     \>      {\fov stat};\\
                \>\>    \ldots \\
        \>\>s1850(\begin{minipg4}
                {\fov curve}, {\fov point}, {\fov normal}, {\fov dim}, {\fov epsco}, {\fov epsge}, \&{\fov numintpt},
                \&{\fov intpar}, \&{\fov numintcu}, \&{\fov intcurve}, \&{\fov stat});
                        \end{minipg4}\\
                \>\>    \ldots \\
                \>      \}
\end{tabbing}

\pgsbreak
\subsection{Intersection between a curve and a 2D circle or a sphere.}
\funclabel{s1371}
\begin{minipg1}
  Find all the intersections between a curve and a sphere
  (if curve dimension and $dim=3$), or a curve and a circle
  (if curve dimension and $dim=2$).
\end{minipg1} \\ \\
SYNOPSIS\\
        \>void s1371(\begin{minipg3}
        {\fov curve}, {\fov centre}, {\fov radius}, {\fov dim}, {\fov epsco}, {\fov epsge}, {\fov numintpt}, {\fov intpar},\\
                        {\fov numintcu}, {\fov intcurve}, {\fov stat})
                \end{minipg3}\\[0.3ex]
                \>\>    SISLCurve       \>      *{\fov curve};\\
                \>\>    double  \>      {\fov centre}[\,];\\
                \>\>    double  \>      {\fov radius};\\
                \>\>    int     \>      {\fov dim};\\
                \>\>    double  \>      {\fov epsco};\\
                \>\>    double  \>      {\fov epsge};\\
                \>\>    int     \>      *{\fov numintpt};\\
                \>\>    double  \>      **{\fov intpar};\\
                \>\>    int     \>      *{\fov numintcu};\\
                \>\>    SISLIntcurve \> ***{\fov intcurve};\\
                \>\>    int     \>      *{\fov stat};\\
\\
ARGUMENTS\\
        \>Input Arguments:\\
        \>\>    {\fov curve}    \> - \> Pointer to the curve.\\
        \>\>    {\fov centre}   \> - \> Centre of the circle/sphere.\\
        \>\>    {\fov radius}   \> - \> Radius of circle or sphere.\\
        \>\>    {\fov dim}      \> - \> \begin{minipg2}
                                Dimension of the space in which the
                                curve and the circle/sphere lies, {\fov dim}
                                should be equal to two or three.
                                \end{minipg2}\\[0.3ex]
        \>\>    {\fov epsco}    \> - \> Computational resolution (not used).\\
        \>\>    {\fov epsge}    \> - \> Geometry resolution.\\
\\
        \>Output Arguments:\\
        \>\>    {\fov numintpt}\> - \>  Number of single intersection points.\\
        \>\>    {\fov intpar}   \> - \> \begin{minipg2}
                        Array containing the parameter values of the
                        single intersection points in the parameter
                        interval of the curve. The points lie in sequence.
                        Intersection curves are stored in intcurve.
                                \end{minipg2}\\[0.8ex]
        \>\>    {\fov numintcu}\> - \>Number of intersection curves.\\
        \>\>    {\fov intcurve}\> - \>  \begin{minipg2}
                        Array of pointers to SISLIntcurve objects
                        containing descriptions of the intersection
                        curves. The curves are only described by start
                        points and end points in
                        the parameter interval of the curve. The curve
                        pointers point to nothing.
                                \end{minipg2}\\[0.8ex]
\newpagetabs
        \>\>    {\fov stat}     \> - \> Status messages\\
                \>\>\>\>\>              $> 0$   : warning\\
                \>\>\>\>\>              $= 0$   : ok\\
                \>\>\>\>\>              $< 0$   : error\\
\\
EXAMPLE OF USE\\
                \>      \{ \\
                \>\>    SISLCurve       \>      *{\fov curve};\\
                \>\>    double  \>      {\fov centre}[3];\\
                \>\>    double  \>      {\fov radius};\\
                \>\>    int     \>      {\fov dim} = 3;\\
                \>\>    double  \>      {\fov epsco};\\
                \>\>    double  \>      {\fov epsge};\\
                \>\>    int     \>      {\fov numintpt};\\
                \>\>    double  \>      *{\fov intpar};\\
                \>\>    int     \>      {\fov numintcu};\\
                \>\>    SISLIntcurve \> **{\fov intcurve};\\
                \>\>    int     \>      {\fov stat};\\
                \>\>    \ldots \\
        \>\>s1371(\begin{minipg4}
        {\fov curve}, {\fov centre}, {\fov radius}, {\fov dim}, {\fov epsco}, {\fov epsge}, \&{\fov numintpt},
                \&{\fov intpar}, \&{\fov numintcu}, \&{\fov intcurve}, \&{\fov stat});
                        \end{minipg4}\\
                \>\>    \ldots \\
                \>      \}
\end{tabbing}

\pgsbreak
\subsection{Intersection between a curve and a cylinder.}
\funclabel{s1372}
\begin{minipg1}
  Find all the intersections between a curve and a cylinder.

\end{minipg1} \\ \\
SYNOPSIS\\
        \>void s1372(\begin{minipg3}
        {\fov curve}, {\fov point}, {\fov dir}, {\fov radius}, {\fov dim}, {\fov epsco}, {\fov epsge}, {\fov numintpt}, {\fov intpar},
                        {\fov numintcu}, {\fov intcurve}, {\fov stat})
                \end{minipg3}\\[0.3ex]
                \>\>    SISLCurve       \>      *{\fov curve};\\
                \>\>    double  \>      {\fov point}[\,];\\
                \>\>    double  \>      {\fov dir}[\,];\\
                \>\>    double  \>      {\fov radius};\\
                \>\>    int     \>      {\fov dim};\\
                \>\>    double  \>      {\fov epsco};\\
                \>\>    double  \>      {\fov epsge};\\
                \>\>    int     \>      *{\fov numintpt};\\
                \>\>    double  \>      **{\fov intpar};\\
                \>\>    int     \>      *{\fov numintcu};\\
                \>\>    SISLIntcurve \> ***{\fov intcurve};\\
                \>\>    int     \>      *{\fov stat};\\
\\
ARGUMENTS\\
        \>Input Arguments:\\
        \>\>    {\fov curve}    \> - \> Pointer to the curve.\\
        \>\>    {\fov point}    \> - \> Point on the cylinder axis.\\
        \>\>    {\fov dir}      \> - \> Direction of the cylinder axis.\\
        \>\>    {\fov radius}   \> - \> Radius of the cylinder.\\
        \>\>    {\fov dim}      \> - \> \begin{minipg2}
                                Dimension of the space in which the
                                cylinder and the curve
                                lie, dim should be equal to three.
                                \end{minipg2}\\[0.3ex]
        \>\>    {\fov epsco}    \> - \> Computational resolution (not used).\\
        \>\>    {\fov epsge}    \> - \> Geometry resolution.\\
\\
        \>Output Arguments:\\
        \>\>    {\fov numintpt}\> - \>  Number of single intersection points.\\
        \>\>    {\fov intpar}   \> - \> \begin{minipg2}
                        Array containing the parameter values of the
                        single intersection points in the parameter
                        interval of the curve. The points lie in sequence.
                        Intersection curves are stored in intcurve.
                                \end{minipg2}\\[0.8ex]
        \>\>    {\fov numintcu}\> - \>Number of intersection curves.\\
        \>\>    {\fov intcurve}\> - \>  \begin{minipg2}
                        Array of pointers to the SISLIntcurve objects
                        containing descriptions of the intersection
                        curves. The curves are only described by start
                        points and end points in
                        the parameter interval of the curve. The curve
                        pointers point to nothing.
                                \end{minipg2}\\[0.8ex]
\newpagetabs
        \>\>    {\fov stat}     \> - \> Status messages\\
                \>\>\>\>\>              $> 0$   : warning\\
                \>\>\>\>\>              $= 0$   : ok\\
                \>\>\>\>\>              $< 0$   : error\\
\\
EXAMPLE OF USE\\
                \>      \{ \\
                \>\>    SISLCurve       \>      *{\fov curve};\\
                \>\>    double  \>      {\fov point}[3];\\
                \>\>    double  \>      {\fov dir}[3];\\
                \>\>    double  \>      {\fov radius};\\
                \>\>    int     \>      {\fov dim} = 3;\\
                \>\>    double  \>      {\fov epsco};\\
                \>\>    double  \>      {\fov epsge};\\
                \>\>    int     \>      {\fov numintpt};\\
                \>\>    double  \>      *{\fov intpar};\\
                \>\>    int     \>      {\fov numintcu};\\
                \>\>    SISLIntcurve \> **{\fov intcurve};\\
                \>\>    int     \>      {\fov stat};\\
                \>\>    \ldots \\
        \>\>s1372(\begin{minipg4}
        {\fov curve}, {\fov point}, {\fov dir}, {\fov radius}, {\fov dim}, {\fov epsco}, {\fov epsge}, \&{\fov numintpt},\\
                \&{\fov intpar}, \&{\fov numintcu}, \&{\fov intcurve}, \&{\fov stat});
                        \end{minipg4}\\
                \>\>    \ldots \\
                \>      \}
\end{tabbing}

\pgsbreak
\subsection{Intersection between a curve and a cone.}
\funclabel{s1373}
\begin{minipg1}
  Find all the intersections between a curve and a cone.
\end{minipg1} \\ \\
SYNOPSIS\\
        \>void s1373(\begin{minipg3}
         {\fov curve}, {\fov top}, {\fov dir}, {\fov conept}, {\fov dim}, {\fov epsco}, {\fov epsge}, {\fov numintpt}, {\fov intpar},
                        {\fov numintcu}, {\fov intcurve}, {\fov stat})
                \end{minipg3}\\[0.3ex]
                \>\>    SISLCurve       \>      *{\fov curve};\\
                \>\>    double  \>      {\fov top}[\,];\\
                \>\>    double  \>      {\fov axispt}[\,];\\
                \>\>    double  \>      {\fov conept}[\,];\\
                \>\>    int     \>      {\fov dim};\\
                \>\>    double  \>      {\fov epsco};\\
                \>\>    double  \>      {\fov epsge};\\
                \>\>    int     \>      *{\fov numintpt};\\
                \>\>    double  \>      **{\fov intpar};\\
                \>\>    int     \>      *{\fov numintcu};\\
                \>\>    SISLIntcurve \> ***{\fov intcurve};\\
                \>\>    int     \>      *{\fov stat};\\
\\
ARGUMENTS\\
        \>Input Arguments:\\
        \>\>    {\fov curve}    \> - \> Pointer to the curve.\\
        \>\>    {\fov top}      \> - \> Top point of the cone.\\
        \>\>    {\fov axispt}   \> - \> Point on the cone axis.\\
        \>\>    {\fov conept}   \> - \> Point on the cone surface, other than the top point.\\
        \>\>    {\fov dim}      \> - \> \begin{minipg2}
                                Dimension of the space in which the
                                cone and the curve
                                lie, dim should be equal to three.
                                \end{minipg2}\\[0.3ex]
        \>\>    {\fov epsco}    \> - \> Computational resolution (not used).\\
        \>\>    {\fov epsge}    \> - \> Geometry resolution.\\
\\
        \>Output Arguments:\\
        \>\>    {\fov numintpt}\> - \>  Number of single intersection points.\\
        \>\>    {\fov intpar}   \> - \> \begin{minipg2}
                        Array containing the parameter values of the
                        single intersection points in the parameter
                        interval of the curve. The points lie in sequence.
                        Intersection curves are stored in intcurve.
                                \end{minipg2}\\[0.8ex]
        \>\>    {\fov numintcu}\> - \>Number of intersection curves.\\
        \>\>    {\fov intcurve}\> - \>  \begin{minipg2}
                        Array of pointers to the SISLIntcurve object
                        containing descriptions of the intersection
                        curves. The curves are only described by start
                        points and end points in the parameter interval
                        of the curve. The curve pointers point
                        to nothing.
                                \end{minipg2}\\[0.8ex]
\newpagetabs
        \>\>    {\fov stat}     \> - \> Status messages\\
                \>\>\>\>\>              $> 0$   : warning\\
                \>\>\>\>\>              $= 0$   : ok\\
                \>\>\>\>\>              $< 0$   : error\\
\\
EXAMPLE OF USE\\
                \>      \{ \\
                \>\>    SISLCurve       \>      *{\fov curve};\\
                \>\>    double  \>      {\fov top}[3];\\
                \>\>    double  \>      {\fov dir}[3];\\
                \>\>    double  \>      {\fov conept}[3];\\
                \>\>    int     \>      {\fov dim} = 3;\\
                \>\>    double  \>      {\fov epsco};\\
                \>\>    double  \>      {\fov epsge};\\
                \>\>    int     \>      {\fov numintpt};\\
                \>\>    double  \>      *{\fov intpar};\\
                \>\>    int     \>      {\fov numintcu};\\
                \>\>    SISLIntcurve \> **{\fov intcurve};\\
                \>\>    int     \>      {\fov stat};\\
                \>\>    \ldots \\
        \>\>s1373(\begin{minipg4}
        {\fov curve}, {\fov top}, {\fov dir}, {\fov conept}, {\fov dim}, {\fov epsco}, {\fov epsge}, \&{\fov numintpt},
                \&{\fov intpar}, \&{\fov numintcu}, \&{\fov intcurve}, \&{\fov stat});
                        \end{minipg4}\\
                \>\>    \ldots \\
                \>      \}
\end{tabbing}

\pgsbreak
\subsection{Intersection between a curve and an elliptic cone.}
\funclabel{s1502}
\begin{minipg1}
  Find all the intersections between a curve and an elliptic cone.
\end{minipg1} \\ \\
SYNOPSIS\\
        \>void s1502(\begin{minipg3}
         {\fov curve}, {\fov basept}, {\fov normdir}, {\fov ellipaxis}, {\fov alpha}, {\fov ratio}, {\fov dim}, {\fov epsco}, {\fov epsge}, {\fov numintpt}, {\fov intpar},
                        {\fov numintcu}, {\fov intcurve}, {\fov stat})
                \end{minipg3}\\[0.3ex]
                \>\>    SISLCurve       \>      *{\fov curve};\\
                \>\>    double  \>      {\fov basept}[\,];\\
                \>\>    double  \>      {\fov normdir}[\,];\\
                \>\>    double  \>      {\fov ellipaxis}[\,];\\
                \>\>    double  \>      {\fov alpha};\\
                \>\>    double  \>      {\fov ratio};\\
                \>\>    int     \>      {\fov dim};\\
                \>\>    double  \>      {\fov epsco};\\
                \>\>    double  \>      {\fov epsge};\\
                \>\>    int     \>      *{\fov numintpt};\\
                \>\>    double  \>      **{\fov intpar};\\
                \>\>    int     \>      *{\fov numintcu};\\
                \>\>    SISLIntcurve \> ***{\fov intcurve};\\
                \>\>    int     \>      *{\fov stat};\\
\\
ARGUMENTS\\
        \>Input Arguments:\\
        \>\>    {\fov curve}    \> - \> Pointer to the curve.\\
        \>\>    {\fov basept}\> - \>    \begin{minipg2}
                                Base point of the cone, centre of elliptic base.
                                \end{minipg2}\\
        \>\>    {\fov normdir}\> - \>   \begin{minipg2}
                                Direction of the cone axis, normal to the elliptic base. The default is pointing from the base point to the top point of the cone.
                                \end{minipg2}\\
        \>\>    {\fov ellipaxis}\> - \> \begin{minipg2}
                                One of the axes of the ellipse (major or minor).
                                \end{minipg2}\\
        \>\>    {\fov alpha}\> - \>     \begin{minipg2}
                                The opening angle of the cone at the ellipaxis.
                                \end{minipg2}\\
        \>\>    {\fov ratio}\> - \>     \begin{minipg2}
                                The ratio of the major and minor
                                axes = ellipaxis/otheraxis.
                                \end{minipg2}\\[0.3ex]
        \>\>    {\fov dim}      \> - \> \begin{minipg2}
                                Dimension of the space in which the
                                cone and the curve
                                lie, dim should be equal to three.
                                \end{minipg2}\\[0.3ex]
        \>\>    {\fov epsco}    \> - \> Computational resolution (not used).\\
        \>\>    {\fov epsge}    \> - \> Geometry resolution.\\
\newpagetabs
        \>Output Arguments:\\
        \>\>    {\fov numintpt}\> - \>  Number of single intersection points.\\
        \>\>    {\fov intpar}   \> - \> \begin{minipg2}
                        Array containing the parameter values of the
                        single intersection points in the parameter
                        interval of the curve. The points lie in sequence.
                        Intersection curves are stored in intcurve.
                                \end{minipg2}\\[0.8ex]
        \>\>    {\fov numintcu}\> - \>Number of intersection curves.\\
        \>\>    {\fov intcurve}\> - \>  \begin{minipg2}
                        Array of pointers to the SISLIntcurve object
                        containing descriptions of the intersection
                        curves. The curves are only described by start
                        points and end points in
                        the parameter interval of the curve. The curve
                        pointers point to nothing.
                                \end{minipg2}\\[0.8ex]
        \>\>    {\fov stat}     \> - \> Status messages\\
                \>\>\>\>\>              $> 0$   : warning\\
                \>\>\>\>\>              $= 0$   : ok\\
                \>\>\>\>\>              $< 0$   : error\\
\\
EXAMPLE OF USE\\
                \>      \{ \\
                \>\>    SISLCurve       \>      *{\fov curve};\\
                \>\>    double  \>      {\fov basept}[3];\\
                \>\>    double  \>      {\fov normdir}[3];\\
                \>\>    double  \>      {\fov ellipaxis}[3];\\
                \>\>    double  \>      {\fov alpha};\\
                \>\>    double  \>      {\fov ratio};\\
                \>\>    int     \>      {\fov dim} = 3;\\
                \>\>    double  \>      {\fov epsco};\\
                \>\>    double  \>      {\fov epsge};\\
                \>\>    int     \>      {\fov numintpt};\\
                \>\>    double  \>      *{\fov intpar};\\
                \>\>    int     \>      {\fov numintcu};\\
                \>\>    SISLIntcurve \> **{\fov intcurve};\\
                \>\>    int     \>      {\fov stat};\\
                \>\>    \ldots \\
        \>\>s1502(\begin{minipg4}
        {\fov curve}, {\fov basept}, {\fov normdir}, {\fov ellipaxis}, {\fov alpha}, {\fov ratio}, {\fov dim}, {\fov epsco}, {\fov epsge}, \&{\fov numintpt},
                \&{\fov intpar}, \&{\fov numintcu}, \&{\fov intcurve}, \&{\fov stat});
                        \end{minipg4}\\
                \>\>    \ldots \\
                \>      \}
\end{tabbing}

\pgsbreak
\subsection{Intersection between a curve and a torus.}
\funclabel{s1375}
\begin{minipg1}
  Find all the intersections between a curve and a torus.
\end{minipg1} \\ \\
SYNOPSIS\\
        \>void s1375(\begin{minipg3}
        {\fov curve}, {\fov centre}, {\fov normal}, {\fov centdist}, {\fov rad}, {\fov dim}, {\fov epsco}, {\fov epsge},\\ {\fov numintpt}, {\fov intpar},
                        {\fov numintcu}, {\fov intcurve}, {\fov stat})
                \end{minipg3}\\[0.3ex]
                \>\>    SISLCurve       \>      *{\fov curve};\\
                \>\>    double  \>      {\fov centre[\,]};\\
                \>\>    double  \>      {\fov normal}[\,];\\
                \>\>    double  \>      {\fov centdist};\\
                \>\>    double  \>      {\fov rad};\\
                \>\>    int             \>      {\fov dim};\\
                \>\>    double  \>      {\fov epsco};\\
                \>\>    double  \>      {\fov epsge};\\
                \>\>    int     \>      *{\fov numintpt};\\
                \>\>    double  \>      **{\fov intpar};\\
                \>\>    int     \>      *{\fov numintcu};\\
                \>\>    SISLIntcurve \> ***{\fov intcurve};\\
                \>\>    int     \>      *{\fov stat};\\
\\
ARGUMENTS\\
        \>Input Arguments:\\
        \>\>    {\fov curve}    \> - \> Pointer to the curve.\\
        \>\>    {\fov centre}   \> - \> \begin{minipg2}
                                The centre of the torus (lying in the symmetry plane)
                                \end{minipg2}\\[0.3ex]
        \>\>    {\fov normal}   \> - \> Normal of symmetry plane.\\
        \>\>    {\fov centdist}\> - \>  \begin{minipg2}
                                Distance from the centre of the cone to the centre circle of the torus.
                                \end{minipg2}\\[0.8ex]
        \>\>    {\fov rad}      \> - \> The radius of the torus surface.\\
        \>\>    {\fov dim}      \> - \> \begin{minipg2}
                                Dimension of the space in which the
                                torus and the curve
                                lie, dim should be equal to  three.
                                \end{minipg2}\\[0.3ex]
        \>\>    {\fov epsco}    \> - \> Computational resolution (not used).\\
        \>\>    {\fov epsge}    \> - \> Geometry resolution.\\
\\
        \>Output Arguments:\\
        \>\>    {\fov numintpt}\> - \>  Number of single intersection points.\\
        \>\>    {\fov intpar}   \> - \> \begin{minipg2}
                        Array containing the parameter values of the
                        single intersection points in the parameter
                        interval of the curve. The points lie in sequence.
                        Intersection curves are stored in intcurve.
                                \end{minipg2}\\[0.8ex]
        \>\>    {\fov numintcu}\> - \>Number of intersection curves.\\
        \>\>    {\fov intcurve}\> - \>  \begin{minipg2}
                        Array of pointers to the SISLIntcurve objects
                        containing descriptions of the intersection
                        curves. The curves are only described by start
                        points and end points in
                        the parameter interval of the curve. The curve
                        pointers point to nothing.
                                \end{minipg2}\\[0.8ex]
        \>\>    {\fov stat}     \> - \> Status messages\\
                \>\>\>\>\>              $> 0$   : warning\\
                \>\>\>\>\>              $= 0$   : ok\\
                \>\>\>\>\>              $< 0$   : error\\
\\
EXAMPLE OF USE\\
                \>      \{ \\
                \>\>    SISLCurve       \>      *{\fov curve};\\
                \>\>    double  \>      {\fov centre}[3];\\
                \>\>    double  \>      {\fov normal}[3];\\
                \>\>    double  \>      {\fov centdist};\\
                \>\>    double  \>      {\fov rad};\\
                \>\>    int     \>      {\fov dim} = 3;\\
                \>\>    double  \>      {\fov epsco};\\
                \>\>    double  \>      {\fov epsge};\\
                \>\>    int     \>      {\fov numintpt};\\
                \>\>    double  \>      *{\fov intpar};\\
                \>\>    int     \>      {\fov numintcu};\\
                \>\>    SISLIntcurve \> **{\fov intcurve};\\
                \>\>    int     \>      {\fov stat};\\
                \>\>    {\fov \ldots} \\
        \>\>s1375(\begin{minipg4}
        {\fov curve}, {\fov centre}, {\fov normal}, {\fov centdist}, {\fov rad}, {\fov dim}, {\fov epsco}, {\fov epsge},\\ \&{\fov numintpt},
                \&{\fov intpar}, \&{\fov numintcu}, \&{\fov intcurve}, \&{\fov stat});
                        \end{minipg4}\\
                \>\>    \ldots \\
                \>      \}
\end{tabbing}

\pgsbreak
\subsection{Intersection between a surface and a point.}
\funclabel{s1870}
\begin{minipg1}
  Find all intersections between a surface and a point.
\end{minipg1} \\ \\
SYNOPSIS\\
        \>void s1870(\begin{minipg3}
        {\fov ps1}, {\fov pt1}, {\fov idim}, {\fov aepsge}, {\fov jpt}, {\fov gpar1}, {\fov jcrv}, {\fov wcurve}, {\fov jstat})
      \end{minipg3}\\[0.3ex]
      \>\>    SISLSurf    \>  *{\fov ps1};\\
      \>\>    double      \>  *{\fov pt1};\\
      \>\>    int         \>  {\fov idim};\\
      \>\>    double      \>  {\fov aepsge};\\
      \>\>    int         \>  *{\fov jpt};\\
      \>\>    double      \>  **{\fov gpar1};\\
      \>\>    int         \>  *{\fov jcrv};\\
      \>\>    SISLIntcurve\>  ***{\fov wcurve};\\
      \>\>    int         \>  *{\fov jstat};\\
\\
ARGUMENTS\\
        \>Input Arguments:\\
        \>\>    {\fov ps1}    \> - \> Pointer to the surface.\\
        \>\>    {\fov pt1}    \> - \> Coordinates of the point.\\
        \>\>    {\fov idim}   \> - \> Number of coordinates in pt1.\\
        \>\>    {\fov aepsge} \> - \> Geometry resolution.\\
\\
        \>Output Arguments:\\
        \>\>    {\fov jpt}    \> - \> Number of single intersection points.\\
        \>\>    {\fov gpar1}  \> - \> \begin{minipg2}
                                        Array containing the parameter values of the
                                        single intersection points in the parameter
                                        interval of the surface. The points lie
                                        continuous. Intersection curves
                                        are stored in wcurve.
                                      \end{minipg2}\\[0.8ex]
        \>\>    {\fov jcrv}   \> - \> Number of intersection curves.\\
        \>\>    {\fov wcurve} \> - \> \begin{minipg2}
                                        Array containing descriptions of
                                        the intersection curves. The
                                        curves are only described by
                                        points in the parameter
                                        plane. The curve-pointers points
                                        to nothing.\\
                                        If the curves given as input are
                                        degnenerate an intersection
                                        point can be returned as an
                                        intersection curve. Use s1327 to
                                        decide if an intersection curve
                                        is a point on one of the
                                        curves.
                                      \end{minipg2}\\[0.8ex]
        \>\>    {\fov jstat}     \> - \> Status messages\\
                \>\>\>\>\>              $> 0$   : Warning.\\
                \>\>\>\>\>              $= 0$   : Ok.\\
                \>\>\>\>\>              $< 0$   : Error.\\
\newpagetabs
EXAMPLE OF USE\\
        \>      \{ \\
        \>\>    SISLSurf    \>  *{\fov ps1};\\
        \>\>    double      \>  *{\fov pt1};\\
        \>\>    int         \>  {\fov idim};\\
        \>\>    double      \>  {\fov aepsge};\\
        \>\>    int         \>  {\fov jpt} = 0;\\
        \>\>    double      \>  *{\fov gpar1} = NULL;\\
        \>\>    int         \>  {\fov jcrv} = 0;\\
        \>\>    SISLIntcurve\>  **{\fov wcurve} = NULL;\\
        \>\>    int         \>  {\fov jstat} = 0;\\
        \>\>    \ldots \\
        \>\>s1870(\begin{minipg4}
        {\fov ps1}, {\fov pt1}, {\fov idim}, {\fov aepsge}, \&{\fov jpt}, \&{\fov gpar1}, \&{\fov jcrv}, \&{\fov wcurve}, \&{\fov jstat});
      \end{minipg4}\\
      \>\>    \ldots \\
      \>      \}
\end{tabbing}

\pgsbreak
\subsection{Intersection between a surface and a straight line.}
\funclabel{s1856}
\begin{minipg1}
  Find all intersections between a tensor-product surface and an infinite straight
  line.
\end{minipg1} \\ \\
SYNOPSIS\\
        \>void s1856(\begin{minipg3}
                {\fov surf}, {\fov point}, {\fov linedir}, {\fov dim}, {\fov epsco}, {\fov epsge}, {\fov numintpt}, {\fov pointpar},
                \linebreak {\fov numintcr}, {\fov intcurves}, {\fov stat})
                \end{minipg3}\\[0.3ex]

                \>\>    SISLSurf        \>      *{\fov surf};\\
                \>\>    double  \>      {\fov point}[\,];\\
                \>\>    double  \>      {\fov linedir}[\,];\\
                \>\>    int     \>      {\fov dim};\\
                \>\>    double  \>      {\fov epsco};\\
                \>\>    double  \>      {\fov epsge};\\
                \>\>    int     \>      *{\fov numintpt};\\
                \>\>    double  \>      **{\fov pointpar};\\
                \>\>    int     \>      *{\fov numintcr};\\
                \>\>    SISLIntcurve\>  ***{\fov intcurves};\\
                \>\>    int     \>      *{\fov stat};\\
\\
ARGUMENTS\\
        \>Input Arguments:\\
        \>\>    {\fov surf}     \> - \> \begin{minipg2}
                                Pointer to the surface.
                                \end{minipg2}\\
        \>\>    {\fov point}    \> - \> \begin{minipg2}
                                Point on the line.
                                \end{minipg2}\\
        \>\>    {\fov linedir}  \> - \> \begin{minipg2}
                                Direction vector of the line.
                                \end{minipg2}\\
        \>\>    {\fov dim}      \> - \> \begin{minipg2}
                                Dimension of the space in which the line lies.
                                \end{minipg2}\\
        \>\>    {\fov epsco}    \> - \> \begin{minipg2}
                                Computational resolution (not used).
                                \end{minipg2}\\
        \>\>    {\fov epsge}    \> - \> \begin{minipg2}
                                Geometry resolution.
                                \end{minipg2}\\
\\
        \>Output Arguments:\\
        \>\>    {\fov numintpt} \> - \> \begin{minipg2}
                                Number of single intersection points.
                                \end{minipg2}\\
        \>\>    {\fov pointpar} \> - \> \begin{minipg2}
                                Array containing the parameter values of the
                                single intersection points in the parameter plane
                                of the surface. The points lie in sequence.
                                Intersection curves are stored in intcurves.
                                \end{minipg2}\\[0.8ex]
        \>\>    {\fov numintcr} \> - \> \begin{minipg2}
                                Number of intersection curves.
                                \end{minipg2}\\
        \>\>    {\fov intcurves}        \> - \> \begin{minipg2}
                                Array containing the description of the intersection
                                curves. The curves are only described by
                                start points and end points in
                                the parameter plane. The curve pointers point to
                                nothing.

                                \end{minipg2}\\[0.3ex]
        \>\>    {\fov stat}     \> - \> Status messages\\
                \>\>\>\>\>              $> 0$   : warning\\
                \>\>\>\>\>              $= 0$   : ok\\
                \>\>\>\>\>              $< 0$   : error\\
\newpagetabs
EXAMPLE OF USE\\
                \>      \{ \\
                \>\>    SISLSurf        \>      *{\fov surf};\\
                \>\>    double  \>      {\fov point}[3];\\
                \>\>    double  \>      {\fov linedir}[3];\\
                \>\>    int     \>      {\fov dim} = 3;\\
                \>\>    double  \>      {\fov epsco};\\
                \>\>    double  \>      {\fov epsge};\\
                \>\>    int     \>      {\fov numintpt};\\
                \>\>    double  \>      *{\fov pointpar};\\
                \>\>    int     \>      {\fov numintcr};\\
                \>\>    SISLIntcurve\>  **{\fov intcurves};\\
                \>\>    int     \>      {\fov stat};\\
                \>\>    \ldots \\
        \>\>s1856(\begin{minipg4}
                {\fov surf}, {\fov point}, {\fov linedir}, {\fov dim}, {\fov epsco}, {\fov epsge}, \&{\fov numintpt}, \&{\fov pointpar},
                \&{\fov numintcr}, \&{\fov intcurves}, \&{\fov stat});
                        \end{minipg4}\\
                \>\>    \ldots \\
                \>      \}
\end{tabbing}

\pgsbreak
\subsection{Newton iteration on the intersection between a 3D NURBS surface and a line.}
\funclabel{s1518}
\begin{minipg1}
Newton iteration on the intersection between
               a 3D NURBS surface and a line.
               If a good initial guess is given, the intersection will
               be found quickly. However if a bad initial guess is given,
               the iteration might not converge.
               We only search in the rectangular subdomain specified
               by "start" and "end". This can be the whole domain if desired.
\end{minipg1} \\ \\
SYNOPSIS\\
        \>     void s1518(\begin{minipg3}
            {\fov surf},  {\fov point},  {\fov dir},  {\fov epsge},  {\fov start},  {\fov end},  {\fov parin},  {\fov parout},  {\fov stat})
                \end{minipg3}\\
                \>\>    SISLSurf    \>  *{\fov surf};\\
                \>\>    double \> point[\,];\\
                \>\>    double \> dir[\,];\\
                \>\>    double \> epsge;\\
                \>\>    double \> start[\,];\\
                \>\>    double \> end[\,];\\
                \>\>    double \> parin[\,];\\
                \>\>    double \> parout[\,];\\
                \>\>    int    \>  *{\fov stat};\\
\\
ARGUMENTS\\
	\>Input Arguments:\\
        \>\>    {\fov surf}\> - \>  \begin{minipg2}
                     The NURBS surface.
                               \end{minipg2}\\
        \>\>    {\fov point}\> - \>  \begin{minipg2}
                     A point on the line.
                               \end{minipg2}\\
        \>\>    {\fov dir}\> - \>  \begin{minipg2}
                     The vector direction of the line
                        (not necessarily normalized).
                               \end{minipg2}\\
        \>\>    {\fov epsge}\> - \>  \begin{minipg2}
                     Geometric resolution.
                               \end{minipg2}\\
        \>\>    {\fov start}\> - \>  \begin{minipg2}
                     Lower limits of search rectangle (umin, vmin).
                               \end{minipg2}\\
        \>\>    {\fov end}\> - \>  \begin{minipg2}
                     Upper limits of search rectangle (umax, vmax).
                               \end{minipg2}\\
        \>\>    {\fov parin}\> - \>  \begin{minipg2}
                     Initial guess (u0,v0) for parameter point of
                        intersection (which should be inside the
                        search rectangle).
                               \end{minipg2}\\
\\
	\>Output Arguments:\\
        \>\>    {\fov parout}\> - \>  \begin{minipg2}
                     Parameter point (u,v) of intersection.
                               \end{minipg2}\\
        \>\>    {\fov jstat}\> - \>  \begin{minipg2}
                     status messages  
                                = 1   : Intersection found.
                                < 0   : error.
                               \end{minipg2}\\
\\
EXAMPLE OF USE\\
		\>      \{ \\

                \>\>    SISLSurf    \>  *{\fov surf};\\
                \>\>    double \> point[\,];\\
                \>\>    double \> dir[\,];\\
                \>\>    double \> epsge;\\
                \>\>    double \> start[\,];\\
                \>\>    double \> end[\,];\\
                \>\>    double \> parin[\,];\\
                \>\>    double \> parout[\,];\\
                \>\>    int    \>  *{\fov stat};\\                \>\>    \ldots \\
        \>    \>s1518(\begin{minipg4}
            {\fov surf},  {\fov point},  {\fov dir},  {\fov epsge},  {\fov start},  {\fov end},  {\fov parin},  {\fov parout},  {\fov stat});
                \end{minipg4}\\
                \>\>    \ldots \\
		\>      \}
\end{tabbing}

\pgsbreak
\subsection{Convert a surface/line intersection into a two-dimensional surface/origo intersection}
\funclabel{s1328}
\begin{minipg1}
Put the equation of the surface pointed at by psold
               into two planes given by the point epoint and the normals
               enorm1 and enorm2. The result is an equation where the 
               new two-dimensional surface rsnew is to be equal to origo.
\end{minipg1} \\ \\
SYNOPSIS\\
        \> void s1328(\begin{minipg3}
            {\fov psold}, {\fov epoint}, {\fov enorm1}, {\fov enorm2}, {\fov idim}, {\fov rsnew}, {\fov jstat})
                \end{minipg3}\\
                \>\>    SISLSurf    \>  *{\fov psold};\\
                \>\>    double \> epoint[\,];\\
                \>\>    double \> enorm1[\,];\\
                \>\>    double \> enorm2[\,];\\
                \>\>    int    \>  {\fov idim};\\
                \>\>    SISLSurf    \>  **{\fov rsnew};\\
                \>\>    int    \>  *{\fov jstat};\\
\\
ARGUMENTS\\
	\>Input Arguments:\\
        \>\>    {\fov psold}\> - \>  \begin{minipg2}
                     Pointer to input surface.
                               \end{minipg2}\\
        \>\>    {\fov epoint}\> - \>  \begin{minipg2}
                     SISLPoint in the planes.
                               \end{minipg2}\\
        \>\>    {\fov enorm1}\> - \>  \begin{minipg2}
                     Normal to the first plane.
                               \end{minipg2}\\
        \>\>    {\fov enorm2}\> - \>  \begin{minipg2}
                     Normal to the second plane.
                               \end{minipg2}\\
        \>\>    {\fov idim}\> - \>  \begin{minipg2}
                     Dimension of the space in which the planes lie.
                               \end{minipg2}\\
\\
	\>Output Arguments:\\
        \>\>    {\fov rsnew}\> - \>  \begin{minipg2}
                    dimensional surface.
                               \end{minipg2}\\
        \>\>    {\fov jstat}\> - \> status messages  \\
	            \>\>\>\>\>          $ > 0 $      : warning\\
		    \>\>\>\>\>          $ = 0 $      : ok \\
		    \>\>\>\>\>          $ < 0 $      : error \\
\\
EXAMPLE OF USE\\
		\>      \{ \\

                \>\>    SISLSurf    \>  *{\fov psold};\\
                \>\>    double \> epoint[\,];\\
                \>\>    double \> enorm1[\,];\\
                \>\>    double \> enorm2[\,];\\
                \>\>    int    \>  {\fov idim};\\
                \>\>    SISLSurf    \>  **{\fov rsnew};\\
                \>\>    int    \>  *{\fov jstat};\\                \>\>    \ldots \\
        \>\>s1328(\begin{minipg4}
            {\fov psold}, {\fov epoint}, {\fov enorm1}, {\fov enorm2}, {\fov idim}, {\fov rsnew}, {\fov jstat});
                \end{minipg4}\\
                \>\>    \ldots \\
		\>      \}
\end{tabbing}

\pgsbreak
\subsection{Intersection between a surface and a circle.}
\funclabel{s1855}
\begin{minipg1}
  Find all intersections between a tensor-product surface and a full circle.
\end{minipg1} \\ \\
SYNOPSIS\\
        \>void s1855(\begin{minipg3}
                {\fov surf}, {\fov centre}, {\fov radius}, {\fov normal}, {\fov dim}, {\fov epsco}, {\fov epsge},
                {\fov numintpt}, \linebreak {\fov pointpar}, {\fov numintcr}, {\fov intcurves}, {\fov stat})
                \end{minipg3}\\[0.3ex]
                \>\>    SISLSurf        \>      *{\fov surf};\\
                \>\>    double  \>      {\fov centre}[\,];\\
                \>\>    double  \>      {\fov radius};\\
                \>\>    double  \>      {\fov normal}[\,];\\
                \>\>    int     \>      {\fov dim};\\
                \>\>    double  \>      {\fov epsco};\\
                \>\>    double  \>      {\fov epsge};\\
                \>\>    int     \>      *{\fov numintpt};\\
                \>\>    double  \>      **{\fov pointpar};\\
                \>\>    int     \>      *{\fov numintcr};\\
                \>\>    SISLIntcurve\>  ***{\fov intcurves};\\
                \>\>    int     \>      *{\fov stat};\\
\\
ARGUMENTS\\
        \>Input Arguments:\\
        \>\>    {\fov surf}\> - \>      \begin{minipg2}
                                Pointer to the surface.
                                \end{minipg2}\\
        \>\>    {\fov centre}\> - \>    \begin{minipg2}
                                Centre of the circle.
                                \end{minipg2}\\
        \>\>    {\fov radius}\> - \>    \begin{minipg2}
                                Radius of the circle.
                                \end{minipg2}\\
        \>\>    {\fov normal}\> - \>    \begin{minipg2}
                                Normal vector to the plane in which the circle
                                lies.
                                \end{minipg2}\\
        \>\>    {\fov epsco}\> - \>     \begin{minipg2}
                                Computational resolution (not used).
                                \end{minipg2}\\
        \>\>    {\fov epsge}\> - \>     \begin{minipg2}
                                Geometry resolution.
                                \end{minipg2}\\
\\
        \>Output Arguments:\\
        \>\>    {\fov numintpt}\> - \>\begin{minipg2}
                                Number of single intersection points.
                                \end{minipg2}\\
        \>\>    {\fov pointpar}\> - \>  \begin{minipg2}
                                Array containing the parameter values of the
                                single intersection points in the parameter plane
                                of the surface. The points lie in sequence.
                                Intersection curves are stored in intcurves.
                                \end{minipg2}\\[0.8ex]
        \>\>    {\fov numintcr}\> - \>  \begin{minipg2}
                                Number of intersection curves.
                                \end{minipg2}\\
        \>\>    {\fov intcurves}\> - \> \begin{minipg2}
                                Array containing the description of the intersection
                                curves. The curves are only described by
                                start points and end points in
                                the parameter plane. The curve pointers point to
                                nothing.

                                \end{minipg2}\\[0.3ex]
        \>\>    {\fov stat}     \> - \> Status messages\\
                \>\>\>\>\>              $> 0$   : warning\\
                \>\>\>\>\>              $= 0$   : ok\\
                \>\>\>\>\>              $< 0$   : error\\
\newpagetabs
EXAMPLE OF USE\\
                \>      \{ \\
                \>\>    SISLSurf        \>      *{\fov surf};\\
                \>\>    double  \>      {\fov centre}[3];\\
                \>\>    double  \>      {\fov radius};\\
                \>\>    double  \>      {\fov normal}[3];\\
                \>\>    int     \>      {\fov dim} = 3;\\
                \>\>    double  \>      {\fov epsco};\\
                \>\>    double  \>      {\fov epsge};\\
                \>\>    int     \>      {\fov numintpt};\\
                \>\>    double  \>      *{\fov pointpar};\\
                \>\>    int     \>      {\fov numintcr};\\
                \>\>    SISLIntcurve\>  **{\fov intcurves};\\
                \>\>    int     \>      {\fov stat};\\
                \>\>    \ldots \\
        \>\>s1855(\begin{minipg4}
                {\fov surf}, {\fov centre}, {\fov radius}, {\fov normal}, {\fov dim}, {\fov epsco}, {\fov epsge},
                        \&{\fov numintpt}, \&{\fov pointpar}, \&{\fov numintcr}, \&{\fov intcurves}, \&{\fov stat});
                        \end{minipg4}\\
                \>\>    \ldots \\
                \>      \}
\end{tabbing}

\pgsbreak
\subsection{Intersection between a surface and a curve.}
\funclabel{s1858}
\begin{minipg1}
  Find all intersections between a surface and a curve.
  Intersection curves are described by guide points.
  To pick the intersection curves use s1712() described on page \pageref{s1712}.
\end{minipg1} \\ \\
SYNOPSIS\\
        \>void s1858(\begin{minipg3}
                {\fov surf}, {\fov curve}, {\fov epsco}, {\fov epsge}, {\fov numintpt}, {\fov pointpar1},
                {\fov pointpar2}, \linebreak {\fov numintcr},
                {\fov intcurves}, {\fov stat})
                \end{minipg3}\\[0.3ex]
                \>\>    SISLSurf        \>      *{\fov surf};\\
                \>\>    SISLCurve       \>      *{\fov curve};\\
                \>\>    double  \>      {\fov epsco};\\
                \>\>    double  \>      {\fov epsge};\\
                \>\>    int     \>      *{\fov numintpt};\\
                \>\>    double  \>      **{\fov pointpar1};\\
                \>\>    double  \>      **{\fov pointpar2};\\
                \>\>    int     \>      *{\fov numintcr};\\
                \>\>    SISLIntcurve\>  ***{\fov intcurves};\\
                \>\>    int     \>      *{\fov stat};\\
\\
ARGUMENTS\\
        \>Input Arguments:\\
        \>\>    {\fov surf}\> - \>      \begin{minipg2}
                                Pointer to the surface.
                                \end{minipg2}\\
        \>\>    {\fov curve}\> - \>     \begin{minipg2}
                                Pointer to the curve.
                                \end{minipg2}\\
        \>\>    {\fov epsco}\> - \>     \begin{minipg2}
                                Computational resolution (not used).
                                \end{minipg2}\\
        \>\>    {\fov epsge}\> - \>     \begin{minipg2}
                                Geometry resolution.
                                \end{minipg2} \\
\\
        \>Output Arguments:\\
        \>\>    {\fov numintpt}\> - \>  \begin{minipg2}
                                Number of single intersection points.
                                \end{minipg2}\\
        \>\>    {\fov pointpar1}\> - \>\begin{minipg2}
                                Array containing the parameter values of the
                                single intersection points in the parameter
                                plane of the surface. The points lie
                                in sequence. Intersection curves are stored in
                                intcurves.
                                \end{minipg2}\\[0.8ex]
        \>\>    {\fov pointpar2}\> - \>\begin{minipg2}
                                Array containing the parameter values of the
                                single intersection points in the parameter
                                interval of the curve.
                                \end{minipg2}\\[0.8ex]
        \>\>    {\fov numintcr}\> - \>  \begin{minipg2}
                                Number of intersection curves.
                                \end{minipg2}\\
        \>\>    {\fov intcurves}\> - \>\begin{minipg2}
                                Array containing the description of the intersection
                                curves. The curves are only described by
                                start points and end points (guide points) in
                                the parameter plane.

                         The curve pointers point to
                                nothing.
                                If the curves given as input are
                                degenerate, an intersection point can be returned
                                as an intersection curve.
                                \end{minipg2}\\[0.8ex]
\newpagetabs
        \>\>    {\fov stat}     \> - \> Status messages\\
                \>\>\>\>\>              $> 0$   : warning\\
                \>\>\>\>\>              $= 0$   : ok\\
                \>\>\>\>\>              $< 0$   : error\\
\\
EXAMPLE OF USE\\
                \>      \{ \\
                \>\>    SISLSurf        \>      *{\fov surf};\\
                \>\>    SISLCurve       \>      *{\fov curve};\\
                \>\>    double  \>      {\fov epsco};\\
                \>\>    double  \>      {\fov epsge};\\
                \>\>    int     \>      {\fov numintpt};\\
                \>\>    double  \>      *{\fov pointpar1};\\
                \>\>    double  \>      *{\fov pointpar2};\\
                \>\>    int     \>      {\fov numintcr};\\
                \>\>    SISLIntcurve\>  **{\fov intcurves};\\
                \>\>    int     \>      {\fov stat};\\
                \>\>    \ldots \\
        \>\>s1858(\begin{minipg4}
                {\fov surf}, {\fov curve}, {\fov epsco}, {\fov epsge}, \&{\fov numintpt}, \&{\fov pointpar1}, \&{\fov pointpar2}, \&{\fov numintcr},
                \&{\fov intcurves}, \&{\fov stat});
                        \end{minipg4}\\
                \>\>    \ldots \\
                \>      \}
\end{tabbing}

\pgsbreak
\section{Find the Topology of the Intersection}
\subsection{Find the topology for the intersection of a surface and a plane.}
\funclabel{s1851}
\begin{minipg1}
  Find all intersections between a tensor-product surface and a plane.
  Intersection curves are described by guide points.
  To make the intersection curves use s1314() described on page \pageref{s1314}.
\end{minipg1} \\ \\
SYNOPSIS\\
        \>void s1851(\begin{minipg3}
                        {\fov surf}, {\fov point}, {\fov normal},
                {\fov dim}, {\fov epsco}, {\fov epsge}, {\fov numintpt},
                {\fov pointpar}, {\fov numintcr},
                        {\fov intcurves}, {\fov stat})
                \end{minipg3}\\[0.3ex]
                \>\>    SISLSurf        \>      *{\fov surf};\\
                \>\>    double  \>      {\fov point}[\,];\\
                \>\>    double  \>      {\fov normal}[\,];\\
                \>\>    int     \>      {\fov dim};\\
                \>\>    double  \>      {\fov epsco};\\
                \>\>    double  \>      {\fov epsge};\\
                \>\>    int     \>      *{\fov numintpt};\\
                \>\>    double  \>      **{\fov pointpar};\\
                \>\>    int     \>      *{\fov numintcr};\\
                \>\>    SISLIntcurve\>  ***{\fov intcurves};\\
                \>\>    int     \>      *{\fov stat};\\
\\
ARGUMENTS\\
        \>Input Arguments:\\
        \>\>    {\fov surf}\> - \>      \begin{minipg2}
                                Pointer to surface
                                \end{minipg2}\\
        \>\>    {\fov point}\> - \>     \begin{minipg2}
                                Point in the plane.
                                \end{minipg2}\\
        \>\>    {\fov normal}\> - \>    \begin{minipg2}
                                Normal to the plane.
                                \end{minipg2}\\
        \>\>    {\fov dim}\> - \>       \begin{minipg2}
                                Dimension of the space in which the plane lies.
                                \end{minipg2}\\
        \>\>    {\fov epsco}\> - \>     \begin{minipg2}
                                Computational resolution (not used).
                                \end{minipg2}\\
        \>\>    {\fov epsge}\> - \>     \begin{minipg2}
                                Geometry resolution.
                                \end{minipg2}\\
\\
        \>Output Arguments:\\
        \>\>    {\fov numintpt}\> - \>  \begin{minipg2}
                                Number of single intersection points.
                                \end{minipg2}\\
        \>\>    {\fov pointpar}\> - \>  \begin{minipg2}
                                Array containing the parameter values of the single
                                intersection points in the parameter plane of the
                                surface. The points lie in sequence. Intersection curves
                                are stored in intcurves.
                                \end{minipg2}\\[0.8ex]
        \>\>    numintcr\> - \> \begin{minipg2}
                                Number of intersection curves.
                                \end{minipg2}\\
        \>\>    intcurves\> - \>        \begin{minipg2}
                                Array containing descriptions of the intersection
                                curves. The curves are only described by
                                start points and end points (guide points) in
                                the parameter plane. The curve pointers
                                point to nothing.
                                \end{minipg2}\\[0.8ex]
\newpagetabs
        \>\>    {\fov stat}     \> - \> Status messages\\
                \>\>\>\>\>              $> 0$   : warning\\
                \>\>\>\>\>              $= 0$   : ok\\
                \>\>\>\>\>              $< 0$   : error\\
\\
EXAMPLE OF USE\\
                \>      \{ \\
                \>\>    SISLSurf        \>      *{\fov surf};\\
                \>\>    double  \>      {\fov point}[3];\\
                \>\>    double  \>      {\fov normal}[3];\\
                \>\>    int     \>      {\fov dim} = 3;\\
                \>\>    double  \>      {\fov epsco};\\
                \>\>    double  \>      {\fov epsge};\\
                \>\>    int     \>      {\fov numintpt};\\
                \>\>    double  \>      *{\fov pointpar};\\
                \>\>    int     \>      {\fov numintcr};\\
                \>\>    SISLIntcurve\>  **{\fov intcurves};\\
                \>\>    int     \>      {\fov stat};\\
                \>\>    \ldots \\
        \>\>s1851(\begin{minipg4}
                {\fov surf}, {\fov point}, {\fov normal}, {\fov dim},
                {\fov epsco}, {\fov epsge}, \&{\fov numintpt},
                \&{\fov pointpar}, \&{\fov numintcr},
                \&{\fov intcurves}, \&{\fov stat});
                        \end{minipg4}\\
                \>\>    \ldots \\
                \>      \}
\end{tabbing}

\pgsbreak
\subsection{Find the topology for the intersection of a surface and a sphere.}
\funclabel{s1852}
\begin{minipg1}
  Find all intersections between a tensor-product surface and a sphere.
  Intersection curves are described by guide points.
  To produce the intersection curves use s1315() described on page \pageref{s1315}.
\end{minipg1} \\ \\
SYNOPSIS\\
        \>void s1852(\begin{minipg3}
                {\fov surf}, {\fov centre}, {\fov radius}, {\fov dim}, {\fov epsco}, {\fov epsge}, {\fov numintpt},
                {\fov pointpar}, \linebreak {\fov numintcr},
                {\fov intcurves}, {\fov stat})
                \end{minipg3}\\[0.3ex]
                \>\>    SISLSurf        \>      *{\fov surf};\\
                \>\>    double  \>      {\fov centre} [];\\
                \>\>    double  \>      {\fov radius};\\
                \>\>    int     \>      {\fov dim};\\
                \>\>    double  \>      {\fov epsco};\\
                \>\>    double  \>      {\fov epsge};\\
                \>\>    int     \>      *{\fov numintpt};\\
                \>\>    double  \>      **{\fov pointpar};\\
                \>\>    int     \>      *{\fov numintcr};\\
                \>\>    SISLIntcurve\>  ***{\fov intcurves};\\
                \>\>    int     \>      *{\fov stat};\\
\\
ARGUMENTS\\
        \>Input Arguments:\\
        \>\>    {\fov surf}\> - \>              \begin{minipg2}
                                Pointer to the surface.
                                \end{minipg2}\\
        \>\>    {\fov centre}\> - \>    \begin{minipg2}
                                Center of the sphere.
                                \end{minipg2}\\
        \>\>    {\fov radius}\> - \>    \begin{minipg2}
                                Radius of the sphere.
                                \end{minipg2}\\
        \>\>    {\fov dim}\> - \>       \begin{minipg2}
                                Dimension of the space in which the sphere lies.
                                \end{minipg2}\\
        \>\>    {\fov epsco}\> - \>     \begin{minipg2}
                                Computational resolution (not used).
                                \end{minipg2}\\
        \>\>    {\fov epsge}\> - \>     \begin{minipg2}
                                Geometry resolution.
                                \end{minipg2}\\
\\
        \>Output Arguments:\\
        \>\>    {\fov numintpt}\> - \>\begin{minipg2}
                                Number of single intersection points.
                                \end{minipg2}\\
        \>\>    {\fov pointpar}\> - \>  \begin{minipg2}
                                Array containing the parameter values of the single
                                intersection points in the parameter plane of the
                                surface. The points lie in sequence. Intersection curves
                                are stored in intcurves.
                                \end{minipg2}\\[0.8ex]
        \>\>    {\fov numintcr}\> - \>  \begin{minipg2}
                                Number of intersection curves.
                                \end{minipg2}\\
        \>\>    {\fov intcurves}\> - \> \begin{minipg2}
                                Array containing description of the intersection
                                curves. The curves are only described by
                                start points and end points (guide points) in
                                the parameter plane. The curve pointers point to
                                nothing.
                                \end{minipg2}\\[0.3ex]
        \>\>    {\fov stat}     \> - \> Status messages\\
                \>\>\>\>\>              $> 0$   : warning\\
                \>\>\>\>\>              $= 0$   : ok\\
                \>\>\>\>\>              $< 0$   : error\\
%\newpagetabs
EXAMPLE OF USE\\
                \>      \{ \\
                \>\>    SISLSurf        \>      *{\fov surf};\\
                \>\>    double  \>      {\fov centre}[3];\\
                \>\>    double  \>      {\fov radius};\\
                \>\>    int     \>      {\fov dim} = 3;\\
                \>\>    double  \>      {\fov epsco};\\
                \>\>    double  \>      {\fov epsge};\\
                \>\>    int     \>      {\fov numintpt};\\
                \>\>    double  \>      *{\fov pointpar};\\
                \>\>    int     \>      {\fov numintcr};\\
                \>\>    SISLIntcurve\>  **{\fov intcurves};\\
                \>\>    int     \>      {\fov stat};\\
                \>\>    \ldots \\
        \>\>s1852(\begin{minipg4}
                        {\fov surf}, {\fov centre}, {\fov radius}, {\fov dim}, {\fov epsco}, {\fov epsge},
                        \&{\fov numintpt}, \&{\fov pointpar}, \&{\fov numintcr},
                        \&{\fov intcurves}, \&{\fov stat});
                        \end{minipg4}\\
                \>\>    \ldots \\
                \>      \}
\end{tabbing}

\pgsbreak
\subsection{Find the topology for the intersection of a surface and a cylinder.}
\funclabel{s1853}
\begin{minipg1}
  Find all intersections between a tensor-product surface and a cylinder.
  Intersection curves are described by guide points.
  To produce the intersection curves use s1316() described on page \pageref{s1316}.
\end{minipg1} \\ \\
SYNOPSIS\\
        \>void s1853(\begin{minipg3}
                        {\fov surf}, {\fov point}, {\fov cyldir}, {\fov radius}, {\fov dim}, {\fov epsco},
                        {\fov epsge}, {\fov numintpt}, \linebreak
                        {\fov pointpar}, {\fov numintcr}, {\fov intcurves}, {\fov stat})
                \end{minipg3}\\[0.3ex]
                \>\>    SISLSurf        \>      *{\fov surf};\\
                \>\>    double  \>      {\fov point}[\,];\\
                \>\>    double  \>      {\fov cyldir}[\,];\\
                \>\>    double  \>      {\fov radius};\\
                \>\>    int     \>      {\fov dim};\\
                \>\>    double  \>      {\fov epsco};\\
                \>\>    double  \>      {\fov epsge};\\
                \>\>    int     \>      *{\fov numintpt};\\
                \>\>    double  \>      **{\fov pointpar};\\
                \>\>    int     \>      *{\fov numintcr};\\
                \>\>    SISLIntcurve\>  ***{\fov intcurves};\\
                \>\>    int     \>      *{\fov stat};\\
\\
ARGUMENTS\\
        \>Input Arguments:\\
        \>\>    {\fov surf}\> - \>      \begin{minipg2}
                                Pointer to the surface.
                                \end{minipg2}\\
        \>\>    {\fov point}\> - \>     \begin{minipg2}
                                Point on the axis of the cylinder.
                                \end{minipg2}\\
        \>\>    {\fov cyldir}\> - \>    \begin{minipg2}
                                The direction vector of the axis of the cylinder.
                                \end{minipg2}\\
        \>\>    {\fov radius}\> - \>    \begin{minipg2}
                                Radius of the cylinder.
                                \end{minipg2}\\
        \>\>    {\fov dim}\> - \>       \begin{minipg2}
                                Dimension of the space in which the cylinder lies.
                                \end{minipg2}\\
        \>\>    {\fov epsco}\> - \>     \begin{minipg2}
                                Computational resolution (not used).
                                \end{minipg2}\\
        \>\>    {\fov epsge}\> - \>     \begin{minipg2}
                                Geometry resolution.
                                \end{minipg2}\\
\\
        \>Output Arguments:\\
        \>\>    {\fov numintpt}\> - \>  \begin{minipg2}
                                Number of single intersection points.
                                \end{minipg2}\\
        \>\>    {\fov pointpar}\> - \>  \begin{minipg2}
                                Array containing the parameter values of the single
                                intersection points in the parameter plane of the
                                surface. The points lie in sequence. Intersection curves
                                are stored in intcurves.
                                \end{minipg2}\\[0.8ex]
        \>\>    {\fov numintcr}\> - \>  \begin{minipg2}
                                Number of intersection curves.
                                \end{minipg2}\\
        \>\>    {\fov intcurves}\> - \> \begin{minipg2}
                                Array containing description of the intersection
                                curves. The curves are only described by
                                start points and end points (guide points) in
                                the parameter plane. The curve pointers point to
                                nothing.

                                \end{minipg2}\\[0.3ex]
\newpagetabs
        \>\>    {\fov stat}     \> - \> Status messages\\
                \>\>\>\>\>              $> 0$   : warning\\
                \>\>\>\>\>              $= 0$   : ok\\
                \>\>\>\>\>              $< 0$   : error\\
\\
EXAMPLE OF USE\\
                \>      \{ \\
                \>\>    SISLSurf        \>      *{\fov surf};\\
                \>\>    double  \>      {\fov point}[3];\\
                \>\>    double  \>      {\fov cyldir}[3];\\
                \>\>    double  \>      {\fov radius};\\
                \>\>    int     \>      {\fov dim} = 3;\\
                \>\>    double  \>      {\fov epsco};\\
                \>\>    double  \>      {\fov epsge};\\
                \>\>    int     \>      {\fov numintpt};\\
                \>\>    double  \>      *{\fov pointpar};\\
                \>\>    int     \>      {\fov numintcr};\\
                \>\>    intcurve\>      **{\fov intcurves};\\
                \>\>    int     \>      {\fov stat};\\
                \>\>    \ldots \\
        \>\>s1853(\begin{minipg4}
                {\fov surf}, {\fov point}, {\fov cyldir}, {\fov radius}, {\fov dim}, {\fov epsco}, {\fov epsge}, \&{\fov numintpt},
                \&{\fov pointpar}, \&{\fov numintcr}, \&{\fov intcurves}, \&{\fov stat});
                        \end{minipg4}\\
                \>\>    \ldots \\
                \>      \}
\end{tabbing}

\pgsbreak
\subsection{Find the topology for the intersection of a surface and a cone.}
\funclabel{s1854}
\begin{minipg1}
  Find all intersections between a tensor-product surface and a cone.
  Intersection curves are described by guide points.
  To produce the intersection curves use s1317() described on page \pageref{s1317}.
\end{minipg1} \\ \\
SYNOPSIS\\
        \>void s1854(\begin{minipg3}
                        {\fov surf}, {\fov toppt}, {\fov axispt}, {\fov conept}, {\fov dim}, {\fov epsco}, {\fov epsge}, {\fov numintpt}, {\fov pointpar},
                        {\fov numintcr}, {\fov intcurves}, {\fov stat})
                \end{minipg3}\\[0.3ex]
                \>\>    SISLSurf        \>      *{\fov surf};\\
                \>\>    double  \>      {\fov toppt}[\,];\\
                \>\>    double  \>      {\fov axispt}[\,];\\
                \>\>    double  \>      {\fov conept}[\,];\\
                \>\>    int     \>      {\fov dim};\\
                \>\>    double  \>      {\fov epsco};\\
                \>\>    double  \>      {\fov epsge};\\
                \>\>    int     \>      *{\fov numintpt};\\
                \>\>    double  \>      **{\fov pointpar};\\
                \>\>    int     \>      *{\fov numintcr};\\
                \>\>    SISLIntcurve\>  ***{\fov intcurves};\\
                \>\>    int     \>      *{\fov stat};\\
\\
ARGUMENTS\\
        \>Input Arguments:\\
        \>\>    {\fov surf}\> - \>      \begin{minipg2}
                                Pointer to the surface
                                \end{minipg2}\\
        \>\>    {\fov toppt}\> - \>     \begin{minipg2}
                                Top point of the cone.
                                \end{minipg2}\\
        \>\>    {\fov axispt}\> - \>    \begin{minipg2}
                                Point on the axis of the cone, axispt must be different from toppt.
                                \end{minipg2}\\[0.8ex]
        \>\>    {\fov conept}\> - \>    \begin{minipg2}
                                Point on the cone surface, conept must be different from toppt.
                                \end{minipg2}\\[0.8ex]
        \>\>    {\fov dim}\> - \>       \begin{minipg2}
                                Dimension of the space in which the cone lies.
                                \end{minipg2}\\
        \>\>    {\fov epsco}\> - \>     \begin{minipg2}
                                Computational resolution (not used).
                                \end{minipg2}\\
        \>\>    {\fov epsge}\> - \>     \begin{minipg2}
                                Geometry resolution.
                                \end{minipg2}\\
\\
        \>Output Arguments:\\
        \>\>    {\fov numintpt}\> - \>  \begin{minipg2}
                                Number of single intersection points.
                                \end{minipg2}\\
        \>\>    {\fov pointpar}\> - \>  \begin{minipg2}
                                Array containing the parameter values of the single
                                intersection points in the parameter plane of the
                                surface. The points lie in sequence. Intersection curves
                                are stored in intcurves.
                                \end{minipg2}\\[0.8ex]
        \>\>    {\fov numintcr}\> - \>  \begin{minipg2}
                                Number of intersection curves.
                                \end{minipg2}\\
        \>\>    {\fov intcurves}\> - \> \begin{minipg2}
                                Array containing the description of the intersection
                                curves. The curves are only described by
                                start points and end points (guide points) in
                                the parameter plane. The curve pointers point to
                                nothing.
                                \end{minipg2}\\[0.3ex]
%\newpagetabs
        \>\>    {\fov stat}     \> - \> Status messages\\
                \>\>\>\>\>              $> 0$   : warning\\
                \>\>\>\>\>              $= 0$   : ok\\
                \>\>\>\>\>              $< 0$   : error\\
\\
EXAMPLE OF USE\\
                \>      \{ \\
                \>\>    SISLSurf        \>      *{\fov surf};\\
                \>\>    double  \>      {\fov toppt}[3];\\
                \>\>    double  \>      {\fov axispt}[3];\\
                \>\>    double  \>      {\fov conept}[3];\\
                \>\>    int     \>      {\fov dim} = 3;\\
                \>\>    double  \>      {\fov epsco};\\
                \>\>    double  \>      {\fov epsge};\\
                \>\>    int     \>      {\fov numintpt};\\
                \>\>    double  \>      *{\fov pointpar};\\
                \>\>    int     \>      {\fov numintcr};\\
                \>\>    SISLIntcurve\>  **{\fov intcurves};\\
                \>\>    int     \>      {\fov stat};\\
                \>\>    \ldots \\
        \>\>s1854(\begin{minipg4}
                {\fov surf}, {\fov toppt}, {\fov axispt}, {\fov conept}, {\fov dim}, {\fov epsco}, {\fov epsge}, \&{\fov numintpt}, \&{\fov pointpar},
                \&numintcr, \&intcurves, \&stat);
                        \end{minipg4}\\
                \>\>    \ldots \\
                \>      \}
\end{tabbing}

\pgsbreak
\subsection{Find the topology for the intersection of a surface and an
\mbox{elliptic} cone.}
\funclabel{s1503}
\begin{minipg1}
  Find all intersections between a tensor-product surface and an elliptic cone.
  Intersection curves are described by guide points.
  To produce the intersection curves use s1501() described on page \pageref{s1501}.
\end{minipg1} \\
SYNOPSIS\\
        \>void s1503(\begin{minipg3}
                        {\fov surf}, {\fov basept}, {\fov normdir}, {\fov ellipaxis}, {\fov alpha}, {\fov ratio}, {\fov dim}, {\fov epsco}, {\fov epsge}, {\fov numintpt}, {\fov pointpar},
                        {\fov numintcr}, {\fov intcurves}, {\fov stat})
                \end{minipg3}\\[0.3ex]
                \>\>    SISLSurf        \>      *{\fov surf};\\
                \>\>    double  \>      {\fov basept}[\,];\\
                \>\>    double  \>      {\fov normdir}[\,];\\
                \>\>    double  \>      {\fov ellipaxis}[\,];\\
                \>\>    double  \>      {\fov alpha};\\
                \>\>    double  \>      {\fov ratio};\\
                \>\>    int     \>      {\fov dim};\\
                \>\>    double  \>      {\fov epsco};\\
                \>\>    double  \>      {\fov epsge};\\
                \>\>    int     \>      *{\fov numintpt};\\
                \>\>    double  \>      **{\fov pointpar};\\
                \>\>    int     \>      *{\fov numintcr};\\
                \>\>    SISLIntcurve\>  ***{\fov intcurves};\\
                \>\>    int     \>      *{\fov stat};\\
\\
ARGUMENTS\\
        \>Input Arguments:\\
        \>\>    {\fov surf}\> - \>      \begin{minipg2}
                                Pointer to the surface
                                \end{minipg2}\\
        \>\>    {\fov basept}\> - \>    \begin{minipg2}
                                Base point of the cone, centre of elliptic base.
                                \end{minipg2}\\
        \>\>    {\fov normdir}\> - \>   \begin{minipg2}
                                Direction of the cone axis, normal to the elliptic base. The default is pointing from the base point to the top point.
                                \end{minipg2}\\[0.8ex]
        \>\>    {\fov ellipaxis}\> - \> \begin{minipg2}
                                One of the axes of the ellipse (major or
                                minor).
                                The other axis will be calculated as
                                $normdir\times ellipaxis$ scaled with
                                {\fov ratio}.
                                \end{minipg2}\\[0.8ex]
        \>\>    {\fov alpha}\> - \>     \begin{minipg2}
                                The opening angle in radians of the cone at the ellipaxis.
                                \end{minipg2}\\[0.8ex]
        \>\>    {\fov ratio}\> - \>     \begin{minipg2}
                                The ratio of the major and minor
                                axes = ellipaxis/otheraxis.
                                \end{minipg2}\\[0.8ex]
        \>\>    {\fov dim}\> - \>       \begin{minipg2}
                                Dimension of the space in which the cone lies.
                                \end{minipg2}\\
        \>\>    {\fov epsco}\> - \>     \begin{minipg2}
                                Computational resolution (not used).
                                \end{minipg2}\\
        \>\>    {\fov epsge}\> - \>     \begin{minipg2}
                                Geometry resolution.
                                \end{minipg2}
\newpagetabs
\\
        \>Output Arguments:\\
        \>\>    {\fov numintpt}\> - \>  \begin{minipg2}
                                Number of single intersection points.
                                \end{minipg2}\\
        \>\>    {\fov pointpar}\> - \>  \begin{minipg2}
                                Array containing the parameter values of the single
                                intersection points in the parameter plane of the
                                surface. The points lie in sequence. Intersection curves
                                are stored in intcurves.
                                \end{minipg2}\\[0.8ex]
        \>\>    {\fov numintcr}\> - \>  \begin{minipg2}
                                Number of intersection curves.
                                \end{minipg2}\\
        \>\>    {\fov intcurves}\> - \> \begin{minipg2}
                                Array containing the description of the intersection
                                curves. The curves are only described by
                                start points and end points (guide points) in
                                the parameter plane. The curve pointers point to
                                nothing.
                                \end{minipg2}\\[0.3ex]
        \>\>    {\fov stat}     \> - \> Status messages\\
                \>\>\>\>\>              $> 0$   : warning\\
                \>\>\>\>\>              $= 0$   : ok\\
                \>\>\>\>\>              $< 0$   : error\\
\\
EXAMPLE OF USE\\
                \>      \{ \\
                \>\>    SISLSurf        \>      *{\fov surf};\\
                \>\>    double  \>      {\fov basept}[3];\\
                \>\>    double  \>      {\fov normdir}[3];\\
                \>\>    double  \>      {\fov ellipaxis}[3];\\
                \>\>    double  \>      {\fov alpha};\\
                \>\>    double  \>      {\fov ratio};\\
                \>\>    double  \>      {\fov alpha};\\
                \>\>    int     \>      {\fov dim} = 3;\\
                \>\>    double  \>      {\fov epsco};\\
                \>\>    double  \>      {\fov epsge};\\
                \>\>    int     \>      {\fov numintpt};\\
                \>\>    double  \>      *{\fov pointpar};\\
                \>\>    int     \>      {\fov numintcr};\\
                \>\>    SISLIntcurve\>  **{\fov intcurves};\\
                \>\>    int     \>      {\fov stat};\\
                \>\>    \ldots \\
        \>\>s1503(\begin{minipg4}
                {\fov surf}, {\fov basept}, {\fov normdir}, {\fov ellipaxis}, {\fov alpha}, {\fov ratio}, {\fov dim}, {\fov epsco}, {\fov epsge}, \&{\fov numintpt}, \&{\fov pointpar},
                \&numintcr, \&intcurves, \&stat);
                        \end{minipg4}\\
                \>\>    \ldots \\
                \>      \}
\end{tabbing}

\pgsbreak
\subsection{Find the topology for the intersection of a surface and a \mbox{torus}.}
\funclabel{s1369}
\begin{minipg1}
  Find all intersections between a surface and a torus.
  Intersection curves are described by guide points.
  To produce the intersection curves use s1318() described on page \pageref{s1318}.
\end{minipg1} \\ \\
SYNOPSIS\\
        \>void s1369(\begin{minipg3}
                {\fov surf}, {\fov centre}, {\fov normal}, {\fov cendist}, {\fov radius}, {\fov dim}, {\fov epsco},
                {\fov epsge}, {\fov \linebreak}
                {\fov numintpt}, {\fov pointpar}, {\fov numintcr}, {\fov intcurves}, {\fov stat})
                \end{minipg3}\\[0.3ex]
                \>\>    SISLSurf        \>      *{\fov surf};\\
                \>\>    double  \>      {\fov centre}[\,];\\
                \>\>    double  \>      {\fov normal}[\,];\\
                \>\>    double  \>      {\fov cendist};\\
                \>\>    double  \>      {\fov radius};\\
                \>\>    int     \>      {\fov dim};\\
                \>\>    double  \>      {\fov epsco};\\
                \>\>    double  \>      {\fov epsge};\\
                \>\>    int     \>      *{\fov numintpt};\\
                \>\>    double  \>      **{\fov pointpar};\\
                \>\>    int     \>      *{\fov numintcr};\\
                \>\>    SISLIntcurve\>  ***{\fov intcurves};\\
                \>\>    int     \>      *{\fov stat};\\
\\
ARGUMENTS\\
        \>Input Arguments:\\
        \>\>    {\fov surf}\> - \>      \begin{minipg2}
                                Pointer to the surface.
                                \end{minipg2}\\
        \>\>    {\fov centre}\> - \>    \begin{minipg2}
                                The centre of the torus (lying in the symmetry
                                plane)
                                \end{minipg2}\\
        \>\>    {\fov normal}\> - \>    \begin{minipg2}
                                Normal to the symmetry plane.
                                \end{minipg2}\\
        \>\>    {\fov cendist}\> - \>   \begin{minipg2}
                                Distance from centre to centre circle of the torus.
                                \end{minipg2} \\
        \>\>    {\fov radius}\> - \>    \begin{minipg2}
                                The radius of the torus surface.
                                \end{minipg2}\\
        \>\>    {\fov di}m      \> - \> \begin{minipg2}
                                Dimension of the space in which the torus lies. dim
                                should be equal to two or three.
                                \end{minipg2}\\[0.3ex]
        \>\>    {\fov epsco}\> - \>     \begin{minipg2}
                                Computational resolution (not used).
                                \end{minipg2}\\
        \>\>    {\fov epsge}\> - \>     \begin{minipg2}
                                Geometry resolution.
                                \end{minipg2}\\
        \>Output Arguments:\\
        \>\>    {\fov numintpt}\> - \>  \begin{minipg2}
                                Number of single intersection points.
                                \end{minipg2}\\
        \>\>    {\fov pointpar}\> - \>  \begin{minipg2}
                                Array containing the parameter values of the single
                                intersection points in the parameter plane of the
                                surface. The points lie in sequence. Intersection curves
                                are stored in intcurves.
                                \end{minipg2}\\[0.8ex]
        \>\>    {\fov numintcr}\> - \>  \begin{minipg2}
                                Number of intersection curves.
                                \end{minipg2}\\
        \>\>    {\fov intcurves}\> - \>\begin{minipg2}
                                Array containing the description of the intersection
                                curves. The curves are only described by
                                start points and end points (guide points) in
                                the parameter planes.
                                The curve pointers point
                                to nothing.
                                \end{minipg2}\\[0.3ex]
%\newpagetabs
        \>\>    {\fov stat}     \> - \> Status messages\\
                \>\>\>\>\>              $> 0$   : warning\\
                \>\>\>\>\>              $= 0$   : ok\\
                \>\>\>\>\>              $< 0$   : error\\
\\
EXAMPLE OF USE\\
                \>      \{ \\
                \>\>    SISLSurf        \>      *{\fov surf};\\
                \>\>    double  \>      {\fov centre}[3];\\
                \>\>    double  \>      {\fov normal}[3];\\
                \>\>    double  \>      {\fov cendist};\\
                \>\>    double  \>      {\fov radius};\\
                \>\>    int     \>      {\fov dim} = 3;\\
                \>\>    double  \>      {\fov epsco};\\
                \>\>    double  \>      {\fov epsge};\\
                \>\>    int     \>      {\fov numintpt};\\
                \>\>    double  \>      *{\fov pointpar};\\
                \>\>    int     \>      {\fov numintcr};\\
                \>\>    SISLIntcurve\>  **{\fov intcurves};\\
                \>\>    int     \>      {\fov stat};\\
                \>\>    \ldots \\
        \>\>s1369(\begin{minipg4}
                {\fov surf}, {\fov centre}, {\fov normal}, {\fov cendist}, {\fov radius}, {\fov dim}, {\fov epsco},
                {\fov epsge}, \linebreak \&{\fov numintpt},
                \&{\fov pointpar}, \&{\fov numintcr}, \&{\fov intcurves}, \&{\fov stat});
                        \end{minipg4}\\
                \>\>    \ldots \\
                \>      \}
\end{tabbing}

\pgsbreak
\subsection{Find the topology for the intersection between two surfaces.}
\funclabel{s1859}
\begin{minipg1}
  Find all intersections between two surfaces.
  Intersection curves are described by guide points.
  To produce the intersection curves use s1310() described on page \pageref{s1310}.
\end{minipg1} \\ \\
SYNOPSIS\\
        \>void s1859    (\begin{minipg3}
                {\fov surfl}, {\fov surf2}, {\fov epsco}, {\fov epsge}, {\fov numintpt},
                {\fov pointpar1}, {\fov pointpar2}, \linebreak {\fov numintcr}, {\fov intcurves},
                {\fov stat})
                \end{minipg3}\\[0.3ex]
                \>\>    SISLSurf        \>      *{\fov surf1};\\
                \>\>    SISLSurf        \>      *{\fov surf2};\\
                \>\>    double  \>      {\fov epsco};\\
                \>\>    double  \>      {\fov epsge};\\
                \>\>    int     \>      *{\fov numintpt};\\
                \>\>    double  \>      **{\fov pointpar1};\\
                \>\>    double  \>      **{\fov pointpar2};\\
                \>\>    int     \>      *{\fov numintcr};\\
                \>\>    SISLIntcurve\>  ***{\fov intcurves};\\
                \>\>    int     \>      *{\fov stat};\\
\\
ARGUMENTS\\
        \>Input Arguments:\\
        \>\>    {\fov surf1}\> - \>     \begin{minipg2}
                                Pointer to the first surface.
                                \end{minipg2}\\
        \>\>    {\fov surf2}\> - \>     \begin{minipg2}
                                Pointer to the second surface.
                                \end{minipg2}\\
        \>\>    {\fov epsco}\> - \>     \begin{minipg2}
                                Computational resolution (not used).
                                \end{minipg2}\\
        \>\>    {\fov epsge}\> - \>     \begin{minipg2}
                                Geometry resolution.
                                \end{minipg2} \\
\\
        \>Output Arguments:\\
        \>\>    {\fov numintpt}\> - \>  \begin{minipg2}
                                Number of single intersection points.
                                \end{minipg2}\\
        \>\>    {\fov pointpar1}\> - \>\begin{minipg2}
                                Array containing the parameter values of the single
                                intersection points in the parameter plane of the
                                first surface. The points lie in sequence. Intersection
                                curves are stored in intcurves.
                                \end{minipg2}\\[0.8ex]
        \>\>    {\fov pointpar2}\> - \>\begin{minipg2}
                                Array containing the parameter values of the single
                                intersection points in the parameter plane of the
                                second surface.
                                \end{minipg2}\\[0.8ex]
        \>\>    {\fov numintcr}\> - \>  \begin{minipg2}
                                Number of intersection curves.
                                \end{minipg2}\\
        \>\>    {\fov intcurves}\> - \>\begin{minipg2}
                                Array containing description of the intersection
                                curves. The curves are only described
                                by start points and end points (guide
                                points) in
                                the parameter planes of the surfaces.
                                The curve pointers point to
                                nothing.
                                \end{minipg2}\\[0.3ex]
        \>\>    {\fov stat}     \> - \> Status messages\\
                \>\>\>\>\>              $> 0$   : warning\\
                \>\>\>\>\>              $= 0$   : ok\\
                \>\>\>\>\>              $< 0$   : error\\
\newpagetabs
EXAMPLE OF USE\\
                \>      \{ \\
                \>\>    SISLSurf        \>      *{\fov surf1};\\
                \>\>    SISLSurf        \>      *{\fov surf2};\\
                \>\>    double  \>      {\fov epsco};\\
                \>\>    double  \>      {\fov epsge};\\
                \>\>    int     \>      {\fov numintpt};\\
                \>\>    double  \>      *{\fov pointpar1};\\
                \>\>    double  \>      *{\fov pointpar2};\\
                \>\>    int     \>      {\fov numintcr};\\
                \>\>    SISLIntcurve\>  **{\fov intcurves};\\
                \>\>    int     \>      {\fov stat};\\
                \>\>    \ldots \\
        \>\>s1859(\begin{minipg4}
                {\fov surfl}, {\fov surf2}, {\fov epsco}, {\fov epsge}, \&{\fov numintpt},
                \&{\fov pointpar1}, \&{\fov pointpar2}, \&{\fov numintcr}, \&{\fov intcurves},
                \&{\fov stat});
                \end{minipg4}\\
                \>\>    \ldots \\
                \>      \}
\end{tabbing}

\pgsbreak
\section{Find the Topology of a Silhouette}
\subsection{Find the topology of the silhouette curves of a surface,
using parallel projection.}
\funclabel{s1860}
\begin{minipg1}
  Find the silhouette curves and points of a surface when the surface is viewed
  from a specific direction (i.e.\ parallel projection).
  In addition to the points and curves found by this routine, break
  curves and edge-curves might be silhouette curves.
  Silhouette curves are described by guide points.
  To produce the silhouette curves use s1319() described on page \pageref{s1319}.
\end{minipg1} \\ \\
NOTE\\
\>     \begin{minipg6}
The silhouette curves are defined as curves on the surface where the inner product of the surface normal and the direction vector of the viewing is 0. This definition will include surface points where the normal is zero.
\end{minipg6}\\ \\
SYNOPSIS\\
        \>void s1860(\begin{minipg3}
                {\fov surf}, {\fov viewdir}, {\fov dim}, {\fov epsco}, {\fov epsge}, {\fov numsilpt}, {\fov pointpar}, {\fov numsilcr}, {\fov silcurves},
                {\fov stat})
                \end{minipg3}\\[0.3ex]
                \>\>    SISLSurf        \>      *{\fov surf};\\
                \>\>    double  \>      {\fov viewdir}[];\\
                \>\>    int     \>      {\fov dim};\\
                \>\>    double  \>      {\fov epsco};\\
                \>\>    double  \>      {\fov epsge};\\
                \>\>    int     \>      *{\fov numsilpt};\\
                \>\>    double  \>      **{\fov pointpar};\\
                \>\>    int     \>      *{\fov numsilcr};\\
                \>\>    SISLIntcurve\>  ***{\fov silcurves};\\
                \>\>    int     \>      *{\fov stat};\\
\\
ARGUMENTS\\
        \>Input Arguments:\\
        \>\>    {\fov surf}\> - \>              \begin{minipg2}
                                Pointer to the surface.
                                \end{minipg2}\\
        \>\>    {\fov viewdir}\> - \>   \begin{minipg2}
                                The direction vector of the viewing.
                                \end{minipg2}\\
        \>\>    {\fov dim}\> - \>       \begin{minipg2}
                                Dimension of the space in which {\fov viewdir} lies.
                                \end{minipg2}\\
        \>\>    {\fov epsco}\> - \>     \begin{minipg2}
                                Computational resolution (not used).
                                \end{minipg2}\\
        \>\>    {\fov epsge}\> - \>     \begin{minipg2}
                                Geometry resolution.
                                \end{minipg2}\\
\\
        \>Output Arguments:\\
        \>\>    {\fov numsilpt}\> - \>  \begin{minipg2}
                                Number of single silhouette points.
                                \end{minipg2}\\
        \>\>    {\fov pointpar}\> - \>  \begin{minipg2}
                                Array containing the parameter values of the
                                single silhouette points in the parameter plane of
                                the surface. The points lie in sequence. Silhouette
                                curves are stored in silcurves.
                                \end{minipg2}\\[0.8ex]
        \>\>    {\fov numsilcr}\> - \>  \begin{minipg2}
                                Number of silhouette curves.
                                \end{minipg2}\\
%\newpagetabs
        \>\>    {\fov silcurves}\> - \>\begin{minipg2}
                                Array containing the description of the silhouette
                                curves. The curves are only described by
                                start points and end points (guide points) in
                                the parameter plane. The curve pointers point to
                                nothing.
                                \end{minipg2}\\[0.3ex]
        \>\>    {\fov stat}     \> - \> Status messages\\
                \>\>\>\>\>              $> 0$   : warning\\
                \>\>\>\>\>              $= 0$   : ok\\
                \>\>\>\>\>              $< 0$   : error\\
\\
EXAMPLE OF USE\\
                \>      \{ \\
                \>\>    SISLSurf        \>      *{\fov surf};\\
                \>\>    double  \>      {\fov viewdir}[3];\\
                \>\>    int     \>      {\fov dim};\\
                \>\>    double  \>      {\fov epsco};\\
                \>\>    double  \>      {\fov epsge};\\
                \>\>    int     \>      {\fov numsilpt} = 0;\\
                \>\>    double  \>      *{\fov pointpar} = NULL;\\
                \>\>    int     \>      {\fov numsilcr} = 0;\\
                \>\>    SISLIntcurve\>  **{\fov silcurves} = NULL;\\
                \>\>    int     \>      {\fov stat} = 0;\\
                \>\>    \ldots \\
        \>\>s1860(\begin{minipg4}
                {\fov surf}, {\fov viewdir}, {\fov dim}, {\fov epsco}, {\fov epsge},
                \&{\fov numsilpt}, \&{\fov pointpar}, \linebreak \&{\fov numsilcr}, \&{\fov silcurves},
                \&{\fov stat});
                \end{minipg4}\\
                \>\>    \ldots \\
                \>      \}
\end{tabbing}

\pgsbreak
\subsection{Find the topology of the silhouette curves of a surface,
 using perspective projection.}
\funclabel{s1510}
\begin{minipg1}
  Find the silhouette curves and points of a surface when
  the surface is viewed perspectively from a specific eye point.
  In addition to the points and curves found by this routine,
  break curves and edge-curves might be silhouette curves.
  To march out the silhouette curves, use s1514() on page~\pageref{s1514}.
\end{minipg1} \\ \\
SYNOPSIS\\
        \>void s1510(\begin{minipg3}
          {\fov ps}, {\fov eyepoint}, {\fov idim},  {\fov aepsco},  {\fov aepsge},  {\fov jpt},  {\fov gpar},  {\fov jcrv},  {\fov wcurve},  {\fov jstat})
        \end{minipg3}\\[0.3ex]
        \>\>    SISLSurf \> *{\fov ps};\\
        \>\>    double   \> {\fov eyepoint}[\,];\\
        \>\>    int      \> {\fov idim};\\
        \>\>    double   \> {\fov aepsco};\\
        \>\>    double   \> {\fov aepsge};\\
        \>\>    int      \> *{\fov jpt};\\
        \>\>    double   \> **{\fov gpar};\\
        \>\>    int      \> *{\fov jcrv};\\
        \>\>    SISLIntcurve \> ***{\fov wcurve};\\
        \>\>    int      \> *{\fov jstat};\\
\\
ARGUMENTS\\
        \>Input Arguments:\\
        \>\>    {\fov ps}\> - \>  \begin{minipg2}
                           Pointer to the surface.
                               \end{minipg2}\\
        \>\>    {\fov eyepoint}\> - \>  \begin{minipg2}
                      The eye point vector.
                               \end{minipg2}\\
        \>\>    {\fov idim}\> - \>  \begin{minipg2}
                        Dimension of the space in which eyepoint lies.
                               \end{minipg2}\\
        \>\>    {\fov aepsco}\> - \>  \begin{minipg2}
                        Computational resolution (not used).
                               \end{minipg2}\\
        \>\>    {\fov aepsge}\> - \>  \begin{minipg2}
                        Geometry resolution.
                               \end{minipg2}\\
\\
        \>Output Arguments:\\
        \>\>    {\fov jpt}\> - \>  \begin{minipg2}
                     Number of single silhouette points.
                               \end{minipg2}\\
        \>\>    {\fov gpar}\> - \>  \begin{minipg2}
                     Array containing the parameter values of the
                       single silhouette points in the parameter
                       plane of the surface. The points lie continuous.
                       Silhouette curves are stored in wcurve.
                               \end{minipg2}\\[0.8ex]
        \>\>    {\fov jcrv}\> - \>  \begin{minipg2}
                     Number of silhouette curves.
                               \end{minipg2}\\
        \>\>    {\fov wcurve}\> - \>  \begin{minipg2}
                     Array containing descriptions of the silhouette
                       curves. The curves are only described by points
                       in the parameter plane. The curve-pointers points
                       to nothing.
                               \end{minipg2}\\[0.8ex]
        \>\>    {\fov jstat}     \> - \> Status messages\\
                \>\>\>\>\>              $> 0$   : warning\\
                \>\>\>\>\>              $= 0$   : ok\\
                \>\>\>\>\>              $< 0$   : error\\
\newpagetabs
EXAMPLE OF USE\\
        \>      \{ \\
        \>\>    SISLSurf \> *{\fov ps};\\
        \>\>    double   \> {\fov eyepoint}[3];\\
        \>\>    int      \> {\fov idim} = 3;\\
        \>\>    double   \> {\fov aepsco};\\
        \>\>    double   \> {\fov aepsge};\\
        \>\>    int      \> {\fov jpt} = 0;\\
        \>\>    double   \> *{\fov gpar} = NULL;\\
        \>\>    int      \> {\fov jcrv} = 0;\\
        \>\>    SISLIntcurve \> **{\fov wcurve} = NULL;\\
        \>\>    int      \> {\fov jstat} = 0;\\
        \>\>    \ldots \\
        \>\>s1510(\begin{minipg4}
          {\fov ps}, {\fov eyepoint}, {\fov idim},  {\fov aepsco},  {\fov aepsge}, \&{\fov jpt}, \&{\fov gpar}, \&{\fov jcrv}, \&{\fov wcurve}, \&{\fov jstat});
        \end{minipg4}\\
        \>\>    \ldots \\
        \>      \}
\end{tabbing}

\pgsbreak
\subsection{Find the topology of the circular silhouette curves of a
\mbox{surface}.}
\funclabel{s1511}
\begin{minipg1}
  Find the circular silhouette curves and points of a surface.
  In addition to the points and curves found by this routine,
  break curves and edge-curves might be silhouette curves.
  To march out the silhouette curves use s1515() on page~\pageref{s1515}.
\end{minipg1} \\ \\
SYNOPSIS\\
        \> void s1511(\begin{minipg3}
          {\fov ps}, {\fov qpoint}, {\fov bvec}, {\fov idim}, {\fov aepsco}, {\fov aepsge}, {\fov jpt},  {\fov gpar},  {\fov jcrv},  {\fov wcurve},  {\fov jstat})
        \end{minipg3}\\[0.3ex]
        \>\>    SISLSurf \> *{\fov ps};\\
        \>\>    double   \> {\fov qpoint}[\,];\\
        \>\>    double   \> {\fov bvec}[\,];\\
        \>\>    int      \> {\fov idim};\\
        \>\>    double   \> {\fov aepsco};\\
        \>\>    double   \> {\fov aepsge};\\
        \>\>    int      \> *{\fov jpt};\\
        \>\>    double   \> **{\fov gpar};\\
        \>\>    int      \> *{\fov jcrv};\\
        \>\>    SISLIntcurve \> ***{\fov wcurve};\\
        \>\>    int      \> *{\fov jstat};\\
\\
ARGUMENTS\\
        \>Input Arguments:\\
        \>\>    {\fov ps}\> - \>  \begin{minipg2}
                          Pointer to the surface.
                               \end{minipg2}\\
        \>\>    {\fov qpoint}\> - \>  \begin{minipg2}
                       A point on the spin axis.
                               \end{minipg2}\\
        \>\>    {\fov bvec}\> - \>  \begin{minipg2}
                        The circular silhouette axis direction.
                               \end{minipg2}\\
        \>\>    {\fov idim}\> - \>  \begin{minipg2}
                       Dimension of the space in which axis lies.
                               \end{minipg2}\\
        \>\>    {\fov aepsco}\> - \>  \begin{minipg2}
                       Computational resolution (not used).
                               \end{minipg2}\\
        \>\>    {\fov aepsge}\> - \>  \begin{minipg2}
                       Geometry resolution.
                               \end{minipg2}\\
\\
        \>Output Arguments:\\
        \>\>    {\fov jpt}\> - \>  \begin{minipg2}
                     Number of single silhouette points.
                               \end{minipg2}\\
        \>\>    {\fov gpar}\> - \>  \begin{minipg2}
                     Array containing the parameter values of the
                       single silhouette points in the parameter
                       plane of the surface. The points lie continuous.
                       Silhouette curves are stored in wcurve.
                               \end{minipg2}\\[0.8ex]
        \>\>    {\fov jcrv}\> - \>  \begin{minipg2}
                     Number of silhouette curves.
                               \end{minipg2}\\
        \>\>    {\fov wcurve}\> - \>  \begin{minipg2}
                     Array containing descriptions of the silhouette
                     curves. The curves are only described by points
                     in the parameter plane. The curve-pointers points
                     to nothing.
                               \end{minipg2}\\[0.8ex]
        \>\>    {\fov jstat}     \> - \> Status messages\\
                \>\>\>\>\>              $> 0$   : warning\\
                \>\>\>\>\>              $= 0$   : ok\\
                \>\>\>\>\>              $< 0$   : error\\
\\
EXAMPLE OF USE\\
        \>      \{ \\
        \>\>    SISLSurf \> *{\fov ps};\\
        \>\>    double   \> {\fov qpoint}[3];\\
        \>\>    double   \> {\fov bvec}[3];\\
        \>\>    int      \> {\fov idim} = 3;\\
        \>\>    double   \> {\fov aepsco};\\
        \>\>    double   \> {\fov aepsge};\\
        \>\>    int      \> {\fov jpt} = 0;\\
        \>\>    double   \> *{\fov gpar} = NULL;\\
        \>\>    int      \> {\fov jcrv} = 0;\\
        \>\>    SISLIntcurve \> **{\fov wcurve} = NULL;\\
        \>\>    int      \> {\fov jstat} = 0;\\
        \>\>    \ldots \\
        \>\>s1511(\begin{minipg4}
          {\fov ps}, {\fov qpoint}, {\fov bvec}, {\fov idim}, {\fov aepsco}, {\fov aepsge}, \&{\fov jpt}, \&{\fov gpar}, \&{\fov jcrv}, \&{\fov wcurve}, \&{\fov jstat});
        \end{minipg4}\\
        \>\>    \ldots \\
        \>      \}
\end{tabbing}

\pgsbreak
\section{Marching}
\subsection{March an intersection curve between a surface and a plane.}
\funclabel{s1314}
\begin{minipg1}
  To march an intersection curve described by parameter pairs in an intersection
  curve object, a surface and a plane.
  The guide points are expected to be found by s1851(), described on
  page \pageref{s1851}.
  The generated geometric curves are represented as B-spline curves.
\end{minipg1} \\ \\
SYNOPSIS\\
        \>void s1314(\begin{minipg3}
                {\fov surf}, {\fov point}, {\fov normal}, {\fov dim}, {\fov epsco}, {\fov epsge}, {\fov maxstep}, {\fov intcurve}, \linebreak
                {\fov makecurv}, {\fov graphic}, {\fov stat})
                \end{minipg3}\\[0.3ex]
                \>\>    SISLSurf        \>      *{\fov surf};\\
                \>\>    double  \>      {\fov point}[\,];\\
                \>\>    double  \>      {\fov normal}[\,];\\
                \>\>    int     \>      {\fov dim};\\
                \>\>    double  \>      {\fov epsco};\\
                \>\>    double  \>      {\fov epsge};\\
                \>\>    double  \>      {\fov maxstep};\\
                \>\>    SISLIntcurve\>  *{\fov intcurve};\\
                \>\>    int     \>      {\fov makecurv};\\
                \>\>    int     \>      {\fov graphic};\\
                \>\>    int     \>      *{\fov stat};\\
\\
ARGUMENTS\\
        \>Input Arguments:\\
        \>\>    {\fov surf}     \> - \> \begin{minipg2}
                                Pointer to the surface.
                                \end{minipg2}\\
        \>\>    {\fov point}    \> - \> \begin{minipg2}
                                Point in the plane.
                                \end{minipg2}\\
        \>\>    {\fov normal}   \> - \> \begin{minipg2}
                                Normal to the plane.
                                \end{minipg2}\\
        \>\>    {\fov dim}      \> - \> \begin{minipg2}
                                Dimension of the space in which the plane lies.
                                Should be 3.
                                \end{minipg2} \\[0.8ex]
        \>\>    {\fov epsco}    \> - \> \begin{minipg2}
                                Computational resolution (not used).
                                \end{minipg2}\\
        \>\>    {\fov epsge}    \> - \> \begin{minipg2}
                                Geometry resolution.
                                \end{minipg2}\\
        \>\>    {\fov maxstep}  \> - \> \begin{minipg2}
                                Maximum step length allowed.
                                If maxstep $\leq$ epsge maxstep is
                                neglected. maxstep = 0.0 is recommended.
                                \end{minipg2}\\[0.8ex]
        \>\>    {\fov makecurv}\> - \>  \begin{minipg2}
                                Indicator telling if a geometric curve is to be made:
                                \end{minipg2}\\
                \>\>\>\>\>      0 -     \>Do not make curves at all.\\
                \>\>\>\>\>      1 -     \>Make only one geometric curve.\\
                \>\>\>\>\>      2 -     \>\begin{minipg5}
                                        Make geometric curve and curve in the parameter
                                        plane.
                                        \end{minipg5} \\[0.3ex]
        \>\>    {\fov graphic}\> - \>   \begin{minipg2}
                                Indicator telling if the function
                                should draw the curve:
                                \end{minipg2}\\
                \>\>\>\>\>      0 -     \>Don't draw the curve.\\
                \>\>\>\>\>      1 -     \>\begin{minipg5}
                                        Draw the geometric curve. If this option
                                        is used see NOTE!
                                        \end{minipg5} \\[0.8ex]
\newpagetabs
        \>Input/Output Arguments:\\
        \>\>    {\fov intcurve}\> - \>  \begin{minipg2}
                                Pointer to the intersection curve.
                                As input, only
                                guide points (points in parameter space)
                                exist. These guide points
                                are used to guide the marching.
                                The routine adds
                                intersection curve and curve in the parameter
                                plane to the SISLIntcurve object, according to the value
                                of makecurv.
                                \end{minipg2}\\[0.8ex]
\\
        \>Output Arguments:\\
        \>\>    {\fov stat}     \> - \> Status messages\\
                \>\>\>\>\>      $= 3$ : \>      \begin{minipg5}
                                                Iteration stopped due to singular
                                                point or degenerate surface. A part of an
                                                intersection curve may have been
                                                traced out. If no curve is traced out
                                                the curve pointers in the SISLIntcurve
                                                object point to NULL.
                                                \end{minipg5} \\[0.3ex]
                \>\>\>\>\>      $= 0$   :\> ok\\
                \>\>\>\>\>      $< 0$   :\> error\\
\\
NOTE\\
\>      \begin{minipg6}
If the draw option is used the empty dummy functions s6move() and
s6line() are called.
Thus if the draw option is used, make sure
you have versions of s6move() and s6line() interfaced to your graphic package.
More about s6move() and s6line() on pages~\pageref{s6move} and~\pageref{s6line}.
\end{minipg6}\\
\\ %\newpagetabs
EXAMPLE OF USE\\
                \>      \{ \\
                \>\>    SISLSurf        \>      *{\fov surf};\\
                \>\>    double  \>      {\fov point}[3];\\
                \>\>    double  \>      {\fov normal}[3];\\
                \>\>    int     \>      {\fov dim} = 3;\\
                \>\>    double  \>      {\fov epsco};\\
                \>\>    double  \>      {\fov epsge};\\
                \>\>    double  \>      {\fov maxstep} = 0.0;\\
                \>\>    SISLIntcurve\>  *{\fov intcurve};\\
                \>\>    int     \>      {\fov makecurv};\\
                \>\>    int     \>      {\fov graphic};\\
                \>\>    int     \>      {\fov stat};\\
                \>\>    \ldots \\
        \>\>s1314(\begin{minipg4}
                {\fov surf}, {\fov point}, {\fov normal}, {\fov dim}, {\fov epsco}, {\fov epsge}, {\fov maxstep}, {\fov intcurve}, \linebreak
                {\fov makecurv}, {\fov graphic}, \&{\fov stat});
                        \end{minipg4}\\
                \>\>    \ldots \\
                \>      \}
\end{tabbing}

\pgsbreak
\subsection{March an intersection curve between a surface and a sphere.}
\funclabel{s1315}
\begin{minipg1}
  To march an intersection curve described by parameter pairs in an intersection
  curve object, a surface and a sphere.
  The guide points are expected to be found by s1852(), described on
  page \pageref{s1852}.
  The generated geometric curves are represented as B-spline curves.
\end{minipg1} \\ \\
SYNOPSIS\\
        \>void s1315(\begin{minipg3}
                {\fov surf}, {\fov centre}, {\fov radius}, {\fov dim}, {\fov epsco}, {\fov epsge}, {\fov maxstep}, {\fov intcurve},
                makecurv, graphic, stat)
                \end{minipg3}\\[0.3ex]
                \>\>    SISLSurf        \>      *{\fov surf};\\
                \>\>    double  \>      {\fov centre}[\,];\\
                \>\>    double  \>      {\fov radius};\\
                \>\>    int     \>      {\fov dim};\\
                \>\>    double  \>      {\fov epsco};\\
                \>\>    double  \>      {\fov epsge};\\
                \>\>    double  \>      {\fov maxstep};\\
                \>\>    SISLIntcurve\>  *{\fov intcurve};\\
                \>\>    int     \>      {\fov makecurv};\\
                \>\>    int     \>      {\fov graphic};\\
                \>\>    int     \>      *{\fov stat};\\
\\
ARGUMENTS\\
        \>Input Arguments:\\
        \>\>    {\fov surf}     \> - \> \begin{minipg2}
                                Pointer to the surface.
                                \end{minipg2}\\
        \>\>    {\fov centre}   \> - \> \begin{minipg2}
                                Center of the sphere.
                                \end{minipg2}\\
        \>\>    {\fov radius}   \> - \> \begin{minipg2}
                                Radius of sphere
                                \end{minipg2}\\
        \>\>    {\fov dim}      \> - \> \begin{minipg2}
                                Dimension of the space in which the sphere lies.
                                Should be 3.
                                \end{minipg2}\\[0.8ex]
        \>\>    {\fov epsco}    \> - \> \begin{minipg2}
                                Computational resolution (not used).
                                \end{minipg2}\\
        \>\>    {\fov epsge}    \> - \> \begin{minipg2}
                                Geometry resolution.
                                \end{minipg2}\\
        \>\>    {\fov maxstep}  \> - \> \begin{minipg2}
                                Maximum step length allowed.
                                If maxstep $\leq$ epsge maxstep is
                                neglected. maxstep = 0.0 is recommended.
                                \end{minipg2}\\[0.3ex]
        \>\>    {\fov makecurv}\> - \>  \begin{minipg2}
                                Indicator specifying if a geometric curve is to be made:
                                \end{minipg2}\\
                \>\>\>\>\>      0 -     \>Do not make curves at all.\\
                \>\>\>\>\>      1 -     \>Make only a geometric curve.\\
                \>\>\>\>\>      2 -     \>\begin{minipg5}
                                        Make geometric curve and curve in parameter
                                        plane.
                                        \end{minipg5} \\[0.3ex]
        \>\>    {\fov graphic}\> - \>   \begin{minipg2}
                                Indicator specifying if the function
                                should draw the curve:
                                \end{minipg2}\\
                \>\>\>\>\>      0 -     \>Don't draw the curve.\\
                \>\>\>\>\>      1 -     \>\begin{minipg5}
                                        Draw the geometric curve. If this option
                                        is used see NOTE!
                                        \end{minipg5} \\[0.8ex]
\newpagetabs
        \>Input/Output Arguments:\\
        \>\>    {\fov intcurve}\> - \>  \begin{minipg2}
                                Pointer to the intersection curve.
                                As input only
                                guide points (points in parameter space)
                                exist. These guide points
                                are used to guide the marching.
                                The routine adds
                                intersection curve and curve in the parameter
                                plane to the SISLIntcurve object according to the value
                                of makecurv.
                                \end{minipg2}\\[0.8ex]
\\
        \>Output Arguments:\\
        \>\>    {\fov stat}     \> - \> Status messages\\
                \>\>\>\>\>      $= 3$ : \>      \begin{minipg5}
                                                Iteration stopped due to singular
                                                point or degenerate surface. A part of an
                                                intersection curve may have been
                                                traced out. If no curve is traced out,
                                                the curve pointers in the SISLIntcurve
                                                object point to NULL.
                                                \end{minipg5} \\[0.3ex]
                \>\>\>\>\>              $= 0$   : ok\\
                \>\>\>\>\>              $< 0$   : error\\
\\
NOTE\\
\>      \begin{minipg6}
If the draw option is used the empty dummy functions s6move() and
s6line() are called.
Thus if the draw option is used, make sure
you have versions of s6move() and s6line() interfaced to your graphic package.
More about s6move() and s6line() on pages~\pageref{s6move} and~\pageref{s6line}.
\end{minipg6}\\
\\ %\newpagetabs
EXAMPLE OF USE\\
                \>      \{ \\
                \>\>    SISLSurf        \>      *{\fov surf};\\
                \>\>    double  \>      {\fov centre}[3];\\
                \>\>    double  \>      {\fov radius};\\
                \>\>    int     \>      {\fov dim} = 3;\\
                \>\>    double  \>      {\fov epsco};\\
                \>\>    double  \>      {\fov epsge};\\
                \>\>    double  \>      {\fov maxstep} = 0;\\
                \>\>    SISLIntcurve\>  *{\fov intcurve};\\
                \>\>    int     \>      {\fov makecurv};\\
                \>\>    int     \>      {\fov graphic};\\
                \>\>    int     \>      {\fov stat};\\
                \>\>    \ldots \\
        \>\>s1315(\begin{minipg4}
                {\fov surf}, {\fov centre}, {\fov radius}, {\fov dim}, {\fov epsco}, {\fov epsge}, {\fov maxstep}, {\fov intcurve},
                {\fov makecurv}, {\fov graphic}, \&{\fov stat});
                        \end{minipg4}\\
                \>\>    \ldots \\
                \>      \}
\end{tabbing}

\pgsbreak
\subsection{March an intersection curve between a surface and a \mbox{cylinder}.}
\funclabel{s1316}
\begin{minipg1}
  To march an intersection curve described by parameter pairs in an intersection
  curve object, a surface and a cylinder.
  The guide points are expected to be found by s1853() described on
  page \pageref{s1853}.
  The generated geometric curves are represented as B-spline curves.
\end{minipg1} \\ \\
SYNOPSIS\\
        \>void s1316(\begin{minipg3}
                {\fov surf}, {\fov point}, {\fov cyldir}, {\fov radius}, {\fov dim}, {\fov epsco}, {\fov epsge}, {\fov maxstep},
                {\fov intcurve}, {\fov makecurv}, {\fov graphic}, {\fov stat})
                \end{minipg3}\\[0.3ex]
                \>\>    SISLSurf        \>      *{\fov surf};\\
                \>\>    double  \>      {\fov point}[\,];\\
                \>\>    double  \>      {\fov cyldir}[\,];\\
                \>\>    double  \>      {\fov radius};\\
                \>\>    int     \>      {\fov dim};\\
                \>\>    double  \>      {\fov epsco};\\
                \>\>    double  \>      {\fov epsge};\\
                \>\>    double  \>      {\fov maxstep};\\
                \>\>    SISLIntcurve\>  *{\fov intcurve};\\
                \>\>    int     \>      {\fov makecurv};\\
                \>\>    int     \>      {\fov graphic};\\
                \>\>    int     \>      *{\fov stat};\\
\\
ARGUMENTS\\
        \>Input Arguments:\\
        \>\>    {\fov surf}\> - \>      \begin{minipg2}
                                Pointer to the surface.
                                \end{minipg2}\\
        \>\>    {\fov point}\> - \>     \begin{minipg2}
                                Point on the axis of the cylinder.
                                \end{minipg2}\\
        \>\>    {\fov cyldir}\> - \>    \begin{minipg2}
                                The direction vector of the axis of the cylinder.
                                \end{minipg2}\\
        \>\>    {\fov radius}\> - \>    \begin{minipg2}
                                Radius of the cylinder.
                                \end{minipg2} \\
        \>\>    {\fov dim}\> - \>       \begin{minipg2}
                                Dimension of the space in which the cylinder lies.
                                Should be 3.
                                \end{minipg2}\\[0.8ex]
        \>\>    {\fov epsco}\> - \>     \begin{minipg2}
                                Computational resolution (not used).
                                \end{minipg2}\\
        \>\>    {\fov epsge}\> - \>     \begin{minipg2}
                                Geometry resolution.
                                \end{minipg2}\\
        \>\>    {\fov maxstep}\> - \>   \begin{minipg2}
                                Maximum step length allowed.
                                If maxstep $\leq$ epsge maxstep is
                                neglected. maxstep = 0.0 is recommended.
                                \end{minipg2}\\[0.8ex]
        \>\>    {\fov makecurv}\> - \>  \begin{minipg2}
                                Indicator specifying if a geometric curve is to be made:
                                \end{minipg2}\\
                \>\>\>\>\>      0 -     \>Do not make curves at all.\\
                \>\>\>\>\>      1 -     \>Make only a geometric curve.\\
                \>\>\>\>\>      2 -     \>\begin{minipg5}
                                        Make geometric curve and curve in the parameter
                                        plane.
                                        \end{minipg5} \\[0.3ex]
        \>\>    {\fov graphic}\> - \>   \begin{minipg2}
                                Indicator specifying if the function
                                should draw the curve:
                                \end{minipg2}\\
                \>\>\>\>\>      0 -     \>Don't draw the curve.\\
                \>\>\>\>\>      1 -     \>\begin{minipg5}
                                        Draw the geometric curve. If this option
                                        is used see NOTE!
                                        \end{minipg5} \\[0.8ex]
\\
        \>Input/Output Arguments:\\
        \>\>    {\fov intcurve}\> - \>  \begin{minipg2}
                                Pointer to the intersection curve.
                                As input only
                                guide points (points in parameter space)
                                exist. These guide points
                                are used to guide the marching.
                                The routine adds
                                intersection curve and curve in the parameter
                                plane to the SISLIntcurve object according to the value
                                of makecurv.
                                \end{minipg2}\\[0.8ex]
\\
        \>Output Arguments:\\
        \>\>    {\fov stat}     \> - \> Status messages\\
                \>\>\>\>\>      $= 3$ : \>      \begin{minipg5}
                                                Iteration stopped due to singular
                                                point or degenerate surface. A part of
                                                an intersection curve may have been
                                                traced out. If no curve is traced out,
                                                the curve pointers in the SISLIntcurve
                                                object point to NULL.
                                                \end{minipg5} \\[0.3ex]
                \>\>\>\>\>      $= 0$   :\> ok\\
                \>\>\>\>\>      $< 0$   :\> error\\
\\
NOTE\\
\>      \begin{minipg6}
If the draw option is used the empty dummy functions s6move() and
s6line() are called.
Thus if the draw option is used, make sure
you have versions of s6move() and s6line() interfaced to your graphic package.
More about s6move() and s6line() on pages~\pageref{s6move} and~\pageref{s6line}.
\end{minipg6}\\
\\ %\newpagetabs
EXAMPLE OF USE\\
                \>      \{ \\
                \>\>    SISLSurf        \>      *{\fov surf};\\
                \>\>    double  \>      {\fov point}[3];\\
                \>\>    double  \>      {\fov cyldir}[3];\\
                \>\>    double  \>      {\fov radius};\\
                \>\>    int     \>      {\fov dim} = 3;\\
                \>\>    double  \>      {\fov epsco};\\
                \>\>    double  \>      {\fov epsge};\\
                \>\>    double  \>      {\fov maxstep} = 0.0;\\
                \>\>    SISLIntcurve\>  *{\fov intcurve};\\
                \>\>    int     \>      {\fov makecurv};\\
                \>\>    int     \>      {\fov graphic};\\
                \>\>    int     \>      {\fov stat} = 0;\\
                \>\>    \ldots \\
        \>\>s1316(\begin{minipg4}
                {\fov surf}, {\fov point}, {\fov cyldir}, {\fov radius}, {\fov dim}, {\fov epsco}, {\fov epsge}, {\fov maxstep},
                {\fov intcurve}, {\fov makecurv}, {\fov graphic}, \&{\fov stat});
                        \end{minipg4}\\
                \>\>    \ldots \\
                \>      \}
\end{tabbing}

\pgsbreak
\subsection{March an intersection curve between a surface and a cone.}
\funclabel{s1317}
\begin{minipg1}
  To march an intersection curve described by parameter pairs in an intersection
  curve object, a surface and a cone.
  The guide points are expected to be found by s1854() described on
  page \pageref{s1854}.
  The generated geometric curves are represented as B-spline curves.
\end{minipg1} \\ \\
SYNOPSIS\\
        \>void s1317(\begin{minipg3}
        {\fov surf}, {\fov toppt}, {\fov axispt}, {\fov conept},
        {\fov dim}, {\fov epsco}, {\fov epsge}, {\fov maxstep},
        {\fov intcurve}, {\fov makecurv}, {\fov graphic}, {\fov stat})
                \end{minipg3}\\[0.3ex]

                \>\>    SISLSurf        \>      *{\fov surf};\\
                \>\>    double  \>      {\fov toppt}[\,];\\
                \>\>    double  \>      {\fov axispt}[\,];\\
                \>\>    double  \>      {\fov conept}[\,];\\
                \>\>    int     \>      {\fov dim};\\
                \>\>    double  \>      {\fov epsco};\\
                \>\>    double  \>      {\fov epsge};\\
                \>\>    double  \>      {\fov maxstep};\\
                \>\>    SISLIntcurve\>  *{\fov intcurve};\\
                \>\>    int     \>      {\fov makecurv};\\
                \>\>    int     \>      {\fov graphic};\\
                \>\>    int     \>      *{\fov stat};\\
\\
ARGUMENTS\\
        \>Input Arguments:\\
        \>\>    {\fov surf}\> - \>      \begin{minipg2}
                                Pointer to the surface.
                                \end{minipg2}\\
        \>\>    {\fov toppt}\> - \>     \begin{minipg2}
                                The top point of the cone.
                                \end{minipg2}\\
        \>\>    {\fov axispt}\> - \>    \begin{minipg2}
                                Point on the axis of the cone; axispt must be different from toppt.
                                \end{minipg2}\\[0.3ex]
        \>\>    {\fov conept}\> - \>    \begin{minipg2}
                                A point on the cone surface that is not the top
                                point.
                                \end{minipg2} \\[0.3ex]
        \>\>    {\fov dim}\> - \>       \begin{minipg2}
                                Dimension of the space in which the cone lies.
                                Should be 3.
                                \end{minipg2}\\[0.8ex]
        \>\>    {\fov epsco}\> - \>     \begin{minipg2}
                                Computational resolution (not used).
                                \end{minipg2}\\
        \>\>    {\fov epsge}\> - \>     \begin{minipg2}
                                Geometry resolution.
                                \end{minipg2}\\
        \>\>    {\fov maxstep}\> - \>   \begin{minipg2}
                                Maximum step length allowed. If maxstep $\leq$ epsge,
                                maxstep is neglected. maxstep = 0.0 is recommended.
                                \end{minipg2}\\[0.8ex]
        \>\>    {\fov makecurv}\> - \>          \begin{minipg2}
                                Indicator specifying if a geometric curve is to be made:
                                \end{minipg2}\\
                \>\>\>\>\>      0 -     \>Do not make curves at all.\\
                \>\>\>\>\>      1 -     \>Make only a geometric curve.\\
                \>\>\>\>\>      2 -     \>\begin{minipg5}
                                        Make geometric curve and curve in the parameter
                                        plane
                                        \end{minipg5} \\[0.3ex]
        \>\>    {\fov graphic}\> - \>   \begin{minipg2}
                                Indicator specifying if the function
                                should draw the curve:
                                \end{minipg2}\\
                \>\>\>\>\>      0 -     \>Don't draw the curve.\\
                \>\>\>\>\>      1 -     \>\begin{minipg5}
                                        Draw the geometric curve. If this option
                                        is used see NOTE!
                                        \end{minipg5} \\[0.8ex]
        \>Input/Output Arguments:\\
        \>\>    {\fov intcurve}\> - \>  \begin{minipg2}
                                Pointer to the intersection curve.
                                As input only
                                guide points (points in parameter space)
                                exist. These guide points
                                are used for guiding the marching.
                                The routine adds the
                                intersection curve and curve in the parameter
                                plane to the SISLIntcurve object according to the value
                                of makecurv.
                                \end{minipg2}\\[0.8ex]
        \>Output Arguments:\\
        \>\>    {\fov stat}     \> - \> Status messages\\
                \>\>\>\>\>      $= 3$ : \>      \begin{minipg5}
                                                Iteration stopped due to singular
                                                point or degenerate surface. A part of
                                                an intersection curve may have been
                                                traced out. If no curve is traced out,
                                                the curve pointers in the SISLIntcurve
                                                object point to NULL.
                                                \end{minipg5} \\[0.3ex]
                \>\>\>\>\>      $= 0$   :\> ok\\
                \>\>\>\>\>      $< 0$   :\> error\\
\\
NOTE\\
\>      \begin{minipg6}
If the draw option is used the empty dummy functions s6move() and
s6line() are called.
Thus if the draw option is used, make sure
you have versions of s6move() and s6line() interfaced to your graphic package.
More about s6move() and s6line() on pages~\pageref{s6move} and~\pageref{s6line}.
\end{minipg6}\\
\\ %\newpagetabs
EXAMPLE OF USE\\
                \>      \{ \\
                \>\>    SISLSurf        \>      *{\fov surf};\\
                \>\>    double  \>      {\fov toppt}[3];\\
                \>\>    double  \>      {\fov axispt}[3];\\
                \>\>    double  \>      {\fov conept}[3];\\
                \>\>    int     \>      {\fov dim} = 3;\\
                \>\>    double  \>      {\fov epsco};\\
                \>\>    double  \>      {\fov epsge};\\
                \>\>    double  \>      {\fov maxstep} = 0.0;\\
                \>\>    SISLIntcurve\>  *{\fov intcurve};\\
                \>\>    int     \>      {\fov makecurv};\\
                \>\>    int     \>      {\fov graphic};\\
                \>\>    int     \>      {\fov stat} = 0;\\
                \>\>    \ldots \\
        \>\>s1317(\begin{minipg4}
                {\fov surf}, {\fov toppt}, {\fov axispt}, {\fov conept}, {\fov dim}, {\fov epsco}, {\fov epsge}, {\fov maxstep},
                {\fov intcurve}, {\fov makecurv}, {\fov graphic}, \&{\fov stat});
                        \end{minipg4}\\
                \>\>    \ldots \\
                \>      \}
\end{tabbing}

\pgsbreak
\subsection{March an intersection curve between a surface and an
\mbox{elliptic} cone.}
\funclabel{s1501}
\begin{minipg1}
  To march an intersection curve described by parameter pairs in an intersection
  curve object, a surface and an elliptic cone.
  The guide points are expected to be found by s1503() described on
  page \pageref{s1503}.
  The generated geometric curves are represented as B-spline curves.
\end{minipg1} \\ \\
SYNOPSIS\\
        \>void s1501(\begin{minipg3}
        {\fov surf}, {\fov basept}, {\fov normdir}, {\fov ellipaxis},
        {\fov alpha}, {\fov ratio}, {\fov dim}, {\fov epsco}, {\fov epsge}, {\fov maxstep},
        {\fov intcurve}, {\fov makecurv}, {\fov graphic}, {\fov stat})
                \end{minipg3}\\[0.3ex]

                \>\>    SISLSurf        \>      *{\fov surf};\\
                \>\>    double  \>      {\fov basept}[\,];\\
                \>\>    double  \>      {\fov normdir}[\,];\\
                \>\>    double  \>      {\fov ellipaxis}[\,];\\
                \>\>    double  \>      {\fov alpha};\\
                \>\>    double  \>      {\fov ratio};\\
                \>\>    int     \>      {\fov dim};\\
                \>\>    double  \>      {\fov epsco};\\
                \>\>    double  \>      {\fov epsge};\\
                \>\>    double  \>      {\fov maxstep};\\
                \>\>    SISLIntcurve\>  *{\fov intcurve};\\
                \>\>    int     \>      {\fov makecurv};\\
                \>\>    int     \>      {\fov graphic};\\
                \>\>    int     \>      *{\fov stat};\\
\\
ARGUMENTS\\
        \>Input Arguments:\\
        \>\>    {\fov surf}\> - \>      \begin{minipg2}
                                Pointer to the surface.
                                \end{minipg2}\\
        \>\>    {\fov basept}\> - \>    \begin{minipg2}
                                Base point of the cone, centre of elliptic base.
                                \end{minipg2}\\
        \>\>    {\fov normdir}\> - \>   \begin{minipg2}
                                Direction of the cone axis, normal to the elliptic base. The default is pointing from the base point to the top point.
                                \end{minipg2}\\[0.3ex]
        \>\>    {\fov ellipaxis}\> - \> \begin{minipg2}
                                One of the axes of the ellipse (major or
                                minor).
                                The other axis will be calculated as
                                $normdir\times ellipaxis$ scaled with
                                {\fov ratio}.
                                \end{minipg2}\\[0.8ex]
        \>\>    {\fov alpha}\> - \>     \begin{minipg2}
                                The opening angle in radians of the cone at the ellipaxis.
                                \end{minipg2}\\[0.3ex]
        \>\>    {\fov ratio}\> - \>     \begin{minipg2}
                                The ratio of the major and minor
                                axes = ellipaxis/otheraxis.
                                \end{minipg2}\\[0.3ex]
        \>\>    {\fov dim}\> - \>       \begin{minipg2}
                                Dimension of the space in which the cone lies.
                                Should be 3.
                                \end{minipg2}\\[0.8ex]
        \>\>    {\fov epsco}\> - \>     \begin{minipg2}
                                Computational resolution (not used).
                                \end{minipg2}\\
        \>\>    {\fov epsge}\> - \>     \begin{minipg2}
                                Geometry resolution.
                                \end{minipg2}\\
        \>\>    {\fov maxstep}\> - \>   \begin{minipg2}
                                Maximum step length allowed. If maxstep $\leq$ epsge,
                                maxstep is neglected. maxstep = 0.0 is recommended.
                                \end{minipg2}\\[0.3ex]
        \>\>    {\fov makecurv}\> - \>          \begin{minipg2}
                                Indicator specifying if a geometric curve is to be made:
                                \end{minipg2}\\
                \>\>\>\>\>      0 -     \>Do not make curves at all.\\
                \>\>\>\>\>      1 -     \>Make only a geometric curve.\\
                \>\>\>\>\>      2 -     \>\begin{minipg5}
                                        Make geometric curve and curve in the parameter
                                        plane
                                        \end{minipg5} \\[0.3ex]
        \>\>    {\fov graphic}\> - \>   \begin{minipg2}
                                Indicator specifying if the function
                                should draw the curve:
                                \end{minipg2}\\
                \>\>\>\>\>      0 -     \>Don't draw the curve.\\
                \>\>\>\>\>      1 -     \>\begin{minipg5}
                                        Draw the geometric curve. If this option
                                        is used see NOTE!
                                        \end{minipg5} \\[0.8ex]
        \>Input/Output Arguments:\\
        \>\>    {\fov intcurve}\> - \>  \begin{minipg2}
                                Pointer to the intersection curve.
                                As input only
                                guide points (points in parameter space)
                                exist. These guide points
                                are used for guiding the marching.
                                The routine adds the
                                intersection curve and curve in the parameter
                                plane to the SISLIntcurve object according to the value
                                of makecurv.
                                \end{minipg2}\\[0.8ex]
        \>Output Arguments:\\
        \>\>    {\fov stat}     \> - \> Status messages\\
                \>\>\>\>\>      $= 3$ : \>      \begin{minipg5}
                                                Iteration stopped due to singular
                                                point or degenerate surface. A part of
                                                an intersection curve may have been
                                                traced out. If no curve is traced out,
                                                the curve pointers in the SISLIntcurve
                                                object point to NULL.
                                                \end{minipg5} \\[0.3ex]
                \>\>\>\>\>      $= 0$   :\> ok\\
                \>\>\>\>\>      $< 0$   :\> error\\
\\
NOTE\\
\>      \begin{minipg6}
If the draw option is used the empty dummy functions s6move() and
s6line() are called.
Thus if the draw option is used, make sure
you have versions of s6move() and s6line() interfaced to your graphic package.
More about s6move() and s6line() on pages~\pageref{s6move} and~\pageref{s6line}.
\end{minipg6}\\
\newpagetabs
EXAMPLE OF USE\\
                \>      \{ \\
                \>\>    SISLSurf        \>      *{\fov surf};\\
                \>\>    double  \>      {\fov basept}[3];\\
                \>\>    double  \>      {\fov normdir}[3];\\
                \>\>    double  \>      {\fov ellipaxis}[3];\\
                \>\>    double  \>      {\fov alpha};\\
                \>\>    double  \>      {\fov ratio};\\
                \>\>    int     \>      {\fov dim} = 3;\\
                \>\>    double  \>      {\fov epsco};\\
                \>\>    double  \>      {\fov epsge};\\
                \>\>    double  \>      {\fov maxstep} = 0.0;\\
                \>\>    SISLIntcurve\>  *{\fov intcurve};\\
                \>\>    int     \>      {\fov makecurv};\\
                \>\>    int     \>      {\fov graphic};\\
                \>\>    int     \>      {\fov stat} = 0;\\
                \>\>    \ldots \\
        \>\>s1501(\begin{minipg4}
                {\fov surf}, {\fov basept}, {\fov normdir}, {\fov ellipaxis}, {\fov alpha}, {\fov ratio}, {\fov dim}, {\fov epsco}, {\fov epsge}, {\fov maxstep},
                {\fov intcurve}, {\fov makecurv}, {\fov graphic}, \&{\fov stat});
                        \end{minipg4}\\
                \>\>    \ldots \\
                \>      \}
\end{tabbing}

\pgsbreak
\subsection{March an intersection curve between a surface and a torus.}
\funclabel{s1318}
\begin{minipg1}
  To march an intersection curve described by parameter pairs in an intersection
  curve object, a surface and a torus.
  The guide points are expected to be found by s1369(), described on
  page \pageref{s1369}.
  The generated geometric curves are represented as B-spline curves.
\end{minipg1} \\ \\
SYNOPSIS\\
        \>void s1318(\begin{minipg3}
                        {\fov surf}, {\fov centre}, {\fov normal}, {\fov cendist}, {\fov radius}, {\fov dim}, {\fov epsco}, {\fov epsge},
                        {\fov maxstep}, {\fov intcurve}, {\fov makecurv}, {\fov graphic}, {\fov stat})
                \end{minipg3}\\[0.3ex]

                \>\>    SISLSurf        \>      *{\fov surf};\\
                \>\>    double  \>      {\fov centre}[\,];\\
                \>\>    double  \>      {\fov normal}[\,];\\
                \>\>    double  \>      {\fov cendist};\\
                \>\>    double  \>      {\fov radius};\\
                \>\>    int     \>      {\fov dim};\\
                \>\>    double  \>      {\fov epsco};\\
                \>\>    double  \>      {\fov epsge};\\
                \>\>    double  \>      {\fov maxstep};\\
                \>\>    SISLIntcurve\>  *{\fov intcurve};\\
                \>\>    int     \>      {\fov makecurv};\\
                \>\>    int     \>      {\fov graphic};\\
                \>\>    int     \>      *{\fov stat};\\
\\
ARGUMENTS\\
        \>Input Arguments:\\
        \>\>    {\fov surf}\> - \>      \begin{minipg2}
                                Pointer to the surface.
                                \end{minipg2}\\
        \>\>    {\fov centre}\> - \>    \begin{minipg2}
                                The centre of the torus (lying in the symmetry
                                plane)
                                \end{minipg2}\\
        \>\>    {\fov normal}\> - \>    \begin{minipg2}
                                Normal to the symmetry plane.
                                \end{minipg2}\\
        \>\>    {\fov cendist}\> - \>   \begin{minipg2}
                                Distance from centre to the centre circle of torus.
                                \end{minipg2} \\
        \>\>    {\fov radius}\> - \>    \begin{minipg2}
                                The radius of the torus surface.
                                \end{minipg2}\\
        \>\>    {\fov dim}\> - \>       \begin{minipg2}
                                Dimension of the space in which the torus lies.
                                Should be 3.
                                \end{minipg2}\\[0.8ex]
        \>\>    {\fov epsco}\> - \>     \begin{minipg2}
                                Computational resolution (not used).
                                \end{minipg2}\\
        \>\>    {\fov epsge}\> - \>     \begin{minipg2}
                                Geometry resolution.
                                \end{minipg2}\\
        \>\>    {\fov maxstep}\> - \>   \begin{minipg2}
                                Maximum step length allowed.
                                If maxstep $\leq$ epsge maxstep is
                                neglected. maxstep = 0.0 is recommended.
                                \end{minipg2}\\[0.8ex]
        \>\>    {\fov makecurv}\> - \>  \begin{minipg2}
                                Indicator specifying if a geometric curve is to be made:
                                \end{minipg2}\\
                \>\>\>\>\>      0 -     \>Do not make curves at all.\\
                \>\>\>\>\>      1 -     \>Make only a geometric curve.\\
                \>\>\>\>\>      2 -     \>\begin{minipg5}
                                        Make geometric curve and curve in the parameter
                                        plane
                                        \end{minipg5} \\[0.3ex]
\newpagetabs
        \>\>    {\fov graphic}\> - \>   \begin{minipg2}
                                Indicator specifying if the function
                                should draw the curve:
                                \end{minipg2}\\
                \>\>\>\>\>      0 -     \>Don't draw the curve.\\
                \>\>\>\>\>      1 -     \>\begin{minipg5}
                                        Draw the geometric curve. If this option
                                        is used see NOTE!
                                        \end{minipg5} \\[0.8ex]
\\
        \>Input/Output Arguments:\\
        \>\>    {\fov intcurve}\> - \>  \begin{minipg2}
                                Pointer to the intersection curve.
                                As input only
                                guide points (points in parameter space)
                                exist. These guide points
                                are used for guiding the marching.
                                The routine adds the
                                intersection curve and curve in the parameter
                                plane to the SISLIntcurve object according to the value
                                of makecurv.
                                \end{minipg2}\\[0.8ex]
\\
        \>Output Arguments:\\
        \>\>    {\fov stat}     \> - \> Status messages\\
                \>\>\>\>\>      $= 3$ : \>      \begin{minipg5}
                                                Iteration stopped due to singular
                                                point or degenerate surface. A part of
                                                an intersection curve may have been
                                                traced out. If no curve is traced out
                                                the curve pointers in the SISLIntcurve
                                                object point to NULL.
                                                \end{minipg5} \\[0.3ex]
                \>\>\>\>\>      $= 0$   :\> ok\\
                \>\>\>\>\>      $< 0$   :\> error\\
\\
NOTE\\
\>      \begin{minipg6}
If the draw option is used the empty dummy functions s6move() and
s6line() are called.
Thus if the draw option is used, make sure
you have versions of s6move() and s6line() interfaced to your graphic package.
More about s6move() and s6line() on pages~\pageref{s6move}
and~\pageref{s6line}.
\end{minipg6}\\
\newpagetabs
EXAMPLE OF USE\\
                \>      \{ \\
                \>\>    SISLSurf        \>      *{\fov surf};\\
                \>\>    double  \>      {\fov centre}[3];\\
                \>\>    double  \>      {\fov normal}[3];\\
                \>\>    double  \>      {\fov cendist};\\
                \>\>    double  \>      {\fov radius};\\
                \>\>    int     \>      {\fov dim} = 3;\\
                \>\>    double  \>      {\fov epsco};\\
                \>\>    double  \>      {\fov epsge};\\
                \>\>    double  \>      {\fov maxstep} = 0.0;\\
                \>\>    SISLIntcurve\>  *{\fov intcurve};\\
                \>\>    int     \>      {\fov makecurv};\\
                \>\>    int     \>      {\fov graphic};\\
                \>\>    int     \>      {\fov stat} = 0;\\
                \>\>    \ldots \\
        \>\>s1318(\begin{minipg4}
                {\fov surf}, {\fov centre}, {\fov normal}, {\fov cendist}, {\fov radius}, {\fov dim}, {\fov epsco}, {\fov epsge},
                {\fov maxstep}, {\fov intcurve}, {\fov makecurv}, {\fov graphic}, \&{\fov stat});
                        \end{minipg4}\\
                \>\>    \ldots \\
                \>      \}
\end{tabbing}

\pgsbreak
\subsection{March an intersection curve between two surfaces.}
\funclabel{s1310}
\begin{minipg1}
  To march an intersection curve between two surfaces.
  The intersection curve is described by guide parameter pairs stored in
  an intersection curve object.
  The guide points are expected to be found by s1859() described on
  page \pageref{s1859}.
  The generated geometric curves are represented as B-spline curves.
\end{minipg1} \\ \\
SYNOPSIS\\
        \>void s1310(\begin{minipg3}
                        {\fov surf1}, {\fov surf2}, {\fov intcurve}, {\fov epsge}, {\fov maxstep}, {\fov makecurv}, {\fov graphic}, {\fov stat})
                \end{minipg3}\\[0.3ex]

                \>\>    SISLSurf        \>      *{\fov surf1};\\
                \>\>    SISLSurf        \>      *{\fov surf2};\\
                \>\>    SISLIntcurve\>  *{\fov intcurve};\\
                \>\>    double  \>      {\fov epsge};\\
                \>\>    double  \>      {\fov maxstep};\\
                \>\>    int     \>      {\fov makecurv};\\
                \>\>    int     \>      {\fov graphic};\\
                \>\>    int     \>      *{\fov stat};\\
\\
ARGUMENTS\\
        \>Input Arguments:\\
        \>\>    {\fov surf1}\> - \>     \begin{minipg2}
                                Pointer to the first surface.
                                \end{minipg2}\\
        \>\>    {\fov surf2}\> - \>     \begin{minipg2}
                                Pointer to the second surface.
                                \end{minipg2}\\
        \>\>    {\fov epsge}\> - \>     \begin{minipg2}
                                Geometry resolution.
                                \end{minipg2} \\
        \>\>    {\fov maxstep}\> - \>   \begin{minipg2}
                                Maximum step length. If maxstep$\leq$0, maxstep is ignored.
                                maxstep = 0.0 is recommended.
                                \end{minipg2}\\[0.8ex]
        \>\>    {\fov makecurv}\> - \>  \begin{minipg2}
                                Indicator specifying if a geometric curve is to be made:
                                \end{minipg2}\\
                \>\>\>\>\>      0 -     \>Do not make curves at all\\
                \>\>\>\>\>      1 -     \>Make only a geometric curve.\\
                \>\>\>\>\>      2 -     \>\begin{minipg5}
                                        Make geometric curve and curves in the parameter
                                        planes
                                        \end{minipg5} \\[0.3ex]
        \>\>    {\fov graphic}\> - \>   \begin{minipg2}
                                Indicator specifying if the function
                                should draw the geometric curve:
                                \end{minipg2}\\
                \>\>\>\>\>      0 -     \>Don't draw the curve\\
                \>\>\>\>\>      1 -     \>\begin{minipg5}
                                        Draw the geometric curve. If this option
                                        is used see NOTE!
                                        \end{minipg5} \\[0.8ex]
\\ %\newpagetabs
        \>Input/Output Arguments:\\
        \>\>    {\fov intcurve}\> - \>  \begin{minipg2}
                                Pointer to the intersection curve.
                                As input only
                                guide points (points in parameter space)
                                exist. These guide points
                                are used for guiding the marching.
                                The routine adds
                                intersection curve and curves in the parameter
                                planes to the SISLIntcurve object, according to the value
                                of makecurv.
                                \end{minipg2}\\
\newpagetabs
        \>Output Arguments:\\
        \>\>    {\fov stat}     \> - \> Status messages\\
                \>\>\>\>\>              $= 3$ : \>      \begin{minipg5}
                                                        Iteration stopped due to singular
                                                        point or degenerate surface. A part of an
                                                        intersection curve may have been
                                                        traced out. If no curve is traced out,
                                                        the curve pointers in the SISLIntcurve
                                                        object point to NULL.
                                                        \end{minipg5} \\[0.3ex]
                \>\>\>\>\>              $= 0$ : \>       ok\\
                \>\>\>\>\>              $< 0$ : \>       error\\
\\
NOTE\\
\>      \begin{minipg6}
If the draw option is used the empty dummy functions s6move() and
s6line() are called.
Thus if the draw option is used, make sure
you have versions of s6move() and s6line() interfaced to your graphic package.
More about s6move() and s6line() on pages~\pageref{s6move}
and~\pageref{s6line}.
\end{minipg6}\\ \\
EXAMPLE OF USE\\
                \>      \{ \\
                \>\>    SISLSurf        \>      *{\fov surf1};\\
                \>\>    SISLSurf        \>      *{\fov surf2};\\
                \>\>    SISLIntcurve \> *{\fov intcurve};\\
                \>\>    double  \>      {\fov epsge};\\
                \>\>    double  \>      {\fov maxstep};\\
                \>\>    int     \>      {\fov makecurv};\\
                \>\>    int     \>      {\fov graphic};\\
                \>\>    int     \>      {\fov stat} = 0;\\
                \>\>    \ldots \\
        \>\>s1310(\begin{minipg4}
                {\fov surf1}, {\fov surf2}, {\fov intcurve}, {\fov epsge}, {\fov maxstep}, {\fov makecurv}, {\fov graphic}, \&{\fov stat});
                        \end{minipg4}\\
                \>\>    \ldots \\
                \>      \}
\end{tabbing}

\pgsbreak
\section{Marching of Silhouettes}
\subsection{\sloppy March a silhouette curve of a surface, using parallel \mbox{projection}.}
\funclabel{s1319}
\begin{minipg1}
  To march the silhouette curve described by an intersection curve object, a
  surface and a view direction (i.e.\ parallel projection).
  The guide points are expected to be found by s1860(), described on
  page \pageref{s1860}.
  The generated geometric curves are represented as B-spline curves.
\end{minipg1} \\ \\
NOTE\\
\>     \begin{minipg6}
The silhouette curves are defined as curves on the surface where the inner product of the surface normal and the direction vector of the viewing is 0. This definition will include surface points where the normal is zero.
\end{minipg6} \\ \\
SYNOPSIS\\
        \>void s1319(\begin{minipg3}
                        {\fov surf}, {\fov viewdir}, {\fov dim}, {\fov epsco}, {\fov epsge}, {\fov maxstep}, {\fov intcurve}, {\fov makecurv},
                        {\fov graphic}, {\fov stat})
                \end{minipg3}\\[0.3ex]
                \>\>    SISLSurf        \>      *{\fov surf};\\
                \>\>    double  \>      {\fov viewdir}[\,];\\
                \>\>    int     \>      {\fov dim};\\
                \>\>    double  \>      {\fov epsco};\\
                \>\>    double  \>      {\fov epsge};\\
                \>\>    double  \>      {\fov maxstep};\\
                \>\>    SISLIntcurve\>  *{\fov intcurve};\\
                \>\>    int     \>      {\fov makecurv};\\
                \>\>    int     \>      {\fov graphic};\\
                \>\>    int     \>      *{\fov stat};\\
\\
ARGUMENTS\\
        \>Input Arguments:\\
        \>\>    {\fov surf}\> - \>              \begin{minipg2}
                                Pointer to the surface.
                                \end{minipg2}\\
        \>\>    {\fov viewdir}\> - \>   \begin{minipg2}
                                View direction.
                                \end{minipg2}\\
        \>\>    {\fov dim}\> - \>       \begin{minipg2}
                                Dimension of the space in which vector describing the view
                                direction lies. Should be 3.
                                \end{minipg2}\\[0.8ex]
        \>\>    {\fov epsco}\> - \>     \begin{minipg2}
                                Computational resolution (not used).
                                \end{minipg2}\\
        \>\>    {\fov epsge}\> - \>     \begin{minipg2}
                                Geometry resolution.
                                \end{minipg2}\\
        \>\>    {\fov maxstep}\> - \>   \begin{minipg2}
                                Maximum step length allowed.
                                If maxstep $\leq$ epsge maxstep is
                                neglected. maxstep = 0.0 is recommended.
                                \end{minipg2}\\
\newpagetabs
        \>\>    {\fov makecurv}\> - \>  \begin{minipg2}
                                Indicator specifying if a geometric curve is to be made:
                                \end{minipg2}\\
                \>\>\>\>\>      0 -     \>Do not make curves at all.\\
                \>\>\>\>\>      1 -     \>Make only a geometric curve.\\
                \>\>\>\>\>      2 -     \>\begin{minipg5}
                                        Make geometric curve and curve in the parameter
                                        plane.
                                        \end{minipg5} \\[0.3ex]
        \>\>    {\fov graphic}\> - \>   \begin{minipg2}
                                Indicator specifying if the function
                                should draw the geometric curve:
                                \end{minipg2}\\
                \>\>\>\>\>      0 -     \>Don't draw the curve.\\
                \>\>\>\>\>      1 -     \>\begin{minipg5}
                                        Draw the geometric curve. If this option
                                        is used see NOTE!
                                        \end{minipg5} \\[0.8ex]
\\
        \>Input/Output Arguments:\\
        \>\>    {\fov intcurve}\> - \>  \begin{minipg2}
                                Pointer to the intersection curve.
                                As input, only
                                guide points (points in parameter space)
                                exist. These guide points
                                are used for guiding the marching.
                                The routine adds
                                intersection curve and curve in the parameter
                                plane to the SISLIntcurve object according to the value
                                of makecurv.
                                \end{minipg2}\\[0.8ex]
\\
        \>Output Arguments:\\
        \>\>    {\fov stat}     \> - \> Status messages\\
                \>\>\>\>\>      $= 3$ : \>      \begin{minipg5}
                                                Iteration stopped due to singular
                                                point or degenerate surface. A part of
                                                an intersection curve may have been
                                                traced out. If no curve is traced out
                                                the curve pointers in the SISLIntcurve
                                                object point to NULL.
                                                \end{minipg5} \\[0.3ex]
                \>\>\>\>\>      $= 0$   :\> ok\\
                \>\>\>\>\>      $< 0$   :\> error\\
\\
NOTE\\
\>      \begin{minipg6}
If the draw option is used the empty dummy functions s6move() and
s6line() are called.
Thus if the draw option is used, make sure
you have versions of s6move() and s6line() interfaced to your graphic package.
More about s6move() and s6line() on pages~\pageref{s6move}
and~\pageref{s6line}.
\end{minipg6}\\
\newpagetabs
EXAMPLE OF USE\\
                \>      \{ \\
                \>\>    SISLSurf        \>      *{\fov surf};\\
                \>\>    double  \>      {\fov viewdir}[3];\\
                \>\>    int     \>      {\fov dim} = 3;\\
                \>\>    double  \>      {\fov epsco};\\
                \>\>    double  \>      {\fov epsge};\\
                \>\>    double  \>      {\fov maxstep} = 0.0;\\
                \>\>    SISLIntcurve\>  *{\fov intcurve};\\
                \>\>    int     \>      {\fov makecurv};\\
                \>\>    int     \>      {\fov graphic};\\
                \>\>    int     \>      {\fov stat} = 0;\\
                \>\>    \ldots \\
        \>\>s1319(\begin{minipg4}
                {\fov surf}, {\fov viewdir}, {\fov dim}, {\fov epsco}, {\fov epsge}, {\fov maxstep}, {\fov intcurve}, {\fov makecurv},
                graphic, \&stat);
                        \end{minipg4}\\
                \>\>    \ldots \\
                \>      \}
\end{tabbing}

\pgsbreak
\subsection{\sloppy March a silhouette curve of a surface, using
perspective \mbox{projection}.}
\funclabel{s1514}
\begin{minipg1}
  To march the perspective silhouette curve described by an intersection
  curve object, a surface and an eye point.
  The generated geometric curves are represented as B-spline curves.
\end{minipg1} \\ \\
SYNOPSIS\\
        \>void s1514(\begin{minipg3}
          {\fov ps1},  {\fov eyepoint},  {\fov idim},  {\fov aepsco},  {\fov aepsge},  {\fov amax},  {\fov pintcr},  {\fov icur},  {\fov igraph},  {\fov jstat})
        \end{minipg3}\\[0.3ex]
        \>\>    SISLSurf \> *{\fov ps1};\\
        \>\>    double   \> {\fov eyepoint}[\,]\\
        \>\>    int      \> {\fov idim};\\
        \>\>    double   \> {\fov aepsco};\\
        \>\>    double   \> {\fov aepsge};\\
        \>\>    double   \> {\fov amax};\\
        \>\>    SISLIntcurve \> *{\fov pintcr};\\
        \>\>    int      \> {\fov icur};\\
        \>\>    int      \> {\fov igraph};\\
        \>\>    int      \> *{\fov jstat};\\
\\
ARGUMENTS\\
        \>Input Arguments:\\
        \>\>    {\fov ps1}\> - \>  \begin{minipg2}
                     Pointer to surface.
                               \end{minipg2}\\
        \>\>    {\fov eyepoint}\> - \>  \begin{minipg2}
                     Eye point for perspective view
                               \end{minipg2}\\
        \>\>    {\fov idim}\> - \>  \begin{minipg2}
                     Dimension of the space in which the {\fov eyepoint}
                       lies.
                               \end{minipg2}\\[0.8ex]
        \>\>    {\fov aepsco}\> - \>  \begin{minipg2}
                     Computational resolution (not used).
                               \end{minipg2}\\
        \>\>    {\fov aepsge}\> - \>  \begin{minipg2}
                     Geometry resolution.
                               \end{minipg2}\\
        \>\>    {\fov amax}\> - \>  \begin{minipg2}
                     Maximal allowed step length.\\ If $amax\leq aepsge$
                       {\fov amax} is neglected.
                               \end{minipg2}\\[0.8ex]
        \>\>    {\fov icur}\> - \>  \begin{minipg2}
                    Indicator telling if a 3D curve is to be made.
                               \end{minipg2}\\
                    \>\>\>\>\> $= 0$ \> : Don't make 3D curve.\\
                    \>\>\>\>\> $= 1$ \> : Make 3D curve.\\
                    \>\>\>\>\> $= 2$ \> : \begin{minipg5}
                                            Make 3D curve and curves in
                                            the parameter plane.
                                          \end{minipg5}\\[0.8ex]
        \>\>    {\fov igraph}\> - \>  \begin{minipg2}
                     Indicator telling if the curve is to be output
                       through function calls:\\
                               \end{minipg2}\\
                    \>\>\>\>\> $= 0$ \> : \begin{minipg5}
                                            Don't output curve through
                                            function call.
                                          \end{minipg5}\\[0.3ex]
                    \>\>\>\>\> $= 0$ \> : \begin{minipg5}
                                             Output as straight line
                                             segments through s6move()
                                             and s6line().
                                          \end{minipg5}\\[0.8ex]
\newpagetabs
        \>Input/Output Arguments:\\
        \>\>    {\fov pintcr}\> - \>  \begin{minipg2}
                     The intersection curve. When coming in as input
                       only parameter values in the parameter plane
                       exist. When coming as output the 3D geometry
                       and possibly the curve in the parameter plane
                       of the surface is added.
                               \end{minipg2}\\[0.8ex]
\\
        \>Output Arguments:\\
        \>\>    {\fov jstat}     \> - \> Status messages\\
        \>\>\>\> $= 3$ \> :
                \begin{minipg5}
                  Iteration stopped due to singular
                  point or degenerate surface. A part
                  of intersection curve may have been
                  traced out. If no curve is traced out
                  the curve pointers in the Intcurve
                  object point to NULL.
                \end{minipg5}\\[0.8ex]
        \>\>\>\> $> 0$ \>\> : Warning.\\
        \>\>\>\> $= 0$ \>\> : Ok.\\
        \>\>\>\> $< 0$ \>\> : Error.\\
        \>\>\>\> $= -185$ \>\> :
        \begin{minipg5}
          No points produced on intersection curve.
        \end{minipg5}\\[0.8ex]
\\
NOTE\\
\>      \begin{minipg6}
If the draw option is used the empty dummy functions s6move() and
s6line() are called.
Thus if the draw option is used, make sure
you have versions of s6move() and s6line() interfaced to your graphic package.
More about s6move() and s6line() on pages~\pageref{s6move}
and~\pageref{s6line}.
\end{minipg6}\\
\\ %\newpagetabs
EXAMPLE OF USE\\
        \>      \{ \\
        \>\>    SISLSurf \> *{\fov ps1};\\
        \>\>    double   \> {\fov eyepoint}[3];\\
        \>\>    int      \> {\fov idim} = 3;\\
        \>\>    double   \> {\fov aepsco};\\
        \>\>    double   \> {\fov aepsge};\\
        \>\>    double   \> {\fov amax};\\
        \>\>    SISLIntcurve \> *{\fov pintcr};\\
        \>\>    int      \> {\fov icur};\\
        \>\>    int      \> {\fov igraph};\\
        \>\>    int      \> {\fov jstat} = 0;\\
        \>\>    \ldots \\
        \>\>s1514(\begin{minipg4}
          {\fov ps1},  {\fov eyepoint},  {\fov idim},  {\fov aepsco},  {\fov aepsge},  {\fov amax},  {\fov pintcr},  {\fov icur},  {\fov igraph},  \&{\fov jstat});
        \end{minipg4}\\
        \>\>    \ldots \\
        \>      \}
\end{tabbing}

\pgsbreak
\subsection{March a circular silhouette curve of a surface.}
\funclabel{s1515}
\begin{minipg1}
  To march the circular silhouette curve described by an intersection
  curve object, a surface, point Q and direction B
  i.e.\ solution of  $f(u,v)=N(u,v)\times (P(u,v)-Q)\cdot B$.\\
  The generated geometric curves are represented as B-spline curves.
\end{minipg1}\\ \\
SYNOPSIS\\
        \>void s1515(\begin{minipg3}
          {\fov ps1},  {\fov qpoint},  {\fov bvec},  {\fov idim},  {\fov aepsco},  {\fov aepsge},  {\fov amax},  {\fov pintcr},  {\fov icur},  {\fov igraph},  {\fov jstat})
        \end{minipg3}\\[0.3ex]
        \>\>    SISLSurf \> *{\fov ps1};\\
        \>\>    double   \> {\fov qpoint}[\,];\\
        \>\>    double   \> {\fov bvec}[\,];\\
        \>\>    int      \> {\fov idim};\\
        \>\>    double   \> {\fov aepsco};\\
        \>\>    double   \> {\fov aepsge};\\
        \>\>    double   \> {\fov amax};\\
        \>\>    SISLIntcurve \> *{\fov pintcr};\\
        \>\>    int      \> {\fov icur};\\
        \>\>    int      \> {\fov igraph};\\
        \>\>    int      \> *{\fov jstat};\\
\\
ARGUMENTS\\
        \>Input Arguments:\\
        \>\>    {\fov ps1}\> - \>  \begin{minipg2}
                     Pointer to surface.
                               \end{minipg2}\\
        \>\>    {\fov qpoint}\> - \>  \begin{minipg2}
                     Point Q for circular silhouette.
                               \end{minipg2}\\
        \>\>    {\fov bvec}\> - \>  \begin{minipg2}
                     Direction B for circular silhouette.
                               \end{minipg2}\\
        \>\>    {\fov idim}\> - \>  \begin{minipg2}
                     Dimension of the space in which Q lies.
                               \end{minipg2}\\
        \>\>    {\fov aepsco}\> - \>  \begin{minipg2}
                     Computational resolution (not used).
                               \end{minipg2}\\
        \>\>    {\fov aepsge}\> - \>  \begin{minipg2}
                     Geometry resolution.
                               \end{minipg2}\\
        \>\>    {\fov amax}\> - \>  \begin{minipg2}
                     Maximal allowed step length. If $amax\leq aepsge$
                       {\fov amax} is neglected.
                               \end{minipg2}\\
        \>\>    {\fov icur}\> - \>  \begin{minipg2}
                    Indicator telling if a 3D curve is to be made.
                               \end{minipg2}\\
                    \>\>\>\>\> $= 0$ \> : Don't make 3D curve.\\
                    \>\>\>\>\> $= 1$ \> : Make 3D curve.\\
                    \>\>\>\>\> $= 2$ \> : \begin{minipg5}
                                            Make 3D curve and curves in
                                            the parameter plane.
                                          \end{minipg5}\\[0.8ex]
        \>\>    {\fov igraph}\> - \>  \begin{minipg2}
                     Indicator telling if the curve is to be output
                       through function calls:\\
                               \end{minipg2}\\
                    \>\>\>\>\> $= 0$ \> : \begin{minipg5}
                                            Don't output curve through
                                            function call.
                                          \end{minipg5}\\[0.3ex]
                    \>\>\>\>\> $= 0$ \> : \begin{minipg5}
                                             Output as straight line
                                             segments through s6move()
                                             and s6line().
                                          \end{minipg5}\\[0.8ex]
\newpagetabs
        \>Input/Output Arguments:\\
        \>\>    {\fov pintcr}\> - \>
        \begin{minipg2}
          The intersection curve. When coming in as input
          only parameter values in the parameter plane
          exist. When coming as output the 3-D geometry
          and possibly the curve in the parameter plane
          of the surface is added.
        \end{minipg2}\\[0.8ex]
\\
        \>Output Arguments:\\
        \>\>    {\fov jstat}     \> - \> Status messages\\
        \>\>\>\> $= 3$ \>\> :
                \begin{minipg5}
                  Iteration stopped due to singular
                  point or degenerate surface. A part
                  of intersection curve may have been
                  traced out. If no curve is traced out
                  the curve pointers in the Intcurve
                  object point to NULL.
                \end{minipg5}\\[0.8ex]
        \>\>\>\> $> 0$ \>\> : Warning.\\
        \>\>\>\> $= 0$ \>\> : Ok.\\
        \>\>\>\> $< 0$ \>\> : Error.\\
        \>\>\>\> $= -185$ \>\> :
        \begin{minipg5}
          No points produced on intersection curve.
        \end{minipg5}\\[0.8ex]
\\
NOTE\\
\>      \begin{minipg6}
If the draw option is used the empty dummy functions s6move() and
s6line() are called.
Thus if the draw option is used, make sure
you have versions of s6move() and s6line() interfaced to your graphic package.
More about s6move() and s6line() on pages~\pageref{s6move}
and~\pageref{s6line}.
\end{minipg6}\\
\\ %\newpagetabs
EXAMPLE OF USE\\
        \>      \{ \\
        \>\>    SISLSurf \> *{\fov ps1};\\
        \>\>    double   \> {\fov qpoint}[3];\\
        \>\>    double   \> {\fov bvec}[3];\\
        \>\>    int      \> {\fov idim};\\
        \>\>    double   \> {\fov aepsco};\\
        \>\>    double   \> {\fov aepsge};\\
        \>\>    double   \> {\fov amax};\\
        \>\>    SISLIntcurve \> *{\fov pintcr};\\
        \>\>    int      \> {\fov icur};\\
        \>\>    int      \> {\fov igraph};\\
        \>\>    int      \> {\fov jstat} = 0;\\
        \>\>    \ldots \\
        \> s1515(\begin{minipg4}
          {\fov ps1},  {\fov qpoint},  {\fov bvec},  {\fov idim},  {\fov aepsco},  {\fov aepsge},  {\fov amax},  {\fov pintcr},  {\fov icur},  {\fov igraph}, \&{\fov jstat});
        \end{minipg4}\\
        \>\>    \ldots \\
        \>      \}
\end{tabbing}

\pgsbreak
% \section{Closed or Degenerate Edges} %, moved into s1450.tex
\section{Check if a Surface is Closed or has Degenerate Edges.}
\funclabel{s1450}
\begin{minipg1}
  To check if a surface is closed or has degenerate boundaries.
  The edge numbers correspond to the  following:
\begin{center}
        \begin{picture}(180,110)(0,0)
                \put(50,15){\framebox(80,80)}
                \put(40,55){\makebox(0,0){4}}
                \put(140,55){\makebox(0,0){2}}
                \put(90,5){\makebox(0,0){1}}
                \put(90,105){\makebox(0,0){3}}

                \put(60,20){\vector(1,0){40}}
                \put(85,28){\makebox(0,0){$(i)$}}
                \put(55,25){\vector(0,1){40}}
                \put(65,50){\makebox(0,0){$(ii)$}}
        \end{picture}\\
        $(i) \; \; \;$ first parameter direction of surface.\\
        $(ii)$   second parameter direction of surface.\\
\end{center}
\end{minipg1}\\ \\
SYNOPSIS\\
        \>void s1450(\begin{minipg3}
                {\fov surf}, {\fov epsge}, {\fov close1}, {\fov close2}, {\fov degen1}, {\fov degen2}, {\fov degen3}, {\fov degen4},
                {\fov stat})
                \end{minipg3}\\[0.3ex]

                \>\>    SISLSurf        \>      *{\fov surf};\\
                \>\>    double  \>      {\fov epsge};\\
                \>\>    int     \>      *{\fov close1};\\
                \>\>    int     \>      *{\fov close2};\\
                \>\>    int     \>      *{\fov degen1};\\
                \>\>    int     \>      *{\fov degen2};\\
                \>\>    int     \>      *{\fov degen3};\\
                \>\>    int     \>      *{\fov degen4};\\
                \>\>    int     \>      *{\fov stat};\\
\\
ARGUMENTS\\
        \>Input Arguments:\\
        \>\>    {\fov surf}\> - \>              \begin{minipg2}
                                Pointer to the surface that is to be checked.
                                \end{minipg2}\\[0.3ex]
        \>\>    {\fov epsge}\> - \>     \begin{minipg2}
                                Tolerance used during testing.
                                \end{minipg2}\\
\newpagetabs
        \>Output Arguments:\\
        \>\>    {\fov close1}\> - \>    \begin{minipg2}
                                Closed indicator in the first parameter direction.
                                \end{minipg2}\\
                \>\>\>\>\>      $=0$ :\>\begin{minipg5}
                                Surface open in first direction
                                \end{minipg5}\\
                \>\>\>\>\>      $=1$ :\>\begin{minipg5}
                                Surface closed in first direction
                                \end{minipg5}\\
        \>\>    {\fov close2}\> - \>    \begin{minipg2}
                                Closed indicator in second direction
                                \end{minipg2}\\
                \>\>\>\>\>      $=0$ :\>\begin{minipg5}
                                Surface open in second direction
                                \end{minipg5}\\[0.8ex]
                \>\>\>\>\>      $=1$ :\>\begin{minipg5}
                                Surface closed in second direction
                                \end{minipg5}\\[0.8ex]
        \>\>    {\fov degen1}\> - \>    \begin{minipg2}
                                Degenerate indicator along standard edge 1
                                \end{minipg2}\\
                \>\>\>\>\>      $=0$ :\>        Edge is not degenerate\\
                \>\>\>\>\>      $=1$ :\>        Edge is degenerate\\
        \>\>    {\fov degen2}\> - \>    \begin{minipg2}
                                Degenerate indicator along standard edge 2
                                \end{minipg2}\\
                \>\>\>\>\>      $=0$ :\>        Edge is not degenerate\\
                \>\>\>\>\>      $=1$ :\>        Edge is degenerate\\
        \>\>    {\fov degen3}\> - \>    \begin{minipg2}
                                Degenerate indicator along standard edge 3
                                \end{minipg2}\\
                \>\>\>\>\>      $=0$ :\>        Edge is not degenerate\\
                \>\>\>\>\>      $=1$ :\>        Edge is degenerate\\
        \>\>    {\fov degen4}\> - \>    \begin{minipg2}
                                Degenerate indicator along standard edge 4
                                \end{minipg2}\\
                \>\>\>\>\>      $=0$ :\>        Edge is not degenerate\\
                \>\>\>\>\>      $=1$ :\>        Edge is degenerate\\
        \>\>    {\fov stat}     \> - \> Status messages\\
                \>\>\>\>\>              $> 0$   : warning\\
                \>\>\>\>\>              $= 0$   : ok\\
                \>\>\>\>\>              $< 0$   : error\\
EXAMPLE OF USE\\
                \>      \{ \\
                \>\>    SISLSurf        \>      *{\fov surf};\\
                \>\>    double  \>      {\fov epsge};\\
                \>\>    int     \>      {\fov close1};\\
                \>\>    int     \>      {\fov close2};\\
                \>\>    int     \>      {\fov degen1};\\
                \>\>    int     \>      {\fov degen2};\\
                \>\>    int     \>      {\fov degen3};\\
                \>\>    int     \>      {\fov degen4};\\
                \>\>    int     \>      {\fov stat};\\
                \>\>    \ldots \\
        \>\>s1450(\begin{minipg4}
                {\fov surf}, {\fov epsge}, \&{\fov close1}, \&{\fov close2}, \&{\fov degen1}, \&{\fov degen2}, \&{\fov degen3}, \&{\fov degen4},
                \&{\fov stat});
                        \end{minipg4}\\
                \>\>    \ldots \\
                \>      \}
\end{tabbing}

\pgsbreak
\section{Pick the Parameter Ranges of a Surface}
\funclabel{s1603}
\begin{minipg1}
  To pick the parameter ranges of a surface.
\end{minipg1} \\ \\
SYNOPSIS\\
        \>void s1603(\begin{minipg3}
                                {\fov surf}, {\fov min1}, {\fov min2}, {\fov max1}, {\fov max2}, {\fov stat})
                \end{minipg3}\\[0.3ex]

                \>\>    SISLSurf        \>      *{\fov surf};\\
                \>\>    double  \>      *{\fov min1};\\
                \>\>    double  \>      *{\fov min2};\\
                \>\>    double  \>      *{\fov max1};\\
                \>\>    double  \>      *{\fov max2};\\
                \>\>    int     \>      *{\fov stat};\\
\\
ARGUMENTS\\
        \>Input Arguments:\\
        \>\>    {\fov surf}\> - \>      \begin{minipg2}
                                The surface.
                                \end{minipg2}\\
\\
        \>Output Arguments:\\
        \>\>    {\fov min1}\> - \>      \begin{minipg2}
                                Start parameter in the first parameter direction.
                                \end{minipg2}\\
        \>\>    {\fov min2}\> - \>      \begin{minipg2}
                                Start parameter in the second parameter direction.
                                \end{minipg2}\\
        \>\>    {\fov max1}\> - \>      \begin{minipg2}
                                End parameter in the first parameter direction.
                                \end{minipg2}\\
        \>\>    {\fov max2}\> - \>      \begin{minipg2}
                                End parameter in the second parameter direction.
                                \end{minipg2}\\
        \>\>    {\fov stat}     \> - \> Status messages\\
                \>\>\>\>\>              $> 0$   : warning\\
                \>\>\>\>\>              $= 0$   : ok\\
                \>\>\>\>\>              $< 0$   : error\\
\\
EXAMPLE OF USE\\
                \>      \{ \\
                \>\>    SISLSurf        \>      *{\fov surf};\\
                \>\>    double  \>      {\fov min1};\\
                \>\>    double  \>      {\fov min2};\\
                \>\>    double  \>      {\fov max1};\\
                \>\>    double  \>      {\fov max2};\\
                \>\>    int     \>      {\fov stat};\\
                \>\>    \ldots \\
        \>\>s1603(\begin{minipg4}
                {\fov surf}, \&{\fov min1}, \&{\fov min2}, \&{\fov max1}, \&{\fov max2}, \&{\fov stat});
                        \end{minipg4}\\
                \>\>    \ldots \\
                \>      \}
\end{tabbing}

\pgsbreak
\section{Closest Points}
\subsection{Find the closest point between a surface and a point.}
\funclabel{s1954}
\begin{minipg1}
  Find the points on a surface lying closest to a given point.
\end{minipg1} \\ \\
SYNOPSIS\\
        \>void s1954(\begin{minipg3}
                        {\fov surf}, {\fov point}, {\fov dim}, {\fov epsco}, {\fov epsge}, {\fov numclopt}, {\fov pointpar},
                        {\fov numclocr}, {\fov clocurves}, {\fov stat})
                \end{minipg3}\\[0.3ex]
                \>\>    SISLSurf        \>      *{\fov surf};\\
                \>\>    double  \>      {\fov point}[\,];\\
                \>\>    int     \>      {\fov dim};\\
                \>\>    double  \>      {\fov epsco};\\
                \>\>    double  \>      {\fov epsge};\\
                \>\>    int     \>      *{\fov numclopt};\\
                \>\>    double  \>      **{\fov pointpar};\\
                \>\>    int     \>      *{\fov numclocr};\\
                \>\>    SISLIntcurve\>  ***{\fov clocurves};\\
                \>\>    int     \>      *{\fov stat};\\
\\
ARGUMENTS\\
        \>Input Arguments:\\
        \>\>    {\fov surf}\> - \>      \begin{minipg2}
                                Pointer to the surface in the closest point
                                problem.
                                \end{minipg2}\\[0.3ex]
        \>\>    {\fov point}\> - \>     \begin{minipg2}
                                The point in the closest point problem.
                                \end{minipg2}\\
        \>\>    {\fov dim}\> - \>       \begin{minipg2}
                                Dimension of the space in which the point lies.
                                \end{minipg2}\\
        \>\>    {\fov epsco}\> - \>     \begin{minipg2}
                                Computational resolution (not used).
                                \end{minipg2}\\
        \>\>    {\fov epsge}\> - \>     \begin{minipg2}
                                Geometry resolution.
                                \end{minipg2}\\
        \>Output Arguments:\\
        \>\>    {\fov numclopt}\> - \>  \begin{minipg2}
                                Number of single closest points.
                                \end{minipg2}\\
        \>\>    {\fov pointpar}\> - \>  \begin{minipg2}
                                Array containing the parameter values of the
                                single closest points in the parameter area of
                                the surface. The points lie in sequence. Closest
                                curves are stored in clocurves.
                                \end{minipg2}\\[0.8ex]
        \>\>    {\fov numclocr}\> - \>  \begin{minipg2}
                                Number of closest curves.
                                \end{minipg2}\\
        \>\>    {\fov clocurves}\> - \> \begin{minipg2}
                                Array containing the description of the closest
                                curves. The curves are only described by points
                                in the parameter area. The curve pointers
                                point to nothing.
                                \end{minipg2}\\[0.3ex]
        \>\>    {\fov stat}     \> - \> Status messages\\
                \>\>\>\>\>              $> 0$   : warning\\
                \>\>\>\>\>              $= 0$   : ok\\
                \>\>\>\>\>              $< 0$   : error\\
\newpagetabs
EXAMPLE OF USE\\
                \>      \{ \\
                \>\>    SISLSurf        \>      *{\fov surf};\\
                \>\>    double  \>      {\fov point}[3];\\
                \>\>    int     \>      {\fov dim} = 3;\\
                \>\>    double  \>      {\fov epsco};\\
                \>\>    double  \>      {\fov epsge};\\
                \>\>    int     \>      {\fov numclopt};\\
                \>\>    double  \>      *{\fov pointpar};\\
                \>\>    int     \>      {\fov numclocr};\\
                \>\>    SISLIntcurve\>  **{\fov clocurves};\\
                \>\>    int     \>      {\fov stat};\\
                \>\>    \ldots \\
        \>\>s1954(\begin{minipg4}
                {\fov surf}, {\fov point}, {\fov dim}, {\fov epsco}, {\fov epsge}, \&{\fov numclopt}, \&{\fov pointpar},
        \&{\fov numclocr}, \&{\fov clocurves}, \&{\fov stat});
                        \end{minipg4}\\
                \>\>    \ldots \\
                \>      \}
\end{tabbing}

\pgsbreak
\subsection{Find the closest point between a surface and a point.
Simple version.}
\funclabel{s1958}
\begin{minipg1}
  Find the closest point between a surface and a point.
  The method is fast and should work well in clear cut cases, but there is
  no guarantee it will find the right solution. As long as it doesn't
  fail, it will find exactly one point.  In other cases, use s1954() on
  page~\pageref{s1954}.
\end{minipg1}\\ \\
SYNOPSIS\\
        \>void s1958(\begin{minipg3}
          {\fov psurf}, {\fov epoint}, {\fov idim}, {\fov aepsco}, {\fov aepsge}, {\fov gpar}, {\fov dist}, {\fov jstat})
        \end{minipg3}\\[0.3ex]
        \>\>    SISLSurf \> *{\fov psurf};\\
        \>\>    double   \> {\fov epoint}[\,];\\
        \>\>    int      \> {\fov idim};\\
        \>\>    double   \> {\fov aepsco};\\
        \>\>    double   \> {\fov aepsge};\\
        \>\>    double   \> {\fov gpar}[\,];\\
        \>\>    double   \> *{\fov dist};\\
        \>\>    int    \>  *{\fov jstat};\\
\\
ARGUMENTS\\
        \>Input Arguments:\\
        \>\>    {\fov psurf}\> - \>  \begin{minipg2}
                     Pointer to the surface in the closest point problem.
                               \end{minipg2}\\
        \>\>    {\fov epoint}\> - \>  \begin{minipg2}
                     The point in the closest point problem.
                               \end{minipg2}\\
        \>\>    {\fov idim}\> - \>  \begin{minipg2}
                     Dimension of the space in which epoint lies.
                               \end{minipg2}\\
        \>\>    {\fov aepsco}\> - \>  \begin{minipg2}
                     Computational resolution (not used).
                               \end{minipg2}\\
        \>\>    {\fov aepsge}\> - \>  \begin{minipg2}
                     Geometry resolution.
                               \end{minipg2}\\
\\
        \>Output Arguments:\\
        \>\>    {\fov gpar}\> - \>  \begin{minipg2}
                     2D array containing the parameter values of the
                       closest point in the parameter space
                       of the surface.
                               \end{minipg2}\\
        \>\>    {\fov dist}\> - \>  \begin{minipg2}
                     The closest distance between point and the surface.
                               \end{minipg2}\\
        \>\>    {\fov jstat}     \> - \> Status messages\\
                \>\>\>\>              $> 2$ \> : Warning.\\
                \>\>\>\>              $= 2$ \> : Solution at a corner.\\
                \>\>\>\>              $= 1$ \> : Solution at an edge.\\
                \>\>\>\>              $= 0$ \> : Solution in interior.\\
                \>\>\>\>              $< 0$ \> : Error.\\
\newpagetabs
EXAMPLE OF USE\\
        \>      \{ \\
        \>\>    SISLSurf  \>  *{\fov psurf};\\
        \>\>    double    \>  {\fov epoint}[3];\\
        \>\>    int       \>  {\fov idim} = 3;\\
        \>\>    double    \>  {\fov aepsco};\\
        \>\>    double    \>  {\fov aepsge};\\
        \>\>    double    \>  {\fov gpar}[2];\\
        \>\>    double    \>  {\fov dist} = 0;\\
        \>\>    int       \>  {\fov jstat} = 0;\\
        \>\>    \ldots \\
        \>\>s1958(\begin{minipg4}
          {\fov psurf}, {\fov epoint}, {\fov idim}, {\fov aepsco}, {\fov aepsge}, {\fov gpar}, \&{\fov dist}, \&{\fov jstat});
        \end{minipg4}\\
        \>\>    \ldots \\
        \>      \}
\end{tabbing}

\pgsbreak
\subsection{Local iteration to closest point bewteen point and surface.}
\funclabel{s1775}
\begin{minipg1}
Newton iteration on the distance function between
               a surface and a point, to find a closest point or an
               intersection point.
               If a bad choice for the guess parameters is given in, the
               iteration may end at a local, not global closest point.
\end{minipg1} \\ \\
SYNOPSIS\\
        \> void s1775(\begin{minipg3}
            {\fov surf},  {\fov point},  {\fov dim},  {\fov epsge},  {\fov start},  {\fov end},  {\fov guess},  {\fov clpar},  {\fov stat})
                \end{minipg3}\\
                \>\>    SISLSurf \> *{\fov surf};\\
                \>\>    double \> {\fov point}[\,];\\
                \>\>    int \> {\fov dim};\\
                \>\>    double \> {\fov epsge};\\
                \>\>    double \> {\fov start}[\,];\\
                \>\>    double \> {\fov end}[\,];\\
                \>\>    double \> {\fov guess}[\,];\\
                \>\>    double \> {\fov clpar}[\,];\\
                \>\>    int \> *{\fov stat};\\
\\
ARGUMENTS\\
	\>Input Arguments:\\
        \>\>    {\fov surf}\> - \>  \begin{minipg2}
                     The surface in the closest point problem.
                               \end{minipg2}\\
        \>\>    {\fov point}\> - \>  \begin{minipg2}
                     The point in the closest point problem.
                               \end{minipg2}\\
        \>\>    {\fov dim}\> - \>  \begin{minipg2}
                     Dimension of the geometry.
                               \end{minipg2}\\
        \>\>    {\fov epsge}\> - \>  \begin{minipg2}
                     Geometry resolution.
                               \end{minipg2}\\
        \>\>    {\fov start}\> - \>  \begin{minipg2}
                     Surface parameters giving the start of the search
                        area (umin, vmin).
                               \end{minipg2}\\
        \>\>    {\fov end}\> - \>  \begin{minipg2}
                     Surface parameters giving the end of the search
                        area (umax, vmax).
                               \end{minipg2}\\
        \>\>    {\fov guess}\> - \>  \begin{minipg2}
                     Surface guess parameters for the closest point
                        iteration.
                               \end{minipg2}\\
\\
	\>Output Arguments:\\
        \>\>    {\fov clpar}\> - \>  \begin{minipg2}
                     Resulting surface parameters from the iteration.
                               \end{minipg2}\\
        \>\>    {\fov stat}     \> - \> Status messages\\
                \>\>\>\>\>              $> 0$   : A minimum distance found.\\
                \>\>\>\>\>              $= 0$   : Intersection found.\\
                \>\>\>\>\>              $< 0$   : Error.\\
\\
EXAMPLE OF USE\\
		\>      \{ \\

                \>\>    SISLSurf \> *{\fov surf};\\
                \>\>    double \> {\fov point}[\,];\\
                \>\>    int \> {\fov dim};\\
                \>\>    double \> {\fov epsge};\\
                \>\>    double \> {\fov start}[\,];\\
                \>\>    double \> {\fov end}[\,];\\
                \>\>    double \> {\fov guess}[\,];\\
                \>\>    double \> {\fov clpar}[\,];\\
                \>\>    int \> *{\fov stat};\\                \>\>    \ldots \\
        \>\>s1775(\begin{minipg4}
            {\fov surf},  {\fov point},  {\fov dim},  {\fov epsge},  {\fov start},  {\fov end},  {\fov guess},  {\fov clpar},  {\fov stat});
                \end{minipg4}\\
                \>\>    \ldots \\
		\>      \}
\end{tabbing}

\pgsbreak
% \section{Calculation of Absolute Extremals on a NURBS Surface} %,
% moved into s1921.tex
\section{Find the Absolute Extremals of a Surface.}
\funclabel{s1921}
\begin{minipg1}
  Find the absolute extremal points/curves of a surface along a given direction.
\end{minipg1}\\ \\
SYNOPSIS\\
        \>void s1921(\begin{minipg3}
          {\fov ps1}, {\fov edir}, {\fov idim}, {\fov aepsco}, {\fov aepsge}, {\fov jpt}, {\fov gpar}, {\fov jcrv}, {\fov wcurve}, {\fov jstat})
        \end{minipg3}\\[0.3ex]
        \>\>    SISLSurf  \>  *{\fov ps1};\\
        \>\>    double    \>  {\fov edir}[\,];\\
        \>\>    int       \>  {\fov idim};\\
        \>\>    double    \>  {\fov aepsco};\\
        \>\>    double    \>  {\fov aepsge};\\
        \>\>    int       \>  *{\fov jpt};\\
        \>\>    double    \>  **{\fov gpar};\\
        \>\>    int       \>  *{\fov jcrv};\\
        \>\>    SISLIntcurve \> ***{\fov wcurve};\\
        \>\>    int       \>  *{\fov jstat};\\
\\
ARGUMENTS\\
        \>Input Arguments:\\
        \>\>    {\fov ps1}\> - \>  \begin{minipg2}
                     Pointer to the surface.
                               \end{minipg2}\\
        \>\>    {\fov edir}\> - \>
        \begin{minipg2}
          The direction in which the extremal point(s)
          and/or interval(s) are to be calculated. If
          $idim=1$ a positive value indicates the maximum
          of the function and a negative value
          the minimum. If the dimension is greater that
          1 the array contains the coordinates of the
          direction vector.
        \end{minipg2}\\[0.8ex]
        \>\>    {\fov idim}\> - \>  \begin{minipg2}
                     Dimension of the space in which the vector {\fov edir}
                       lies.
                               \end{minipg2}\\
        \>\>    {\fov aepsco}\> - \>  \begin{minipg2}
                     Computational resolution (not used).
                               \end{minipg2}\\
        \>\>    {\fov aepsge}\> - \>  \begin{minipg2}
                     Geometry resolution.
                               \end{minipg2}\\
\\
        \>Output Arguments:\\
        \>\>    {\fov jpt}\> - \>  \begin{minipg2}
                     Number of single extremal points.
                               \end{minipg2}\\
        \>\>    {\fov gpar}\> - \>
        \begin{minipg2}
          Array containing the parameter values of the
          single extremal points in the parameter
          area of the surface. The points lie continuous.
          Extremal curves are stored in {\fov wcurve}.
        \end{minipg2}\\[0.8ex]
        \>\>    {\fov jcrv}\> - \>  \begin{minipg2}
                     Number of extremal curves.
                               \end{minipg2}\\
        \>\>    {\fov wcurve}\> - \>
        \begin{minipg2}
          Array containing descriptions of the extremal
          curves. The curves are only described by points
          in the parameter area. The curve-pointers point
          to nothing.
        \end{minipg2}\\[0.8ex]
\newpagetabs
        \>\>    {\fov jstat}     \> - \> Status messages\\
                \>\>\>\>              $> 0$ \> : Warning.\\
                \>\>\>\>              $= 0$ \> : Ok.\\
                \>\>\>\>              $< 0$ \> : Error.\\
\\
EXAMPLE OF USE\\
        \>      \{ \\
        \>\>    SISLSurf  \>  *{\fov ps1};\\
        \>\>    double    \>  {\fov edir}[3];\\
        \>\>    int       \>  {\fov idim} = 3;\\
        \>\>    double    \>  {\fov aepsco};\\
        \>\>    double    \>  {\fov aepsge};\\
        \>\>    int       \>  {\fov jpt} = 0;\\
        \>\>    double    \>  *{\fov gpar} = NULL;\\
        \>\>    int       \>  {\fov jcrv} = 0;\\
        \>\>    SISLIntcurve \> **{\fov wcurve} = NULL;\\
        \>\>    int       \>  {\fov jstat} = 0;\\
        \>\>    \ldots \\
        \>\>s1921(\begin{minipg4}
          {\fov ps1}, {\fov edir}, {\fov idim}, {\fov aepsco}, {\fov aepsge}, \&{\fov jpt}, \&{\fov gpar}, \&{\fov jcrv}, \&{\fov wcurve}, \&{\fov jstat});
        \end{minipg4}\\
        \>\>    \ldots \\
        \>      \}
\end{tabbing}

\pgsbreak
\section{Bounding Box}
Both curves and surfaces have bounding boxes. These are boxes surrounding an object not only parallel to the main axis, but also rotated 45 degrees around each main axis. These bounding boxes are used by the intersection functions to decide if an intersection is possible or not. They might also be used to find the position of objects under other circumstances.


\subsection{Bounding box object.}

In the library a bounding box is stored in a struct SISLbox
containing the following:
\typelabel{SISLBox}
 \> double      \>*emax;        \>\> \begin{minipg2}
                                Allocated array containing
                                the minimum values of the bounding box
                                \end{minipg2}\\[0.3ex]
 \> double      \>*emin;        \>\> \begin{minipg2}
                                Allocated array containing
                                the maximum values of the bounding box
                                \end{minipg2}\\[0.3ex]
 \> int         \>imin;         \>\> \begin{minipg2}
                                The index of the minimum coefficient
                                {\fov ecoef}[{\fov imin}].
                                Only used in dimension one.
                                {\fov ecoef} is the control polygon of
                                the curve/surface.
                                \end{minipg2}\\[0.3ex]
 \> int         \>imax;         \>\> \begin{minipg2}
                                The index of the maximum coefficient
                                {\fov ecoef}[{\fov imax}].
                                Only used in dimension one.
                                {\fov ecoef} is the control polygon of
                                the curve/surface.
                                \end{minipg2}\\[0.3ex]
\end{tabbing}

\pgsbreak
\subsection{Create and initialize a curve/surface bounding box instance.}
\funclabel{newbox}
\begin{minipg1}
Create and initialize a curve/surface bounding box instance.
\end{minipg1} \\ \\
SYNOPSIS\\
        \>SISLbox *newbox(\begin{minipg3}
          {\fov idim})
        \end{minipg3}\\
        \>\>    int \> {\fov idim};\\
\\
ARGUMENTS\\
        \>Input Arguments:\\
        \>\>    {\fov idim}\> - \>
        \begin{minipg2}
          Dimension of geometry space.
        \end{minipg2}\\
\\
        \>Output Arguments:\\
        \>\>    {\fov newbox}\> - \>
        \begin{minipg2}
          Pointer to new SISLbox structure. If it is
          impossible to allocate space for the structure,
          newbox will return a NULL value.
        \end{minipg2}\\
\\
EXAMPLE OF USE\\
        \>      \{ \\
        \>\>int   \> {\fov idim};\\
        \>\>SISLbox *{\fov box};\\
        \>\>    \ldots \\
        \>\>{\fov box} = newbox(\begin{minipg4}
          {\fov idim});
        \end{minipg4}\\
        \>\>    \ldots \\
        \>      \}
\end{tabbing}



\pgsbreak
\subsection{Find the bounding box of a surface.}
\funclabel{s1989}
\begin{minipg1}
  Find the bounding box of a surface.\\
  NOTE: The geometric
  bounding box is returned also in the rational case, that
  is the box in homogeneous coordinates is NOT computed.
\end{minipg1} \\ \\
SYNOPSIS\\
        \>void s1989(\begin{minipg3}
          {\fov ps}, {\fov emax}, {\fov emin}, {\fov jstat})
        \end{minipg3}\\[0.3ex]
        \>\>    SISLSurf \> *{\fov ps};\\
        \>\>    double   \> **{\fov emax};\\
        \>\>    double   \> **{\fov emin};\\
        \>\>    int      \> *{\fov jstat};\\
\\
ARGUMENTS\\
        \>Input Arguments:\\
        \>\>    {\fov ps}\> - \>  \begin{minipg2}
                     Surface to treat.
                               \end{minipg2}\\
\\
        \>Output Arguments:\\
        \>\>    {\fov emin}\> - \>  \begin{minipg2}
                     Array of dimension {\fov idim} containing
                          the minimum values of the bounding box,
                          i.e.\ bottom-left corner of the box.
                               \end{minipg2}\\[0.8ex]
        \>\>    {\fov emax}\> - \>  \begin{minipg2}
                     Array of dimension {\fov idim} containing
                          the maximum values of the bounding box,
                          i.e.\ upper-right corner of the box.
                               \end{minipg2}\\[0.8ex]
        \>\>    {\fov jstat}     \> - \> Status messages\\
                \>\>\>\>              $> 0$ \> : Warning.\\
                \>\>\>\>              $= 0$ \> : Ok.\\
                \>\>\>\>              $< 0$ \> : Error.\\
\\
EXAMPLE OF USE\\
        \>      \{ \\
        \>\>    SISLSurf \> *{\fov ps};\\
        \>\>    double   \> *{\fov emax} = NULL;\\
        \>\>    double   \> *{\fov emin} = NULL;\\
        \>\>    int      \> {\fov jstat} = 0;\\
        \>\>    \ldots \\
        \>\>s1989(\begin{minipg4}
          {\fov ps}, \&{\fov emax}, \&{\fov emin}, \&{\fov jstat});
        \end{minipg4}\\
        \>\>    \ldots \\
        \>      \}
\end{tabbing}

\pgsbreak
\section{Normal Cone}
Both curves and surfaces have normal cones. These are the cones that are convex hull of all normalized tangents of a curve and all normalized normals of a surface.

These normal cones are used by the intersection functions to decide if only one intersection is possible. They might also be used to find directions of objects for other reasons.


\subsection{Normal cone object.}

In the library a direction cone is stored in a struct SISLdir
containing the following:

\typelabel{SISLdir}
 \> int         \>{\fov igtpi}; \>\> \begin{minipg2}
                                To mark if the angle of direction cone
                                is greater than $\pi$.
                                \end{minipg2}\\
                \>\>\>\>\>      $= 0$   :\>\hspace*{0.3em}\begin{minipg5}
                                The direction of a surface
                                and its boundary curves or a curve
                                is not greater than $\pi$ in any
                                parameter direction.
                                \end{minipg5}\\[0.3ex]
                \>\>\>\>\>      $= 1$   :\>\hspace*{0.3em}\begin{minipg5}
                                The direction of a surface or a curve
                                is greater than $\pi$ in the first
                                parameter direction.
                                \end{minipg5}\\[0.8ex]
                \>\>\>\>\>      $= 2$   :\>\hspace*{0.3em}\begin{minipg5}
                                The angle of direction cone of a surface is greater
                                than $\pi$ in the second parameter
                                direction.
                                \end{minipg5}\\[0.8ex]
                \>\>\>\>\>      $= 10:$\>\hspace*{0.3em}\begin{minipg5}
                                The angle of direction cone of a boundary curve
                                in first parameter direction of a
                                surface is greater than $\pi$.
                                \end{minipg5}\\[0.8ex]
                \>\>\>\>\>      $= 20:$\>\hspace*{0.3em}\begin{minipg5}
                                The angle of direction cone of a boundary curve
                                in second parameter direction of a
                                surface is greater than $\pi$.
                                \end{minipg5}\\[0.3ex]
 \> double      \>*{\fov ecoef};        \>\> \begin{minipg2}
                                Allocated array containing
                                the coordinates of the centre of the cone.
                                \end{minipg2}\\[0.3ex]
 \> double      \>{\fov aang};          \>\> \begin{minipg2}
                                The angle from the centre which
                                describes the cone.
                                \end{minipg2}\\[0.3ex]
\end{tabbing}

\pgsbreak
\subsection{Create and initialize a curve/surface direction instance.}
\funclabel{newdir}
\begin{minipg1}
Create and initialize a curve/surface direction instance.
\end{minipg1} \\ \\
SYNOPSIS\\
        \>SISLdir *newdir(\begin{minipg3}
          {\fov idim})
        \end{minipg3}\\[0.3ex]
        \>\>    int \> {\fov idim};\\
\\
ARGUMENTS\\
        \>Input Arguments:\\
        \>\>    {\fov idim}\> - \>
        \begin{minipg2}
          Dimension of the space in which the object lies.
        \end{minipg2}\\
\\
        \>Output Arguments:\\
        \>\>    {\fov newdir}\> - \>
        \begin{minipg2}
          Pointer to new direction structure. If it is
          impossible to allocate space for the structure,
          newdir will return a NULL value.
        \end{minipg2}\\
\\
EXAMPLE OF USE\\
        \>      \{ \\
        \>\>int     \> {\fov idim};\\
        \>\>SISLdir \> *{\fov dir};\\
        \>\>    \ldots \\
        \>\>{\fov dir} = newdir(\begin{minipg4}
          {\fov idim});
        \end{minipg4}\\
        \>\>    \ldots \\
        \>      \}
\end{tabbing}



\pgsbreak
\subsection{Find the direction cone of a surface.}
\funclabel{s1987}
\begin{minipg1}
  Find the direction cone of a surface.
\end{minipg1} \\ \\
SYNOPSIS\\
        \>void s1987(\begin{minipg3}
          {\fov ps}, {\fov aepsge}, {\fov jgtpi}, {\fov gaxis}, {\fov cang}, {\fov jstat})
        \end{minipg3}\\[0.ex]
        \>\>    SISLSurf \> *{\fov ps};\\
        \>\>    double   \> {\fov aepsge};\\
        \>\>    int      \> *{\fov jgtpi};\\
        \>\>    double   \> **{\fov gaxis};\\
        \>\>    double   \> *{\fov cang};\\
        \>\>    int      \> *{\fov jstat};\\
\\
ARGUMENTS\\
        \>Input Arguments:\\
        \>\>    {\fov ps}\> - \>  \begin{minipg2}
                     Surface to treat.
                               \end{minipg2}\\
        \>\>    {\fov aepsge}\> - \>  \begin{minipg2}
                     Geometry tolerance.
                               \end{minipg2}\\
\\
        \>Output Arguments:\\
        \>\>    {\fov jgtpi}\> - \>
        \begin{minipg2}
          To mark if the angle of the direction cone is
          greater than $\pi$.
        \end{minipg2}\\[0.8ex]
                \>\>\>\> $=0$ \> :
                \begin{minipg5}
                  The direction cone of the surface
                  is not greater than $\pi$ in any parameter direction.
                \end{minipg5}\\[0.8ex]
                \>\>\>\> $=1$ \> :
                \begin{minipg5}
                  The direction cone of the surface
                  is greater than $\pi$ in the first parameter direction.
                \end{minipg5}\\[0.8ex]
                \>\>\>\> $=2$ \> :
                \begin{minipg5}
                  The direction cone of the surface is greater
                  than $\pi$ in the second parameter direction.
                \end{minipg5}\\[0.8ex]
                \>\>\>\> $=10$ \> :
                \begin{minipg5}
                  The direction cone of a boundary curve of
                  the surface is greater than $\pi$ in the first
                  parameter direction.
                \end{minipg5}\\[0.8ex]
                \>\>\>\> $=20$ \> :
                \begin{minipg5}
                  The direction cone of a boundary curve of
                  the surface is greater than $\pi$ in the second
                  parameter direction.
                \end{minipg5}\\[0.8ex]
        \>\>    {\fov gaxis}\> - \>
        \begin{minipg2}
          Allocated array containing the coordinates of the
          centre of the cone. It is only computed if $jgtpi=0$.
        \end{minipg2}\\[0.8ex]
        \>\>    {\fov cang}\> - \>
        \begin{minipg2}
          The angle from the centre to the boundary of the
          cone. It is only computed if $jgtpi=0$.
        \end{minipg2}\\[0.8ex]
        \>\>    {\fov jstat}     \> - \> Status messages\\
                \>\>\>\>              $> 0$ \> : Warning.\\
                \>\>\>\>              $= 0$ \> : Ok.\\
                \>\>\>\>              $< 0$ \> : Error.\\
\newpagetabs
EXAMPLE OF USE\\
        \>      \{ \\
        \>\>    SISLSurf \> *{\fov ps};\\
        \>\>    double   \> {\fov aepsge};\\
        \>\>    int      \> {\fov jgtpi} = 0;\\
        \>\>    double   \> *{\fov gaxis} = NULL;\\
        \>\>    double   \> {\fov cang} = 0.0;\\
        \>\>    int      \> {\fov jstat} = 0;\\
        \>\>    \ldots \\
        \>\>s1987(\begin{minipg4}
          {\fov ps}, {\fov aepsge}, \&{\fov jgtpi}, \&{\fov gaxis}, \&{\fov cang}, \&{\fov jstat});
        \end{minipg4}\\
        \>\>    \ldots \\
        \>      \}
\end{tabbing}

